% !TeX root = ../main-conf.tex
\section{Hypergraphs for traced commutative comonoid categories}

\begin{figure}
    \centering
    \begin{minipage}{0.2\textwidth}
        \begin{equation}
            \tag{\(\mathsf{C1}\)}
            \iltikzfig{strings/structure/comonoid/unitality-l-lhs}
            =
            \iltikzfig{strings/structure/comonoid/unitality-l-rhs}
            \label{eq:comonoid-unitality-l}
        \end{equation}
    \end{minipage}
    \begin{minipage}{0.22\textwidth}
        \begin{equation}
            \tag{\(\mathsf{C2}\)}
            \iltikzfig{strings/structure/comonoid/unitality-r-lhs}
            =
            \iltikzfig{strings/structure/comonoid/unitality-r-rhs}
            \label{eq:comonoid-unitality-r}
        \end{equation}
    \end{minipage}
    \begin{minipage}{0.27\textwidth}
        \begin{equation}
            \tag{\(\mathsf{C3}\)}
            \iltikzfig{strings/structure/comonoid/associativity-lhs}
            =
            \iltikzfig{strings/structure/comonoid/associativity-rhs}
            \label{eq:comonoid-associativity}
        \end{equation}
    \end{minipage}
    \begin{minipage}{0.27\textwidth}
        \begin{equation}
            \tag{\(\mathsf{C4}\)}
            \iltikzfig{strings/structure/comonoid/commutativity-lhs}
            =
            \iltikzfig{strings/structure/comonoid/commutativity-rhs}
            \label{eq:comonoid-commutativity}
        \end{equation}
    \end{minipage}
    \caption{Equations of \(\ccomon\).}
    \label{fig:comonoid-equations}
\end{figure}

In addition to the trace, there is another element of structure we are
interested in: the ability to \emph{copy} and \emph{discard} wires.
This is known as a \emph{(commutative) comonoid structure}: categories equipped
with such a structure are also known as \emph{gs-monoidal}
(\emph{garbage-sharing}) categories~\cite{fritz2022free}.

\begin{definition}
    Let \((\generators[\ccomon], \equations[\ccomon])\) be the symmetric
    monoidal theory of \emph{commutative comonoids}, with \(\Sigma_\ccomon := \{
        \iltikzfig{strings/structure/comonoid/copy}[white],
        \iltikzfig{strings/structure/comonoid/discard}[white]
    \}\) and \(\mce_\ccomon\) defined as the equations in
    \cref{fig:comonoid-equations}.
    We write \(\ccomon := \smc{\generators[\ccomon], \equations[\ccomon]}\).
\end{definition}

From now on, we write `comonoid' to mean `commutative comonoid'.
There has already been work into hypergraphs at the `halfway point' between
symmetric monoidal and Frobenius terms: \cite{fritz2022free} examines a general
construction for constructing free gs-categories as hypergraphs, and
\cite{milosavljevic2022string} presensts results for rewriting with a
commutative \emph{monoid} structure.
However, both of these papers consider \emph{acyclic} hypergraphs: we must
ensure that removing the acyclicity condition does not lead to any degeneracies.

\begin{definition}[Partial left-monogamy]
    For a cospan \(\cospan{m}[f]{H}[g]{n}\), we say it is
    \emph{partial left-monogamous} if \(f\) is mono and, for all nodes
    \(v \in H_\star\), the degree of \(v\) is \((0,m)\) if \(v \in f_\star\) and
    \((0,m)\) or \((1,m)\) otherwise, for some \(m \in \nat\).
\end{definition}

\begin{remark}
    As with the vertices not in the interfaces with degree \((0, 0)\) in the
    vanilla traced case, the vertices not in the interface with degree
    \((0, m)\) allow for terms such as \(
        \trace{}{\iltikzfig{strings/structure/comonoid/copy}[white]}
    \).
\end{remark}

\begin{lemma}
    \(\pmcspdhyp\) is a sub-prop of \(\plmcspdhyp\).
\end{lemma}

\iftoggle{conf}{}{
    \begin{lemma}
        \label{lem:trace-in-degree}
        Given a partial left-monogamous cospan \(\cospan{x+m}[f+h]{K}[g+k]{x+n}\)
        and its trace \(\cospan{m}[h \seq p]{pK}[k \seq p]{n}\), let vertices
        \(v_0, v_1, \cdots, v_n\) such that each \(v_i\) is in the image of \(g\)
        and \(p(v_0) = p(v_1) = \cdots = p(v_n)\).
        Then, there exists at most one \(v_i\) with in-degree \(1\).
    \end{lemma}
    \iftoggle{proofs}{
        \begin{proof}
            Assume that there exist vertices \(g(i),g(j)\) with in-degree \(1\).
            For \(p(g(i)) = p(g(j))\) to hold, then there must either exist a sequence
            \(f(i) = g(i_0), f(i_0) = g(i_1), \cdots, f(i_n) = g(j)\), or vice versa.
            But \(f(i_n) = g(j)\) must have in-degree \(0\) by partial left-monogamy, a
            contradiction.
            Therefore at most one \(v_i\) can have in-degree \(1\).
        \end{proof}
    }{}
}

\begin{lemma}
    \label{lem:partial-monogamous-ops}
    Let \(\cospan{m}{F}{n}\), \(\cospan{n}{G}{p}\), \(\cospan{p}{H}{q}\) and
    \(\cospan{x+m}{K}{x+n}\) be partial left-monogamous cospans.
    Then,
    \begin{itemize}
        \item Identities and symmetries are partial left-monogamous.
        \item \(\cospan{m}[f]{F}[g]{n} \seq \cospan{n}[h]{G}[k]{p}\) is partial
                left-monogamous.
        \item \(\cospan{m}{F}{n} \tensor \cospan{p}{H}{q}\) is partial
                left-monogamous.
        \item \(
                \trace{x}{\cospan{x+m}[f+h]{K}[g+k]{x+n}}
                =
                \cospan{m}[h \seq p]{pK}[k \seq p]{n}
              \) is partial left-monogamous.
    \end{itemize}
\end{lemma}
\iftoggle{proofs}{
    \begin{proof}
        Identities and symmetries are monogamous, and as such they are also
        partial left-monogamous.
        For composition, the vertices in the image of \(g\) and \(h\) are
        identified.
        Let \(v = p(g(i)) = p(h(i))\).
        We must show that \(v\) has in-degree \(0\) if it is in the image of
        \(f\), and \(0\) or \(1\) otherwise.
        \(h(i)\) has in-degree \(1\) by definition, so the in-degree of \(v\) is
        entirely determined by \(g(i)\).
        If \(v\) is in the image of \(f\), then \(g(i)\) must also be in the
        image of \(f\), so it has in-degree \(0\), and hence so does \(v\).
        Conversely, if \(v\) is not in the image of \(f\), \(g(i)\) has
        in-degree of either \(0\) or \(1\), and hence so does \(v\).

        For tensor, the elements of the original two graphs are unaffected so
        the degrees remain unchanged.

        For trace, let \(v = p(f(i)) = p(g(i))\).
        \(v\) cannot be in the image of \(h \seq p\) as this would mean that
        \(f + h\) is not mono.
        Therefore we must show that \(v\) has either degree of either \((0,m)\)
        or \((1,m)\).
        The degree of \(v\) is the sum of the degrees of each \(v_{fi}\) and
        \(v_{gi}\).
        Let \(v_{f0},\cdots,v_{fn}\) be the vertices in the image of \(f\) such
        that \(p(v_{fi}) = v\), and similarly for \(v_{gi}\).
        The in-degree of each \(v_{fi}\) must be \(0\) so all the in-degree is
        contributed by each \(v_{gi}\).
        By \cref{lem:trace-in-degree}, at most \(v_{fi}\) can have in-degree
        \(1\), so the in-degree of \(v\) can either be \(0\) or \(1\).
        Therefore the cospan is partial left-monogamous.
    \end{proof}
}{}

Therefore cospans of partial left-monogamous hypergraphs form a category, which
we name \(\plmcspfihyp\).
This category can be equipped with a comonoid structure.

\iftoggle{conf}{
    \begin{definition}
        Let \(\morph{\comonoidtohyp{\Sigma}}{\ccomon}{\cspdhyp}\) be the
        restriction of \(\frobtohyp{\Sigma}\) from \cref{def:hyp-morphisms} to the
        generators in \(\generators[\ccomon]\).
    \end{definition}
}{
    \begin{proposition}[\cite{lack2004composing}, Example 5.2]
        \label{prop:ccomon-iso-finsetop}
        \(\ccomon \cong \op{\finset}\).
    \end{proposition}

    \iftoggle{conf}{}{
        The following is a corollary of \cref{thm:cospan-homomorphism}.
    }

    \begin{corollary}
        \label{cor:prop-homomorphism-finset}
        There is a faithful prop homomorphism \(
            \morph{\tilde{D}}{\csp{\op{\finset}}}{\csp[D]{\hypsigma}}
        \).
    \end{corollary}

    \begin{definition}
        Let \(\morph{\comonoidtohyp{\Sigma}}{\ccomon}{\cspdhyp}\) be the homomorphism
        obtained by composing the isomorphism of \cref{prop:ccomon-iso-finsetop}
        with the homomorphism of \cref{cor:prop-homomorphism-finset}.
        Concretely, it is defined on objects in the obvious way and on morphisms as
        \(
            \comonoidtohyp[\iltikzfig{strings/structure/comonoid/copy}[white]]
            :=
            \cospan{1}{1}{2}
        \) and \(
            \comonoidtohyp[\iltikzfig{strings/structure/comonoid/discard}[white]]
            :=
            \cospan{1}{1}{0}
        \).
    \end{definition}
}

\begin{lemma}
    The image of \(\comonoidtohyp{\Sigma}\) is in \(\plmcspdhyp\).
\end{lemma}
\iftoggle{proofs}{
    \begin{proof}
        By definition.
    \end{proof}
}{}

\begin{definition}
    Let \(\morph{\termandfrobtohypsigma}{\stmcsigma + \ccomon}{\plmcspfihyp}\) be
    defined as the copairing of \(\tracedtosymandfrob\) and \(\comonoidtohyp{\Sigma}\).
\end{definition}

\begin{lemma}\label{lem:ccomon-term}
    Given a left-monogamous cospan \(\cospan{m}[f]{m}[g]{n}\), there exists a
    unique term \(\morph{h}{m}{n} \in \ccomon\) up to the axioms of SMCs such
    that \(\comonoidtohyp{\Sigma}\) = \(\cospan{m}[f]{m}[g]{n}\).
\end{lemma}
\iftoggle{proofs}{
    \begin{proof}

    \end{proof}
}{}

\begin{theorem}
    \(\stmcsigma + \ccomon \cong \plmcspfihyp\).
\end{theorem}
\begin{proof}
    Since \(\termandfrobtohypsigma\) and \(\comonoidtohyp{\Sigma}\) are faithful,
    it suffices to show that every cospan \(\cospan{m}{F}{n} \in \plmcspfihyp\)
    can be decomposed in such a way that each component is in the image of
    either \(\termandfrobtohypsigma\) or \(\comonoidtohyp{\Sigma}\).
    This is achieved by taking the construction of \cref{thm:termtohyp-image}
    and allowing the component made of symmetries and identities to also contain
    the copy and discard generator: by \cref{lem:ccomon-term} a term in
    \(\ccomon\) can be retrieved from this.
\end{proof}
