% !TeX root = ../main-arxiv.tex
\section{Hypergraphs for traced commutative comonoid categories}

\begin{figure}
    \centering
    \includestandalone{figures/graphs/roadmap}
    \caption{The various PROP morphisms at play.}
    \label{fig:roadmap}
\end{figure}


We are interested in another element of structure in addition to the trace: the
ability to \emph{copy} and \emph{discard} wires.
This is known as a \emph{(commutative) comonoid structure}: categories equipped
with such a structure are also known as \emph{gs-monoidal}
(\emph{garbage-sharing}) categories~\cite{fritz2022free}.

\begin{definition}
    Let \((\generators[\ccomon], \equations[\ccomon])\) be the symmetric
    monoidal theory of \emph{commutative comonoids}, with \(\Sigma_\ccomon := \{
        \iltikzfig{strings/structure/comonoid/copy}[white],
        \iltikzfig{strings/structure/comonoid/discard}[white]
    \}\) and \(\mce_\ccomon\) defined as in \cref{fig:comonoid-equations}.
    We write \(\ccomon := \smc{\generators[\ccomon], \equations[\ccomon]}\).
\end{definition}

From now on, we write `comonoid' to mean `commutative comonoid'.
There has already been work using hypergraphs for PROPs with a (co)monoid
structure~\cite{fritz2022free,milosavljevic2022string} but these consider
\emph{acyclic} hypergraphs: we must ensure that removing the acyclicity
condition does not lead to any degeneracies.

\begin{definition}[Partial left-monogamy]
    For a cospan \(\cospan{m}[f]{H}[g]{n}\), we say it is
    \emph{partial left-monogamous} if \(f\) is mono and, for all nodes
    \(v \in H_\star\), the degree of \(v\) is \((0,m)\) if \(v \in f_\star\) and
    \((0,m)\) or \((1,m)\) otherwise, for some \(m \in \nat\).
\end{definition}

Partial left-monogamy is a weakening of partial monogamy that allows vertices
to have multiple `out' connections, which represent the use of the comonoid
structure to fork wires.

\begin{figure}
    \centering
    \[
        \underbrace{
            \iltikzfig{graphs/monogamy/yes-comonoid-0}
            \iltikzfig{graphs/monogamy/yes-comonoid-1}
        }_{\text{partial left-monogamous}}
        \qquad
        \underbrace{
            \iltikzfig{graphs/monogamy/no-comonoid-0}
            \iltikzfig{graphs/monogamy/no-comonoid-1}
        }_{\text{not partial left-monogamous}}
    \]
    \caption{Examples of cospans that are and are not partial left-monogamous.}
    \label{fig:partial-left-monogamous-examples}
\end{figure}

\begin{example}
    Examples of cospans that are and are not partial left-monogamous are shown
    in \cref{fig:partial-left-monogamous-examples}.
\end{example}

\begin{remark}
    As with the vertices not in the interfaces with degree \((0, 0)\) in the
    vanilla traced case, the vertices not in the interface with degree
    \((0, m)\) allow for terms such as \(
        \trace{}{\iltikzfig{strings/structure/comonoid/copy}[white]}
    \).
\end{remark}

\begin{lemma}
    \label{lem:partial-monogamous-ops}
    Let \(\cospan{m}{F}{n}\), \(\cospan{n}{G}{p}\), \(\cospan{p}{H}{q}\) and
    \(\cospan{x+m}{K}{x+n}\) be partial left-monogamous cospans.
    Then,
    \begin{itemize}
        \item identities and symmetries are partial left-monogamous;
        \item \(\cospan{m}{F}{n} \seq \cospan{n}{G}{p}\) is partial
                left-monogamous;
        \item \(\cospan{m}{F}{n} \tensor \cospan{p}{H}{q}\) is partial
                left-monogamous; and
        \item \(\trace{x}{\cospan{x+m}{K}{x+n}}\) is partial left-monogamous.
    \end{itemize}
\end{lemma}
\iftoggle{proofs}{
    \begin{proof}
        Identities and symmetries are monogamous, and as such they are also
        partial left-monogamous.
        For the operations, we only need to check the in-degrees of vertices, as
        the out-degree can be arbitrary.
        For composition, only the vertices in the image of \(n \to F\) and
        \(n \to G\) have their in-degrees modified.
        By partial left-monogamy we can conclude that:
        \begin{itemize}
            \item the in-degree of vertices in the image of \(n \to F\) is
                    \(0\) if they are in the image of \(m \to F\), and \(0\) or
                    \(1\) otherwise; and
            \item the in-degree of vertices in the image of \(n \to G\) is
                    \(0\).
        \end{itemize}
        This means that vertices in the image of \(m \to F \seq G\) will have
        in-degree \(0\), and the other vertices will have \(0\) or \(1\)
        otherwise.
        For tensor, the elements of the original two graphs are unaffected so
        the degrees remain unchanged.
        For trace, we must show that for any set of vertices in the image
        of \(x + n \to K\) merged by the trace, at most one of them can have
        in-degree \(1\).
        But this must be the case because any image in the image of
        \(x + m \to K\) must have in-degree \(0\), and \(x + m \to K\) is
        moreover mono so it cannot merge vertices in the image of
        \(x + n \to K\).
    \end{proof}
}{}

\begin{definition}
    Let \(\plmcspdhyp\) be the sub-PROP of \(\cspdhyp\) containing only the
    partial left-monogamous cospans of hypergraphs.
\end{definition}

This category can be equipped with a comonoid structure.

\begin{definition}
    Let \(
        \morph{
            \comonoidtofrob
        }{
            \ccomon
        }{
            \frob
        }
    \) be the obvious embedding of \(\ccomon\) into \(\frob\), and let \(
        \morph{
            \tracedandcomonoidtofrob{\Sigma}
        }{
            \stmc{\Sigma} + \comon
        }{
            \smc{\Sigma} + \frob
        }
    \) be the copairing of \(\tracedtosymandfrob{\Sigma}\) and
    \(\comonoidtofrob\).
\end{definition}

As before, these PROP morphisms are summarised in \cref{fig:roadmap}.
To show that partial left-monogamy is the correct notion to characterise terms
in a traced comonoid setting, it is necessary to ensure that the image of these
PROP morphisms lands in \(\plmcspdhyp\).

\begin{lemma}
    The image of \(\frobtohyp{\Sigma} \circ \comonoidtofrob\) is in
    \(\plmcspdhyp\).
\end{lemma}
\iftoggle{proofs}{
    \begin{proof}
        By definition.
    \end{proof}
}{}

\begin{corollary}
    The image of \(
        \termandfrobtohypsigma \circ \tracedandcomonoidtofrob{\Sigma}
    \) is in \(\plmcspdhyp\).
\end{corollary}

\iftoggle{conf}{}{
    \begin{lemma}[\cite{lack2004composing}]\label{lem:finop-ccomon}
        \(\ccomon \cong \op{\finset}\).
    \end{lemma}
}

\begin{lemma}\label{lem:ccomon-term}
    Given a partial left-monogamous cospan \(\cospan{m}[f]{m}[g]{n}\), there
    exists a unique term \(
        \iltikzfig{strings/category/f}[h][white][m][n] \in \ccomon
    \) up to the axioms of
    SMCs and comonoids such that \(
        \frobtohyp[\comonoidtofrob[\iltikzfig{strings/category/f}[h][white]]]{\Sigma}
        =
        \cospan{m}[f]{m}[g]{n}
    \).
\end{lemma}
\iftoggle{proofs}{
    \begin{proof}
        Given a partial left-monogamous cospan \(\cospan{m}[f]{m}[g]{n}\), there
        exists a unique function \(\morph{h}{[n]}{[m]}\) such that \(h(i) = j\) if
        \(g(i) = f(j)\).
        Since \(\finset \cong \op{\finset}\), by \cref{lem:finop-ccomon}, there
        exists a corresponding term in \(\ccomon\).
    \end{proof}
}{}

\begin{theorem}\label{thm:comonoid-fully-complete}
    \(\stmcsigma + \ccomon \cong \plmcspfihyp\).
\end{theorem}
\begin{proof}
    Since \(\termandfrobtohypsigma\) and \(\comonoidtohyp{\Sigma}\) are faithful,
    it suffices to show that a cospan \(\cospan{m}{F}{n}\) in
    \(\plmcspdhyp\) can be decomposed in such a way that each component is in
    the image of either \(\termandfrobtohypsigma\) or
    \(\comonoidtohyp{\Sigma}\).
    This is achieved by taking the construction of \cref{thm:termtohyp-image}
    and allowing the first component to be partial left-monogamous; by
    \cref{lem:ccomon-term} a term in \(\ccomon\) can be retrieved from this.
\end{proof}
