% !TeX root = ../main-conf.tex
\section{Monoidal theories and hypergraphs}

When modelling a system using monoidal categories, its components and
properties is specified using a \emph{monoidal theory}.

\begin{definition}[Symmetric monoidal theory]
    A \emph{(single-sorted) symmetric monoidal theory} (SMT) is a tuple \(
        (\generators,\equations)
    \) where \(\generators\) is a set of \emph{generators} and \(\equations\)
    is a set of \(\equations\).
    Given a SMT \((\generators,\equations)\), let \(
        \smc{\generators}
    \) be the symmetric monoidal category freely generated over \(\generators\)
    and let \(
        \smc{\generators,\equations}
    \) be \(\smc{\generators}\) quotiented by the equations in \(\equations\).
    We write \(\smc{} := \smc{\emptyset}\), for the SMC with terms
    constructed solely from identities and symmetries.
\end{definition}

While one could reason using one-dimensional text strings representing morphisms
in \(\smc{\generators.\equations}\), it is more intuitive to reason with
\emph{string diagrams}~\cite{joyal1991geometry,selinger2011survey}, which
represent \emph{equivalence classes} of terms up to the axioms of SMCs.
In the language of string diagrams, a generator \(\morph{\phi}{X}{Y}\) is drawn
as a box \(
    \iltikzfig{strings/category/f}[\phi][white][X][Y]
\), the identity \(\id[x]\) as \(
    \iltikzfig{strings/category/identity}[white][X]
\), the symmetry \(\swap{X}{Y}\) as \(
    \iltikzfig{strings/symmetric/symmetry}[white][X][Y]
\), (diagrammatic order) composition \(f \seq g\) as horizontal juxtaposition \(
    \iltikzfig{strings/category/composition}[f][g][white][X][Y][Z]
\) and tensor \(f \tensor g\) as vertical juxtaposition \(
    \iltikzfig{strings/monoidal/tensor}[f][g][white][X][Y][Z][W]
\).
When clear from context, wire labels may be omitted.

A class of SMC particularly interesting to us is that of
\emph{props}~\cite{maclane1965categorical} (`categories of \emph{PRO}ducts and
\emph{P}ermutations'), which have natural numbers as objects and addition as
tensor product.
In a prop, a morphism \(m \to n\) can be drawn as a box with \(m\) input wires
and \(n\) output wires.

Reasoning equationally using string diagrams is certainly attractive
as a pen-and-paper method, but for larger systems it quickly becomes intractible
to do this by hand.
Instead, it is desirable to perform equational reasoning \emph{computationally}.
Unfortunately, string diagrams as topological objects are not particular suited
for this purpose: instead, we require a combinatorial representation.
Fortunately, this has been well studied
recently, first with
\emph{string graphs}~\cite{dixon2013opengraphs,kissinger2012pictures}
and later with
\emph{hypergraphs}~\cite{bonchi2022string,bonchi2021string,bonchi2022stringa},
a generalisation of regular graphs in which edges can be the source or target of
an arbitrary number of vertices.
In this paper we are concerned in the latter.

Hypergraphs are formally defined as objects in a functor category.

\begin{definition}[Hypergraph]
    Let \(\mathbf{X}\) be the category containing objects \((k, l)\) for
    \(k, l \in \nat\) and one additional object \(\star\).
    For each \((k, l)\) there are \(k + l\) morphisms from \((k, l) \to \star\).
    Let \(\hyp\) be the functor category \([\mathbf{X},\set]\).
\end{definition}

An object in \(\hyp\) maps \(\star\) to a set of vertices, and each \((k,l)\) to
a set of hyperedges with \(k\) sources and \(l\) targets,
Given a hypergraph \(F \in \hyp\), we write \(\vertices{F}\) for its set of
vertices and \(\edges{F}{k}{l}\) for the set of edges with \(k\) sources and
\(l\) targets.

A morphism of hypergraphs \(\morph{f}{F}{G} \in \hyp\) consists of functions
\(\vertices{f}\) and \(\edges{f}{k}{l}\) for each \(k,l \in \nat\) preserving
sources and targets in the obvious way.
With such morphisms a hypergraph can be \emph{labelled}.

\begin{definition}[Slice category]
    For a category \(\mathbf{C}\) and an object \(C \in \mathbf{C}\), the
    \emph{slice category} \(\mathbb{C} / C\) is the category with objects the
    morphisms of \(\mathbf{C}\) with target \(C\), and morphisms \(
        (\morph{f}{X}{C}) \to (\morph{f^\prime}{X^\prime}{C})
    \) are the morphisms \(\morph{g}{X}{X^\prime} \in \mathbf{C}\) such that
    \(f^\prime \circ g = f\).
\end{definition}


\begin{definition}[Hypergraph signature~\cite{bonchi2022string}]
    For a given monoidal signature \(\signature\), its corresponding
    \emph{hypergraph signature} \(\hypsignature{\Sigma}\) is the hypergraph with
    edges \(
        e_\phi \in \edges{\hypsignature{\Sigma}}{\dom{\phi}}{\cod{\phi}}
    \) for each \(\phi \in \Sigma\), and a vertex \(v\).
    For a hyperedge \(e_\phi\), \(i < \dom{\phi}\) and \(j < \cod{\phi}\), \(
        \sources{i}(e_\phi) = \targets{j}(e_\phi) = v
    \).
\end{definition}

\begin{definition}[Labelled hypergraph]
    For a monoidal signature \(\Sigma\), let the category \(\hypsigma\) be
    defined as the slice category \(\hyp \setminus \hypsignature{\Sigma}\).
\end{definition}

While (labelled) hypergraphs may have dangling vertices, they do not have
\emph{interfaces}.
These can be provided using \emph{cospans}.

\begin{definition}[Categories of cospans~\cite{bonchi2021string}]\label{def:cospans}
    For a finitely cocomplete category \(\mathbf{C}\), a \emph{cospan} from
    \(X \to Y\) is a pair of arrows \(X \to A \leftarrow Y\).
    A \emph{cospan morphism} \(
        (\cospan{X}[f]{A}[g]{Y}) \to (\cospan{X}[h]{B}[k]{Y})
    \) is a morphism \(\morph{\alpha}{A}{B} \in \mathbf{C}\)%
    \iftoggle{conf}{
        such that \(f \seq \alpha = g\) and \(h \seq \alpha = k\).
    }{
        such that the following diagram commutes:
        \begin{center}
            \includestandalone{figures/graphs/cospans/morphism}
        \end{center}
    }

    Two cospans \(\cospan{X}{A}{Y}\) and \(\cospan{X}{B}{Y}\) are
    \emph{isomorphic} if there exists a morphism of cospans as above where
    \(\alpha\) is an isomorphism.%
    \iftoggle{conf}{
        Composition is by pushout.
    }{
        Composition is by pushout:

        \begin{center}
            \includestandalone{figures/graphs/cospans/composition}
        \end{center}
    }
    The identity is \(X \xrightarrow{\id[X]} X \xleftarrow{\id[X]} X\).
    The category of cospans over \(\mathbf{C}\), denoted \(\csp{\mathbf{C}}\)
    has as objects the objects of \(\mathbf{C}\) and as morphisms the
    isomorphism classes of cospans.
    This category has monoidal product given by the coproduct in \(\mathbf{C}\)
    with unit the initial object \(0 \in \mathbf{C}\).
\end{definition}

The interfaces of a hypergraph can be specified as cospans by having the `legs'
of the cospan pick vertices in the graph at the apex.
For this purpose a special type of hypergraph is required.

\begin{definition}[Discrete hypergraph]
    A hypergraph is called \emph{discrete} if it has no edges.
\end{definition}

A discrete hypergraph \(F\) with \(|\vertices{F}| = n\) is often written as
\(n\) when clear from context.
Morphisms from discrete hypergraphs to a main graph pick out the vertices in the
interface, but there is still the notion of \emph{ordering} to consider.

\begin{theorem}[\cite{bonchi2022string}, Thm. 3.6]
    Let \(\mathbb{X}\) be a prop whose monoidal product is a coproduct,
    \(\mathbf{C}\) a category with finite colimits, and \(
        \morph{F}{\mathbb{X}}{\mathbf{C}}
    \) a coproduct-preserving functor.
    Then there exists a prop \(\csp[F]{\mathbf{C}}\) whose arrows \(m \to n\)
    are isomorphism classes of \(\mathbf{C}\) cospans \(\cospan{Fm}{C}{Fn}\).
\end{theorem}

\iftoggle{conf}{}{
    \begin{theorem}[\cite{bonchi2022string}, Thm. 3.8]
        \label{thm:cospan-homomorphism}
        Let \(\mathbb{X}\) be a prop whose monoidal product is a coproduct,
        \(\mathbf{C}\) a category with finite colimits, and
        \(\morph{F}{\mathbb{X}}{\mathbf{C}}\) a coproduct-preserving functor.
        Then there is a homomorphism of props \(
            \morph{\tilde{F}}{\csp{\mathbb{X}}}{\csp[F]{\mathbf{C}}}
        \) that sends \(\cospan{m}[f]{X}[g]{n}\) to \(\cospan{Fm}[Ff]{FX}[Fg]{Fn}\).
        If \(F\) is full and faithful, then \(\tilde{F}\) is faithful.
    \end{theorem}
}

\begin{definition}
    Let \(\finset\) be the prop with morphisms \(m \to n\) the functions
    between finite sets \([m] \to [n]\).
\end{definition}

\begin{definition}[\cite{bonchi2022string}]
    Let \(\morph{D}{\finset}{\hypsigma}\) be the faithful, coproduct-preserving
    functor that sends each object \(m \in \finset\) to the discrete hypergraph
    \(m \in \hypsigma\) and each morphism to the induced homomorphism of
    discrete hypergraphs.
\end{definition}

\noindent
From this we define the category \(\cspdhyp\) with objects
\emph{discrete cospans of hypergraphs}.
Since the legs of each cospan are discrete hypergraphs containing some number of
vertices, the objects of this category can be viewed as natural numbers, making
this another prop.
