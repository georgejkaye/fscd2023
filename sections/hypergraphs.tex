% !TeX root = ../main-conf.tex
\section{Hypergraphs}

When modelling a system using monoidal categories, its components and
properties is specified using a \emph{monoidal theory}.

\begin{definition}[Symmetric monoidal theory]
    A \emph{(single-sorted) symmetric monoidal theory} (SMT) is a tuple \(
        (\generators,\equations)
    \) where \(\generators\) is a set of \emph{generators} and \(\equations\)
    is a set of \(\equations\).
    Given a SMT \((\generators,\equations)\), let \(
        \smc{\generators}
    \) be the symmetric monoidal category freely generated over \(\generators\)
    and let \(
        \smc{\generators,\equations}
    \) be \(\smc{\generators}\) quotiented by the equations in \(\equations\).
    We write \(\smc{} := \smc{\emptyset}\), for the SMC with terms
    constructed solely from identities and symmetries.
\end{definition}

While one could reason using one-dimensional text strings representing morphisms
in \(\smc{\generators.\equations}\), it is more intuitive to reason with
\emph{string diagrams}~\cite{joyal1991geometry,selinger2011survey}, which
represent \emph{equivalence classes} of terms up to the axioms of SMCs.
In the language of string diagrams, a generator \(\morph{\phi}{X}{Y}\) is drawn
as a box \(
    \iltikzfig{strings/category/f}[\phi][white][X][Y]
\), the identity \(\id[x]\) as \(
    \iltikzfig{strings/category/identity}[white][X]
\), the symmetry \(\swap{X}{Y}\) as \(
    \iltikzfig{strings/symmetric/symmetry}[white][X][Y]
\), (diagrammatic order) composition \(f \seq g\) as horizontal juxtaposition \(
    \iltikzfig{strings/category/composition}[f][g][white][X][Y][Z]
\) and tensor \(f \tensor g\) as vertical juxtaposition \(
    \iltikzfig{strings/monoidal/tensor}[f][g][white][X][Y][Z][W]
\).
When clear from context, wire labels may be omitted.

A class of SMC particularly interesting to us is that of
\emph{props}~\cite{maclane1965categorical} (`categories of \emph{PRO}ducts and
\emph{P}ermutations'), which have natural numbers as objects and addition as
tensor product.
In a prop, a morphism \(m \to n\) can be drawn as a box with \(m\) input wires
and \(n\) output wires.

Reasoning equationally using string diagrams is certainly attractive
as a pen-and-paper method, but for larger systems it quickly becomes intractible
to do this by hand.
Instead, it is desirable to perform equational reasoning \emph{computationally}.
Unfortunately, string diagrams as topological objects are not particular suited
for this purpose: instead, we require a combinatorial representation.
Fortunately, this has been well studied
recently, first with
\emph{string graphs}~\cite{dixon2013opengraphs,kissinger2012pictures}
and later with
\emph{hypergraphs}~\cite{bonchi2022string,bonchi2021string,bonchi2022stringa},
a generalisation of regular graphs in which edges can be the source or target of
an arbitrary number of vertices.
In this paper we are concerned in the latter.

Hypergraphs are formally defined as objects in a functor category.

\begin{definition}[Hypergraph]
    Let \(\mathbf{X}\) be the category containing objects \((k, l)\) for
    \(k, l \in \nat\) and one additional object \(\star\).
    For each \((k, l)\) there are \(k + l\) morphisms from \((k, l) \to \star\).
    Let \(\hyp\) be the functor category \([\mathbf{X},\set]\).
\end{definition}

An object in \(\hyp\) maps \(\star\) to a set of vertices, and each \((k,l)\) to
a set of hyperedges with \(k\) sources and \(l\) targets,
Given a hypergraph \(F \in \hyp\), we write \(\vertices{F}\) for its set of
vertices and \(\edges{F}{k}{l}\) for the set of edges with \(k\) sources and
\(l\) targets.

A morphism of hypergraphs \(\morph{f}{F}{G} \in \hyp\) consists of functions
\(\vertices{f}\) and \(\edges{f}{k}{l}\) for each \(k,l \in \nat\) preserving
sources and targets in the obvious way.
With such morphisms a hypergraph can be \emph{labelled}.

\begin{definition}[Slice category]
    For a category \(\mathbf{C}\) and an object \(C \in \mathbf{C}\), the
    \emph{slice category} \(\mathbb{C} / C\) is the category with objects the
    morphisms of \(\mathbf{C}\) with target \(C\), and morphisms \(
        (\morph{f}{X}{C}) \to (\morph{f^\prime}{X^\prime}{C})
    \) are the morphisms \(\morph{g}{X}{X^\prime} \in \mathbf{C}\) such that
    \(f^\prime \circ g = f\).
\end{definition}


\begin{definition}[Hypergraph signature~\cite{bonchi2022string}]
    For a given monoidal signature \(\signature\), its corresponding
    \emph{hypergraph signature} \(\hypsignature{\Sigma}\) is the hypergraph with
    edges \(
        e_\phi \in \edges{\hypsignature{\Sigma}}{\dom{\phi}}{\cod{\phi}}
    \) for each \(\phi \in \Sigma\), and a vertex \(v\).
    For a hyperedge \(e_\phi\), \(i < \dom{\phi}\) and \(j < \cod{\phi}\), \(
        \sources{i}(e_\phi) = \targets{j}(e_\phi) = v
    \).
\end{definition}

\begin{definition}[Labelled hypergraph]
    For a monoidal signature \(\Sigma\), let the category \(\hypsigma\) be
    defined as the slice category \(\hyp \setminus \hypsignature{\Sigma}\).
\end{definition}

While (labelled) hypergraphs may have dangling vertices, they do not have
\emph{interfaces}.
These can be provided using \emph{cospans}.

\begin{definition}[Categories of cospans~\cite{bonchi2021string}]\label{def:cospans}
    For a finitely cocomplete category \(\mathbf{C}\), a \emph{cospan} from
    \(X \to Y\) is a pair of arrows \(X \to A \leftarrow Y\).
    A \emph{cospan morphism} \(
        (\cospan{X}[f]{A}[g]{Y}) \to (\cospan{X}[h]{B}[k]{Y})
    \) is a morphism \(\morph{\alpha}{A}{B} \in \mathbf{C}\)%
    \iftoggle{conf}{
        such that \(f \seq \alpha = g\) and \(h \seq \alpha = k\).
    }{
        such that the following diagram commutes:
        \begin{center}
            \includestandalone{figures/graphs/cospans/morphism}
        \end{center}
    }

    Two cospans \(\cospan{X}{A}{Y}\) and \(\cospan{X}{B}{Y}\) are
    \emph{isomorphic} if there exists a morphism of cospans as above where
    \(\alpha\) is an isomorphism.%
    \iftoggle{conf}{
        Composition is by pushout.
    }{
        Composition is by pushout:

        \begin{center}
            \includestandalone{figures/graphs/cospans/composition}
        \end{center}
    }
    The identity is \(X \xrightarrow{\id[X]} X \xleftarrow{\id[X]} X\).
    The category of cospans over \(\mathbf{C}\), denoted \(\csp{\mathbf{C}}\)
    has as objects the objects of \(\mathbf{C}\) and as morphisms the
    isomorphism classes of cospans.
    This category has monoidal product given by the coproduct in \(\mathbf{C}\)
    with unit the initial object \(0 \in \mathbf{C}\).
\end{definition}

The interfaces of a hypergraph can be specified as cospans by having the `legs'
of the cospan pick vertices in the graph at the apex.
For this purpose a special type of hypergraph is required.

\begin{definition}[Discrete hypergraph]
    A hypergraph is called \emph{discrete} if it has no edges.
\end{definition}

A discrete hypergraph \(F\) with \(|\vertices{F}| = n\) is often written as
\(n\) when clear from context.
Morphisms from discrete hypergraphs to a main graph pick out the vertices in the
interface, but there is still the notion of \emph{ordering} to consider.

\begin{theorem}[\cite{bonchi2022string}, Thm. 3.6]
    Let \(\mathbb{X}\) be a prop whose monoidal product is a coproduct,
    \(\mathbf{C}\) a category with finite colimits, and \(
        \morph{F}{\mathbb{X}}{\mathbf{C}}
    \) a coproduct-preserving functor.
    Then there exists a prop \(\csp[F]{\mathbf{C}}\) whose arrows \(m \to n\)
    are isomorphism classes of \(\mathbf{C}\) cospans \(\cospan{Fm}{C}{Fn}\).
\end{theorem}

\iftoggle{conf}{}{
    \begin{theorem}[\cite{bonchi2022string}, Thm. 3.8]
        \label{thm:cospan-homomorphism}
        Let \(\mathbb{X}\) be a prop whose monoidal product is a coproduct,
        \(\mathbf{C}\) a category with finite colimits, and
        \(\morph{F}{\mathbb{X}}{\mathbf{C}}\) a coproduct-preserving functor.
        Then there is a homomorphism of props \(
            \morph{\tilde{F}}{\csp{\mathbb{X}}}{\csp[F]{\mathbf{C}}}
        \) that sends \(\cospan{m}[f]{X}[g]{n}\) to \(\cospan{Fm}[Ff]{FX}[Fg]{Fn}\).
        If \(F\) is full and faithful, then \(\tilde{F}\) is faithful.
    \end{theorem}
}

\begin{definition}
    Let \(\finset\) be the prop with morphisms \(m \to n\) the functions
    between finite sets \([m] \to [n]\).
\end{definition}

\begin{definition}[\cite{bonchi2022string}]
    Let \(\morph{D}{\finset}{\hypsigma}\) be the faithful, coproduct-preserving
    functor that sends each object \(m \in \finset\) to the discrete hypergraph
    \(m \in \hypsigma\) and each morphism to the induced homomorphism of
    discrete hypergraphs.
\end{definition}

\noindent
From this we define the category \(\cspdhyp\) with objects
\emph{discrete cospans of hypergraphs}.
Since the legs of each cospan are discrete hypergraphs containing some number of
vertices, the objects of this category can be viewed as natural numbers, making
this another prop.

\subsection{Adding a trace}

We wish to use the hypergraph framework for a setting with a \emph{trace}.

\begin{definition}[\cite{joyal1996traced,hasegawa2009traced}]
    A \emph{symmetric traced monoidal category} (STMC) is a symmetric monoidal
    category \(\mathbf{C}\) equipped with a family of functions \(
        \morph{
            \trace{X}[A][B]{-}
        }{
            \mathbf{C}(X \tensor A, X \tensor B)
        }{
            \mathbf{C}(X, Y)
        }
    \) satisfying the axioms of STMCs listed in \cref{app:equations},
    \cref{fig:stmc-axioms}.
\end{definition}

In the string diagram notation, the trace is represented by joining some of the
inputs of a circuit to its outputs.
%
\begin{center}
    \(
        \trace{x}[m][n]{\iltikzfig{strings/traced/trace-lhs}[F][comb][x][m][n]}
        \xrightarrow{\text{def}}
        \iltikzfig{strings/traced/trace-rhs}[F][comb][m][n]
    \)
\end{center}

The category of interfaced hypergraphs we have just defined already
contains the structure necessary to define a trace.

\iftoggle{conf}{}{
    \begin{definition}[Hypergraph category \cite{fong2019hypergraph}]
        A \emph{hypergraph category} is a symmetric monoidal category in which each
        object \(X\) is equipped with a Frobenius structure.
    \end{definition}

    \begin{proposition}\label{prop:hypergraph-category}
        \(\cspdhyp\) is a hypergraph category.
    \end{proposition}
    \begin{proof}
        The Frobenius structure is defined as follows:
        \begin{gather*}
            \iltikzfig{strings/structure/monoid/merge}[white]
            :=
            \cospan{m + m}{m}{m}
            \quad
            \iltikzfig{strings/structure/monoid/init}[white]
            :=
            \cospan{0}{m}{m}
            \\
            \iltikzfig{strings/structure/comonoid/copy}[white]
            :=
            \cospan{m}{m}{m+m}
            \quad
            \iltikzfig{strings/structure/comonoid/discard}[white]
            :=
            \cospan{m}{m}{0}
        \end{gather*}
    \end{proof}
    \begin{corollary}[\cite{bonchi2022string}]
        \(\cspdhyp\) is compact closed.
    \end{corollary}
}{
    \begin{proposition}[\cite{carboni1987cartesian,bonchi2022string}]
        \(\cspdhyp\) is compact closed.
    \end{proposition}

}

It is well known that every compact closed category is equipped with a
\emph{canonical trace}~\cite{joyal1996traced}, constructed by combining the cup,
cap and an identity.
So in some sense, \(\cspfihyp\) is already traced.
However, it is useful to consider a(n equivalent) formulation in which the trace
is constructed directly.
To take the \((f,g)\)-trace of a cospan \(\cospan{x+m}[f+h]{H}[g+k]{x+n}\), we
coalesce the \(i\)th elements of the image of \(f\) and \(g\).
\begin{center}
    \(
        \trace{1}{\iltikzfig{graphs/trace/before-trace}}
        \equiv
        \iltikzfig{graphs/trace/after-trace}
    \)
\end{center}

This is defined formally using a pushout, in a similar vein
to~\cite{dixon2013opengraphs}.

\begin{definition}
    For a cospan of hypergraphs \(
        \cospan{x+m}[f+h]{H}[g+k]{x+n},
    \) its trace on \(x\) is \(
        \cospan{m}[h \seq p]{H^\prime}[k \seq p]{n},
    \) where \(p\) and \(H^\prime\) are computed%
    \iftoggle{conf}{
        as the pushout of \(f + h\) and \([\id, \id]\).
    }{
        by the following pushout:
        \begin{center}
            \tikzfig{graphs/trace/trace-pushout}
        \end{center}
    }
\end{definition}

\begin{example}
    Below is an example of tracing a simple hypergraph in this manner.
    \begin{center}
        \iltikzfig{graphs/trace/trace-pushout-example}
    \end{center}
\end{example}

\subsection{Monogamy}

In~\cite{bonchi2016rewriting}, it is shown that terms in a (non-traced)
symmetric monoidal category are interpreted as a certain sub-prop of
\(\cspdhyp\), in which all hypergraphs are \emph{acyclic}.
Clearly, to model trace this condition must be dropped.

However, there is also another condition known as \emph{monogamy}: informally,
this means that every vertex has exactly one `in' and `out' connection, be it to
an edge or an interface.
For the most part, this condition also applies to the traced case: wires cannot
arbitrarily fork and join.
There is one nuance: the trace of the identity.
This is depicted as a closed loop \(
    \trace{1}{\iltikzfig{strings/category/identity}[white]}
    =
    \iltikzfig{strings/traced/trace-id}
\), and one might think that it can be discarded, i.e. \(
    \iltikzfig{strings/traced/trace-id}
    =
    \iltikzfig{strings/monoidal/empty}
\).
This is \emph{not} always the case, such as in the category of finite
dimensional vector spaces~\cite[Sec. 6.1]{hasegawa1997recursion}.
Closed loops must therefore be represented in the hypergraph framework:
there is a natural representation of such a loop as a lone vertex disconnected
from either interface.

\begin{definition}
    For a hypergraph \(H \in \hyp\), the \emph{degree} of a vertex
    \(v \in \vertices{H}\) is a tuple \((i,o)\) where \(i\) is the number of
    pairs \((e,i)\) where \(e\) is a hyperedge with \(v\) as its \(i\)th target,
    and \(o\) is similarly the number of pairs \((e,j)\) where \(e\) is a
    hyperedge with \(v\) as its \(j\)th target.
\end{definition}

\begin{definition}
    For a cospan \(\cospan{m}[f]{H}[g]{n}\), we say it is
    \emph{partial monogamous} if \(f\) and \(g\) are mono and, for all nodes
    \(v \in H_\star\), the degree of \(v\) is

    \begin{center}
        \begin{tabular}{rlcrl}
            \((0,0)\)
            &
            if \(v \in f_\star \wedge v \in g_\star\)
            &
            \quad
            &
            \((0,1)\)
            &
            if \(v \in f_\star\)
            \\
            \((1,0)\)
            &
            if \(v \in g_\star\)
            &
            \quad
            &
            \((0,0)\)
            or \((1,1)\)
            &
            otherwise
        \end{tabular}
    \end{center}
\end{definition}

\iftoggle{conf}{}{
    \begin{lemma}\label{lem:trace-degree}
        Given a cospan \(\cospan{x+m}[f+h]{F}[g+k]{x+n}\), and its corresponding
        traced cospan \(\cospan{m}[g \seq p]{F^\prime}[h \seq p]{n}\).
        For \(i < x\), let \((k_1,l_1)\) be the degree of \(f(i)\) and \((k_2,l_2)\)
        be the degree of \(g(i)\).
        Then, the degree of \(p(f(i)) = p(h(i))\) is \((k_1 + k_2, l_1 + l_2)\).
    \end{lemma}
    \iftoggle{proofs}{
        \begin{proof}
            \(p(f(i))\) and \(p(g(i))\) are coalesced by the pushout.
            The pushout cannot coalesce the edges in \(H\) as \(x\) contains no
            edges,
            so the degree of \(f(i)\) and \(h(i)\) must be preserved.
        \end{proof}
    }{}

    \begin{lemma}\label{lem:trace-interface}
        Given a partial monogamous cospan \(\cospan{x+m}[f+h]{F}[g+k]{x+n}\) and its
        trace \(\cospan{m}[h \seq p]{F^\prime}[k \seq p]{n},\) any vertex in the
        image of \(f\) is not in the image of \(h \seq p\), and similarly any vertex
        in the image of \(g\) is not in the image of \(k \seq p\).
    \end{lemma}
    \iftoggle{proofs}{
        \begin{proof}
            Since this is a partial monogamous cospan, the images of \(f\) and \(h\)
            are disjoint, as are the images of \(g\) and \(k\).
            Therefore, \(f(v)\) is not in the image of \(h\), so it cannot be in the
            image of \(h \seq p\), and similarly for \(g(v)\) and \(k\).
        \end{proof}
    }{}
}

\begin{lemma}~\label{lem:partial monogamous-ops}
    Let \(\cospan{m}{F}{n}\), \(\cospan{n}{G}{p}\), \(\cospan{p}{H}{q}\) and
    \(\cospan{x+m}{K}{x+n}\) be partial monogamous cospans.
    Then,
    \begin{itemize}
        \item Identities and symmetries are partial monogamous.
        \item \(\cospan{m}{F}{n} \seq \cospan{n}{G}{p}\) is partial monogamous.
        \item \(\cospan{m}{F}{n} \tensor \cospan{p}{H}{q}\) is partial
        monogamous.
        \item \(
            \trace{x}{\cospan{x+m}[f+h]{K}[g+k]{x+n}}
            =
            \cospan{m}[h \seq p]{pK}[k \seq p]{n}
        \) is partial monogamous.
    \end{itemize}
\end{lemma}
\iftoggle{proofs}{
    \begin{proof}
        Since any monogamous hypergraph is also partial monogamous, the first
        three points hold due to~\cite[Prop.16]{bonchi2022string}, dropping the
        acyclicity condition.
        For the final condition, consider the image of \(f\) and \(g\).
        For each \(i \in x\), there are two cases to consider: \(f(i) = g(i)\)
        and \(f(i) \neq g(i)\).

        In the former, the degree of \(v^\prime = f(v) = h(v)\) must be
        \((0,0)\) by definition of semi-monogamicity.
        Therefore in the traced hypergraph \(
            \cospan{m}[g \seq p]{K^\prime}[h \seq p]{n}
        \), \(v^\prime\) will still have degree \((0,0)\), and will not be in
        the image of \(g \seq p\) or \(h \seq p\) by \cref{lem:trace-interface}.
        So the cospan is partial monogamous.

        In the latter case, \(f(i)\) must have degree of \((0,0)\) if it in the
        image of \(k\) or \((0,1)\) otherwise.
        Similarly \(g(i)\) either has degree \((0,0)\) or \((1,0)\).
        Let \(v := p(f(i)) = p(g(i))\); we now consider the degree of \(v\)
        computed using \cref{lem:trace-degree}:
        \begin{itemize}
            \item If \(f(i)\) is in the image of \(k\) and \(g(i)\) is in the
                    image of \(h\), then \(v\) has degree \((0,0)\).
                    \(v\) is in the image of \(h \seq p\) and \(h \seq p\), so
                    the cospan is partial monogamous.
            \item If \(f(i)\) is in the image of \(k\), then \(v\) has degree
                    \((1, 0)\); since \(v\) is in the image of \(k \seq p\), the
                    cospan is partial monogamous.
            \item If \(g(i)\) is in the image of \(h\), then \(v\) has degree
                    \((0, 1)\); since \(v\) is in the image of \(h \seq p\), the
                    cospan is partial monogamous.
            \item Otherwise, \(v\) will have degree \((1, 1)\), and is not in
                    the image of either interface so the cospan is partial
                    monogamous.
        \end{itemize}
    \end{proof}
}{}

\noindent
This means that the partial monogamous hypergraphs form a sub-prop of
\(\cspdhyp\), which we name \(\pmcspdhyp\).

\subsection{From terms to graphs}

For \(\pmcspfihyp\) to be suitable for reasoning with traced categories, it must
be \emph{sound and complete}: there must be a two-way correspondence between
terms in a traced monoidal theory and interfaced hypergraphs.

\begin{definition}
    For a SMT \((\generators,\equations)\), let
    \(\stmc{\generators}\) be the STMC freely generated over the
    generators in \(\generators\).
    Let \(\stmc{\generators, \equations}\) be \(\stmc{\generators}\) quotiented
    by equations in \(\equations\).
\end{definition}

A \emph{(traced) prop morphism} is a strict (traced) symmetric monoidal functor
between props .
Since \(\stmc{\Sigma}\) and \(\pmcspfihyp\) are freely generated, to define a
prop morphism between them it suffices to define it on the generators in the
former.

\begin{definition}
    Let \(\morph{\termtohyp{\Sigma}}{\stmc{\Sigma}}{\pmcspfihyp}\) be a prop
    morphism defined on the generators of \(\Sigma\) as shown in
    \cref{fig:termtohyp}.
    \begin{figure}
        \begin{gather*}
            \termtohyp[\iltikzfig{strings/category/f}[\phi][white][m][n]]{\Sigma}
            :=
            \iltikzfig{graphs/terms/generator}
            \quad
            \termtohyp[\iltikzfig{strings/category/identity}[white][m][n]]{\Sigma}
            :=
            \iltikzfig{graphs/terms/identity}[white]
            \\[1em]
            \termtohyp[\iltikzfig{strings/symmetric/symmetry}[white][m][n]]{\Sigma}
            :=
            \iltikzfig{graphs/terms/symmetry}[white]
        \end{gather*}
        \caption{Definition of \(\termtohyp{\Sigma}\) on generators in \(\smc{\Sigma}\).}
        \label{fig:termtohyp}
    \end{figure}
\end{definition}

\iftoggle{conf}{}{
    Recall one of the key theorems from~\cite{bonchi2021string}.

    \begin{theorem}[\cite{bonchi2021string}, Corollary 20]\label{thm:smc-graph-iso}
        \(\smc{\Sigma} \cong \macspfihyp\).
    \end{theorem}
}

Our goal is to extend the results of \cite{bonchi2021string} to terms in
\(\stmcsigma\) and \emph{partial monogamous hypergraphs}.
We first characterise the image of \(\termtohypsigma\) and show that it is
exactly \(\pmcspfihyp\).
Before progressing to the main theorem, we must show a result about terms in
\(\smc{}\), i.e.\ terms constructed from just symmetries and identities.

In particular, there is a correspondence between \(\smc{}\) and
\emph{bijective} functions.

\begin{definition}
    Let \(\perms\) be the subprop of \(\finset\) containing only the
    bijective functions.
\end{definition}

\begin{lemma}\label{lem:symmetries-prop}
    \(\smc{} \cong \perms\).
\end{lemma}
\iftoggle{proofs}{
    \begin{proof}
        The morphism \(\morph{\phi}{\smc{}}{\perms}\) is defined over
        generators in \(\smc{}\) as \[
            \phi(\iltikzfig{strings/monoidal/empty}) = \{\}
            \quad
            \phi(\iltikzfig{strings/category/identity}[white])
            =
            \{0 \mapsto 0\}
            \quad
            \phi(\iltikzfig{strings/symmetric/symmetry}[white])
            =
            \{0 \mapsto 1, 1 \mapsto 0\}
        \]
        Since any term in \(\smc{}\) can be expressed using these generators,
        this defines the complete transformation.

        The reverse morphism \(\morph{\psi}{\finset}{\smc{}}\) is inductively
        over the size of \(m\).
        For the base case \(\morph{f}{[0]}{[0]}\), let \(
            \phi(f) := \iltikzfig{strings/monoidal/empty}
        \).
        For \(
            \morph{f}{[k+1]}{[k+1]}
        \), let \(i\) such that \(f(i) = k+1\), and define the function \(
            \morph{f^\prime}{\nat_{k}}{\nat_{k}}
        \) as the function such that \(
            f^\prime(j) = f(j)
        \) if \(j < i\), and \(f(j+1)\) otherwise.
        Then \[
            \psi(f) := \iltikzfig{strings/symmetric/f-construction}.
        \]

        \noindent
        These are shown to be inverses by a simple induction in both directions.
    \end{proof}
}{}

This allows us to establish a correspondence between terms in \(\smc{}\)
and cospans of discrete hypergraphs.

\begin{lemma}\label{lem:monog-discrete-cospan}
    Given a monogamous cospan \(\cospan{m}[f]{m}[g]{m}\), there exists a unique
    term \(\morph{h}{m}{m} \in \smc{}\) up to the axioms of SMCs such that
    \(\termtohyp[h]{\Sigma} = \cospan{m}[f]{m}[g]{m}\).
\end{lemma}
\iftoggle{proofs}{
    \begin{proof}
        Since the cospan is monogamous, \(f\) and \(g\) are mono.
        As the cospan is also discrete, there exists a (unique) bijective
        function \(\morph{h^\prime}{[m]}{[m]}\) such that \(h^\prime(i) = j\) if
        \(f(i) = g(j)\).
        By \cref{lem:symmetries-prop}, there is a corresponding term
        \(h \in \smc{}\) that is unique up to SMC axioms: a simple induction
        shows that \(\termtohyp[h]{\Sigma} = \cospan{m}[f]{m}[g]{m}\).
    \end{proof}
}{}

We now proceed to the first main result.

\begin{theorem}\label{thm:termtohyp-image}
    A cospan \(\cospan{m}{F}{n}\) is in the image of \(\termtohyp{\Sigma}\) if
    and only if it is partial monogamous.
\end{theorem}
\iftoggle{proofs}{
    \begin{proof}
        % !TeX root = ../../main-conf.tex
To show that \(\termtohyp[f]{\Sigma}\) is partial monogamous for any
\(f \in \smc{\Sigma}\) we use induction on the structure of \(f\).
Generators, identities and symmetries are partial monogamous, as
semi-monogamicity is preserved by composition, tensor and trace.
So \(\termtohyp[f]{\Sigma}\) is partial monogamous.

Now we show that any partial monogamous cospan \(
    \cospan{m}[f]{F}[g]{n}
\)
must be in the image of \(\termtohyp{\Sigma}\).
To do this, we will now construct a cospan that is isomorphic to
\(\cospan{m}[f]{F}[g]{n}\), but from which it is possible to read off a
unique term in \(\smc{\Sigma}\).
The components of the new cospan are as follows:
\begin{itemize}
    \item let \(L\) be the hypergraph containing vertices with degree
            \((0,0)\) that are not in the image of \(f\) or \(g\);
    \item let \(E\) be the hypergraph containing hyperedges of \(F\) and
            their source and target vertices, but disconnected;
    \item let \(V\) be the discrete hypergraph containing all the
            vertices of \(F\); and
    \item let \(S\) and \(T\) be the discrete hypergraphs containing
            the source and target vertices of hyperedges in \(F\)
            respectively, with the ordering determined by some order
            \(e_1,e_2,\cdots,e_n\) on the edges in \(F\).
\end{itemize}

These parts can be composed and a trace applied to obtain the follow
cospan:
\begin{gather}
    \trace{T}{
        \cospan{T + m}[\id + f]{V}[\id + g]{S + n}
        \,\seq\,
        \cospan{\emptyset + S + n}[\id]{L + E + n}[\id]{\emptyset + T + n}
    }
    \label{gat:cospan}
\end{gather}

This can be checked to be isomorphic to the original cospan
\(\cospan{m}[f]{F}[g]{n}\) by applying the pushouts.
From this we can read off a term in \(\smc{\Sigma}\):
Since the first cospan is monogamous, it corresponds to a term \(
    \iltikzfig{strings/category/f-2-2}[f][white][|\vertices{T}|][m][|\vertices{S}|][n]
\) by \cref{lem:monog-discrete-cospan}.
The second cospan corresponds to \(
    \iltikzfig{strings/category/f}[g][white][|\vertices{S}][\vertices{T}]
    :=
    \bigtensor_{v \in \vertices{L}}
    \iltikzfig{strings/traced/trace-id}
    \tensor
    \bigtensor_{e \in 0 \leq i \leq n}
    \iltikzfig{graphs/isomorphism/label-e}
    \tensor
    \iltikzfig{strings/category/identity}[white][n]
\), where \(\elabel{e}\) is the generator in \(\generators\) that \(e\) is
labelled with.
Putting this all together yields \(
    h := \termtohypsigma[\iltikzfig{graphs/isomorphism/construction}]
\).
While there may be multiple orderings on the edges, the possible terms
are equal by sliding and the naturality of symmetry, so there is one
unique term \(
    \iltikzfig{strings/category/f}[H][white]
\) that corresponds to cospan (\ref{gat:cospan}).

It is clear by definition that \(
    \termtohypsigma[\iltikzfig{strings/category/f}[H][white]]
\) produces (\ref{gat:cospan}), which is isomorphic to the original
cospan \(\cospan{m}[f]{F}[g]{n}\), so it is in the image of
\(\termtohypsigma\).
    \end{proof}
}{
    \begin{proof}[Proof (Sketch)]
        The \(\onlyifdir\) direction is by induction on the structure of the term.
        For the \(\ifdir\) direction, a cospan isomorphic to the original
        cospan can be constructed from which a term in \(\stmcsigma\) can be
        read off.
        Informally, this cospan is
        \begin{gather}
            \trace{}{
                \cospan{x + m}{V}{x + n}
                \seq
                \cospan{x + n}{L + E + n}{x + n}
            }
        \end{gather}
        where \(V\) is all the vertices in \(F\), \(L\) is the vertices with
        degree \((0, 0)\) not in the image of the interfaces, and \(E\) is the
        all the hyperedges in \(F\).
        The morphisms then connect the components appropriately.
        The first component corresponds to a term in \(\smc{}\) by
        \cref{lem:symmetries-prop}, and the stack of edges to a tensor of
        generators in \(\stmc{\Sigma}\).
    \end{proof}
}

The final step is to show that \(\termtohypsigma\) is \emph{faithful}, i.e.\
no two terms \(\stmcsigma\) equal up to the equations of STMCs corresponds to
the same cospan of hypergraphs.
First we isolate the terms that actually contain traces:

\begin{lemma}
    If \(\iltikzfig{strings/category/f}[H][white]\) is a term in \(
        \stmc{\Sigma} \setminus \smc{\Sigma}
    \), then \(
        \termtohypsigma[\iltikzfig{strings/category/f}[H][white]]
    \) either fails to be monogamous, acyclic, or both.
\end{lemma}
\iftoggle{proofs}{
    \begin{proof}
        Assume that \(\iltikzfig{strings/category/f}[H][white]\) is monogamous
        acyclic, so it is in \(\macspfihyp\).
        By \cref{thm:smc-graph-iso}, any graph in \(\macspfihyp\) corresponds to
        a unique term in \(\smc{\Sigma}\) up to axioms of SMCs: a contradiction,
        as \(
            \iltikzfig{strings/category/f}[H][white]
            \in
            \stmc{\Sigma} \setminus \smc{\Sigma}
        \).
    \end{proof}
}{}

\begin{corollary}
    The images of \(\smc{\Sigma}\) and \(\stmc{\Sigma} \setminus \smc{\Sigma}\)
    are disjoint under \(\termtohypsigma\).
\end{corollary}

Finally, we use a result from \cite{bonchi2022string} for the symmetric monoidal
case to conclude a similar result about our traced case.

\begin{corollary}[\cite{bonchi2022string}, Corollary 3.11]\label{cor:termtohyp-faithful-smc}
    \(\morph{\termtohypsigma}{\stmc{\Sigma}}{\pmcspfihyp}\) is faithful when the
    domain is restricted to \(\smc{\Sigma}\).
\end{corollary}

\begin{theorem}
    \(\termtohypsigma\) is faithful when the domain is restricted to \(
        \stmc{\Sigma} \setminus \smc{\Sigma}
    \).
\end{theorem}
\iftoggle{proofs}{
    \begin{proof}
        Let \(
            \iltikzfig{strings/category/f}[F][white]
        \) and \(
            \iltikzfig{strings/category/f}[G][white]
        \) be terms in \(\stmc{\Sigma}\) such that \(
            \iltikzfig{strings/category/f}[F][white]
            \neq
            \iltikzfig{strings/category/f}[G][white]
        \).
        By applying axioms of STMCs, these terms can be rewritten as \(
            \iltikzfig{strings/traced/trace-rhs}[f^\prime][white]
        \) and \(
            \iltikzfig{strings/traced/trace-rhs}[g^\prime][white]
        \) respectively, where \(
            \iltikzfig{strings/traced/trace-lhs}[f^\prime][white]
        \) and \(
            \iltikzfig{strings/traced/trace-lhs}[g^\prime][white]
        \) are terms in \(\smc{\Sigma}\).
        Since \(
            \iltikzfig{strings/category/f}[F][white]
            \neq
            \iltikzfig{strings/category/f}[F][white]
        \), then \(
            \iltikzfig{strings/traced/trace-lhs}[f^\prime][white]
            \neq
            \iltikzfig{strings/traced/trace-lhs}[g^\prime][white]
        \).
        Assume that \(
            \termtohypsigma[\iltikzfig{strings/category/f}[F][white]]
        \) and \(
            \termtohypsigma[\iltikzfig{strings/category/f}[F][white]]
        \) are isomorphic as cospans.
        Then \(
            \termtohypsigma[\iltikzfig{strings/traced/trace-rhs}[f^\prime][white]]
        \) and \(
            \termtohypsigma[\iltikzfig{strings/traced/trace-rhs}[g^\prime][white]]
        \) are also isomorphic.
        However, since \(\termtohypsigma\) is faithful on terms in \(\smc{\Sigma}\)
        by \cref{cor:termtohyp-faithful-smc}, \(
            \termtohypsigma[\iltikzfig{strings/traced/trace-lhs}[f^\prime][white]]
        \) and \(
            \termtohypsigma[\iltikzfig{strings/traced/trace-lhs}[g^\prime][white]]
        \) are not isomorphic.

        The only axiom of STMCs that allows for the trace of non-equal morphisms
        to be equal is the sliding axiom.
        But for such a situation to be valid here, the original terms must also
        be equal by sliding, a contradiction.
        So \(
            \termtohypsigma[\iltikzfig{strings/category/f}[F][white]]
        \) and \(
            \termtohypsigma[\iltikzfig{strings/category/f}[G][white]]
        \) are not isomorphic: \(\termtohypsigma\) is faithful.
    \end{proof}
}{}

\begin{corollary}
    \(\termtohypsigma\) is faithful.
\end{corollary}

\begin{corollary}\label{cor:stmc-graph-iso}
    \(\stmc{\Sigma} \cong \pmcspfihyp\).
\end{corollary}