% !TeX root = ../main-conf.tex

\section{Digital circuits}

We use the circuits framework of~\cite{ghica2022compositional}.
However, we augment it with the \emph{strictifiers} of~\cite{wilson2022stringa}

\begin{definition}[Circuit signature, value, gate symbol~\cite{ghica2022compositional}]
    A \emph{circuit signature} \(\circuitsignature\) is a tuple \((
        \circuitsignaturevalues,
        \disconnected,
        \shortcircuit,
        \circuitsignaturegates,
        \circuitsignaturearity,
        \circuitsignaturecoarity
    )\) where \(\circuitsignaturevalues\) is a finite set of \emph{values}, \(
        \disconnected \in \circuitsignaturevalues
    \) is a \emph{disconnected} value, \(
        \shortcircuit \in \circuitsignaturevalues
    \) is a \emph{short-circuit} value, \(\circuitsignaturegates\) is a (usually
    finite) set of \emph{gate symbols}, \(
        \morph{\circuitsignaturearity}{\circuitsignaturegates}{\nat}
    \) is an \emph{arity} function and \(
        \morph{\circuitsignaturecoarity}{\circuitsignaturegates}{\nat}
    \) is a \emph{coarity} function.
\end{definition}

\begin{definition}[Combinational circuits~\cite{ghica2022compositional}]
    Let \(\ccircsigma\) be the symmetric monoidal category generated over:
    \begin{gather*}
        \iltikzfig{circuits/components/gates/gate}[g][\circuitsignaturearity(g)][\circuitsignaturecoarity(g)]
        \text{ for each }
        g \in \circuitsignaturegates
        ,
        \iltikzfig{strings/structure/monoid/init}[comb]
        ,
        \iltikzfig{strings/structure/comonoid/copy}[comb]
        ,
        \iltikzfig{strings/structure/monoid/merge}[comb]
        ,
        \iltikzfig{strings/structure/comonoid/discard}[comb]
        \text{, and }
        \iltikzfig{strings/strictifiers/expand}[comb][m][n]
        ,
        \iltikzfig{strings/strictifiers/collapse}[comb][m][n]
        \text{ for } m, n \in \nat
    \end{gather*}
\end{definition}

\begin{definition}[Sequential circuits~\cite{ghica2022compositional}]
    For a circuit signature \(\Sigma = (
        \circuitsignaturevalues,
        \disconnected,
        \shortcircuit,
        \circuitsignaturegates,
        \circuitsignaturearity,
        \circuitsignaturecoarity
    )\), let \(\scircsigma\) be the symmetric traced monoidal category generated
    over the generators of \(\ccircsigma\) and \(
        \iltikzfig{circuits/components/waveforms/delay}
    \).
\end{definition}

\(\ccircsigma\) and \(\scircsigma\) define the \emph{syntax} of digital
circuits.
Since these are symmetric (traced) categories, the axioms of STMCs in
\cref{fig:stmc-equations} hold by default; since we are using string diagrams
they can be `applied' by moving boxes and wires around without altering their
connectivity.
However, more interesting equations that show the effects of \emph{computation}
must be applied explicitly.
In \cite{ghica2022compositional} several equational theories for digital
circuits were presented: we will focus on one in this paper.

\begin{definition}[\cite{ghica2022compositional}]
    Let \(\scircsigmal\) be defined as \(\scircsigma\) quotiented by the
    equations in \cref{fig:bialgebra,fig:local-equations}.
\end{definition}

\begin{figure*}
    \centering
    \combinationalequationslist
    \begin{minipage}[b]{0.215\textwidth}
        \forkgateeqn
    \end{minipage}
    \hspace{-0.6em}
    \begin{minipage}[b]{0.174\textwidth}
        \stubgateeqn
    \end{minipage}
    \hspace{-0.6em}
    \begin{minipage}[b]{0.171\textwidth}
        \stubdelayeqn
    \end{minipage}
    \hspace{-0.6em}
    \begin{minipage}[b]{0.164\textwidth}
        \forkjoininverseeqn
    \end{minipage}
    \hspace{-0.6em}
    \begin{minipage}[b]{0.2714\textwidth}
        \streamingeqn
    \end{minipage}
    \begin{minipage}[b]{0.16\textwidth}
        \disconnecteqn
    \end{minipage}
    \begin{minipage}[b]{0.2\textwidth}
        \forkdelayeqn
    \end{minipage}
    \begin{minipage}[b]{0.2\textwidth}
        \joindelayeqn
    \end{minipage}
    \begin{minipage}[b]{0.19\textwidth}
        \instantfeedbackeqn
    \end{minipage}
    \begin{minipage}[b]{0.21\textwidth}
        \delaydiscardeqn
    \end{minipage}
    \caption{
        Axioms of \(\scircsigmal\).
        See also \cref{app:equations}, \cref{fig:bialgebra-axioms}.
    }
    \label{fig:local-equations}
\end{figure*}