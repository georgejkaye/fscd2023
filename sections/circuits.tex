% !TeX root = ../main-conf.tex

\section{Digital circuits}

We use the circuits framework of~\cite{ghica2022compositional}.

\begin{definition}[Circuit signature, value, gate symbol~\cite{ghica2022compositional}]
    A \emph{circuit signature} \(\circuitsignature\) is a tuple \((
        \circuitsignaturevalues,
        \disconnected,
        \shortcircuit,
        \circuitsignaturegates,
        \circuitsignaturearity
    )\) where \(\circuitsignaturevalues\) is a finite set of \emph{values}, \(
        \disconnected \in \circuitsignaturevalues
    \) is a \emph{disconnected} value, \(
        \shortcircuit \in \circuitsignaturevalues
    \) is a \emph{short-circuit} value, \(
        \circuitsignaturegates
    \) is a (usually finite) set of \emph{gate symbols}, and \(
        \morph{\circuitsignaturearity}{\circuitsignaturegates}{\nat^\star}
    \) is an \emph{arity} function.
\end{definition}

The distinct elements \(\bullet\) and \(\circ\) represent a
\emph{disconnected wire} (a \emph{lack} of information) and a
\emph{short circuit} (\emph{inconsistent} information) respectively.
We have a made slight generalisation, motivated by the desire to succinctly
represent wires of arbitrary bit-width.
The arity of a gate is now expressed as a \emph{list} of natural numbers.
This makes a clear distinction between

\begin{example}[Gate-level circuits]\label{ex:sig}
    The signature for \emph{gate-level circuits} is \(
        \belnapsignature = (
            \belnapvalues,
            \belnapnone,
            \belnapboth,
            \belnapgates,
            \belnaparity
    )\), where \(
        \belnapvalues := \{\belnapnone, \belnapfalse, \belnaptrue, \belnapboth\}
    \), respectively representing \emph{no} signal, a \emph{false} signal, a
        \emph{true} signal and \emph{both} signals at once, \(
        \belnapgates := \{\andgate,\orgate,\notgate\}
    \), and \(
        \belnaparity :=
            \andgate \mapsto [1, 1],
            \orgate \mapsto [1, 1],
            \notgate \mapsto [1]
    \).
\end{example}

We are particularly interested in \emph{props}, symmetric monoidal categories
with natural numbers as objects and addition as tensor product.
Traditionally, morphisms \(m \to n\) in a prop are drawn string diagrammatically
as a box with \(m\) wires as inputs and \(n\) wires as outputs.
We will adopt the more conventional string diagram of each object (i.e.\ each
natural number) with its own (labelled) wire.
This ties the notation closer with the ports of components on circuits,
which may have arbitrary widths.
When discussing circuits in general, we may label wires with lists of natural
numbers: this is merely a syntactic sugar for drawing multiple wires in parallel.


\begin{definition}[Combinational circuits~\cite{ghica2022compositional}]
    Let \(\ccircsigma\) be the prop generated over \(
        \iltikzfig{circuits/components/gates/gate}[g][\circuitsignaturearity(g)]
    \) for each \(
        g \in \circuitsignaturegates
    \) and \(
        \iltikzfig{strings/structure/monoid/init}[comb][n]
    \), \(
        \iltikzfig{strings/structure/comonoid/copy}[comb][n]
    \), \(
        \iltikzfig{strings/structure/monoid/merge}[comb][n]
    \), \(
        \iltikzfig{strings/structure/comonoid/discard}[comb][n]
    \), \(
        \iltikzfig{strings/strictifiers/expand}[comb][m][n]
    \), \(
        \iltikzfig{strings/strictifiers/collapse}[comb][m][n]
    \) for \(m, n \in \nat\).
\end{definition}

The first generators are \emph{gates}, and the remainder are \emph{strucutural}
generators for \emph{introducing}, \emph{forking}, \emph{joining}, \emph{stubbing},
\emph{expanding} or \emph{collapsing} wires.
The latter two generators were not present in the original formulation
in~\cite{ghica2022compositional}: they have been adapted
from~\cite{wilson2022stringa}.

\begin{definition}[Sequential circuits~\cite{ghica2022compositional}]
    For a circuit signature \(\Sigma = (
        \circuitsignaturevalues,
        \disconnected,
        \shortcircuit,
        \circuitsignaturegates,
        \circuitsignaturearity,
        \circuitsignaturecoarity
    )\), let \(\scircsigma\) be the traced prop generated
    over the generators of \(\ccircsigma\) and \(
        \iltikzfig{circuits/components/waveforms/delay}[n]
    \) for \(m \in \nat\).
\end{definition}

\(\ccircsigma\) and \(\scircsigma\) define the \emph{syntax} of digital
circuits.
Since these are symmetric (traced) categories, the axioms of STMCs in
\cref{fig:stmc-equations} hold by default; since we are using string diagrams
they can be `applied' by moving boxes and wires around without altering their
connectivity.
However, more interesting equations that show the effects of \emph{computation}
must be applied explicitly.
In \cite{ghica2022compositional} several equational theories for digital
circuits were presented: we will focus on the \emph{local} equational theory
in this paper.

\begin{definition}[\cite{ghica2022compositional}]
    Let \(\scircsigmal\) be defined as \(\scircsigma\) quotiented by the
    equations in \cref{fig:bialgebra,fig:local-equations}.
\end{definition}

\begin{figure*}
    \centering
    \combinationalequationslist
    \begin{minipage}[b]{0.215\textwidth}
        \forkgateeqn
    \end{minipage}
    \hspace{-0.6em}
    \begin{minipage}[b]{0.174\textwidth}
        \stubgateeqn
    \end{minipage}
    \hspace{-0.6em}
    \begin{minipage}[b]{0.171\textwidth}
        \stubdelayeqn
    \end{minipage}
    \hspace{-0.6em}
    \begin{minipage}[b]{0.164\textwidth}
        \forkjoininverseeqn
    \end{minipage}
    \hspace{-0.6em}
    \begin{minipage}[b]{0.2714\textwidth}
        \streamingeqn
    \end{minipage}
    \begin{minipage}[b]{0.16\textwidth}
        \disconnecteqn
    \end{minipage}
    \begin{minipage}[b]{0.2\textwidth}
        \forkdelayeqn
    \end{minipage}
    \begin{minipage}[b]{0.2\textwidth}
        \joindelayeqn
    \end{minipage}
    \begin{minipage}[b]{0.19\textwidth}
        \instantfeedbackeqn
    \end{minipage}
    \begin{minipage}[b]{0.21\textwidth}
        \delaydiscardeqn
    \end{minipage}
    \caption{
        Axioms of \(\scircsigmal\).
        See also \cref{app:equations}, \cref{fig:bialgebra-axioms}.
    }
    \label{fig:local-equations}
\end{figure*}

Reasoning with circuits can then be performed by applying equations in
\(\scircsigmal\).
In particular, \cite{ghica2022compositional} sketches an
\emph{operational semantics} for digital circuits by applying an equation
that shows how a sequential circuit processes an input.

\begin{theorem}[\cite{ghica2022compositional}]
    The \eqref{eq:cycle} equation holds in \(\scircsigmal\).
\end{theorem}

\begin{figure*}
    \centering
    \begin{equation*}
        \tag{\(\mathsf{Cycle}\)}
        \iltikzfig{circuits/axioms/cycle-lhs}
        =
        \iltikzfig{circuits/axioms/cycle-lhs}
        \label{eq:cycle}
    \end{equation*}
    \caption{The }
\end{figure*}