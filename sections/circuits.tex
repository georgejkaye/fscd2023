% !TeX root = ../main-conf.tex

\subsection{Digital circuits}

A traced monoidal theory with a comonoid structure that is of particular
interest to us is the \emph{local theory of sequential circuits} from
\cite[Sec. VI]{ghica2022compositional}.

\begin{definition}[Gate-level circuits]
    Let the monoidal theory of \emph{gate-level sequential circuits} be defined
    as \(
        (\generators[\mathbf{SCirc}], \equations[\mathbf{SCirc}])
    \), where \[
        \generators[\mathbf{SCirc}]
        :=
        \{
            \iltikzfig{circuits/components/gates/and},
            \iltikzfig{circuits/components/gates/or},
            \iltikzfig{circuits/components/gates/not},
            \iltikzfig{strings/structure/comonoid/copy}[comb],
            \iltikzfig{strings/structure/monoid/merge}[comb],
            \iltikzfig{strings/structure/comonoid/discard}[comb],
            \iltikzfig{circuits/components/values/v}[\belnapnone],
            \iltikzfig{circuits/components/values/v}[\belnaptrue],
            \iltikzfig{circuits/components/values/v}[\belnapfalse],
            \iltikzfig{circuits/components/values/v}[\belnapboth],
            \iltikzfig{circuits/components/waveforms/delay}
        \}
    \] and \[
        \equations[\mathbf{SCirc}]
    \] are listed in \cref{fig:local-equations}, where \(
        \iltikzfig{circuits/components/circuits/f-1-2}[F^n][comb][m][x][n]
    \) is defined inductively as \(
        \iltikzfig{circuits/instant-feedback/f0-box}
        :=
        \iltikzfig{circuits/instant-feedback/f0-definition}
    \) and \(
        \iltikzfig{circuits/instant-feedback/fkp1-box}
        :=
        \iltikzfig{circuits/instant-feedback/fkp1-definition}
    \).
\end{definition}

The first generators are \emph{gates}, and the remainder are \emph{structural}
generators for \emph{introducing}, \emph{forking}, \emph{joining}, \emph{stubbing},
\emph{expanding} or \emph{collapsing} wires.
The latter two generators were not present in the original formulation
in~\cite{ghica2022compositional}: they have been adapted
from~\cite{wilson2022stringa}.


The smaller generators with no inputs are \emph{instantaneous values}: these
specify the initial state of a circuit.
The diamond is a \emph{delay} generator, which can be thought of as delaying its
input by one unit of time.

\begin{figure*}
    \centering
    \begin{minipage}[b]{0.28\textwidth}
        \begin{equation}
            \tag{\(\mathsf{Gate}\)}
            \iltikzfig{circuits/axioms/gate-lhs}
            =
            \iltikzfig{circuits/axioms/gate-rhs}
            \label{eq:gate}
        \end{equation}
    \end{minipage}%
    \begin{minipage}[b]{0.24\textwidth}
        \begin{equation}\textbf{}
            \tag{\(\mathsf{Fork}\)}
            \iltikzfig{circuits/axioms/fork-lhs}[v]
            =
            \iltikzfig{circuits/axioms/fork-rhs}[v]
            \label{eq:fork}
        \end{equation}
    \end{minipage}%
    \begin{minipage}[b]{0.25\textwidth}
        \begin{equation}
            \tag{\(\mathsf{Join}\)}
            \iltikzfig{circuits/axioms/join-lhs}[v][w]
            =
            \iltikzfig{circuits/axioms/join-rhs}[v][w]
            \label{eq:join}
        \end{equation}
    \end{minipage}

    \begin{minipage}[b]{0.22\textwidth}
        \begin{equation}
            \tag{\(\mathsf{Stub}\)}
            \iltikzfig{circuits/axioms/stub-lhs}[v]
            =
            \iltikzfig{strings/monoidal/empty}
            \label{eq:stub}
        \end{equation}
    \end{minipage}
    \begin{minipage}[b]{0.3\textwidth}
        \forkgateeqn
    \end{minipage}
    \begin{minipage}[b]{0.25\textwidth}
        \stubgateeqn
    \end{minipage}
    \begin{minipage}[b]{0.25\textwidth}
        \stubdelayeqn
    \end{minipage}
    \begin{minipage}[b]{0.22\textwidth}
        \forkjoininverseeqn
    \end{minipage}
    \begin{minipage}[b]{0.4\textwidth}
        \streamingeqn
    \end{minipage}
    \begin{minipage}[b]{0.25\textwidth}
        \disconnecteqn
    \end{minipage}
    \begin{minipage}[b]{0.25\textwidth}
        \forkdelayeqn
    \end{minipage}
    \begin{minipage}[b]{0.25\textwidth}
        \joindelayeqn
    \end{minipage}
    \begin{minipage}[b]{0.3\textwidth}
        \instantfeedbackeqn
    \end{minipage}
    \begin{minipage}[b]{0.35\textwidth}
        \delaydiscardeqn
    \end{minipage}
    \caption{
        Axioms of \(\scircsigmal\).
        See also \cref{app:equations}, \cref{fig:bialgebra-axioms}.
    }
    \label{fig:local-equations}
\end{figure*}

Reasoning with circuits can then be performed by applying equations in
\(\scircsigmal\).
In particular, \cite{ghica2022compositional} sketches an
\emph{operational semantics} for digital circuits by applying an equation
that shows how a sequential circuit processes an input.

\begin{theorem}[\cite{ghica2022compositional}]
    The \eqref{eq:cycle} equation in \cref{fig:cycle} holds in \(\scircsigmal\).
\end{theorem}

\begin{figure*}
    \centering
    \begin{equation*}
        \tag{\(\mathsf{Cycle}\)}
        \iltikzfig{circuits/productivity/productive-lhs}[F][s][v]
        =
        \iltikzfig{circuits/productivity/productive-step-9}
        \label{eq:cycle}
    \end{equation*}
    \caption{The cycle equation, which holds in \(\scircsigmal\)}
    \label{fig:cycle}
\end{figure*}