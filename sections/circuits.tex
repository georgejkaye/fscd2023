% !TeX root = ../main-conf.tex

\subsection{Digital circuits}
\label{sec:digital-circuits}

A traced monoidal theory with a comonoid structure that is of particular
interest to us is the \emph{local theory of sequential circuits} from
\cite[Sec. VI]{ghica2022compositional}.

\begin{definition}[Gate-level circuits]
    Let the monoidal theory of \emph{gate-level sequential circuits} be defined
    as \(
        (\generators[\mathbf{SCirc}], \equations[\mathbf{SCirc}])
    \), where \[
        \generators[\mathbf{SCirc}]
        :=
        \{
            \iltikzfig{circuits/components/gates/and},
            \iltikzfig{circuits/components/gates/or},
            \iltikzfig{circuits/components/gates/not},
            \iltikzfig{strings/structure/monoid/init}[comb]
            \iltikzfig{strings/structure/comonoid/copy}[comb],
            \iltikzfig{strings/structure/monoid/merge}[comb],
            \iltikzfig{strings/structure/comonoid/discard}[comb],
            \iltikzfig{circuits/components/values/v}[\belnaptrue],
            \iltikzfig{circuits/components/values/v}[\belnapfalse],
            \iltikzfig{circuits/components/values/v}[\belnapboth],
            \iltikzfig{circuits/components/waveforms/delay}
        \}
    \] and the equations of \(
        \equations[\mathbf{SCirc}]
    \) are listed in
    \cref{fig:monoid-equations,fig:comonoid-equations,fig:bialgebra-equations,fig:circuit-equations},
    where
    \(
        \gateinterpretation
    \) maps gates to the corresponding truth table, \(\ljoin\) is the join in
    the information lattice, and
    \(
        \iltikzfig{circuits/components/circuits/f-1-2}[F^n][comb][m][x][n]
    \) is defined inductively as \(
        \iltikzfig{circuits/instant-feedback/f0-box}
        :=
        \iltikzfig{circuits/instant-feedback/f0-definition}
    \) and \(
        \iltikzfig{circuits/instant-feedback/fkp1-box}
        :=
        \iltikzfig{circuits/instant-feedback/fkp1-definition}
    \).
\end{definition}

\begin{figure}
    \centering
    \begin{minipage}{0.28\textwidth}
        \begin{equation}
            \tag{\(\mathsf{B1}\)}
            \iltikzfig{strings/structure/bialgebra/merge-copy-lhs}
            =
            \iltikzfig{strings/structure/bialgebra/merge-copy-rhs}
            \label{eq:bialgebra-merge-copy}
        \end{equation}
    \end{minipage}
    \begin{minipage}{0.23\textwidth}
        \begin{equation}
            \tag{\(\mathsf{B2}\)}
            \iltikzfig{strings/structure/bialgebra/init-copy-lhs}
            =
            \iltikzfig{strings/structure/bialgebra/init-copy-rhs}
            \label{eq:bialgebra-init-copy}
        \end{equation}
    \end{minipage}
    \begin{minipage}{0.23\textwidth}
        \begin{equation}
            \tag{\(\mathsf{B3}\)}
            \iltikzfig{strings/structure/bialgebra/merge-discard-lhs}
            =
            \iltikzfig{strings/structure/bialgebra/merge-discard-rhs}
            \label{eq:bialgebra-merge-discard}
        \end{equation}
    \end{minipage}
    \begin{minipage}{0.2\textwidth}
        \begin{equation}
            \tag{\(\mathsf{B4}\)}
            \iltikzfig{strings/structure/bialgebra/init-discard-lhs}
            =
            \iltikzfig{strings/structure/bialgebra/init-discard-rhs}
            \label{eq:bialgebra-init-discard}
        \end{equation}
    \end{minipage}
    \caption{
        Equations \(\equations[\bialg]\) of a \emph{bialgebra}, in
        addition to those in
        \cref{fig:monoid-equations,fig:comonoid-equations}.
    }
    \label{fig:bialgebra-equations}
\end{figure}
\begin{figure}[t]
    \centering
    \iltikzfig{circuits/axioms/gate-lhs}
    \(=\)
    \iltikzfig{circuits/axioms/gate-rhs}
    \quad
    \iltikzfig{circuits/axioms/fork-lhs}[v]
    \(=\)
    \iltikzfig{circuits/axioms/fork-rhs}[v]
    \quad
    \iltikzfig{circuits/axioms/join-lhs}[v][w]
    \(=\)
    \iltikzfig{circuits/axioms/join-rhs}[v][w]
    \quad
    \iltikzfig{circuits/axioms/stub-lhs}[v]
    \(=\)
    \iltikzfig{strings/monoidal/empty}

    \vspace{1em}

    \iltikzfig{circuits/axioms/fork-gate-lhs}
    \(=\)
    \iltikzfig{circuits/axioms/fork-gate-rhs}
    \quad
    \iltikzfig{circuits/axioms/gate-stub-lhs}
    \(=\)
    \iltikzfig{circuits/axioms/gate-stub-rhs}
    \quad
    \iltikzfig{circuits/axioms/unobservable-lhs}
    \(=\)
    \iltikzfig{circuits/axioms/unobservable-rhs}
    \quad
    \iltikzfig{strings/structure/frobenius/copy-merge-lhs}
    \(=\)
    \iltikzfig{strings/structure/frobenius/copy-merge-rhs}
    \quad
    \iltikzfig{circuits/axioms/streaming-lhs-verbose}[g][v][m]
    \(=\)
    \iltikzfig{circuits/axioms/streaming-rhs}[g][v][n]

    \vspace{1em}

    \iltikzfig{circuits/axioms/bottom-delay-lhs}
    \(=\)
    \iltikzfig{circuits/axioms/bottom-delay-rhs}
    \quad
    \iltikzfig{circuits/axioms/delay-fork-lhs}
    \(=\)
    \iltikzfig{circuits/axioms/delay-fork-rhs}
    \quad
    \iltikzfig{circuits/axioms/delay-join-lhs}
    \(=\)
    \iltikzfig{circuits/axioms/delay-join-rhs}
    \quad
    \iltikzfig{circuits/instant-feedback/equation-lhs}[F][m][n][x]
    \(=\)
    \iltikzfig{circuits/instant-feedback/fixpoint-concrete}
    \quad
    \iltikzfig{circuits/axioms/delay-discard-lhs}[F][m][x]
    \(=\)
    \iltikzfig{circuits/axioms/delay-discard-rhs}[m]
    \caption{
        The equations of \(\equations[\mathbf{SCirc}]\), from the monoidal
        theory of gate-level sequential circuits.
    }
    \label{fig:circuit-equations}
\end{figure}
\begin{figure*}
    \centering
    \begin{equation*}
        \tag{\(\mathsf{Cycle}\)}
        \iltikzfig{circuits/productivity/productive-lhs-verbose}[F][s][v][m][n]
        =
        \iltikzfig{circuits/productivity/productive-step-9}[F][s][v][m][n]
        \label{eq:cycle}
    \end{equation*}
    \caption{
        The cycle equation, which is derivable from the equations in
        \(\equations[\mathbf{SCirc}]\)
    }
    \label{fig:cycle}
\end{figure*}

The generators in \(\generators[\mathbf{SCirc}]\) are, respectively:
\(\andgate\), \(\orgate\) and \(\notgate\) gates; constructs for introducing,
forking, joining and stubbing wires; \emph{values} representing a true signal,
a false signal, and both signals at once; and a delay of one unit of time.

The equations of \(\equations[\mathbf{SCirc}]\) contain the equations of a
commutative comonoid, so this is a perfect use case for rewriting modulo
trace commutative comonoid structure.
Using graph rewriting, we can sketch out an \emph{operational semantics} for
sequential circuits.
For the interests of brevity, we will only consider circuits of the form \(
    \iltikzfig{circuits/productivity/mealy-form-verbose}[f][v][m][n]
\): circuits with no `non-delay-guarded feedback' in which the registers of the
circuit have been isolated from a core containing only `blue'
(\emph{combinational}) components, which models a function.
Any sequential circuit can be translated into such a form by the equational
theory.

We can `apply' such a circuit to an input as shown in the left-hand side of
\cref{fig:cycle}; the equations in
\(\equations[\mathbf{SCirc}]\) can be used to derive the right-hand side.
More equations can then be applied to reduce the two `new' cores down to
values, representing the output and new state of the circuit.

When the circuits are interpreted as hypergraphs and the equations as rewrites,
a computer could perform this sequence of rewrites in order to evaluate circuits
in a step-by-step manner.
