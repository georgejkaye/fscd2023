% !TeX root = ../main-conf.tex

\section{Digital circuits}

We use the circuits framework of~\cite{ghica2022compositional}.

\begin{definition}[Circuit signature, value, gate symbol~\cite{ghica2022compositional}]
    A \emph{circuit signature} \(\circuitsignature\) is a tuple \((
        \circuitsignaturevalues,
        \disconnected,
        \shortcircuit,
        \circuitsignaturegates,
        \circuitsignaturearity
    )\) where \(\circuitsignaturevalues\) is a finite set of \emph{values}, \(
        \disconnected \in \circuitsignaturevalues
    \) is a \emph{disconnected} value, \(
        \shortcircuit \in \circuitsignaturevalues
    \) is a \emph{short-circuit} value, \(
        \circuitsignaturegates
    \) is a (usually finite) set of \emph{gate symbols}, and \(
        \morph{\circuitsignaturearity}{\circuitsignaturegates}{\nat^\star}
    \) is an \emph{arity} function.
\end{definition}

The distinct elements \(\bullet\) and \(\circ\) represent a
\emph{disconnected wire} (a \emph{lack} of information) and a
\emph{short circuit} (\emph{inconsistent} information) respectively.
We have a made slight generalisation, motivated by the desire to succinctly
represent wires of arbitrary bit-width.
The arity of a gate is now expressed as a \emph{list} of natural numbers.
This makes a clear distinction between

\begin{example}[Gate-level circuits]\label{ex:sig}
    The signature for \emph{gate-level circuits} is \(
        \belnapsignature = (
            \belnapvalues,
            \belnapnone,
            \belnapboth,
            \belnapgates,
            \belnaparity
    )\), where \(
        \belnapvalues := \{\belnapnone, \belnapfalse, \belnaptrue, \belnapboth\}
    \), respectively representing \emph{no} signal, a \emph{false} signal, a
        \emph{true} signal and \emph{both} signals at once, \(
        \belnapgates := \{\andgate,\orgate,\notgate\}
    \), and \(
        \belnaparity :=
            \andgate \mapsto [1, 1],
            \orgate \mapsto [1, 1],
            \notgate \mapsto [1]
    \).
\end{example}

A circuit signature generates two \emph{monoidal} signatures: sets of
generators with associated lists of (co)arities.

\begin{definition}[Monoidal signature]
    A monoidal signature is a set of \emph{generators} \(\mathcal{G}\) along
    with two functions \(
        \morph{\dom,\cod}{G}{nat^\star}
    \).
\end{definition}

Since we are working with string diagrams, we will draw the generators as
tiles, and sometimes merge wires into one, labelled when appropriate.

\begin{definition}
    Let the monoidal signature for \emph{combinational signatures} for a given
    circuit signature be defined as \(
        G^C_\Sigma := \{
            \iltikzfig{circuits/components/gates/gate}[g][\circuitsignaturearity(g)]
            \, | \,
            g \in \circuitsignaturegates
        \}
        +
        \{
            \iltikzfig{strings/structure/monoid/init}[comb][n],
            \iltikzfig{strings/structure/comonoid/copy}[comb][n],
            \iltikzfig{strings/structure/monoid/merge}[comb][n],
            \iltikzfig{strings/structure/comonoid/discard}[comb][n],
            \iltikzfig{strings/strictifiers/expand}[comb][m][n],
            \iltikzfig{strings/strictifiers/collapse}[comb][m][n]
        \}
    \).
\end{definition}

The first generators are \emph{gates}, and the remainder are \emph{structural}
generators for \emph{introducing}, \emph{forking}, \emph{joining}, \emph{stubbing},
\emph{expanding} or \emph{collapsing} wires.
The latter two generators were not present in the original formulation
in~\cite{ghica2022compositional}: they have been adapted
from~\cite{wilson2022stringa}.

Monoidal signatures specify a freely generated \emph{prop}: a category of
\emph{pro}ducts and \emph{p}ermutations.
Props are symmetric monoidal categories with natural numbers as objects and
addition as tensor product.
Terms in a prop freely generated over a monoidal signature are defined by
juxtaposing the generators in a signature horizontally or vertically with each
other, the identity \(
    \iltikzfig{strings/category/identity}[comb][n]
\) and the symmetry \(
    \iltikzfig{strings/symmetric/symmetry}[comb][m][n]
\).

\begin{definition}
    Let \(\ccircsigma\) be the prop freely generated over
    \(G^C_\Sigma\).
\end{definition}

To model real-world circuits, we need a notion of \emph{state} into circuits.

\begin{definition}
    Let the monoidal signature for \emph{sequential circuits} be defined as \(
    G^S_\Sigma
    +
    \{
        \iltikzfig{circuits/components/values/v}[v]
        \, | \,
        v \in \circuitsignaturevalues \setminus \bullet
    \}
    +
    \iltikzfig{circuits/components/waveforms/delay}
\).
\end{definition}

The smaller generators with no inputs are \emph{instantaneous values}: these
specify the initial state of a circuit.
The diamond is a \emph{delay} generator, which can be thought of as delaying its
input by one unit of time.
We also need a notion of \emph{feedback}, which can be handled by adding some
extra structure to the category.

\begin{definition}[\cite{joyal1996traced,hasegawa2009traced}]
    A \emph{symmetric traced monoidal category} (STMC) is a
    symmetric monoidal category \(\mcc\) equipped with functions
    \(
        \morph{
            \trace{X}[A][B]{-}
        }{
            \mcc(X \tensor A, X \tensor B)
        }{
            \mcc(X, Y)
        }
    \) satisfying the axioms of STMCs listed in \cref{app:equations},
    \cref{fig:stmc-axioms}.
\end{definition}

\begin{definition}
    Let \(\scircsigma\) be the traced prop freely generated over \(G^S_\Sigma\).
\end{definition}

\(\ccircsigma\) and \(\scircsigma\) define the \emph{syntax} of digital
circuits.
Semantics are specified with an \emph{interpretation}.
Recall that a function \(f\) between two posets is \emph{monotone}
if \(x \leq y \Rightarrow f(x) \leq f(y)\): monotone functions are the
interpretation of combinational circuits.
We write \(\bot^n\) for the \(n\)-tuple containing only \(\bot\) values.

\begin{definition}[Interpretation~\cite{ghica2022compositional}]\label{def:interpretation}
    An interpretation of a signature \(
        \circuitsignature = (
            \circuitsignaturevalues,
            \bullet,
            \circ,
            \circuitsignaturegates,
            \circuitsignaturearity
        )
    \) is a tuple \(
        \interpretation = (
            \values,
            \valueinterpretation,
            \gateinterpretation
        )
    \) where \((\values,\sqsubseteq, \bot,\top)\) is a finite lattice,
    \(\valueinterpretation\) is a function
    \(\circuitsignaturevalues \to \values\),
    and \(\gateinterpretation\) is a map sending each \(
        g \in \circuitsignaturegates
    \) to a monotone function \(
        \valuetuple{\circuitsignaturearity(g)} \to \values
    \). These functions are required to satisfy \(
        \valueinterpretation[\bullet] = \bot
    \); \(
        \valueinterpretation[\circ] = \top
    \); \(
        \gateinterpretation[g](\bot^m) = \bot
    \); and \(
        \gateinterpretation[g](\overline{v})
    \) is in the image of \(\valueinterpretation\) for all \(
        \overline{v} \in \valuetuple{m}
    \).
\end{definition}

The semantics of digital circuits are defined using a prop morphism into
a category of \emph{stream functions}.
However, it is more intuitive to reason \emph{equationally}.
Since these are symmetric (traced) categories, the axioms of STMCs in
\cref{fig:stmc-equations} hold by default; since we are using string diagrams
they can be `applied' by moving boxes and wires around without altering their
connectivity.
However, more interesting equations that show the effects of \emph{computation}
must be applied explicitly.
In \cite{ghica2022compositional} several equational theories for digital
circuits were presented: we will focus on the \emph{local} equational theory
in this paper.
In particular, this includes an equation for eliminating
\emph{non-delay-guarded} feedback from circuits.

\begin{figure*}
    \centering
    \combinationalequationslist
    \begin{minipage}[b]{0.215\textwidth}
        \forkgateeqn
    \end{minipage}
    \hspace{-0.6em}
    \begin{minipage}[b]{0.174\textwidth}
        \stubgateeqn
    \end{minipage}
    \hspace{-0.6em}
    \begin{minipage}[b]{0.171\textwidth}
        \stubdelayeqn
    \end{minipage}
    \hspace{-0.6em}
    \begin{minipage}[b]{0.164\textwidth}
        \forkjoininverseeqn
    \end{minipage}
    \hspace{-0.6em}
    \begin{minipage}[b]{0.2714\textwidth}
        \streamingeqn
    \end{minipage}
    \begin{minipage}[b]{0.16\textwidth}
        \disconnecteqn
    \end{minipage}
    \begin{minipage}[b]{0.2\textwidth}
        \forkdelayeqn
    \end{minipage}
    \begin{minipage}[b]{0.2\textwidth}
        \joindelayeqn
    \end{minipage}
    \begin{minipage}[b]{0.19\textwidth}
        \instantfeedbackeqn
    \end{minipage}
    \begin{minipage}[b]{0.21\textwidth}
        \delaydiscardeqn
    \end{minipage}
    \caption{
        Axioms of \(\scircsigmal\).
        See also \cref{app:equations}, \cref{fig:bialgebra-axioms}.
    }
    \label{fig:local-equations}
\end{figure*}

\begin{definition}[Iteration~\cite{ghica2022compositional}]
    For a combinational circuit \(
        \iltikzfig{circuits/components/circuits/f-2-2}[F][comb][x][m][x][n],
    \)
    let its \emph{\(n\)th iteration} \(
        \iltikzfig{circuits/components/circuits/f-1-2}[F^n][comb][m][x][n]
    \) be defined inductively as \(
        \iltikzfig{circuits/instant-feedback/f0-box}
        :=
        \iltikzfig{circuits/instant-feedback/f0-definition}
    \) and \(
        \iltikzfig{circuits/instant-feedback/fkp1-box}
        :=
        \iltikzfig{circuits/instant-feedback/fkp1-definition}
    \).
\end{definition}

\begin{definition}[Fixpoint]
    Given an interpretation with value lattice \(\values\), the \emph{fixpoint}
    of a combinational circuit \(
        \iltikzfig{circuits/components/circuits/f-2-2}[F][comb][x][m][x][n]
    \), denoted as \(
        \iltikzfig{circuits/components/circuits/f-1-2}[F^\dagger][comb][m][x][n]
    \), is defined as \(
        \iltikzfig{circuits/instant-feedback/fc-box}[m][x][n]
    \) where \(c\) is the length of the longest chain in \(\valuetuple{x}\).
\end{definition}

\begin{definition}[\cite{ghica2022compositional}]
    Let \(\scircsigmal\) be defined as \(\scircsigma\) quotiented by the
    equations in \cref{fig:bialgebra,fig:local-equations}.
\end{definition}

Reasoning with circuits can then be performed by applying equations in
\(\scircsigmal\).
In particular, \cite{ghica2022compositional} sketches an
\emph{operational semantics} for digital circuits by applying an equation
that shows how a sequential circuit processes an input.

\begin{theorem}[\cite{ghica2022compositional}]
    The \eqref{eq:cycle} equation in \cref{fig:cycle} holds in \(\scircsigmal\).
\end{theorem}

\begin{figure*}
    \centering
    \begin{equation*}
        \tag{\(\mathsf{Cycle}\)}
        \iltikzfig{circuits/productivity/productive-lhs}[F][s][v]
        =
        \iltikzfig{circuits/productivity/productive-step-9}
        \label{eq:cycle}
    \end{equation*}
    \caption{The cycle equation, which holds in \(\scircsigmal\)}
    \label{fig:cycle}
\end{figure*}