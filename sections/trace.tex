% !TeX root = ../main-conf.tex
\section{Hypergraphs for traced categories}

We wish to use the hypergraph framework for a setting with a \emph{trace}.

\begin{definition}[Symmetric tracd monoidal category \cite{joyal1996traced,hasegawa2009traced}]
    A \emph{symmetric traced monoidal category} (STMC) is a symmetric monoidal
    category \(\mathbf{C}\) equipped with a family of functions \(
        \morph{
            \trace{X}[A][B]{-}
        }{
            \mathbf{C}(X \tensor A, X \tensor B)
        }{
            \mathbf{C}(A, B)
        }
    \) for any objects \(A,B\) and \(X\) satisfying the axioms of STMCs listed
    in \cref{app:equations}, \cref{fig:stmc-axioms}.
\end{definition}

In the string diagram notation, the trace is represented by joining some of the
inputs of a circuit to its outputs.
%
\begin{center}
    \(
        \trace{X}[A][B]{\iltikzfig{strings/traced/trace-lhs}[F][white][X][A][B]}
        \stackrel{\text{def}}{=}
        \iltikzfig{strings/traced/trace-rhs}[F][white][A][B]
    \)
\end{center}

Traced monoidal categories are not the only kind of category in which wires can
`bend'.

\begin{definition}[Compact closed category]
    A \emph{compact closed category} (CCC) is a symmetric monoidal category in
    which every object \(X\) has a \emph{dual} \(\dual{X}\) equipped with
    morphisms called the \emph{unit} \(
        \iltikzfig{strings/compact-closed/cup}[white][\dual{X}][X]
    \) and the \emph{counit} \(
        \iltikzfig{strings/compact-closed/cap}[white][X][\dual{X}]
    \) satisfying the equations of CCCs listed in \cref{app:equations},
    \cref{fig:ccc-axioms}.
\end{definition}

Dual objects are conventionally drawn as wires flowing the `other way', but in
this paper this will not be necessary as all categories will be
\emph{self-dual}: any object \(X\) is isomorphic to \(\dual{X}\).

\begin{proposition}[Canonical trace (\cite{joyal1996traced}, Prop. 3.1)]
    \label{prop:canonical-trace}
    Any compact closed category has a trace called the \emph{canonical trace} \(
        \trace{X}[A][B]{\iltikzfig{strings/category/f-2-2}[f][white][X][A][X][B]}
    \) defined for the self-dual case as \[
        \left(
            \iltikzfig{strings/compact-closed/cup}[white][X][X]
            \tensor
            \iltikzfig{strings/category/identity}[white][A]
        \right)
        \seq
        \left(
            \iltikzfig{strings/category/identity}[white][X]
            \tensor
            \iltikzfig{strings/category/f-2-2}[f][white][X][A][X][B]
        \right)
        \seq
        \left(
            \iltikzfig{strings/compact-closed/cap}[white][X][X]
            \tensor
            \iltikzfig{strings/category/identity}[white][B]
        \right).
    \]
\end{proposition}

The category of interfaced hypergraphs as defined in the previous section
already contains the structure necessary to define a trace.

\begin{definition}[Hypergraph category \cite{fong2019hypergraph}]
    A \emph{hypergraph category} is a symmetric monoidal category in which
    each object \(X\) has a special commutative Frobenius structure in the sense
    of \cref{ex:frobenius} satisfying the equations in \cref{app:equations},
    \cref{fig:hypergraph-category}.
\end{definition}

\begin{proposition}[\cite{rosebrugh2005generic}]
    Any hypergraph category is self-dual compact closed.
\end{proposition}
\begin{proof}
    The cup is constructed as \(
        \iltikzfig{strings/structure/monoid/init}[white]
        \seq
        \iltikzfig{strings/structure/comonoid/copy}[white]
    \) and the cap as \(
        \iltikzfig{strings/structure/monoid/merge}[white]
        \seq
        \iltikzfig{strings/structure/comonoid/discard}[white]
    \).
\end{proof}

A generic `hypergraph category' should not be confused with the
category of hypergraphs \(\hyp\), which is not itself a hypergraph category.
However, the category of \emph{cospans} of hypergraphs is such a category.

\begin{proposition}[\cite{carboni1987cartesian,bonchi2022string}]\label{prop:frobenius-map}
    \(\cspdhyp\) is a hypergraph category.
\end{proposition}
\begin{proof}
    A Frobenius structure can be defined on \(\cspdhyp\) for each \(m \in \nat\)
    as follows:
    \begin{gather*}
        \iltikzfig{strings/structure/monoid/merge}[white]
        :=
        \cospan{m + m}{m}{m}
        \quad
        \iltikzfig{strings/structure/monoid/init}[white]
        :=
        \cospan{0}{m}{m}
        \\
        \iltikzfig{strings/structure/comonoid/copy}[white]
        :=
        \cospan{m}{m}{m+m}
        \quad
        \iltikzfig{strings/structure/comonoid/discard}[white]
        :=
        \cospan{m}{m}{0}
        \qedhere
    \end{gather*}
\end{proof}

\begin{corollary}
    \(\cspdhyp\) is compact closed.
\end{corollary}

\begin{corollary}
    \(\cspdhyp\) has a trace.
\end{corollary}

This means that a STMC freely generated over a signature faithfully embeds into
a CCC generated over the same signature, mapping the trace in the former to the
canonical trace in the latter.
However, this mapping is not \emph{full}: there are terms in a CCC that are not
terms in a STMC, such as \(\iltikzfig{strings/traced/invalid}\).
This means we must still restrict the cospans of hypergraphs in \(\cspdhyp\) we
use for \emph{traced} terms.

\subsection{Monogamy}

In~\cite{bonchi2016rewriting}, it is shown that terms in a (non-traced)
symmetric monoidal category are interpreted via a faithful functor into a
sub-PROP of \(\cspdhyp\).
One condition on this sub-PROP is that all hypergraphs are \emph{acyclic}.
Clearly, to model trace this condition must be dropped.

However, there is also another condition known as \emph{monogamy}: informally,
this means that every vertex has exactly one `in' and `out' connection, be it to
an edge or an interface.
For the most part, this condition also applies to the traced case: wires cannot
arbitrarily fork and join.
There is one nuance: the trace of the identity.
This is depicted as a closed loop \(
    \trace{1}{\iltikzfig{strings/category/identity}[white]}
    =
    \iltikzfig{strings/traced/trace-id}
\), and one might think that it can be discarded, i.e. \(
    \iltikzfig{strings/traced/trace-id}
    =
    \iltikzfig{strings/monoidal/empty}
\).
This is \emph{not} always the case, such as in the category of finite
dimensional vector spaces~\cite[Sec. 6.1]{hasegawa1997recursion}.
These closed loops must be represented in the hypergraph framework:
there is a natural representation as a lone vertex disconnected
from either interface.
In fact, this is exactly how the canonical trace applied to an identity is
interpreted in \(\cspdhyp\).

\begin{definition}
    For a hypergraph \(F \in \hyp\), the \emph{degree} of a vertex
    \(v \in \vertices{F}\) is a tuple \((i,o)\) where \(i\) is the number of
    pairs \((e,i)\) where \(e\) is a hyperedge with \(v\) as its \(i\)th target,
    and \(o\) is similarly the number of pairs \((e,j)\) where \(e\) is a
    hyperedge with \(v\) as its \(j\)th target.
\end{definition}

\begin{definition}
    For a cospan \(\cospan{m}[f]{F}[g]{n} \in \cspdhyp\), we say it is
    \emph{partial monogamous} if \(f\) and \(g\) are mono and, for all nodes
    \(v \in \vertices{F}\), the degree of \(v\) is

    \begin{center}
        \begin{tabular}{rlcrl}
            \((0,0)\)
            &
            if \(v \in f_\star \wedge v \in g_\star\)
            &
            \quad
            &
            \((0,1)\)
            &
            if \(v \in f_\star\)
            \\
            \((1,0)\)
            &
            if \(v \in g_\star\)
            &
            \quad
            &
            \((0,0)\)
            or \((1,1)\)
            &
            otherwise
        \end{tabular}
    \end{center}
\end{definition}

\iftoggle{conf}{}{
    \begin{lemma}\label{lem:trace-degree}
        Given a cospan \(\cospan{x+m}[h+f]{F}[k+g]{x+n} \in \cspdhyp\), let \(p\) be
        the map sending vertices in \(\vertices{F}\) to where they are sent by
        the pushouts constructing \(\trace{x}{\cospan{x+m}{F}{x+n}}\).
        Then if the degree of \(h(i)\) is \((p_1, q_1)\) and the degree of
        \(k(i)\) is \((p_2, q_2)\), then the degree of \(p(h(i)) = p(k(i))\) is
        \((p_1 + p_2, q_1 + q_2)\).
    \end{lemma}
    \iftoggle{proofs}{
        \begin{proof}

        \end{proof}
    }{}
}

\begin{theorem}~\label{thm:partial-monogamous-ops}
    Let \(\cospan{m}{F}{n}\), \(\cospan{n}{G}{p}\), \(\cospan{p}{H}{q}\) and
    \(\cospan{x+m}{K}{x+n}\) be partial monogamous cospans in \(\cspdhyp\).
    Then,
    \begin{itemize}
        \item identities and symmetries are partial monogamous;
        \item \(\cospan{m}{F}{n} \seq \cospan{n}{G}{p}\) is partial monogamous;
        \item \(\cospan{m}{F}{n} \tensor \cospan{p}{H}{q}\) is partial
                monogamous; and
        \item \(\trace{x}{\cospan{x+m}{K}{x+n}} \) is partial monogamous.
    \end{itemize}
\end{theorem}
\iftoggle{proofs}{
    \begin{proof}
        Since any monogamous hypergraph is also partial monogamous, the first
        three points hold due to~\cite[Prop.16]{bonchi2022string}, dropping the
        acyclicity condition.
        For the final condition, consider the image of \(f\) and \(g\).
        For each \(i \in x\), there are two cases to consider: \(f(i) = g(i)\)
        and \(f(i) \neq g(i)\).

        In the former, the degree of \(v^\prime = f(v) = h(v)\) must be
        \((0,0)\) by definition of partial monogamicity.
        Therefore in the traced hypergraph \(
            \cospan{m}{K^\prime}{n}
        \), \(v^\prime\) will still have degree \((0,0)\) by
        \cref{lem:trace-degree}, and cannot be in the interfaces, so the
        hypergraph is still partial monogamous.

        In the latter case, \(f(i)\) must have degree of \((0,0)\) if it in the
        image of \(k\) or \((0,1)\) otherwise.
        Similarly \(g(i)\) either has degree \((0,0)\) or \((1,0)\).
        Let \(v := p(f(i)) = p(g(i))\); we now consider the degree of \(v\)
        computed using \cref{lem:trace-degree}:
        \begin{itemize}
            \item If \(f(i)\) is in the image of \(k\) and \(g(i)\) is in the
                    image of \(h\), then \(v\) has degree \((0,0)\).
                    \(v\) is in the image of \(h \seq p\) and \(h \seq p\), so
                    the cospan is partial monogamous.
            \item If \(f(i)\) is in the image of \(k\), then \(v\) has degree
                    \((1, 0)\); since \(v\) is in the image of \(k \seq p\), the
                    cospan is partial monogamous.
            \item If \(g(i)\) is in the image of \(h\), then \(v\) has degree
                    \((0, 1)\); since \(v\) is in the image of \(h \seq p\), the
                    cospan is partial monogamous.
            \item Otherwise, \(v\) will have degree \((1, 1)\), and is not in
                    the image of either interface so the cospan is partial
                    monogamous.
        \end{itemize}
    \end{proof}
}{}

This means that the partial monogamous hypergraphs form a sub-PROP of
\(\cspdhyp\).

\begin{definition}
    Let \(\pmcspdhyp\) be the sub-PROP of \(\cspdhyp\) containing only the
    partial monogamous cospans of hypergraphs.
\end{definition}

Crucially, while we leave \(\pmcspdhyp\) in order to construct the trace using
the cup and cap, the resulting cospan \emph{is} in \(\pmcspdhyp\).

\subsection{From terms to graphs}

\begin{definition}
    For a SMT \((\generators,\equations)\), let
    \(\stmc{\generators}\) be the STMC freely generated over the
    generators in \(\generators\).
    Let \(\stmc{\generators, \equations}\) be \(\stmc{\generators}\) quotiented
    by equations in \(\equations\).
\end{definition}

A \emph{(traced) PROP morphism} is a strict (traced) symmetric monoidal functor
between PROPs.
For \(\pmcspfihyp\) to be suitable for reasoning with a traced category
\(\stmc{\Sigma}\) for some given signature, there must be a
\emph{fully complete} PROP morphism \(\stmc{\Sigma} \to \pmcspfihyp\): a full
and faithful functor from terms to cospans of hypergraphs.

Since \(\stmc{\Sigma}\) and \(\pmcspdhyp\) are freely generated, to define a
PROP morphism between them it suffices to define it on the generators in the
former.

\begin{definition}[\cite{bonchi2022string}]\label{def:hyp-morphisms}
    Let \(\morph{\termtohyp{\Sigma}}{\smc{\Sigma}}{\cspdhyp}\) be a prop
    morphism defined as \begin{gather*}
        \termtohyp[\iltikzfig{strings/category/generator}[\phi][white][m][n]]{\Sigma}
            :=
            \cospan{m}{
                \iltikzfig{graphs/terms/generator}
            }{n}
        \\
        \termtohyp[\iltikzfig{strings/category/identity}[white][n][n]]{\Sigma}
        :=
        \cospan{n}[\id]{n}[\id]{n}
        \quad
        \termtohyp[\iltikzfig{strings/symmetric/symmetry}[white][m][n]]{\Sigma}
            :=
        \cospan{m+n}[[\id,\id]]{m+n}[[\id,\id]]{n+m}
    \end{gather*}
    Let \(\morph{\frobtohyp{\Sigma}}{\frob}{\cspdhyp}\) be a PROP morphism
    defined as in \cref{prop:frobenius-map}.
    Then, let \(
        \morph{\termandfrobtohypsigma}{\smc{\Sigma} + \frob}{\cspdhyp}
    \)
    be the copairing of \(\termtohyp{\Sigma} + \frobtohyp{\Sigma}\).
\end{definition}

We exploit the interplay between compact closed and traced categories in
order to use these existing PROP morphisms for the traced case.

\begin{lemma}\label{lem:smc-core}
    Let \(\iltikzfig{strings/category/f}[f][white][m][n]\) be a term in
    \(\stmc{\Sigma}\).
    Then there exists at least one \(
        \iltikzfig{strings/category/f-2-2}[g][white][x][m][x][n] \in \smc{\Sigma}
    \) such that \(
        \trace{x}{\iltikzfig{strings/category/f-2-2}[g][white]}
        =
        \iltikzfig{strings/category/f}[f][white]
    \).
\end{lemma}
\iftoggle{proofs}{
    \begin{proof}
        Any trace can be made a `global trace' by applying \eqref{eq:tightening}
        and \eqref{eq:superposing}.
    \end{proof}
}{}

\begin{proposition}\label{prop:trace-to-sym-frob}
    There exists a faithful PROP morphism \(
        \morph{\tracedtosymandfrob{\Sigma}}{\stmc{\Sigma}}{\smc{\Sigma} + \frob}
    \).
\end{proposition}
\begin{proof}
    \cref{lem:smc-core} is used to isolate a term in \(\smc{\Sigma}\).
    The corresponding term in \(\smc{\Sigma} + \frob\) is then the canonical
    trace of this term.
    There may be many such terms in \(\smc{\Sigma}\), but the canonical trace
    being a trace means that any possible outcomes post-trace are all equal.
    The equations of \(\frob\) do not merge any morphisms
    since the only use of the generators of \(\frob\) is in the canonical trace,
    to which the Frobenius equations do not apply.
\end{proof}

\iftoggle{conf}{}{
    Recall one of the key theorems from~\cite{bonchi2022stringa}.

    \begin{theorem}[\cite{bonchi2022stringa}, Corollary 20]\label{thm:smc-graph-iso}
        \(\smc{\Sigma} \cong \macspfihyp\).
    \end{theorem}
}

Before progressing to the main theorem, we must show a result about terms in
\(\smc{}\), i.e.\ terms constructed from just symmetries and identities.
In particular, there is a correspondence between \(\smc{}\) and
\emph{bijective} functions.

\begin{definition}
    Let \(\perms\) be the sub-PROP of \(\finset\) containing only the
    bijective functions.
\end{definition}

\begin{lemma}\label{lem:symmetries-prop}
    \(\smc{} \cong \perms\).
\end{lemma}
\iftoggle{proofs}{
    \begin{proof}
        The morphism \(\morph{\phi}{\smc{}}{\perms}\) is defined over
        generators in \(\smc{}\) as \[
            \phi(\iltikzfig{strings/monoidal/empty}) = \{\}
            \quad
            \phi(\iltikzfig{strings/category/identity}[white])
            =
            \{0 \mapsto 0\}
            \quad
            \phi(\iltikzfig{strings/symmetric/symmetry}[white])
            =
            \{0 \mapsto 1, 1 \mapsto 0\}
        \]
        Since any term in \(\smc{}\) can be expressed using these generators,
        this defines the complete transformation.

        The reverse morphism \(\morph{\psi}{\finset}{\smc{}}\) is inductively
        over the size of \(m\).
        For the base case \(\morph{f}{[0]}{[0]}\), let \(
            \phi(f) := \iltikzfig{strings/monoidal/empty}
        \).
        For \(
            \morph{f}{[k+1]}{[k+1]}
        \), let \(i\) such that \(f(i) = k+1\), and define the function \(
            \morph{f^\prime}{\nat_{k}}{\nat_{k}}
        \) as the function such that \(
            f^\prime(j) = f(j)
        \) if \(j < i\), and \(f(j+1)\) otherwise.
        Then \[
            \psi(f) := \iltikzfig{strings/symmetric/f-construction}.
        \]

        These are shown to be inverses by a simple induction in both directions.
    \end{proof}
}{}

\begin{lemma}\label{lem:monog-discrete-cospan}
    Given a monogamous cospan \(\cospan{m}[f]{m}[g]{m}\), there exists a unique
    term \(\morph{h}{m}{m} \in \smc{}\) up to the axioms of SMCs such that
    \(\termtohyp[h]{\Sigma} = \cospan{m}[f]{m}[g]{m}\).
\end{lemma}
\iftoggle{proofs}{
    \begin{proof}
        Since the cospan is monogamous, \(f\) and \(g\) are mono.
        As the cospan is also discrete, there exists a (unique) bijective
        function \(\morph{h^\prime}{[m]}{[m]}\) such that \(h^\prime(i) = j\) if
        \(f(i) = g(j)\).
        By \cref{lem:symmetries-prop}, there is a corresponding term
        \(h \in \smc{}\) that is unique up to SMC axioms: a simple induction
        shows that \(\termtohyp[h]{\Sigma} = \cospan{m}[f]{m}[g]{m}\).
    \end{proof}
}{}

\begin{theorem}\label{thm:termtohyp-image}
    A cospan \(\cospan{m}{F}{n}\) is in the image of \(
        \termandfrobtohypsigma \circ \tracedtosymandfrob{\Sigma}\) if
    and only if it is partial monogamous.
\end{theorem}
\iftoggle{proofs}{
    \begin{proof}
        % !TeX root = ../../main-conf.tex
To show that \(\termtohyp[f]{\Sigma}\) is partial monogamous for any
\(f \in \smc{\Sigma}\) we use induction on the structure of \(f\).
Generators, identities and symmetries are partial monogamous, as
semi-monogamicity is preserved by composition, tensor and trace.
So \(\termtohyp[f]{\Sigma}\) is partial monogamous.

Now we show that any partial monogamous cospan \(
    \cospan{m}[f]{F}[g]{n}
\)
must be in the image of \(\termtohyp{\Sigma}\).
To do this, we will now construct a cospan that is isomorphic to
\(\cospan{m}[f]{F}[g]{n}\), but from which it is possible to read off a
unique term in \(\smc{\Sigma}\).
The components of the new cospan are as follows:
\begin{itemize}
    \item let \(L\) be the hypergraph containing vertices with degree
            \((0,0)\) that are not in the image of \(f\) or \(g\);
    \item let \(E\) be the hypergraph containing hyperedges of \(F\) and
            their source and target vertices, but disconnected;
    \item let \(V\) be the discrete hypergraph containing all the
            vertices of \(F\); and
    \item let \(S\) and \(T\) be the discrete hypergraphs containing
            the source and target vertices of hyperedges in \(F\)
            respectively, with the ordering determined by some order
            \(e_1,e_2,\cdots,e_n\) on the edges in \(F\).
\end{itemize}

These parts can be composed and a trace applied to obtain the follow
cospan:
\begin{gather}
    \trace{T}{
        \cospan{T + m}[\id + f]{V}[\id + g]{S + n}
        \,\seq\,
        \cospan{\emptyset + S + n}[\id]{L + E + n}[\id]{\emptyset + T + n}
    }
    \label{gat:cospan}
\end{gather}

This can be checked to be isomorphic to the original cospan
\(\cospan{m}[f]{F}[g]{n}\) by applying the pushouts.
From this we can read off a term in \(\smc{\Sigma}\):
Since the first cospan is monogamous, it corresponds to a term \(
    \iltikzfig{strings/category/f-2-2}[f][white][|\vertices{T}|][m][|\vertices{S}|][n]
\) by \cref{lem:monog-discrete-cospan}.
The second cospan corresponds to \(
    \iltikzfig{strings/category/f}[g][white][|\vertices{S}][\vertices{T}]
    :=
    \bigtensor_{v \in \vertices{L}}
    \iltikzfig{strings/traced/trace-id}
    \tensor
    \bigtensor_{e \in 0 \leq i \leq n}
    \iltikzfig{graphs/isomorphism/label-e}
    \tensor
    \iltikzfig{strings/category/identity}[white][n]
\), where \(\elabel{e}\) is the generator in \(\generators\) that \(e\) is
labelled with.
Putting this all together yields \(
    h := \termtohypsigma[\iltikzfig{graphs/isomorphism/construction}]
\).
While there may be multiple orderings on the edges, the possible terms
are equal by sliding and the naturality of symmetry, so there is one
unique term \(
    \iltikzfig{strings/category/f}[H][white]
\) that corresponds to cospan (\ref{gat:cospan}).

It is clear by definition that \(
    \termtohypsigma[\iltikzfig{strings/category/f}[H][white]]
\) produces (\ref{gat:cospan}), which is isomorphic to the original
cospan \(\cospan{m}[f]{F}[g]{n}\), so it is in the image of
\(\termtohypsigma\).
    \end{proof}
}{
    \begin{proof}[Proof (Sketch)]
        The \(\onlyifdir\) direction is by induction on the structure of the term.
        For the \(\ifdir\) direction, a cospan isomorphic to the original
        cospan can be constructed from which a term in \(\stmcsigma\) can be
        read off.
        Informally, this cospan is
        \begin{gather}
            \trace{}{
                \cospan{x + m}{V}{x + n}
                \seq
                \cospan{x + n}{L + E + n}{x + n}
            }
        \end{gather}
        where \(V\) is all the vertices in \(F\), \(L\) is the vertices with
        degree \((0, 0)\) not in the image of the interfaces, and \(E\) is the
        all the hyperedges in \(F\).
        The first component corresponds to a term in \(\smc{}\) by
        \cref{lem:symmetries-prop}, and the stack of edges to a tensor of
        generators in \(\stmc{\Sigma}\).
    \end{proof}
}

\begin{proposition}[\cite{bonchi2022string}]
    \(\termtohypsigma\) is faithful.
\end{proposition}

\begin{corollary}\label{cor:stmc-graph-iso}
    \(\stmc{\Sigma} \cong \pmcspfihyp\).
\end{corollary}