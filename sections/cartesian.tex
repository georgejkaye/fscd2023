% !TeX root = ../main-conf.tex
\section{Case studies}

\subsection{Cartesian structure}

One important class of categories with a traced comonoid structure are
\emph{traced Cartesian categories}~\cite{cazanescu1990new,hasegawa1997recursion}.
These categories are interesting because any traced Cartesian
category admits a fixpoint operator~\cite[Thm. 3.1]{hasegawa1997recursion}.

\begin{definition}[Cartesian category~\cite{fox1976coalgebras}]
    A monoidal category is \emph{Cartesian} if its tensor is given by the
    Cartesian product.
\end{definition}

As a result of this, the unit is a terminal object in any Cartesian category,
and any object has a comonoid structure.
Cartesian categories can also be expressed as a monoidal theory:

\begin{definition}
    For a given base PROP \(\stmc{\Sigma_\mathbf{C}}\) with a comonoid
    structure, the monoidal theory \((
        \generators[\mathbf{Cart}_\mathbf{C}],
        \equations[\mathbf{Cart}_\mathbf{C}]
    )\) is defined with \(
        \generators[\mathbf{Cart}_\mathbf{C}] := \Sigma_\mathbf{C}
    \) and \(
        \equations[\mathbf{Cart}_\mathbf{C}]
    \) as the equations in \cref{fig:cartesian-equations}.
\end{definition}

As the equations in \(\equations[\mathbf{Cart}_\mathbf{C}]\) are parameterised
over any morphism \(\iltikzfig{strings/category/f}[F][white][m][n]\), a separate
DPO rewrite rule is required for every combination of generators as in
\cref{fig:cartesian-equations}.

\begin{remark}
    The combination of Cartesian equations with the underlying compact closed
    structure of \(\cspdhyp\) may prompt alarm bells, as a compact closed
    category in which the tensor is the Cartesian product is trivial.
    However, it is important to note that \(\cspdhyp\) is \emph{not} subject to
    these equations: it is only a setting for performing graph
    rewrites.
\end{remark}

\begin{remark}
    One might define a finite set of Cartesian rewrite rules by assuming that
    the edge actually represents a subgraph, but what do the sources and targets
    of this edge mean if we would like to match on a single vertex?
    One solution would be to use the \emph{hierarchical hypergraphs}
    of~\cite{alvarez-picallo2022rewriting} to replace the concrete generator
    edge with a single edge representing an arbitrary subgraph: this remains
    as future work.
\end{remark}

\begin{figure}
    \centering
    \begin{minipage}{0.45\textwidth}
        \begin{equation}
            \tag{\(\mathsf{Copy}\)}
            \iltikzfig{strings/structure/cartesian/naturality-copy-lhs}[F][white][m][n]
            =
            \iltikzfig{strings/structure/cartesian/naturality-copy-rhs}[F][white][m][n]
            \label{eq:cartesian-copy}
        \end{equation}
    \end{minipage}
    \qquad
    \begin{minipage}{0.4\textwidth}
        \centering
        \begin{equation}
            \tag{\(\mathsf{Discard}\)}
            \iltikzfig{strings/structure/cartesian/naturality-discard-lhs}[F][white][m]
            =
            \iltikzfig{strings/structure/cartesian/naturality-discard-rhs}[m]
            \label{eq:cartesian-discard}
        \end{equation}
    \end{minipage}

    \begin{minipage}{0.45\textwidth}
        \begin{center}
            \includestandalone{figures/graphs/dpo/cartesian/copy/rule}
        \end{center}
    \end{minipage}
    \begin{minipage}{0.4\textwidth}
        \begin{center}
            \includestandalone{figures/graphs/dpo/cartesian/discard/rule}
        \end{center}
    \end{minipage}

    \caption{
        Equations of the monoidal theory \(\mathbf{Cart}_\mathcal{C}\),
        where \(\iltikzfig{strings/category/f}[F][white][m][n]\) is an arbitrary
        morphism in \(\mathcal{C}\), and the interpretations of these equations
        as rewrite rules for an arbitrary generator \(e\).
    }
    \label{fig:cartesian-equations}
\end{figure}