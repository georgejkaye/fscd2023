% !TeX root = ../main-conf.tex
\section{Cartesian structure}

One important class of categories with a traced comonoid structure are
\emph{traced Cartesian categories}~\cite{cazanescu1990new,hasegawa1997recursion}.
These categories are particularly interesting because any traced Cartesian
category admits a fixpoint operator~\cite[Thm. 3.1]{hasegawa1997recursion}.

\begin{definition}[Cartesian category]
    A monoidal category is \emph{Cartesian} if its tensor is given by the
    Cartesian product.
\end{definition}

Given a base category \(\mathcal{C}\) with a comonoid structure, the monoidal
theory \(\mathbf{Cart}_\mathcal{C}\) is shown in \cref{fig:cartesian-equations}.
Rather than two concrete equations, this actually specifies two \emph{families}
of equations over any morphism \(\iltikzfig{strings/category/f}[F][white][m][n]\)
in \(\mathcal{C}\).
When implemented as DPO rewrite rules, a separate concrete rule is required for
every combination of generators.
Future work could use the \emph{hierarchical hypergraphs}
of~\cite{alvarez-picallo2022rewriting} to express the complete families of rules
in terms of two rewrite rules, as with the equational representation.

\begin{figure}
    \centering
    \begin{minipage}{0.45\textwidth}
        \begin{equation}
            \tag{\(\mathsf{Copy}\)}
            \iltikzfig{strings/structure/cartesian/naturality-copy-lhs}[F][white][m][n]
            =
            \iltikzfig{strings/structure/cartesian/naturality-copy-rhs}[F][white][m][n]
            \label{eq:cartesian-copy}
        \end{equation}
        \begin{center}
            \includestandalone{figures/graphs/dpo/cartesian/copy/rule}
        \end{center}
    \end{minipage}
    \qquad
    \begin{minipage}{0.4\textwidth}
        \centering
        \begin{equation}
            \tag{\(\mathsf{Discard}\)}
            \iltikzfig{strings/structure/cartesian/naturality-discard-lhs}[F][white][m]
            =
            \iltikzfig{strings/structure/cartesian/naturality-discard-rhs}[m]
            \label{eq:cartesian-discard}
        \end{equation}
        \begin{center}
            \includestandalone{figures/graphs/dpo/cartesian/discard/rule}
        \end{center}
    \end{minipage}
    \caption{
        Equations of the monoidal theory \(\mathbf{Cart}_\mathcal{C}\),
        where \(\iltikzfig{strings/category/f}[F][white][m][n]\) is an arbitrary
        morphism in \(\mathcal{C}\), and the concrete interpretations of
        these rules for a generator \(e\).
    }
    \label{fig:cartesian-equations}
\end{figure}