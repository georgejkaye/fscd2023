% !TeX root = ../main-conf.tex
\section{Case studies}

\subsection{Cartesian structure}

One important class of categories with a traced comonoid structure are
\emph{traced Cartesian categories}~\cite{cazanescu1990new,hasegawa1997recursion}.
These categories are interesting when considering \emph{data flow} because any
traced Cartesian category admits a fixpoint
operator~\cite[Thm. 3.1]{hasegawa1997recursion}.

\begin{definition}[Cartesian category~\cite{fox1976coalgebras}]
    A monoidal category is \emph{Cartesian} if its tensor is given by the
    Cartesian product.
\end{definition}

As a result of this, the unit is a terminal object in any Cartesian category,
and any object has a comonoid structure.
Cartesian categories can also be expressed as a monoidal theory:

\begin{definition}
    For a given base PROP \(\stmc{\Sigma_\mathbf{C}}\) with a comonoid
    structure, the monoidal theory \((
        \generators[\mathbf{Cart}_\mathbf{C}],
        \equations[\mathbf{Cart}_\mathbf{C}]
    )\) is defined with \(
        \generators[\mathbf{Cart}_\mathbf{C}] := \Sigma_\mathbf{C}
    \) and \(
        \equations[\mathbf{Cart}_\mathbf{C}]
    \) as the equations in \cref{fig:cartesian-equations}.
\end{definition}

Note that as the equations in \(\equations[\mathbf{Cart}_\mathbf{C}]\) are
parameterised over any morphism
\(\iltikzfig{strings/category/f}[f][white][m][n]\), a separate
DPO rewrite rule is required for every combination of generators as in
\cref{fig:cartesian-equations}.
However, as is the case in the next section, it is often possible to
characterise the copying behaviour through a finite number of equations.

\begin{remark}
    The combination of Cartesian equations with the underlying compact closed
    structure of \(\cspdhyp\) may prompt alarm bells, as a compact closed
    category in which the tensor is the Cartesian product is trivial.
    However, it is important to note that \(\cspdhyp\) is \emph{not} subject to
    these equations: it is only a setting for performing graph
    rewrites.
\end{remark}

\begin{figure}
    \centering
    \begin{tabular}{cc}
        \iltikzfig{strings/structure/cartesian/naturality-copy-lhs}[f][white][m][n]
        \(=\)
        \iltikzfig{strings/structure/cartesian/naturality-copy-rhs}[f][white][m][n]
        &
        \iltikzfig{strings/structure/cartesian/naturality-discard-lhs}[f][white][m]
        \(=\)
        \iltikzfig{strings/structure/cartesian/naturality-discard-rhs}[m]
        \\[2em]
        \includestandalone{figures/graphs/dpo/cartesian/copy/rule}
        &
        \raisebox{1em}{\includestandalone{figures/graphs/dpo/cartesian/discard/rule}}
    \end{tabular}
    \caption{
        Equations of the monoidal theory \(\mathbf{Cart}_\mathcal{C}\),
        where \(\iltikzfig{strings/category/f}[f][white][m][n]\) is an arbitrary
        morphism in \(\mathcal{C}\), and the interpretations of these equations
        as rewrite rules for an arbitrary generator \(e\).
    }
    \label{fig:cartesian-equations}
\end{figure}

Fixpoints can be reasoned about using the \emph{unfolding} rule, which holds in
any traced Cartesian category.

\begin{center}
    \iltikzfig{strings/traced/trace-rhs}[f][white][m][n][x]
    \(=\)
    \iltikzfig{circuits/examples/reasoning/unfolding/unfolding-1}[f][white][m][n][x]
    \(=\)
    \iltikzfig{circuits/examples/reasoning/unfolding/unfolding-2}[f][white][m][n][x]
    \(=\)
    \iltikzfig{circuits/examples/reasoning/unfolding/unfolding-3}[f][white][m][n][x]
\end{center}

In the syntactic setting, this requires the application of multiple
equations: the two counitality equations followed by the copy equation and
finally some axioms of STMCs for `housekeeping'.
However, if we interpret this in the hypergraph setting, the comonoid equations
are absorbed into the notation so only the copy equation needs to be applied.

\begin{center}
    \includestandalone{figures/graphs/dpo/unfolding/rewrite-1}
\end{center}



The dual notion of traced \emph{cocartesian}
categories~\cite{bainbridge1976feedback} are also important in computer science:
a trace in a traced cocartesian category corresponds to \emph{iteration} in the
context of \emph{control flow}.
The details of this section could also be applied to the cocartesian case by
flipping all the directions and working with partial \emph{right}-monogamous
cospans.

However, attempting to combine the product and coproduct approaches for settings
with a \emph{biproduct} would simply yield the category \(\cspdhyp\), a
hypergraph category (\cref{prop:frobenius-map}) subject to the Frobenius
equations in \cref{fig:frobenius-equations}.
A category with biproducts is not necessarily subject to such equations, so this
would not be a suitable approach.