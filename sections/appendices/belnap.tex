% !TeX root = ../../main-conf.tex
\section{Belnap logic}
\label{app:belnap}

The equational theory of gate-level sequential circuits in
\cref{sec:digital-circuits} is based on the four-valued
\emph{Belnap logic}~\cite{belnap1977useful}.
The four values in question, \(\belnapnone\), \(\belnapfalse\), \(\belnaptrue\)
and \(\belnapboth\) form a lattice in two ways.
The first is the lattice of \emph{information order} shown in the left of
\cref{tab:truths}, which dictates how much `information content' a particular value
has.
The second is the lattice of \emph{logical order}, which can be seen as the
lattice in \cref{tab:truths} rotated by 90 degrees clockwise.
This lattice determines the outputs of the gates: the \(\andgate\) gate \(
    \iltikzfig{circuits/components/gates/and}
\) is the meet in this lattice and the \(\orgate\) gate \(
    \iltikzfig{circuits/components/gates/or}
\) is the join.
The \(\notgate\) gate \(
    \iltikzfig{circuits/components/gates/not}
\) acts in the usual way on \(\belnapfalse\) and \(\belnaptrue\), and as the
identity otherwise.
The truth tables are shown in the right of \cref{tab:truths}.

\begin{figure*}
    \centering
    \tikzfig{circuits/a4}
    \quad
    \begin{tabular}{|c|cccc|}
        \hline
        \(\iltikzfig{circuits/components/gates/and}\) & \(\belnapnone\) & \(\belnapfalse\) & \(\belnaptrue\) & \(\belnapboth\) \\
        \hline
        \(\belnapnone\)  & \(\belnapnone\) & \(\belnapfalse\) & \(\belnapnone\) & \(\belnapfalse\) \\
        \(\belnapfalse\) & \(\belnapfalse\) & \(\belnapfalse\) & \(\belnapfalse\) & \(\belnapfalse\) \\
        \(\belnaptrue\) & \(\belnapnone\) & \(\belnapfalse\) & \(\belnaptrue\) & \(\belnapboth\) \\
        \(\belnapboth\) & \(\belnapfalse\) & \(\belnapfalse\) & \(\belnapboth\) & \(\belnapboth\) \\
        \hline
    \end{tabular}
    \quad
    \begin{tabular}{|c|c|}
        \hline
        \(\iltikzfig{circuits/components/gates/not}\) & \\
        \hline
        \(\belnapnone\) & \(\belnapnone\) \\
        \(\belnaptrue\) & \(\belnapfalse\) \\
        \(\belnapfalse\) & \(\belnaptrue\) \\
        \(\belnapboth\) & \(\belnapboth\) \\
        \hline
    \end{tabular}
    \quad
    \begin{tabular}{|c|cccc|}
        \hline
        \(\iltikzfig{circuits/components/gates/or}\) & \(\belnapnone\) & \(\belnapfalse\) & \(\belnaptrue\) & \(\belnapboth\) \\
        \hline
        \(\belnapnone\)  & \(\belnapnone\) & \(\belnapnone\) & \(\belnaptrue\) & \(\belnaptrue\) \\
        \(\belnapfalse\) & \(\belnapnone\) & \(\belnapfalse\) & \(\belnaptrue\) & \(\belnapboth\) \\
        \(\belnaptrue\) & \(\belnaptrue\) & \(\belnaptrue\) & \(\belnaptrue\) & \(\belnaptrue\) \\
        \(\belnapboth\) & \(\belnaptrue\) & \(\belnapboth\) & \(\belnaptrue\) & \(\belnapboth\) \\
        \hline
    \end{tabular}

    \caption{The lattice structure on Belnap values, and the truth tables
    of Belnap logic gates~\cite{belnap1977useful}.}
    \label{tab:truths}
\end{figure*}