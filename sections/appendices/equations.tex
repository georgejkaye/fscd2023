\section{Equations}
\label{app:equations}

Equations that hold in any symmetric traced monoidal category are listed in
\cref{fig:stmc-axioms}.
These axioms were originally presented in~\cite{joyal1996traced}: an additional
vanishing axiom was originally also included but it was shown in
\cite{hasegawa2009traced} that this was redundant.

Equations in the monoidal theory of gate-level sequential circuits are listed in
\cref{fig:circuit-equations}.
These include the equations of a \emph{bialgebra}, which themselves include the
equations of a commutative comonoid.

\begin{figure}
    \centering
    \begin{minipage}{0.52\textwidth}
        \begin{align*}
            \tag{\(\mathsf{Tighten}\)}
            \iltikzfig{strings/traced/naturality-lhs}
            &=
            \iltikzfig{strings/traced/naturality-rhs}
            \label{eq:tightening}
            \\[0.75em]
            \tag{\(\mathsf{Slide}\)}
            \iltikzfig{strings/traced/sliding-lhs}
            &=
            \iltikzfig{strings/traced/sliding-rhs}
            \label{eq:sliding}
            \\[0.75em]
            \tag{\(\mathsf{Vanish}\)}
            \iltikzfig{strings/traced/vanishing-lhs}
            &=
            \iltikzfig{strings/traced/vanishing-rhs}
            \label{eq:vanishing}
        \end{align*}
    \end{minipage}
    \hspace{-1em}
    \begin{minipage}{0.455\textwidth}
        \begin{align*}
            \tag{\(\mathsf{Superpose}\)}
            \iltikzfig{strings/traced/superposing-lhs}
            &=
            \iltikzfig{strings/traced/superposing-rhs}
            \label{eq:superposing}
            \\[0.75em]
            \tag{\(\mathsf{Yank}\)}
            \iltikzfig{strings/traced/yanking-lhs}
            &=
            \iltikzfig{strings/traced/yanking-rhs}
            \label{eq:yanking}
        \end{align*}
    \end{minipage}
    \caption{Equations that hold in any STMC.}
    \label{fig:stmc-axioms}
\end{figure}

\begin{figure}
    \centering
    \begin{minipage}{0.3\textwidth}
        \begin{align}
            \tag{\(\mathsf{C1}\)}
            \iltikzfig{strings/structure/comonoid/unitality-l-lhs}
            &=
            \iltikzfig{strings/structure/comonoid/unitality-l-rhs}
            \label{eq:comonoid-unitality-l}
            \\[0.5em]
            \tag{\(\mathsf{C2}\)}
            \iltikzfig{strings/structure/comonoid/unitality-r-lhs}
            &=
            \iltikzfig{strings/structure/comonoid/unitality-r-rhs}
            \label{eq:comonoid-unitality-r}
            \\[0.5em]
            \tag{\(\mathsf{C3}\)}
            \iltikzfig{strings/structure/comonoid/associativity-lhs}
            &=
            \iltikzfig{strings/structure/comonoid/associativity-rhs}
            \label{eq:comonoid-associativity}
            \\[0.5em]
            \tag{\(\mathsf{C4}\)}
            \iltikzfig{strings/structure/comonoid/commutativity-lhs}
            &=
            \iltikzfig{strings/structure/comonoid/commutativity-rhs}
            \label{eq:comonoid-commutativity}
        \end{align}
    \end{minipage}
    \begin{minipage}{0.3\textwidth}
        \begin{align}
            \tag{\(\mathsf{M1}\)}
            \iltikzfig{strings/structure/monoid/unitality-l-lhs}
            &=
            \iltikzfig{strings/structure/monoid/unitality-l-rhs}
            \label{eq:monoid-unitality-l}
            \\[0.5em]
            \tag{\(\mathsf{M2}\)}
            \iltikzfig{strings/structure/monoid/unitality-r-lhs}
            &=
            \iltikzfig{strings/structure/monoid/unitality-r-rhs}
            \label{eq:monoid-unitality-r}
            \\[0.5em]
            \tag{\(\mathsf{M3}\)}
            \iltikzfig{strings/structure/monoid/associativity-lhs}
            &=
            \iltikzfig{strings/structure/monoid/associativity-rhs}
            \label{eq:monoid-associativity}
            \\[0.5em]
            \tag{\(\mathsf{M4}\)}
            \iltikzfig{strings/structure/monoid/commutativity-lhs}
            &=
            \iltikzfig{strings/structure/monoid/commutativity-rhs}
            \label{eq:monoid-commutativity}
        \end{align}
    \end{minipage}
    \begin{minipage}{0.3\textwidth}
        \begin{align}
            \tag{\(\mathsf{B1}\)}
            \iltikzfig{strings/structure/bialgebra/merge-copy-lhs}
            &=
            \iltikzfig{strings/structure/bialgebra/merge-copy-rhs}
            \label{eq:bialgebra-merge-copy}
            \\[0.5em]
            \tag{\(\mathsf{B2}\)}
            \iltikzfig{strings/structure/bialgebra/init-copy-lhs}
            &=
            \iltikzfig{strings/structure/bialgebra/init-copy-rhs}
            \label{eq:bialgebra-init-copy}
            \\[0.5em]
            \tag{\(\mathsf{B3}\)}
            \iltikzfig{strings/structure/bialgebra/merge-discard-lhs}
            &=
            \iltikzfig{strings/structure/bialgebra/merge-discard-rhs}
            \label{eq:bialgebra-merge-discard}
            \\[0.5em]
            \tag{\(\mathsf{B4}\)}
            \iltikzfig{strings/structure/bialgebra/init-discard-lhs}
            &=
            \iltikzfig{strings/structure/bialgebra/init-discard-rhs}
            \label{eq:bialgebra-init-discard}
        \end{align}
    \end{minipage}
    \begin{minipage}[b]{0.28\textwidth}
        \begin{equation}
            \tag{\(\mathsf{Gate}\)}
            \iltikzfig{circuits/axioms/gate-lhs}
            =
            \iltikzfig{circuits/axioms/gate-rhs}
            \label{eq:gate}
        \end{equation}
    \end{minipage}%
    \begin{minipage}[b]{0.24\textwidth}
        \begin{equation}\textbf{}
            \tag{\(\mathsf{Fork}\)}
            \iltikzfig{circuits/axioms/fork-lhs}[v]
            =
            \iltikzfig{circuits/axioms/fork-rhs}[v]
            \label{eq:fork}
        \end{equation}
    \end{minipage}%
    \begin{minipage}[b]{0.25\textwidth}
        \begin{equation}
            \tag{\(\mathsf{Join}\)}
            \iltikzfig{circuits/axioms/join-lhs}[v][w]
            =
            \iltikzfig{circuits/axioms/join-rhs}[v][w]
            \label{eq:join}
        \end{equation}
    \end{minipage}

    \begin{minipage}[b]{0.22\textwidth}
        \begin{equation}
            \tag{\(\mathsf{Stub}\)}
            \iltikzfig{circuits/axioms/stub-lhs}[v]
            =
            \iltikzfig{strings/monoidal/empty}
            \label{eq:stub}
        \end{equation}
    \end{minipage}
    \begin{minipage}[b]{0.3\textwidth}
        \forkgateeqn
    \end{minipage}
    \begin{minipage}[b]{0.25\textwidth}
        \stubgateeqn
    \end{minipage}
    \begin{minipage}[b]{0.25\textwidth}
        \stubdelayeqn
    \end{minipage}
    \begin{minipage}[b]{0.22\textwidth}
        \forkjoininverseeqn
    \end{minipage}
    \begin{minipage}[b]{0.4\textwidth}
        \streamingeqn
    \end{minipage}
    \begin{minipage}[b]{0.25\textwidth}
        \disconnecteqn
    \end{minipage}
    \begin{minipage}[b]{0.25\textwidth}
        \forkdelayeqn
    \end{minipage}
    \begin{minipage}[b]{0.25\textwidth}
        \joindelayeqn
    \end{minipage}
    \begin{minipage}[b]{0.3\textwidth}
        \instantfeedbackeqn
    \end{minipage}
    \begin{minipage}[b]{0.35\textwidth}
        \delaydiscardeqn
    \end{minipage}
    \caption{
        The equations of \(\equations[\mathbf{SCirc}]\), from the monoidal
        theory of gate-level sequential circuits.
        \cref{sec:digital-circuits}.
    }
    \label{fig:circuit-equations}
\end{figure}