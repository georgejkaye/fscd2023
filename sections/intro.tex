% !TeX root = ../main-conf.tex
\section{Introduction}

\begin{itemize}
    \item We want to reason/rewrite in traced categories with comonoid structure
            (e.g. dataflow categories are traced comonoid + cartesian,
            digital circuits is traced + comonoid + circuit axioms)
    \item Previous work on rewriting for traced monoidal categories with
            \emph{string diagrams} \cite{kissinger2012pictures,dixon2013opengraphs}
            but for these you have to work with rewriting modulo wire homeomorphisms
            which is a pain
    \item More recent work on rewriting with \emph{Frobenius} \cite{bonchi2022string}
            structure with hypergraphs, and subsequently for \emph{symmetric monoidal}
            by restricting the type of graphs and rewrites \cite{bonchi2022stringa}
    \item Comonoid structure is `half' Frobenius, (co)monoid structure has been studied
        too \cite{fritz2022free,milosavljevic2022string}
    \item The trace sits in the middle of compact closed/frobenius and symmetric
        monoidal (notion of causal vs relational)
    \item The trace can be constructed using the Frobenius structure (and this
        is fine in the land of hypergraphs because e.g. Cartesian equations do
        not hold by default, and since we do the trace all in one go we don't
        end up with degeneracies)
    \item \textbf{Contribution 1:} show that a subclass of cospans of hypergraphs
        called \emph{partial monogamous} are in correspondence with
        traced terms, and \emph{partial left-monogamous} cospans are in
        correspondence with traced comonoid terms.
    \item \textbf{Design choice:} what precisely should a partial monogamous
        hypergraph be? Sticking point e.g. how to represent trace of the
        identity (closed loop). In the frobenius realm this is a disconnected
        vertex not in the interfaces: is this okay for traced? When adding
        comonoid structure does this still work with the trace of the fork?
    \item When rewriting with Frobenius, every pushout complement is valid.
        When rewriting with symmetric monoidal, exactly one pushout complement
        is valid.
        In the traced case, \emph{some} are valid.
    \item \textbf{Contribution 2:} show that \emph{traced boundary complements}
        correspond to valid traced rewrites
    \item \textbf{Example}: there is a pushout complement below that is valid
        in Frobenius because it uses the monoid structure, but not in a
        traced or even traced comonoid setting
    \begin{itemize}
        \item Imagine we have a rule \(\rrule{
            \iltikzfig{graphs/dpo/non-valid/rule-lhs}
        }{
            \iltikzfig{graphs/dpo/non-valid/rule-rhs}
        }\) and a term \(
            \iltikzfig{graphs/dpo/non-valid/term}
        \).
        When performing graph rewriting modulo \emph{Frobenius}, the following
        DPO rewrite is valid:
    \end{itemize}
    \begin{center}
        \includestandalone{figures/graphs/dpo/non-valid/rewrite}
    \end{center}
    This corresponds to the term rewrite \(
        \iltikzfig{graphs/dpo/non-valid/term}
        =
        \iltikzfig{graphs/dpo/non-valid/term-rewriting}
        =
        \iltikzfig{graphs/dpo/non-valid/term-rewritten}
    \), which holds in a Frobenius setting, but not a setting without a
    commutative monoid structure.
    On the other hand, the rewriting system for symmetric monoidal categories~\cite{bonchi2022stringa}
    is too restricting, and prevents matchings such as \(
        \iltikzfig{graphs/dpo/matchings/trace-rule}
    \) in \(
        \iltikzfig{graphs/dpo/matchings/trace-match}
    \).
    We need something in the middle.
    \item \textbf{Design choices:} what precisely is a traced boundary
        complement? In the end it turned out to be very similar to the symmetric
        monoidal case but everything was partial monogamous instead of vanilla
        monogamous. What happens when we make the matching non-mono (it is mono
        for symmetric monoidal)
\end{itemize}