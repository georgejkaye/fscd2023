% !TeX root = ../main-conf.tex
\section{Introduction}

String diagrams constitute a useful and elegant conceptual bridge between term
rewriting and graph rewriting.
We will not reprise here, for lack of space, the by now impressive body of
theoretical and applied work for and with string diagrams but will take it as a
given. The survey~\cite{selinger2011survey} is a suitable starting point into
the literature.

The purpose of this paper is to support reasoning (via graph rewrite) in
\emph{traced categories with a comonoid structure} but without a monoid
structure. Prior art on this topic
exists~\cite{kissinger2012pictures,dixon2013opengraphs}, but it is based on the
framework of `framed point graphs' which requires rewriting modulo so called
\emph{wire homeomorphisms}.
This style of rewriting is awkward and is increasingly considered as obsolete as
compared to more recent work on rewriting with
hypergraphs~\cite{bonchi2022string,bonchi2022stringa,bonchi2022stringb},
possibly modulo \emph{Frobenius} structure.
The variation to just a (co)monoid structure (`half a Frobenius') has been
studied as well~ \cite{fritz2022free,milosavljevic2022string}.

The study of rewriting for traced categories with a comonoid structure is
motivated by an important application,
\emph{dataflow categories}~\cite{cazanescu1990new,cazanescu1994feedback,hasegawa1997recursion},
which represent a categorical foundation for the semantics of digital
circuits~\cite{ghica2022compositional}.
It is also technically challenging, as it falls in a gap between compact closed
structures constructible via Frobenius and symmetric monoidal categories
(without trace) so `off the shelf' solutions cannot be currently used.
In fact the gap between the kind of semantic models which use an underlying
compact closed structure and those which use a traced monoidal structure is
significant: the former have a \emph{relational} nature with subtle causality
(e.g. quantum or electrical circuits) whereas the latter are \emph{functional}
with clear input-output causality (e.g. digital or logical circuits) so it is
not surprising that the underlying rewrite frameworks should differ.

A key feature of compact closed categories is that the Cartesian product, if it
exists, is degenerate and identified with the co-product.
Even without invoking copying, we will see how trying to perform rewriting in a
traced category with a comonoid structure can also lead to inconsistencies.
This is a firm indication that a bespoke rewriting framework needs to be
constructed to fill this particular situation.

\emph{Contributions.}
This paper makes two distinct technical contributions.
The first is to show that one subclass of cospans of hypergraphs (`partial
monogamous') are fully complete for traced terms, and another class
(`partial left-monogamous') fully complete for traced comonoid terms.
The challenge is not so much in proving the correctness of the construction but
in defining precisely what these combinatorial structures should be.
In particular, the extremal point of tracing the identity: \(
    \trace{}{
        \iltikzfig{strings/category/identity}[white]
    }
    =
    \iltikzfig{strings/traced/trace-id}
\), corresponding graphically to a closed loop, provides a litmus test.
The way this is resolved must be robust enough to handle the addition of the
comonoid structure, which graphically corresponds to `tracing a forking wire' \(
    \trace{}{
        \iltikzfig{strings/structure/comonoid/copy}[white]
    }
    =
    \iltikzfig{strings/structure/comonoid/trace-fork}
\).

When rewriting with Frobenius using double pushout (DPO) rewriting, every
pushout complement is valid~\cite{bonchi2022string} whereas when rewriting with
symmetric monoidal structure exactly one pushout complement is
valid~\cite{bonchi2022stringa}.
For the traced case some pushout complements are valid and some are not.
The second contribution is to characterise the valid pushout complements as
`traced boundary complements'.

This is best illustrated with an example in which there is a pushout
complement that is valid in a Frobenius setting because it uses the monoid
structure, but it is not valid neither in a traced, nor even in a traced
comonoid setting.
Imagine we have a rule \(\rrule{
    \iltikzfig{graphs/dpo/non-valid/rule-lhs}
}{
    \iltikzfig{graphs/dpo/non-valid/rule-rhs}
}\) and a term \(
    \iltikzfig{graphs/dpo/non-valid/term}
\), and rewrite it as follows.
\begin{center}
    \includestandalone{figures/graphs/dpo/non-valid/rewrite}
\end{center}
This corresponds to the term rewrite \(
    \iltikzfig{graphs/dpo/non-valid/term}
    =
    \iltikzfig{graphs/dpo/non-valid/term-rewriting}
    =
    \iltikzfig{graphs/dpo/non-valid/term-rewritten}
\), which holds in a Frobenius setting but not a setting without a commutative
monoid structure.
On the other hand, the rewriting system for symmetric monoidal
categories~\cite{bonchi2022stringa} is too restrictive as it enforces that any
matching must be mono: this prevents matchings such as \(
    \iltikzfig{graphs/dpo/matchings/trace-rule}
\) in \(
    \iltikzfig{graphs/dpo/matchings/trace-match}
\).
Here again the challenge is precisely identifying the concept of traced boundary
complement mathematically.
The solution, although not immediately obvious, is not complicated, again
requiring a generalisation from monogamy to partial monogamy.
