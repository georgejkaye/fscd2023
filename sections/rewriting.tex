% !TeX root = ../main-conf.tex
\section{Graph rewriting}

While we can reason about terms in a monoidal category with \emph{equations},
we can reason using hypergraphs with \emph{graph rewrites}.

\begin{definition}[Rewrite rule]
    Given interfaced hypergraphs \(
        \cospan{i}[a_1]{L}[a_2]{j}
    \) and \(
        \cospan{i}[b_1]{R}[b_2]{j}
    \), their rewrite rule is a span in \(\hypsigma\): \(
        \spann{L}[[a_1,a_2]]{i+j}[[b_1,b_2]]{R}
    \).
\end{definition}

\begin{definition}[DPO]\label{def:dpo}
    \begin{gather}
        \label{gath:dpo}
        \includestandalone{figures/graphs/dpo/dpo}
    \end{gather}
\end{definition}

\begin{definition}[\cite{bonchi2022string}, Definition 3.16]
    Let \(K \xrightarrow{i} L \xrightarrow{m} G\) be morphisms in \(\hypsigma\).
    They satisfy the \emph{no-dangling} condition if, for every hyperedge \(e\)
    not in the image of \(m\), every node of its source and target is either in
    not in the image of \(m\) or in the image of \(i \seq m\).

    They satisfy the \emph{no-identification} condition if, for any two distinct
    vertices \(v_1\) and \(v_2\) such that \(m(v_1) = m(v_2)\), \(v_1\) and
    \(v_2\) are in the image of \(i\).
\end{definition}

\begin{proposition}[\cite{bonchi2022string}, Proposition 3.17]
    \label{prop:pushout-complement}
    A pushout complement of \(K \xrightarrow{i} L \xrightarrow{m} G\) exists
    if and only if it satisfies the no-dangling and no-identification
    conditions.
\end{proposition}

\subsection{Rewriting with traced structure}

While in the Frobenius case, all valid pushout complements correspond to a valid
rewrite, this is not the case for the traced monoidal case: for example, a wire
cannot connect the outputs of two boxes together.
In \cite{bonchi2021string}, pushout complements that correspond to a valid
rewrite are identified as \emph{boundary complements}.
Crucially, boundary complements rely on the matching \(L \to G\) being mono.
Restricting to these matchings actually cuts off potential rewrites in the
\emph{traced} setting, such as the occurrence of a rewrite rule inside a loop:
\begin{center}
    \iltikzfig{graphs/dpo/matchings/non-mono-matching}
\end{center}
This means that we need a slightly different definition that allows such
rewrites.

\begin{definition}[Traced candidate]
    Given morphism \(\morph{a}{i + j}{L}\), a morphism \(\morph{f}{L}{G}\) is
    called a \emph{traced candidate} if is injective on vertices and edges not
    in the image of \(a\).
\end{definition}

\begin{definition}[Traced boundary complement]
    \label{def:traced-boundary-complement}
    For partial monogamous cospans \(
        \cospan{i}[a_1]{L}[a_2]{j}
    \) and \(
        \cospan{n}[b_1]{G}[b_2]{m} \in \mathcal{C}
    \) and traced candidate \(
        \morph{f}{L}{G} \in \hypsigma
    \), a pushout complement as in
    \begin{center}
        \includestandalone{figures/graphs/dpo/boundary-complement}
    \end{center}
    is called a \emph{traced boundary complement} if \(c_1\) and \(c_2\) are
    mono and \(
        \cospan{j+m}[[c_2,d_1]]{\pushoutcomplement{L}}[[d_2,c_1]]{{n+i}}
    \) is a partial monogamous cospan.
\end{definition}

% \begin{lemma}
%     In a traced boundary complement, if \(
%         a_1(v) = a_2(w)
%     \), then \(
%         c_1(v) = c_2(w)
%     \) if and only if \(c_1(v)\) and \(c_2(w)\) are not in the image of \(d\)
%     and \(f(a_1(v)) = f(a_2(w))\) has degree \((0,0)\).
% \end{lemma}
% \iftoggle{proofs}{
%     \begin{proof}
%         For the \(\onlyifdir\) direction, assume that \(c_1(v) = c_2(w)\): let this
%         vertex be \(v^\prime\).
%         Since (\ref{gat:traced-complement}) is partial monogamous, \([c_2,d_1]\) and
%         \([c_1,d_2]\) must be mono, and thus the images of \(c_1\) and \(d_2\) must
%         be disjoint, as must the images of \(c_2\) and \(d_1\).
%         Therefore \(v^\prime\) cannot be in the image of \(d\); moreover, \(c_1(v)\)
%         must have out-degree \(0\) and \(c_2(w)\) must have in-degree \(0\), so
%         \(v^\prime\) has degree \((0,0)\).
%         Let \(v^{\prime\prime}\) be \(a_1(v) = a_2(w)\): since \(
%             \cospan{i}[a_1]{\iltikzfig{strings/rewriting/l}}[a_2]{j}
%         \) is a partial monogamous cospan \(v^\prime\) must have degree \((0,0)\).
%         Since \(G\) is computed by pushout, \(
%             f(v^{\prime\prime}) = g(v^{\prime})
%         \) and the degree of this vertex is contributed wholly by \(v^\prime\) and
%         \(v^{\prime\prime}\).
%         So \(f(a_1(v)) = f(a_2(w))\) also has degree \((0,0)\).

%         Now for the \(\ifdir\) direction, assume that \(c_1(v)\) and \(c_2(w)\) are
%         not in the image of \(d\), and \(f(a_1(v)) = f(a_2(w))\) has degree
%         \((0,0)\).
%         Assume that \(c_1(v) \neq c_2(w)\).
%         By semi-monogamy of (\ref{gat:traced-complement}), \(c_1(v)\) must have
%         degree \((0,1)\) since it is solely in the image of \([d_2, c_1]\).
%         But \(f(a_1(v)) = f(a_2(w)) = g(c_1(v)) = g(c_2(w))\), \(c_1(v)\) and
%         \(c_2(w)\) must have degree \((0,0)\), a contradiction.
%         The same holds for \(c_2(w)\), so \(c_1(v) = c_2(w)\).
%     \end{proof}
% }{}

\begin{lemma}
    Given partial monogamous cospan \(\cospan{m}[b_1]{G}[b_2]{n}\) and morphisms
    \(
        i + j \xrightarrow{[a_1, a_2]} L \xrightarrow{f} G
    \) satisfying the no-dangling and no-identification conditions where \(f\)
    is a traced candidate, there exists at least one traced boundary complement
    \(L^\bot\) as in \cref{def:traced-boundary-complement}.
\end{lemma}
\iftoggle{proofs}{
    \begin{proof}
        Since the morphisms satisfy the no-dangling and no-identification
        conditions, pushout complements exist: we just need to show that at
        least one of these is a traced boundary complement.

        The pushout complement is constructed by deleting elements of \(L\)
        from \(G\) while retaining the interfaces, i.e.\ the vertices in the
        image of \(a\).
        Since the morphism \(f\) is a traced candidate, it may not be injective
        on these interface vertices, and as such these vertices may also be
        merged or unmerged by \(c\).
        We just need to show that, depending on whether vertices are merged
        by \(a\) or \(f\), there is a pushout complement \(L^\bot\) such that
        \(c(v_1)\) and \(c(v_2)\) have degree suitable for the traced boundary
        complement condition.

        First assume vertices \(a_1(v_1), a_2(v_2)\) are not merged by \(a\) but
        are merged by \(f\).
        This means that in the pushout complement \(c_1(v_1) = c_2(v_2)\), as
        otherwise this would not be a pushout.
        Since \(\cospan{i}[a_1]{L}[a_2]{j}\) is partial monogamous, this means
        that \(a_1(v_1)\) has degree \((0,1)\) and \(a_2(v_2)\) has degree
        \((1,0)\); subsequently \(f(a_1(v_1)) = f(a_2(v_2))\) has degree
        \((1,1)\), both of which are contributed by the same edge \(e\).
        By the no-dangling condition the edge must be in the image of \(f\), so
        it is not present in \(L^\bot\), so \(c(v_1)\) and \(c(v_2)\) must have
        degree \((0,0)\).
        By partial monogamicity, \(f(a_1(v_1)) = f(a_2(v_2))\) is not in the
        image of \(b\) so \(c(v_1)\) and \(c(v_2)\) cannot be in the image of
        \(d\).
        This means that \(
            \cospan{m+j}{L^\bot}{n+i}
        \) is a partial monogamous cospan.

        Conversely, assume vertices \(a_1(v_1) = a_2(v_2)\) are merged by \(a\)
        and not by \(f\).
        This means the pushout does not enforce that \(c_1(v_1) = c_2(v_2)\).
        Pick a pushout complement where \(c_1(v_1) \neq c_2(v_2)\): by
        considering each case it can be seen that this is a valid traced
        boundary complement since the `gap' in between the vertices allows for
        the \(i\) and the \(j\) components of the partial monogamous cospan
        to contribute while keeping the leg mono.

        If a vertex \(v \in G\) is in the image of more than two vertices from
        \(i + j\), this means that there has been merging by both \(a\) and
        \(f\).
        However, note that since \(a_1\) and \(a_2\) are mono, \(a\) can only
        merge vertices from \(i\) and \(j\) together, not from both \(i\) and
        \(j\).
        Following the cases above, valid pushout complements have the property
        that any vertices not merged by \(a\) must be merged by \(c\), and
        otherwise they should be split apart.
    \end{proof}
}{
    \begin{proof}[Proof (Sketch)]
        There is always at least one pushout complement by
        \cref{prop:pushout-complement} as the morphisms satisfy the no-dangling
        and no-identification conditions.
        The traced boundary conditions can be found by ensuring any
        vertices merged by \(a\) are not merged by \(c\), and any vertices
        not merged by \(a\) are merged by \(c\).
    \end{proof}
}


\begin{definition}[Traced DPO]
    For morphisms \(G \leftarrow m+n\) and \(H \leftarrow m+n\) in
    \(\hypsigma\), there is a traced rewrite \(G \trgrewrite[\mcr] H\) if there
    exists a rule \(
        \spann{L}{i+j}{G} \in \mcr
    \) and cospan \(
        \cospan{i+j}{C}{n+m} \in \hypsigma
    \) such that diagram in \cref{def:dpo} commutes and \(i+j \to C\) is a
    traced boundary complement.
\end{definition}



A crucial part of~\cite{bonchi2022string} is that (non-traced) boundary
complements are \emph{unique}.
One might assume the same for traced boundary complements, but this is not the
case.

\begin{example}
    Consider the rule and its interpretation.
    \begin{gather}
        \rrule{
            \iltikzfig{graphs/dpo/non-unique/rule-lhs}
        }{
            \iltikzfig{graphs/dpo/non-unique/rule-rhs}
        }
        \qquad
        \raisebox{-2.1em}{\includestandalone{figures/graphs/dpo/non-unique/rule}}
        \label{gath:non-unique-rule}
    \end{gather}

    Now consider the following hypergraph with interface \(
        \iltikzfig{graphs/dpo/non-unique/g-unlabelled}
        \leftarrow
        \iltikzfig{graphs/dpo/non-unique/j-unlabelled}
    \).
    There are \emph{two} valid traced boundary complements for the above rule in
    this graph!

    \begin{center}
        \scalebox{0.95}{
            \includestandalone{figures/graphs/dpo/non-unique/rewrite-1}
            \includestandalone{figures/graphs/dpo/non-unique/rewrite-2}
        }
    \end{center}

    Both derivations are valid and arise since we are rewriting modulo
    \emph{yanking}:
    \begin{gather*}
        \iltikzfig{graphs/dpo/non-unique/derivation-1}
        =
        \iltikzfig{graphs/dpo/non-unique/derivation-2}
        =
        \iltikzfig{graphs/dpo/non-unique/derivation-3a}
        =
        \iltikzfig{graphs/dpo/non-unique/derivation-4a}
        =
        \iltikzfig{graphs/dpo/non-unique/derivation-5a}
        \\
        \iltikzfig{graphs/dpo/non-unique/derivation-1}
        =
        \iltikzfig{graphs/dpo/non-unique/derivation-2}
        =
        \iltikzfig{graphs/dpo/non-unique/derivation-3b}
        =
        \iltikzfig{graphs/dpo/non-unique/derivation-4b}
        =
        \iltikzfig{graphs/dpo/non-unique/derivation-5b}
    \end{gather*}
\end{example}

Rewriting modulo yanking also eliminates another foible of rewriting modulo
(non-traced) symmetric monoidal structure.
In the latter, matchings must be \emph{convex}: there cannot be a path from the
outputs of a match back to its inputs.
However, in the traced case this is redundant, as illustrated in the following
example.

\begin{example}
    Consider the following rewrite rule and its interpretation.
    %
    \begin{gather}
        \rrule{
            \iltikzfig{graphs/dpo/convex/example-l}
        }{
            \iltikzfig{graphs/dpo/convex/example-r}
        }
        \qquad
        \iltikzfig{graphs/dpo/convex/example-rule-graph}
        \label{gath:convex-rule}
    \end{gather}
    %
    Now consider the following term and interpretation:
    %
    \begin{gather}
        \iltikzfig{graphs/dpo/convex/example-g}
        \qquad
        \iltikzfig{graphs/dpo/convex/example-g-graph}
        \label{gath:convex-term}
    \end{gather}
    Although it is not obvious in the original string diagram, there is in fact
    a matching of (\ref{gath:convex-rule}) in (\ref{gath:convex-term}).
    Performing the DPO procedure yields the following:
    %
    \begin{gather}
        \iltikzfig{graphs/dpo/convex/example-h-graph}
        \qquad
        \iltikzfig{graphs/dpo/convex/example-h}
    \end{gather}
    In a non-traced setting this is an invalid rule!
    However, it is possible with yanking.
    \begin{gather*}
        \iltikzfig{graphs/dpo/convex/example-g}
        =
        \iltikzfig{graphs/dpo/convex/rewrite-1}
        =
        \iltikzfig{graphs/dpo/convex/rewrite-2}
        =
        \iltikzfig{graphs/dpo/convex/rewrite-3}
        \\[1em]
        =
        \iltikzfig{graphs/dpo/convex/rewrite-4}
        =
        \iltikzfig{graphs/dpo/convex/rewrite-5}
        =
        \iltikzfig{graphs/dpo/convex/example-h}
    \end{gather*}
\end{example}

We are almost ready to show the soundness and completeness of this DPO rewriting
system.
The final prerequisite is a decomposition lemma, akin to a similar result
in~\cite{bonchi2022string} for the symmetric monoidal case.

\begin{lemma}[Traced decomposition]\label{lem:decomposition}
    Given partial monogamous cospans \(\cospan{m}[p_1]{G}[p_2]{n}\) and \(
        \cospan{i}[a_1]{L}[a_2]{j}
    \), along with a morphism \(
        L \xrightarrow{f} G
    \) such that \(i+j \rightarrow L \rightarrow G\) satisfies the no-dangling
    and no-identification condition, then \(
        \cospan{m}[p_1]{G}[p_2]{n}
    \) can be factored as
    \begin{gather}
        \trace{i}{
            \begin{array}{cc}
                \cospan{i}[a_1]{L}[a_2]{j} \\
                \tensor \\
                \cospan{m}{m}{m}
            \end{array}
            \seq
            \cospan{j+m}{C}{i+n}
        }
        \label{gath:decomposition}
    \end{gather}
    where all cospans are partial monogamous and \(C\) is a traced boundary
    complement.
\end{lemma}
\iftoggle{proofs}{
    \begin{proof}
        Let \(C\) be defined as a traced boundary complement of \(
            i+j \xrightarrow{[a_1,a_2]} L \xrightarrow{f} G
        \), which exists as the no-dangling and no-identification condition is
        satisfied.
        We assign names to the various cospans in play, and reason string
        diagrammatically:
        \begin{align*}
            \iltikzfig{graphs/dpo/lhat} &:= \cospan{i}{L}{j}
            &
            \iltikzfig{graphs/dpo/ltilde} &:= \cospan{0}{L}{i+j} \\
            \iltikzfig{graphs/dpo/chat} &:= \cospan{j+m}{L}{i+n}
            &
            \iltikzfig{graphs/dpo/ctilde} &:= \cospan{i+j}{C}{m+n} \\
            \iltikzfig{graphs/dpo/ghat} &:= \cospan{m}{G}{n}
            &
            \iltikzfig{graphs/dpo/gtilde} &:= \cospan{0}{G}{m+n}
        \end{align*}
        By using the compact closed structure of \(\cspfihyp\), we also have that \(
            \iltikzfig{graphs/dpo/ltilde} = \iltikzfig{graphs/dpo/lhat-bent}
        \), \(
            \iltikzfig{graphs/dpo/ctilde} = \iltikzfig{graphs/dpo/chat-bent}
        \) and \(
            \iltikzfig{graphs/dpo/gtilde} = \iltikzfig{graphs/dpo/ghat-bent}
        \).
        Since \(
            \iltikzfig{graphs/dpo/gtilde} = \iltikzfig{graphs/dpo/lctilde}
        \), it follows that \(
            \iltikzfig{graphs/dpo/ghat-bent} = \iltikzfig{graphs/dpo/lchat-bent}
        \) and subsequently \(
            \iltikzfig{graphs/dpo/ghat} = \iltikzfig{graphs/dpo/lchat}
        \).
        The `loop' is constructed in the same manner as the canonical trace on
        \(\cspfihyp\), so this is a term in the form of (\ref{gath:decomposition}).
        Moreover, all cospans involved are partial monogamous by definition of
        rewrite rules and traced boundary complements.
    \end{proof}
}{}

\begin{theorem}
    Let \(\mcr\) be a rewriting system on \(\stmcsigma\).
    Then, \(
        \iltikzfig{strings/category/f}[g][white]
        \rewrite[\mcr]
        \iltikzfig{strings/category/f}[H][white]
    \) if and only if \(
        \termandfrobtohypsigma[\foldinterfaces[\iltikzfig{strings/category/f}[g][white]]]
        \grewrite[\termandfrobtohypsigma[\foldinterfaces[\mcr]]]
        \termandfrobtohypsigma[\foldinterfaces[\iltikzfig{strings/category/f}[H][white]]].
    \)
\end{theorem}
% \iftoggle{proofs}{
    \begin{proof}
        First the \((\Rightarrow)\) direction.
        If \(
            \iltikzfig{strings/category/f}[g][white]
            \rewrite[\mcr]
            \iltikzfig{strings/category/f}[H][white]
        \) then we have \(
            \iltikzfig{strings/category/f}[g][white]
            =
            \iltikzfig{strings/rewriting/rewrite-l}
        \) and \(
            \iltikzfig{strings/rewriting/rewrite-r}
            =
            \iltikzfig{strings/category/f}[H][white].
        \)
        Define the following cospans:
        \begin{align}
            \label{gath:l-cospan}
            \cospan{0}{L}{i+j}
            &:=
            \termandfrobtohypsigma[\foldinterfaces[\iltikzfig{strings/category/f}[l][white]]]
            &&=
            \termandfrobtohypsigma[\iltikzfig{strings/rewriting/l-folded}]
            \\
            \cospan{0}{R}{i+j}
            &:=
            \termandfrobtohypsigma[\foldinterfaces[\iltikzfig{strings/category/f}[r][white]]]
            &&=
            \termandfrobtohypsigma[\iltikzfig{strings/rewriting/r-folded}]
            \\
            \cospan{0}{G}{m+n}
            &:=
            \termandfrobtohypsigma[\foldinterfaces[\iltikzfig{strings/category/f}[f][white]]]
            &&=
            \termandfrobtohypsigma[\iltikzfig{strings/rewriting/lc-folded}]
            \\
            \label{gath:h-cospan}
            \cospan{0}{H}{m+n}
            &:=
            \termandfrobtohypsigma[\foldinterfaces[\iltikzfig{strings/category/f}[H][white]]]
            &&=
            \termandfrobtohypsigma[\iltikzfig{strings/rewriting/rc-folded}]
            \\
            \cospan{i+j}{C}{m+n}
            &:=
            \termandfrobtohypsigma[\iltikzfig{strings/rewriting/c-folded}]
            &&
        \end{align}
        By functoriality, since \(
            \foldinterfaces[\iltikzfig{strings/category/f}[f][white]]
            =
            \iltikzfig{strings/rewriting/l-folded}
            \seq
            \iltikzfig{strings/rewriting/c-folded}
        \) then \[
            \cospan{0}{G}{m+n} = \cospan{0}{L}{i+j} \seq \cospan{i+j}{C}{m+n}.
        \]
        Cospan composition is pushout, so \(\cospan{L}{G}{C}\) is a pushout.
        Using the same reasoning, \(\cospan{R}{G}{C}\) is also a pushout: this gives
        us the DPO diagram.
        All that remains is to check that the aforementioned pushouts are traced
        boundary complements: this follows by inspecting components.

        Now the \(\ifdir\) direction: assume we have a DPO diagram (\ref{gath:dpo})
        where \(L \leftarrow i + j\), \(i + j \rightarrow R\), \(m + n \to G\) and
        \(m + n \to H\) are defined as in (\ref{gath:l-cospan}-\ref{gath:h-cospan})
        above.
        We must show that \(
            \iltikzfig{strings/category/f}[f][white]
            =
            \iltikzfig{strings/rewriting/rewrite-l}
        \) and \(
            \iltikzfig{strings/category/f}[H][white]
            =
            \iltikzfig{strings/rewriting/rewrite-r}
        \).
        By definition of traced boundary complement \(\cospan{j+m}{C}{i+n}\) is a
        partial monogamous cospan, so by fullness of \(\termandfrobtohypsigma\),
        there exists a term \(
            \iltikzfig{strings/category/f-2-2}[c][white][j][m][i][n]
            \in \stmcsigma
        \) such that \(
            \iltikzfig{strings/category/f-2-2}[c][white]
            =
            \cospan{j+m}{C}{i+n}
        \).
        By traced decomposition (\cref{lem:decomposition}), we have that for any
        traced boundary complement \(\cospan{i+j}{C}{m+n}\) and morphism
        \(L \to G\), \(\cospan{m}{G}{n}\) can be factored as in
        (\ref{gath:decomposition}), i.e.\ \[
            \termandfrobtohypsigma[\iltikzfig{strings/category/f}[f][white]]
            =
            \trace{j}{\termandfrobtohypsigma[\iltikzfig{strings/category/f}[l][white]]
            \tensor
            \id[n]
            \seq
            \termandfrobtohypsigma[\iltikzfig{strings/category/f-2-2}[c][white]]}.
        \]
        So by functoriality, we have that \(
            \iltikzfig{strings/category/f}[f][white]
            =
            \iltikzfig{strings/rewriting/rewrite-l}
        \).
        The same reasoning follows for \(
            \iltikzfig{strings/category/f}[H][white]
            =
            \iltikzfig{strings/rewriting/rewrite-r}
        \).
    \end{proof}
% }{}

\subsection{Rewriting with traced comonoid structure}

To extend rewriting with traced structure to the comonoid case, the traced
boundary complement conditions need to be weakened to the case of
\emph{left}-monogamous cospans.

\begin{definition}[Traced left-boundary complement]
    \label{def:traced-left-boundary-complement}
    For partial left-monogamous cospans \(
        \cospan{i}[a_1]{L}[a_2]{j}
    \) and \(
        \cospan{n}[b_1]{G}[b_2]{m} \in \mathcal{C}
    \) and (not necessarily mono) \(
        \morph{f}{L}{G} \in \hypsigma
    \), a pushout complement as in \cref{gath:boundary-complement}
    is called a \emph{traced left-boundary complement} if \(c_1\) and \(c_2\)
    are mono, \(f\) is injective on edges and vertices not in the image of
    \(a\), and there exist morphisms \(
        \morph{d_1}{m}{\pushoutcomplement{L}}
    \) and \(
        \morph{d_2}{n}{\pushoutcomplement{L}}
    \) making the above triangle commute such that \(
        \cospan{j+m}[[c_2,d_1]]{\pushoutcomplement{L}}[[d_2,c_1]]{{n+i}}
    \) is a partial left-monogamous cospan.
\end{definition}