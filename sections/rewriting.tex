% !TeX root = ../main-conf.tex
\section{Graph rewriting}

We have now shown that we can reason up to the axioms of symmetric traced
categories with a comonoid structure using hypergraphs: string diagrams equal by
topological deformations are translated into isomorphic hypergraphs.
However, to reason about an \emph{monoidal theory} with extra equations we must
actually rewrite the components of the graph.
In the syntactic realm this is performed with \emph{term rewriting}.

\begin{definition}[Term rewriting]\label{def:term-rewriting}
    A \emph{rewriting system} \(\mcr\) for a traced PROP \(\stmc{\Sigma}\)
    consists of a set of \emph{rewrite rules} \(
        \rrule{
            \iltikzfig{strings/category/f}[l][white][i][j]
        }{
            \iltikzfig{strings/category/f}[r][white][i][j]
        }
    \).
    Given terms \(
        \iltikzfig{strings/category/f}[g][white][m][n]
    \) and \(
        \iltikzfig{strings/category/f}[h][white][m][n]
    \) in \(\stmc{\generators}\) we write \(
        \iltikzfig{strings/category/f}[g][white]
        \rewrite[\mcr]
        \iltikzfig{strings/category/f}[h][white]
    \) if there exists rewrite rule \((
        \iltikzfig{strings/category/f}[l][white][i][j],
        \iltikzfig{strings/category/f}[r][white][i][j]
    )\) in \(\mcr\) and \(
        \iltikzfig{strings/category/f-2-2}[c][white][j][m][i][n]
    \) in \(\stmc{\Sigma}\) such that \(
        \iltikzfig{strings/category/f}[g][white]
        =
        \iltikzfig{strings/rewriting/rewrite-l}
    \) and \(
        \iltikzfig{strings/category/f}[h][white]
        =
        \iltikzfig{strings/rewriting/rewrite-r}
    \) by axioms of STMCs.
\end{definition}

The graph equivalent is \emph{graph rewriting}.
A common framework is that of \emph{double pushout rewriting} (DPO rewriting).
We use an extension, known as \emph{double pushout rewriting with interfaces}
(DPOI rewriting).

\begin{definition}[DPO rule]
    Given interfaced hypergraphs \(
        \cospan{i}[a_1]{L}[a_2]{j}
    \) and \(
        \cospan{i}[b_1]{R}[b_2]{j}
    \), their \emph{DPO rule} in \(\hypsigma\) is a span \(
        \spann{L}[[a_1,a_2]]{i+j}[[b_1,b_2]]{R}
    \).
\end{definition}

\begin{definition}[DPO(I) rewriting]\label{def:dpo}
    Let \(\mcr\) be a set of DPO rules.
    Then, for morphisms \(G \leftarrow m+n\) and \(H \leftarrow m+n\) in
    \(\hypsigma\), there is a rewrite \(G \trgrewrite[\mcr] H\) if there
    exists a rule \(
        \spann{L}{i+j}{R} \in \mcr
    \) and cospan \(
        \cospan{i+j}{C}{n+m} \in \hypsigma
    \) such that diagram in the left of \cref{fig:dpo} commutes.
\end{definition}

\begin{figure}
    \centering
    \raisebox{1em}{\includestandalone{figures/graphs/dpo/dpo}}
    \qquad
    \includestandalone{figures/graphs/dpo/pushout-complement}
    \caption{The DPO diagram and a pushout complement.}
    \label{fig:dpo}
\end{figure}

The first thing to note is that the graphs in the DPO diagram have a
\emph{single} interface \(G \leftarrow m + n\) instead of the cospans \(
    \cospan{m}{G}{n}
\) we are used to.
Before performing DPO rewriting in \(\hypsigma\), the interfaces must be
`folded' into one.

\begin{definition}[\cite{bonchi2022stringa}]
    Let \(\morph{\foldinterfaces}{\smc{\Sigma} + \frob}{\smc{\Sigma} + \frob}\)
    be defined as having action \(
        \iltikzfig{strings/category/f}[f][white][m][n]
        \mapsto
        \iltikzfig{strings/rewriting/folding}[f][white][m][n]
    \).
\end{definition}

Note that the result of applying \(\foldinterfaces\) is not in the image of \(
    \termandfrobtohypsigma \circ \tracedtosymandfrob{\Sigma}
\) any more.
This is not an issue, so long as we `unfold' the interfaces once rewriting is
completed.

\begin{proposition}[\cite{bonchi2022string}, Prop. 4.8]
    If \(
        \termandfrobtohypsigma[\iltikzfig{strings/category/f}[f][white][m][n]]
        =
        \cospan{m}[i]{F}[o]{n}
    \) then \(
        \foldinterfaces[
            \termandfrobtohypsigma[\iltikzfig{strings/category/f}[f][white]]
        ]
    \) is isomorphic to \(
        \cospan{0}[]{F}[i+o]{m+n}
    \).
\end{proposition}

In order to apply a given DPO rule \(\spann{L}{i+j}{R}\) in some larger
graph \(\cospan{m}{G}{n}\), a morphism \(L \to G\) must first be identified.
The next step is to `cut out' the components of \(L\) that exist in \(G\).

\begin{definition}[Pushout complement]\label{def:pushout-complement}
    Let \(i + j \to L \to G \rightarrow m + n\) be morphisms in \(\hypsigma\).
    Then the \emph{pushout complement} of these morphisms is an object \(C\)
    with morphisms \(i + j \to C \to G\) such that \(\cospan{L}{G}{C}\) is a
    pushout and the diagram on the right of \cref{fig:dpo} commutes.
\end{definition}

Once a pushout complement \(C\) is computed, the pushout of
\(\spann{C}{i+j}{R}\) can be performed to obtain the completed rewrite \(H\).
However, a pushout complement may not exist for a given rule and matching.

\begin{definition}[\cite{bonchi2022string}, Def. 3.16]
    Let \(i + j \xrightarrow{a} L \xrightarrow{f} G\) be morphisms in
    \(\hypsigma\).
    The morphisms satisfy the \emph{no-dangling} condition if, for every
    hyperedge not in the image of \(f\), each of its source and target vertices
    is either not in the image of \(f\) or are in the image of \(f \circ a\).
    The morphisms satisfy the \emph{no-identification} condition if any two
    distinct elements merged by \(f\) are also in the image of \(f \circ a\).
\end{definition}

\begin{proposition}[\cite{bonchi2022string}, Prop. 3.17]
    \label{prop:pushout-complement}
    The morphisms \(i + j \to L \to G\) have at least one pushout complement if
    and only if they satisfy the no-dangling and no-identification conditions.
\end{proposition}

\begin{definition}
    Given a partial monogamous cospan \(\cospan{i}{L}{j}\), a morphism
    \(L \to G\) is called a \emph{matching} if it has at least one pushout
    complement.
\end{definition}

In certain settings, known as
\emph{adhesive categories}~\cite{lack2004adhesive}, it is possible to be more
precise about the number of pushout complements for a given matching and rewrite
rule.

\begin{proposition}[\cite{lack2004adhesive}]
    In an adhesive category, pushout complements of \(
        i + j \xrightarrow{a} L \to G\)
    are unique if they exist and \(a\) is mono.
\end{proposition}

\begin{proposition}[\cite{lack2005adhesive}]
    \(\hypsigma\) is adhesive.
\end{proposition}

Since a given pushout complement uniquely determines the rewrite performed, it
might seem advantageous to always have exactly one pushout complement.
However, as shall be detailed later, when writing modulo traced comonoid
structure there are settings where having multiple pushout complements is
beneficial.

\subsection{Rewriting with traced structure}

While in the Frobenius case considered in~\cite{bonchi2022string}, all valid
pushout complements correspond to a valid rewrite, this is not the case for the
traced monoidal case.
In \cite{bonchi2022stringa}, pushout complements that correspond to a valid
rewrite in the non-traced symmetric monoidal case are identified as
\emph{boundary complements}.
We will use a weakening of this definition.

\begin{definition}[Traced boundary complement]
    \label{def:traced-boundary-complement}
    For partial monogamous cospans \(
        \cospan{i}[a_1]{L}[a_2]{j}
    \) and \(
        \cospan{n}[b_1]{G}[b_2]{m} \in \hypsigma
    \), a pushout complement as in \cref{def:pushout-complement} is called a
    \emph{traced boundary complement} if \(c_1\) and \(c_2\) are mono and \(
        \cospan{j+m}[[c_2,d_1]]{C}[[d_2,c_1]]{{n+i}}
    \) is a partial monogamous cospan.
\end{definition}

Unlike~\cite{bonchi2022stringa}, we do not enforce that the matching is mono,
as restricting to these matchings actually cuts off potential rewrites in the
\emph{traced} setting, such as the occurrence of a rewrite rule inside a loop: \(
    \iltikzfig{graphs/dpo/matchings/trace-l}
    \to
    \iltikzfig{graphs/dpo/matchings/trace-g}
\).

\begin{definition}[Traced DPO]
    For morphisms \(G \leftarrow m+n\) and \(H \leftarrow m+n\) in
    \(\hypsigma\), there is a traced rewrite \(G \trgrewrite[\mcr] H\) if there
    exists a rule \(
        \spann{L}{i+j}{G} \in \mcr
    \) and cospan \(
        \cospan{i+j}{C}{n+m} \in \hypsigma
    \) such that diagram in \cref{def:dpo} commutes and \(i+j \to C\) is a
    traced boundary complement.
\end{definition}

Some intuition on the construction of traced boundary complements may be
required: this will be provided through a lemma and some examples.

\begin{lemma}
    In a traced boundary complement, let \(v \in i\) and
    \(w_0, w_1, \cdots w_k\) such that \(f(a_1(v)) = f(a_2(w_0))\),
    \(f(a_1(v)) = f(a_2(w_1))\) and so on.
    Then either (1) there exists exactly one \(w_l\) not in the image of \(d_1\)
    such that \(c_1(v) = c_2(w_l)\); (2) \(c_1(v)\) is in the image of \(d_1\);
    or (3) \(c_1(v)\) has degree \((1,0)\).
    The same also holds for \(w \in j\), with the interface map as \(d_2\) and
    the degree as \((0,1)\).
\end{lemma}
\iftoggle{proofs}{
    \begin{proof}
        Since \(c_1(v)\) is in the image of \(c\), it must have either degree
        \((0,0)\) or \((1,0)\) by partial monogamy.
        For it to have degree \((0, 0)\), it must either be in the image of one
        of \(c_2\) or \(d_2\).
        In the case of the former, this means that there must be a \(w_l\) such
        that \(c_1(v) = c_2(w_l)\), and only one such vertex as \(c_2\) is mono,
        so (1) is satisfied.
        In the case of the latter, (2) is immediately satisfied.
        Otherwise, (3) is satisfied.

        The proof for the flipped case is exactly the same.
    \end{proof}
}{}

Often there can be valid rewrites in the realm of graphs that are non-obvious
in the term language.
This is because we are rewriting modulo \emph{yanking}.

\begin{example}
    Consider the rule \(
        \rrule{
            \iltikzfig{graphs/dpo/split-loop/rule-lhs}
        }{
            \iltikzfig{graphs/dpo/split-loop/rule-rhs}
        }
    \).
    The interpretation of this as a DPO rule in a valid traced boundary
    complement is illustrated below.
    \begin{center}
        \includestandalone{figures/graphs/dpo/split-loop/rewrite}
    \end{center}
    This corresponds to a valid term rewrite:
    \[
        \iltikzfig{graphs/dpo/split-loop/derivation-1}
        =
        \iltikzfig{graphs/dpo/split-loop/derivation-2}
        =
        \iltikzfig{graphs/dpo/split-loop/derivation-3}
        =
        \iltikzfig{graphs/dpo/split-loop/derivation-4}
    \]
\end{example}

Unlike regular boundary complements, traced boundary complements need not be
unique.
This is not an issue, as all pushout complements can be enumerated for a given
rule and matching~\cite{heumuller2011construction}.

\begin{example}
    Consider the rule \(
        \rrule{
            \iltikzfig{graphs/dpo/non-unique/rule-lhs}
        }{
            \iltikzfig{graphs/dpo/non-unique/rule-rhs}
        }
    \).
    Below are two valid traced boundary complements involving a matching of this
    rule.

    \begin{center}
        \includestandalone{figures/graphs/dpo/non-unique/rewrite-1}
        \includestandalone{figures/graphs/dpo/non-unique/rewrite-2}
    \end{center}

    These derivations arise since we are rewriting modulo \emph{yanking}:
    \begin{gather*}
        \iltikzfig{graphs/dpo/non-unique/derivation-1}
        =
        \iltikzfig{graphs/dpo/non-unique/derivation-2}
        =
        \iltikzfig{graphs/dpo/non-unique/derivation-3a}
        =
        \iltikzfig{graphs/dpo/non-unique/derivation-4a}
        =
        \iltikzfig{graphs/dpo/non-unique/derivation-5a}
        \\
        \iltikzfig{graphs/dpo/non-unique/derivation-1}
        =
        \iltikzfig{graphs/dpo/non-unique/derivation-2}
        =
        \iltikzfig{graphs/dpo/non-unique/derivation-3b}
        =
        \iltikzfig{graphs/dpo/non-unique/derivation-4b}
        =
        \iltikzfig{graphs/dpo/non-unique/derivation-5b}
    \end{gather*}
\end{example}


Rewriting modulo yanking also eliminates another foible of rewriting modulo
(non-traced) symmetric monoidal structure.
In the SMC case, the image of the matching must be \emph{convex}: any
path between vertices must also be captured.
This is not necessary in the traced case.

\begin{example}
    Consider the following rewrite rule and its interpretation.
    %
    \begin{gather}
        \rrule{
            \iltikzfig{graphs/dpo/convex/example-l}
        }{
            \iltikzfig{graphs/dpo/convex/example-r}
        }
        \qquad
        \iltikzfig{graphs/dpo/convex/example-rule-graph}
        \label{gath:convex-rule}
    \end{gather}
    %
    Now consider the following term and interpretation:
    %
    \begin{gather}
        \iltikzfig{graphs/dpo/convex/example-g}
        \qquad
        \iltikzfig{graphs/dpo/convex/example-g-graph}
        \label{gath:convex-term}
    \end{gather}
    Although it is not obvious in the original string diagram, there is in fact
    a matching of (\ref{gath:convex-rule}) in (\ref{gath:convex-term}).
    Performing the DPO procedure yields the following:
    %
    \begin{gather}
        \iltikzfig{graphs/dpo/convex/example-h-graph}
        \qquad
        \iltikzfig{graphs/dpo/convex/example-h}
    \end{gather}
    In a non-traced setting this is an invalid rule!
    However, it is possible with yanking.
    \begin{gather*}
        \iltikzfig{graphs/dpo/convex/example-g}
        =
        \iltikzfig{graphs/dpo/convex/rewrite-2}
        =
        \iltikzfig{graphs/dpo/convex/rewrite-4}
        =
        \iltikzfig{graphs/dpo/convex/rewrite-5}
        =
        \iltikzfig{graphs/dpo/convex/example-h}
    \end{gather*}
\end{example}

We are almost ready to show the soundness and completeness of this DPO rewriting
system.
The final prerequisite is a decomposition lemma, akin to a similar result
in~\cite{bonchi2022string}.

\begin{lemma}[Traced decomposition]\label{lem:traced-decomposition}
    Given partial monogamous cospans \(\cospan{m}[p_1]{G}[p_2]{n}\) and \(
        \cospan{i}[a_1]{L}[a_2]{j}
    \), along with a morphism \(
        L \xrightarrow{f} G
    \) such that \(i+j \rightarrow L \rightarrow G\) satisfies the no-dangling
    and no-identification conditions, then there exists \(
        \cospan{j+m}[c_2,d_1]{C}[c_1,d_2]{i+n}
    \) such that \(
        \cospan{m}{G}{n}
    \) can be factored as
    \begin{gather}
        \trace{i}{
            \begin{array}{cc}
                \cospan{i}[a_1]{L}[a_2]{j} \\
                \tensor \\
                \cospan{m}{m}{m}
            \end{array}
            \seq
            \cospan{j+m}[c_2,d_1]{C}[c_1,d_2]{i+n}
        }
        \label{gath:decomposition}
    \end{gather}
    where all cospans are partial monogamous and \(
        \cospan{j+m}[c_2,d_1]{C}[c_1,d_2]{i+n}
    \) is a traced boundary complement.
\end{lemma}
\iftoggle{proofs}{
    \begin{proof}
        Let \(C\) be defined as a traced boundary complement of \(
            i+j \xrightarrow{[a_1,a_2]} L \xrightarrow{f} G
        \), which exists as the no-dangling and no-identification condition is
        satisfied.
        We assign names to the various cospans in play, and reason string
        diagrammatically:
        \begin{align*}
            \iltikzfig{graphs/dpo/lhat} &:= \cospan{i}{L}{j}
            &
            \iltikzfig{graphs/dpo/ltilde} &:= \cospan{0}{L}{i+j} \\
            \iltikzfig{graphs/dpo/chat} &:= \cospan{j+m}{L}{i+n}
            &
            \iltikzfig{graphs/dpo/ctilde} &:= \cospan{i+j}{C}{m+n} \\
            \iltikzfig{graphs/dpo/ghat} &:= \cospan{m}{G}{n}
            &
            \iltikzfig{graphs/dpo/gtilde} &:= \cospan{0}{G}{m+n}
        \end{align*}
        By using the compact closed structure of \(\cspfihyp\), we also have that \(
            \iltikzfig{graphs/dpo/ltilde} = \iltikzfig{graphs/dpo/lhat-bent}
        \), \(
            \iltikzfig{graphs/dpo/ctilde} = \iltikzfig{graphs/dpo/chat-bent}
        \) and \(
            \iltikzfig{graphs/dpo/gtilde} = \iltikzfig{graphs/dpo/ghat-bent}
        \).
        Since \(
            \iltikzfig{graphs/dpo/gtilde} = \iltikzfig{graphs/dpo/lctilde}
        \), it follows that \(
            \iltikzfig{graphs/dpo/ghat-bent} = \iltikzfig{graphs/dpo/lchat-bent}
        \) and subsequently \(
            \iltikzfig{graphs/dpo/ghat} = \iltikzfig{graphs/dpo/lchat}
        \).
        The `loop' is constructed in the same manner as the canonical trace on
        \(\cspfihyp\), so this is a term in the form of (\ref{gath:decomposition}).
        Moreover, all cospans involved are partial monogamous by definition of
        rewrite rules and traced boundary complements.
    \end{proof}
}{}

We write \(
    \foldinterfaces[\tracedtosymandfrob[\mcr]{\Sigma}]
\) for the pointwise map \(
    (
        \iltikzfig{strings/category/f}[l][white],
        \iltikzfig{strings/category/f}[r][white]
    )
    \mapsto
    (
        \foldinterfaces[
            \tracedtosymandfrob[
                \iltikzfig{strings/category/f}[l][white]
            ]{\Sigma}
        ],
        \foldinterfaces[
            \tracedtosymandfrob[
                \iltikzfig{strings/category/f}[r][white]
            ]{\Sigma}
        ]
    ).
\)

\begin{theorem}\label{thm:traced-rewriting}
    Let \(\mcr\) be a rewriting system on \(\stmcsigma\).
    Then, \(
        \iltikzfig{strings/category/f}[g][white][m][n]
        \rewrite[\mcr]
        \iltikzfig{strings/category/f}[h][white][m][n]
    \) if and only if \(
        \termandfrobtohypsigma[
            \foldinterfaces[
                \tracedtosymandfrob[
                    \iltikzfig{strings/category/f}[g][white]
                ]{\Sigma}
            ]
        ]
        \grewrite[
            \termandfrobtohypsigma[
                \foldinterfaces[
                    \tracedtosymandfrob[\mcr]{\Sigma}
                ]
            ]
        ]
        \termandfrobtohypsigma[
            \foldinterfaces[
                \tracedtosymandfrob[
                    \iltikzfig{strings/category/f}[h][white]
                ]{\Sigma}
            ]
        ].
    \)
\end{theorem}
\iftoggle{proofs}{
    \begin{proof}
        First the \((\Rightarrow)\) direction.
If \(
    \iltikzfig{strings/category/f}[g][white]
    \rewrite[\mcr]
    \iltikzfig{strings/category/f}[H][white]
\) then we have \(
    \iltikzfig{strings/category/f}[g][white]
    =
    \iltikzfig{strings/rewriting/rewrite-l}
\) and \(
    \iltikzfig{strings/rewriting/rewrite-r}
    =
    \iltikzfig{strings/category/f}[H][white].
\)
Define the following cospans:
\begin{alignat}{3}
    \label{gath:l-cospan}
    \cospan{0}{L}{i+j}
    &:=
    \termandfrobtohypsigma[\foldinterfaces[\iltikzfig{strings/category/f}[l][white]]]
    &&=
    \termandfrobtohypsigma[\iltikzfig{strings/rewriting/l-folded}]
    \\
    \cospan{0}{R}{i+j}
    &:=
    \termandfrobtohypsigma[\foldinterfaces[\iltikzfig{strings/category/f}[r][white]]]
    &&=
    \termandfrobtohypsigma[\iltikzfig{strings/rewriting/r-folded}]
    \\
    \cospan{0}{G}{m+n}
    &:=
    \termandfrobtohypsigma[\foldinterfaces[\iltikzfig{strings/category/f}[f][white]]]
    &&=
    \termandfrobtohypsigma[\iltikzfig{strings/rewriting/lc-folded}]
    \\
    \label{gath:h-cospan}
    \cospan{0}{H}{m+n}
    &:=
    \termandfrobtohypsigma[\foldinterfaces[\iltikzfig{strings/category/f}[H][white]]]
    &&=
    \termandfrobtohypsigma[\iltikzfig{strings/rewriting/rc-folded}]
    \\
    \cospan{i+j}{C}{m+n}
    &:=
    \termandfrobtohypsigma[\iltikzfig{strings/rewriting/c-folded}]
    &&
\end{alignat}
By functoriality, since \(
    \foldinterfaces[\iltikzfig{strings/category/f}[f][white]]
    =
    \iltikzfig{strings/rewriting/l-folded}
    \seq
    \iltikzfig{strings/rewriting/c-folded}
\) then \[
    \cospan{0}{G}{m+n} = \cospan{0}{L}{i+j} \seq \cospan{i+j}{C}{m+n}.
\]
Cospan composition is pushout, so \(\cospan{L}{G}{C}\) is a pushout.
Using the same reasoning, \(\cospan{R}{G}{C}\) is also a pushout: this
gives us the DPO diagram.
All that remains is to check that the aforementioned pushouts are traced
boundary complements: this follows by inspecting components.

Now the \(\ifdir\) direction: assume we have a DPO diagram (\ref{def:dpo})
where \(L \leftarrow i + j\), \(i + j \rightarrow R\), \(m + n \to G\) and
\(m + n \to H\) are defined as in (\ref{gath:l-cospan}-\ref{gath:h-cospan})
above.
We must show that \(
    \iltikzfig{strings/category/f}[f][white]
    =
    \iltikzfig{strings/rewriting/rewrite-l}
\) and \(
    \iltikzfig{strings/category/f}[H][white]
    =
    \iltikzfig{strings/rewriting/rewrite-r}
\).
By definition of traced boundary complement \(\cospan{j+m}{C}{i+n}\) is a
partial monogamous cospan, so by fullness of \(\termandfrobtohypsigma\),
there exists a term \(
    \iltikzfig{strings/category/f-2-2}[c][white][j][m][i][n]
    \in \smcsigma
\) such that \(
    \termtohyp[
        \iltikzfig{strings/category/f-2-2}[c][white]
    ]{\Sigma}
    =
    \cospan{j+m}{C}{i+n}
\).
By traced decomposition (\cref{lem:traced-decomposition}), we have that for any
traced boundary complement \(\cospan{i+j}{C}{m+n}\) and morphism
\(L \to G\), \(\cospan{m}{G}{n}\) can be factored as in
(\ref{gath:decomposition}), i.e.\ \[
    \termandfrobtohypsigma[\iltikzfig{strings/category/f}[f][white]]
    =
    \trace{j}{\termandfrobtohypsigma[\iltikzfig{strings/category/f}[l][white]]
    \tensor
    \id[n]
    \seq
    \termandfrobtohypsigma[\iltikzfig{strings/category/f-2-2}[c][white]]}.
\]
So by functoriality, we have that \(
    \iltikzfig{strings/category/f}[f][white]
    =
    \iltikzfig{strings/rewriting/rewrite-l}
\).
The same reasoning follows for \(
    \iltikzfig{strings/category/f}[H][white]
    =
    \iltikzfig{strings/rewriting/rewrite-r}
\).
    \end{proof}
}{
    \begin{proof}
        \(\onlyifdir\) follows by defining cospans corresponding each part
        of \cref{def:term-rewriting} and composing them together:
        since composition of cospans is by pushout, the DPO diagram can be
        recovered and the pushouts checked to be traced boundary complements.
        \(\ifdir\) follows by using \cref{lem:traced-decomposition} and the
        fullness of \(\termandfrobtohypsigma\) to obtain the pieces of
        \cref{def:term-rewriting}.
    \end{proof}
}

\subsection{Rewriting with traced comonoid structure}

To extend rewriting with traced structure to the comonoid case, the traced
boundary complement conditions need to be weakened to the case of
\emph{left}-monogamous cospans.

\begin{definition}[Traced left-boundary complement]
    \label{def:traced-left-boundary-complement}
    For partial left-monogamous cospans \(
        \cospan{i}[a_1]{L}[a_2]{j}
    \) and \(
        \cospan{n}[b_1]{G}[b_2]{m} \in \hypsigma
    \), a pushout complement as in \cref{def:traced-boundary-complement}
    is called a \emph{traced left-boundary complement} if \(c_2\)
    is mono and \(
        \cospan{j+m}[[c_2,d_1]]{C}[[c_1,d_2]]{{i+n}}
    \) is a partial left-monogamous cospan.
\end{definition}

\begin{definition}[Traced comonoid DPO]
    For morphisms \(G \leftarrow m+n\) and \(H \leftarrow m+n\) in
    \(\hypsigma\), there is a traced comonoid rewrite \(G \trgrewrite[\mcr] H\)
    if there exists a rule \(
        \spann{L}{i+j}{G} \in \mcr
    \) and cospan \(
        \cospan{i+j}{C}{n+m} \in \hypsigma
    \) such that diagram in \cref{def:dpo} commutes and \(i+j \to C \to G\) is a
    traced left-boundary complement.
\end{definition}

\begin{lemma}[Traced comonoid decomposition]\label{lem:traced-comonoid-decomposition}
    \cref{lem:traced-decomposition} holds when all cospans are partial
    left-monogamous and \(
        \cospan{j+m}[c_2,d_1]{C}[c_1,d_2]{i+n}
    \) is a traced left-boundary complement.
\end{lemma}
\iftoggle{proofs}{
    \begin{proof}

    \end{proof}
}{}

\begin{theorem}\label{thm:traced-comonoid-rewriting}
    Let \(\mcr\) be a rewriting system on \(\stmcsigma + \ccomon\).
    Then, \(
        \iltikzfig{strings/category/f}[g][white]
        \rewrite[\mcr]
        \iltikzfig{strings/category/f}[h][white]
    \) in \(\stmcsigma + \ccomon\) if and only if \(
        \termandfrobtohypsigma[
            \foldinterfaces[
                \tracedandcomonoidtofrob[
                    \iltikzfig{strings/category/f}[g][white]
                ]{\Sigma}
            ]
        ]
        \grewrite[
            \termandfrobtohypsigma[
                \foldinterfaces[
                    \tracedandcomonoidtofrob[\mcr]{\Sigma}
                ]
            ]
        ]
        \termandfrobtohypsigma[
            \foldinterfaces[
                \tracedandcomonoidtofrob[
                    \iltikzfig{strings/category/f}[h][white]
                ]{\Sigma}
            ]
        ].
    \).
\end{theorem}
\begin{proof}
    This is the same as for \cref{thm:traced-rewriting} but with the replacement
    of all traced boundary complements with traced left-boundary complements.
\end{proof}

\begin{example}
    As with the traced case, there may be multiple valid rewrites given a
    particular interface.
    The comonoid structure adds more possibilities, as there are the equations
    of commutative comonoids to consider.
    Consider the following rule and its interpretation.
    \begin{gather}
        \rrule{
            \iltikzfig{graphs/dpo/non-unique-comonoid/rule-lhs}
        }{
            \iltikzfig{graphs/dpo/non-unique-comonoid/rule-rhs}
        }
        \qquad
        \raisebox{-1.7em}{\includestandalone{figures/graphs/dpo/non-unique-comonoid/rule}}
        \label{gath:non-unique-rule-comonoid}
    \end{gather}
    Two valid rewrites are as follows:
    \begin{center}
        \includestandalone{figures/graphs/dpo/non-unique-comonoid/rewrite-1}
        \quad
        \includestandalone{figures/graphs/dpo/non-unique-comonoid/rewrite-2}
    \end{center}
    The first rewrite is the `obvious' one, but the second also holds by
    cocommutativity:
    \begin{gather*}
        \iltikzfig{graphs/dpo/non-unique-comonoid/rewrite-1}
        =
        \iltikzfig{graphs/dpo/non-unique-comonoid/rewrite-2a}
        \qquad
        \iltikzfig{graphs/dpo/non-unique-comonoid/rewrite-1}
        =
        \iltikzfig{graphs/dpo/non-unique-comonoid/rewrite-2b}
        =
        \iltikzfig{graphs/dpo/non-unique-comonoid/rewrite-3b}
    \end{gather*}
\end{example}