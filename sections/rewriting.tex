% !TeX root = ../main-conf.tex
\section{Graph rewriting}

\begin{definition}[Rewrite rule]
    Given cospans \(
        \cospan{i}[a_1]{L}[a_2]{j}
    \) and \(
        \cospan{i}[b_1]{R}[b_2]{j}
    \), their rewrite rule is a span in \(\hypsigma\) \(
        \spann{L}[[a_1,a_2]]{i+j}[[b_1,b_2]]{R}
    \).
\end{definition}

\begin{lemma}
    Given terms \(
        \iltikzfig{strings/rewriting/l}
    \), \(
        \iltikzfig{strings/rewriting/r}
    \) in \(\stmcsigma\) and their corresponding rewrite rule \[
        \spann{
            \termtohypsigma[\foldinterfaces[\iltikzfig{strings/rewriting/l}]]
        }[[a_1, a_2]]{
            i + j
        }[[b_1, b_2]]{
            \termtohypsigma[\foldinterfaces[\iltikzfig{strings/rewriting/r}]]
        }
    \] in \(\hypsigma\), \(a_1\), \(a_2\), \(b_1\) and \(b_2\) are mono.
\end{lemma}
\iftoggle{proofs}{
    \begin{proof}
        By definition of partial monogamous cospan.
    \end{proof}
}{}

\begin{definition}[DPO]
    \begin{gather}
        \label{gath:dpo}
        \includestandalone{figures/graphs/dpo/dpo}
    \end{gather}
\end{definition}

\begin{definition}[Boundary complement \cite{bonchi2021string}, Definition 23]\label{def:boundary-complement}
    For \(
        \cospan{i}[a_1]{L}[a_2]{j}
    \) and \(
        \cospan{n}[b_1]{G}[b_2]{m} \in \macspfihyp
    \) and mono \(
        \morph{f}{L}{G} \in \hypsigma
    \), a pushout complement as below
    \begin{gather}
        \includestandalone{figures/graphs/dpo/boundary-complement}
        \label{gath:boundary-complement}
    \end{gather}
    is called a \emph{boundary complement in \(\mathcal{C}\)} if \([c_1, c_2]\)
    is mono and there exist morphisms \(
        \morph{d_1}{n}{\pushoutcomplement{L}}
    \) and \(
        \morph{d_2}{m}{\pushoutcomplement{L}}
    \) making the above triangle commute and such that \[
        \cospan{j+n}[[c_2,d_1]]{\pushoutcomplement{L}}[[c_1,d_2]]{{m+i}}
    \] is a monogamous cospan.
\end{definition}

\begin{proposition}[\cite{bonchi2021string}, Proposition 24]\label{prop:boundary-complement-unique}
    Boundary complements in \(\macspfihyp\) are unique, when they exist.
\end{proposition}

\noindent
Crucially, boundary complements rely on the matching \(L \to G\) being mono.
But restricting to these matchings cuts off potential rewrites in the
\emph{traced} setting, such as the occurrence of a rewrite rule inside a loop:
\[
    \iltikzfig{graphs/dpo/matchings/non-mono-matching}
\]

\begin{definition}
    For partial monogamous cospans \(
        \cospan{i}[a_1]{L}[a_2]{j}
    \) and \(
        \cospan{n}[b_1]{G}[b_2]{m} \in \mathcal{C}
    \) and (not necessarily mono) \(
        \morph{f}{L}{G} \in \hypsigma
    \), a pushout complement as in \cref{gath:boundary-complement}
    is called a \emph{traced boundary complement} if \(c_1\) and \(c_2\) are
    mono, \(f\) is injective on edges and vertices not in the image of \(a\),
    and there exist morphisms \(
        \morph{d_1}{m}{\pushoutcomplement{L}}
    \) and \(
        \morph{d_2}{n}{\pushoutcomplement{L}}
    \) making the above triangle commute such that
    \begin{gather}
        \cospan{j+m}[[c_2,d_1]]{\pushoutcomplement{L}}[[d_2,c_1]]{{n+i}}
        \label{gat:traced-complement}
    \end{gather} is a partial monogamous cospan.
\end{definition}

\noindent
Using the semi-monogamicity of (\ref{gat:traced-complement}) we can prove some
useful lemmas regarding when vertices can be coalesced.

\begin{lemma}
    In a traced boundary complement, if \(
        a_1(v) = a_2(w)
    \), then \(
        c_1(v) = c_2(w)
    \) if and only if \(c_1(v)\) and \(c_2(w)\) are not in the image of \(d\)
    and \(f(a_1(v)) = f(a_2(w))\) has degree \((0,0)\).
\end{lemma}
\iftoggle{proofs}{
    \begin{proof}
        For the \(\onlyifdir\) direction, assume that \(c_1(v) = c_2(w)\): let this
        vertex be \(v^\prime\).
        Since (\ref{gat:traced-complement}) is partial monogamous, \([c_2,d_1]\) and
        \([c_1,d_2]\) must be mono, and thus the images of \(c_1\) and \(d_2\) must
        be disjoint, as must the images of \(c_2\) and \(d_1\).
        Therefore \(v^\prime\) cannot be in the image of \(d\); moreover, \(c_1(v)\)
        must have out-degree \(0\) and \(c_2(w)\) must have in-degree \(0\), so
        \(v^\prime\) has degree \((0,0)\).
        Let \(v^{\prime\prime}\) be \(a_1(v) = a_2(w)\): since \(
            \cospan{i}[a_1]{\iltikzfig{strings/rewriting/l}}[a_2]{j}
        \) is a partial monogamous cospan \(v^\prime\) must have degree \((0,0)\).
        Since \(G\) is computed by pushout, \(
            f(v^{\prime\prime}) = g(v^{\prime})
        \) and the degree of this vertex is contributed wholly by \(v^\prime\) and
        \(v^{\prime\prime}\).
        So \(f(a_1(v)) = f(a_2(w))\) also has degree \((0,0)\).

        Now for the \(\ifdir\) direction, assume that \(c_1(v)\) and \(c_2(w)\) are
        not in the image of \(d\), and \(f(a_1(v)) = f(a_2(w))\) has degree
        \((0,0)\).
        Assume that \(c_1(v) \neq c_2(w)\).
        By semi-monogamy of (\ref{gat:traced-complement}), \(c_1(v)\) must have
        degree \((0,1)\) since it is solely in the image of \([d_2, c_1]\).
        But \(f(a_1(v)) = f(a_2(w)) = g(c_1(v)) = g(c_2(w))\), \(c_1(v)\) and
        \(c_2(w)\) must have degree \((0,0)\), a contradiction.
        The same holds for \(c_2(w)\), so \(c_1(v) = c_2(w)\).
    \end{proof}
}{}

A crucial part of~\cite{bonchi2021string} is that (non-traced) boundary
complements are \emph{unique}.
One might assume the same for traced boundary complements, but this is not the
case.

\begin{example}
    Consider the rule and its interpretation.
    \begin{gather}
        \rrule{
            \iltikzfig{graphs/dpo/non-unique/rule-lhs}
        }{
            \iltikzfig{graphs/dpo/non-unique/rule-rhs}
        }
        \qquad
        \raisebox{-2.1em}{\includestandalone{figures/graphs/dpo/non-unique/rule}}
        \label{gath:non-unique-rule}
    \end{gather}

    Now consider the following hypergraph with interface \(
        \iltikzfig{graphs/dpo/non-unique/g-unlabelled}
        \leftarrow
        \iltikzfig{graphs/dpo/non-unique/j-unlabelled}
    \).
    There are \emph{two} valid traced boundary complements for the above rule in
    this graph!

    \begin{center}
        \scalebox{0.95}{
            \includestandalone{figures/graphs/dpo/non-unique/rewrite-1}
            \includestandalone{figures/graphs/dpo/non-unique/rewrite-2}
        }
    \end{center}
    \noindent
    Both derivations are valid and arise since we are rewriting modulo
    \emph{yanking}:
    \begin{gather*}
        \iltikzfig{graphs/dpo/non-unique/derivation-1}
        =
        \iltikzfig{graphs/dpo/non-unique/derivation-2}
        =
        \iltikzfig{graphs/dpo/non-unique/derivation-3a}
        =
        \iltikzfig{graphs/dpo/non-unique/derivation-4a}
        =
        \iltikzfig{graphs/dpo/non-unique/derivation-5a}
        \\
        \iltikzfig{graphs/dpo/non-unique/derivation-1}
        =
        \iltikzfig{graphs/dpo/non-unique/derivation-2}
        =
        \iltikzfig{graphs/dpo/non-unique/derivation-3b}
        =
        \iltikzfig{graphs/dpo/non-unique/derivation-4b}
        =
        \iltikzfig{graphs/dpo/non-unique/derivation-5b}
    \end{gather*}
\end{example}

\noindent
Rewriting modulo yanking also eliminates another foible of rewriting modulo
(non-traced) symmetric monoidal structure.
In the latter, matchings must be \emph{convex}: there cannot be a path from the
outputs of a match back to its inputs.
However, in the traced case this is redundant, as illustrated in the following
example.

\begin{example}
    Consider the following rewrite rule and its interpretation.
    %
    \begin{gather}
        \rrule{
            \iltikzfig{graphs/dpo/convex/example-l}
        }{
            \iltikzfig{graphs/dpo/convex/example-r}
        }
        \qquad
        \iltikzfig{graphs/dpo/convex/example-rule-graph}
        \label{gath:convex-rule}
    \end{gather}
    %
    \noindent
    Now consider the following term and interpretation:
    %
    \begin{gather}
        \iltikzfig{graphs/dpo/convex/example-g}
        \qquad
        \iltikzfig{graphs/dpo/convex/example-g-graph}
        \label{gath:convex-term}
    \end{gather}
    %
    \noindent
    Although it is not obvious in the original string diagram, there is in fact
    a matching of (\ref{gath:convex-rule}) in (\ref{gath:convex-term}).
    Performing the DPO procedure yields the following:
    %
    \begin{gather}
        \iltikzfig{graphs/dpo/convex/example-h-graph}
        \qquad
        \iltikzfig{graphs/dpo/convex/example-h}
    \end{gather}
    %
    \noindent
    In a non-traced setting this is an invalid rule!
    However, it is possible with yanking.
    \begin{gather*}
        \iltikzfig{graphs/dpo/convex/example-g}
        =
        \iltikzfig{graphs/dpo/convex/rewrite-1}
        =
        \iltikzfig{graphs/dpo/convex/rewrite-2}
        =
        \iltikzfig{graphs/dpo/convex/rewrite-3}
        \\[1em]
        =
        \iltikzfig{graphs/dpo/convex/rewrite-4}
        =
        \iltikzfig{graphs/dpo/convex/rewrite-5}
        =
        \iltikzfig{graphs/dpo/convex/example-h}
    \end{gather*}
\end{example}

\begin{definition}[Traced DPO]
    For morphisms \(G \leftarrow m+n\) and \(H \leftarrow m+n\) in
    \(\hypsigma\), there is a traced rewrite \(G \trgrewrite{\mcr} H\) if there
    exists a rule \(
        \spann{L}{i+j}{G} \in \mcr
    \) and cospan \(
        \cospan{i+j}{C}{n+m} in \hypsigma
    \) such that the following diagram commutes:
    \begin{gather}
        \includestandalone{figures/graphs/dpo/dpo}
    \end{gather}
    and \(i+j \to C\) is a traced boundary complement.
\end{definition}

\noindent
We are almost ready to show the soundness and completeness of this DPO rewriting
system.
The final prerequisite is a decomposition lemma, akin to a similar result
in~\cite{bonchi2021string} for the symmetric monoidal case.

\begin{lemma}[Traced decomposition]\label{lem:decomposition}
    Given partial monogamous cospans \(\cospan{m}[p_1]{G}[p_2]{n}\) and \(
        \cospan{i}[a_1]{L}[a_2]{j}
    \), along with a morphism \(
        L \xrightarrow{f} G
    \) satisfying the no-dangling-hyperedges condition, then \(
        \cospan{m}[p_1]{G}[p_2]{n}
    \) can be factored as
    \begin{gather}
        \trace{i}{
            \begin{array}{cc}
                \cospan{i}[a_1]{L}[a_2]{j} \\
                \tensor \\
                \cospan{m}{m}{m}
            \end{array}
            \seq
            \cospan{j+m}{C}{i+n}
        }
        \label{gath:decomposition}
    \end{gather}
    where all cospans are partial monogamous and \(C\) is a traced boundary
    complement.
\end{lemma}
\iftoggle{proofs}{
    \begin{proof}
        Let \(C\) be defined as a traced boundary complement of \(
            i+j \xrightarrow{[a_1,a_2]} L \xrightarrow{f} G
        \), which exists as the no-dangling-hyperedges condition is satisfied.
        We assign names to the various cospans in play, and reason string
        diagrammatically:
        \begin{align*}
            \iltikzfig{graphs/dpo/lhat} &:= \cospan{i}{L}{j}
            &
            \iltikzfig{graphs/dpo/ltilde} &:= \cospan{0}{L}{i+j} \\
            \iltikzfig{graphs/dpo/chat} &:= \cospan{j+m}{L}{i+n}
            &
            \iltikzfig{graphs/dpo/ctilde} &:= \cospan{i+j}{C}{m+n} \\
            \iltikzfig{graphs/dpo/ghat} &:= \cospan{m}{G}{n}
            &
            \iltikzfig{graphs/dpo/gtilde} &:= \cospan{0}{G}{m+n}
        \end{align*}
        By using the compact closed structure of \(\cspfihyp\), we also have that \(
            \iltikzfig{graphs/dpo/ltilde} = \iltikzfig{graphs/dpo/lhat-bent}
        \), \(
            \iltikzfig{graphs/dpo/ctilde} = \iltikzfig{graphs/dpo/chat-bent}
        \) and \(
            \iltikzfig{graphs/dpo/gtilde} = \iltikzfig{graphs/dpo/ghat-bent}
        \).
        Since \(
            \iltikzfig{graphs/dpo/gtilde} = \iltikzfig{graphs/dpo/lctilde}
        \), it follows that \(
            \iltikzfig{graphs/dpo/ghat-bent} = \iltikzfig{graphs/dpo/lchat-bent}
        \) and subsequently \(
            \iltikzfig{graphs/dpo/ghat} = \iltikzfig{graphs/dpo/lchat}
        \).
        The `loop' is constructed in the same manner as the canonical trace on
        \(\cspfihyp\), so this is a term in the form of (\ref{gath:decomposition}).
        Moreover, all cospans involved are partial monogamous by definition of
        rewrite rules and traced boundary complements.
    \end{proof}
}{}

\begin{theorem}
    Let \(\mcr\) be a rewriting system on \(\stmcsigma\).
    Then, \(
        \iltikzfig{strings/category/f}[F][white]
        \rewrite[\mcr]
        \iltikzfig{strings/category/f}[H][white]
    \) if and only if \(
        \termandfrobtohypsigma[\foldinterfaces[\iltikzfig{strings/category/f}[F][white]]]
        \grewrite[\termandfrobtohypsigma[\foldinterfaces[\mcr]]]
        \termandfrobtohypsigma[\foldinterfaces[\iltikzfig{strings/category/f}[H][white]]].
    \)
\end{theorem}
\iftoggle{proofs}{
    \begin{proof}
        First the \((\Rightarrow)\) direction.
        If \(
            \iltikzfig{strings/category/f}[F][white]
            \rewrite[\mcr]
            \iltikzfig{strings/category/f}[H][white]
        \) then we have \(
            \iltikzfig{strings/category/f}[F][white]
            =
            \iltikzfig{strings/rewriting/rewrite-l}
        \) and \(
            \iltikzfig{strings/rewriting/rewrite-r}
            =
            \iltikzfig{strings/category/f}[H][white].
        \)
        Define the following cospans:
        \begin{align}
            \label{gath:l-cospan}
            \cospan{0}{L}{i+j}
            &:=
            \termandfrobtohypsigma[\foldinterfaces[\iltikzfig{strings/rewriting/l}]]
            &&=
            \termandfrobtohypsigma[\iltikzfig{strings/rewriting/l-folded}]
            \\
            \cospan{0}{R}{i+j}
            &:=
            \termandfrobtohypsigma[\foldinterfaces[\iltikzfig{strings/rewriting/r}]]
            &&=
            \termandfrobtohypsigma[\iltikzfig{strings/rewriting/r-folded}]
            \\
            \cospan{0}{G}{m+n}
            &:=
            \termandfrobtohypsigma[\foldinterfaces[\iltikzfig{strings/category/f}[F][white]]]
            &&=
            \termandfrobtohypsigma[\iltikzfig{strings/rewriting/lc-folded}]
            \\
            \label{gath:h-cospan}
            \cospan{0}{H}{m+n}
            &:=
            \termandfrobtohypsigma[\foldinterfaces[\iltikzfig{strings/category/f}[H][white]]]
            &&=
            \termandfrobtohypsigma[\iltikzfig{strings/rewriting/rc-folded}]
            \\
            \cospan{i+j}{C}{m+n}
            &:=
            \termandfrobtohypsigma[\iltikzfig{strings/rewriting/c-folded}]
            &&
        \end{align}
        By functoriality, since \(
            \foldinterfaces[\iltikzfig{strings/category/f}[F][white]]
            =
            \iltikzfig{strings/rewriting/l-folded}
            \seq
            \iltikzfig{strings/rewriting/c-folded}
        \) then \[
            \cospan{0}{G}{m+n} = \cospan{0}{L}{i+j} \seq \cospan{i+j}{C}{m+n}.
        \]
        Cospan composition is pushout, so \(\cospan{L}{G}{C}\) is a pushout.
        Using the same reasoning, \(\cospan{R}{G}{C}\) is also a pushout: this gives
        us the DPO diagram.
        All that remains is to check that the aforementioned pushouts are traced
        boundary complements: this follows by inspecting components.

        Now the \(\ifdir\) direction: assume we have a DPO diagram (\ref{gath:dpo})
        where \(L \leftarrow i + j\), \(i + j \rightarrow R\), \(m + n \to G\) and
        \(m + n \to H\) are defined as in (\ref{gath:l-cospan}-\ref{gath:h-cospan})
        above.
        We must show that \(
            \iltikzfig{strings/category/f}[F][white]
            =
            \iltikzfig{strings/rewriting/rewrite-l}
        \) and \(
            \iltikzfig{strings/category/f}[H][white]
            =
            \iltikzfig{strings/rewriting/rewrite-r}
        \).
        By definition of traced boundary complement \(\cospan{j+m}{C}{i+n}\) is a
        partial monogamous cospan, so by fullness of \(\termandfrobtohypsigma\),
        there exists a term \(\iltikzfig{strings/rewriting/c} \in \stmcsigma\) such
        that \(
            \termandfrobtohypsigma[\iltikzfig{strings/rewriting/c}]
            =
            \cospan{j+m}{C}{i+n}
        \).
        By traced decomposition (\cref{lem:decomposition}), we have that for any
        traced boundary complement \(\cospan{i+j}{C}{m+n}\) and morphism
        \(L \to G\), \(\cospan{m}{G}{n}\) can be factored as in
        (\ref{gath:decomposition}), i.e.\ \[
            \termandfrobtohypsigma[\iltikzfig{strings/category/f}[F][white]]
            =
            \trace{j}{\termandfrobtohypsigma[\iltikzfig{strings/rewriting/l}]
            \tensor
            \id[n]
            \seq
            \termandfrobtohypsigma[\iltikzfig{strings/rewriting/c}]}.
        \]
        So by functoriality, we have that \(
            \iltikzfig{strings/category/f}[F][white]
            =
            \iltikzfig{strings/rewriting/rewrite-l}
        \).
        The same reasoning follows for \(
            \iltikzfig{strings/category/f}[H][white]
            =
            \iltikzfig{strings/rewriting/rewrite-r}
        \).
    \end{proof}
}{}
