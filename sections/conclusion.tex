% !TeX root = ../main-conf.tex
\section{Conclusion, related and further work}

\begin{figure}
    \centering
    \includestandalone{figures/graphs/roadmap}
    \caption{The various PROP morphisms at play.}
    \label{fig:roadmap}
\end{figure}

We have shown how the frameworks for rewriting string diagrams modulo
Frobenius~\cite{bonchi2022string} and symmetric
monoidal~\cite{bonchi2022stringa} structure using hypergraphs can also be
adapted for rewriting modulo traced comonoid structure, by using
hypergraphs that sit between the two settings.

Graphical languages for traced categories have seen many applications, such as
to illustrate cyclic lambda calculi~\cite{hasegawa1997recursion}, or to reason
graphically about programs~\cite{schweimeier1999categorical}.
The presentation of traced categories as \emph{string diagrams} has existed
since the 90s~\cite{joyal1991geometry,joyal1996traced}; a soundness and
completeness theorem for traced string diagrams, folklore for many years
but only proven for certain signatures~\cite{selinger2011survey}, was finally
shown in~\cite{kissinger2014abstract}.
Combinatorial languages predate even this, having existed since at least the 80s
in the guise of
\emph{flowchart schemes}~\cite{stefanescu1990feedback,cazanescu1990new,cazanescu1994feedback}.
These diagrams have also been used to show the completeness of finite dimensional
vector spaces~\cite{hasegawa2008finite} with respect to traced categories and,
when equipped with a dagger, Hilbert spaces~\cite{selinger2012finite}.

We are not just concerned with diagrammatic languages as a standalone concept:
we are interested in performing \emph{graph rewriting} with them to reason about
monoidal theories.
This has been been studied in the context of traced categories before using
\emph{string graphs}~\cite{kissinger2012pictures,dixon2013opengraphs}.
We have instead opted to use the \emph{hypergraph} framework
of~\cite{bonchi2022string,bonchi2022stringa,bonchi2022stringb} instead, as it
allows rewriting modulo \emph{yanking}, is more extensible for rewriting modulo
comonoid structure, and one does not need to awkwardly reason modulo wire
homeomorphisms.

As mentioned during the case studies, there are still elements of the rewriting
framework that are somewhat informal.
One such issue involves defining rewrite spans for arbitrary subgraphs: this is
hard to do at a general level because the edges must be concretely specified in
DPO rewriting.
However, if we performed rewriting with
\emph{hierarchical hypergraphs}~\cite{alvarez-picallo2021functorial}, in
which edges can have hypergraphs as labels, we could `compress' the subgraph
into a single edge that can be rewritten: this is future work.

In regular PROP notation, wires are annotated with numbers in order to avoid
drawing multiple wires in parallel: when interpreted as hypergraphs a vertex is
created for each wire, and simple diagrams can quickly get very large.
The results of \cite{bonchi2022stringa} also extend to the multi-sorted case, in
which vertices are labelled in addition to wires.
We could use this in combination with the \emph{strictifiers}
of~\cite{wilson2022string}: these are additional generators for transforming
buses of wires into thinner or thicker ones.
This could drastically reduce the number of elements in a hypergraph, which is
ideal from a computational point of view.
Work has already begun on implementing the rewriting system for digital circuits
using these techniques.

\section*{Acknowledgements}

Thanks to Chris Barrett for helpful comments.