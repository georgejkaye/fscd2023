% !TeX root = ../main-conf.tex
\section{Conclusion, related and further work}

We have shown how the existing frameworks for rewriting string diagrams modulo
Frobenius~\cite{bonchi2022string} and symmetric
monoidal~\cite{bonchi2022stringa} structure can also be adapted for rewriting
modulo traced comonoid structure, by looking at graphs that sit between
the two settings.

Graphical languages for traced categories have seen many applications, such as
to illustrate cyclic lambda calculi~\cite{hasegawa1997recursion}, or to reason
graphically about programs~\cite{schweimeier1999categorical}.
The presentation as \emph{string diagrams} has existed since the
90s~\cite{joyal1991geometry,joyal1996traced}; a soundness and completeness
theorem for traced string diagrams, while folklore for many years but only
proven for certain signatures~\cite{selinger2011survey}, was finally shown
in~\cite{kissinger2014abstract}.
Combinatorial languages predate even this, having existed since at least the 80s
in the guise of
\emph{flowchart schemes}~\cite{stefanescu1990feedback,cazanescu1990new,cazanescu1994feedback}
These diagrams have also been used to show the completeness of finite dimensional
vector spaces~\cite{hasegawa2008finite} and, when equipped with a dagger,
Hilbert spaces~\cite{selinger2012finite}.

We are not just concerned with diagrammatic languages as a standalone concept:
we are interested in performing \emph{graph rewriting} with them in order to
reason about monoidal theories.
This has been been studied in the context of traced categories before with
\emph{string graphs}~\cite{kissinger2012pictures,dixon2013opengraphs}.
We have instead opted to use the \emph{hypergraph} framework
of~\cite{bonchi2022string,bonchi2022stringa,bonchi2022stringb} as a basis for our
rewriting framework, as it allows rewriting modulo \emph{yanking}, is more
extensible for rewriting modulo comonoid structure, and there is less hassle
involved with wire homeomorphisms.

As mentioned during the case studies, there are still elements of the rewriting
framework that are somewhat informal.
One such issue involves defining rewrite spans for arbitrary subgraphs: this is
hard to do at a general level because the edges must be concretely specified in
DPO rewriting.
However, if we performed rewriting with
\emph{hierarchical hypergraphs}~\cite{alvarez-picallo2021functorial}, in
which edges can have hypergraphs as labels, we could `compress' the subgraph
into a single edge that can be rewritten.

In situations such as the sequential circuits study in \cref{sec:digital-circuits},
wires of width greater than one often represent bundles (or \emph{buses}) of
wires.
In regular PROP notation this is merely syntactic sugar to avoid drawing
multiple wires in parallel: moreover, when interpreted as hypergraphs a vertex
is created for each wire.
This means that simple circuits can quickly get very large.
Usefully, the results of \cite{bonchi2022stringa} are also shown to extend to the
multi-sorted case, in which vertices are labelled in addition to wires; we could
use this in combination with the \emph{strictifiers} of~\cite{wilson2022string}
which can transform buses of wires into thinner or thicker ones in order to
drastically reduce the number of elements in a hypergraph, ideal from a
computational point of view.
We are already working on implementing the rewriting system for digital circuits
using these techniques.


\begin{figure}
    \centering
    \includestandalone{figures/graphs/roadmap}
    \caption{The various PROP morphisms at play.}
    \label{fig:roadmap}
\end{figure}