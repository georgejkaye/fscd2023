Let \(C\) be defined as a traced boundary complement of \(
    i+j \xrightarrow{[a_1,a_2]} L \xrightarrow{f} G
\), which exists as the no-dangling and no-identification condition is
satisfied.
We assign names to the various cospans in play, and reason string
diagrammatically:
\begin{align*}
    \iltikzfig{strings/category/f}[l][white][i][j] &:= \cospan{i}{L}{j}
    &
    \iltikzfig{strings/category/f-0-2}[\hat{l}][white][i][j] &:= \cospan{0}{L}{i+j} \\
    \iltikzfig{strings/category/f-2-2}[c][white][j][m][i][n] &:= \cospan{j+m}{L}{i+n}
    &
    \iltikzfig{strings/category/f-2-2}[\hat{c}][white][i][j][m][n] &:= \cospan{i+j}{C}{m+n} \\
    \iltikzfig{strings/category/f}[g][white][m][n] &:= \cospan{m}{G}{n}
    &
    \iltikzfig{strings/category/f-0-2}[\hat{g}][white][m][n] &:= \cospan{0}{G}{m+n}
\end{align*}
By using the compact closed structure of \(\cspdhyp\), we also have that \(
    \iltikzfig{strings/category/f-0-2}[\hat{l}][white][i][j]
    =
    \iltikzfig{strings/compact-closed/f-bent-input}[l][white][i][j]
\), \(
    \iltikzfig{strings/category/f-2-2}[\hat{c}][white][i][j][m][n]
    =
    \iltikzfig{graphs/dpo/chat-bent}
\) and \(
    \iltikzfig{strings/category/f-0-2}[\hat{g}][white][i][j]
    =
    \iltikzfig{strings/compact-closed/f-bent-input}[g][white][i][j]
\).
Since \(
    \iltikzfig{graphs/dpo/gtilde} = \iltikzfig{graphs/dpo/lctilde}
\), it follows that \(
    \iltikzfig{graphs/dpo/ghat-bent} = \iltikzfig{graphs/dpo/lchat-bent}
\) and subsequently \(
    \iltikzfig{graphs/dpo/ghat} = \iltikzfig{graphs/dpo/lchat}
\).
The `loop' is constructed in the same manner as the canonical trace on
\(\cspfihyp\), so this is a term in the form of (\ref{gath:decomposition}).
Moreover, all cospans involved are partial monogamous by definition of
rewrite rules and traced boundary complements.