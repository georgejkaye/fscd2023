Let \(
    i + j \xrightarrow{[c_1, c_2]} C \xleftarrow{[d_1, d_2]} m+n
\) be defined as a traced boundary complement of \(
    i+j \xrightarrow{[a_1,a_2]} L \xrightarrow{f} G
\), which exists as the no-dangling and no-identification condition is
satisfied.
We assign names to the various cospans in play, and reason string
diagrammatically:
\begin{align*}
    \iltikzfig{strings/category/f}[l][white][i][j] &:= \cospan{i}{L}{j}
    &
    \iltikzfig{strings/category/f-0-2}[\hat{l}][white][i][j]
    &:=
    \cospan{0}{L}{i+j}
    \\
    \iltikzfig{strings/category/f-2-2}[c][white][j][m][i][n]
    &:=
    \cospan{j+m}[[c_2, d_1]]{C}[[c_1, d_2]]{i+n}
    &
    \iltikzfig{strings/category/f-2-2}[\hat{c}][white][i][j][m][n]
    &:=
    \cospan{i+j}[[c_1, c_2]]{C}[[d_1, d_2]]{m+n}
    \\
    \iltikzfig{strings/category/f}[g][white][m][n]
    &:=
    \cospan{m}{G}{n}
    &
    \iltikzfig{strings/category/f-0-2}[\hat{g}][white][m][n]
    &:=
    \cospan{0}{G}{m+n}
\end{align*}
Note that the cospans in the left column are partial monogamous by definition
of rewrite rules and traced boundary complements.
We will show that  \(
    \iltikzfig{strings/category/f}[g][white]
\) can be decomposed into a form using the two cospans \(
    \iltikzfig{strings/category/f}[l][white]
\) and \(
    \iltikzfig{strings/category/f-2-2}[c][white]
\), along with identities.

By using the compact closed structure of \(\cspdhyp\), we have that \(
    \iltikzfig{strings/category/f}[g][white][m][n]
    =
    \iltikzfig{graphs/dpo/g-bent}
\), \(
    \iltikzfig{strings/category/f-2-2}[c][white][i][j][m][n]
    =
    \iltikzfig{graphs/dpo/chat-bent}
\) and \(
    \iltikzfig{strings/category/f-0-2}[\hat{l}][white][i][j]
    =
    \iltikzfig{strings/compact-closed/f-bent-input}[l][white][i][j]
\).
Since \(G\) is the pushout of \(
    L \xleftarrow{[a_1, a_2]} i+j \xrightarrow{[c_1, c_2]} C
\) and pushout is cospan composition, we have \(
    \iltikzfig{strings/category/f-0-2}[\hat{g}][white]
    =
    \iltikzfig{graphs/dpo/lctilde}
\).

Putting this all together we have that
\begin{gather*}
    \iltikzfig{strings/category/f}[g][white]
    =
    \iltikzfig{graphs/dpo/g-bent}
    =
\end{gather*}
The `loop' is constructed in the same manner as the canonical trace on
\(\cspfihyp\), so this is a term in the form of (\ref{gath:decomposition}).
Moreover, all cospans involved are partial monogamous by definition of
rewrite rules and traced boundary complements.