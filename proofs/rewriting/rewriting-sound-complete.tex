% !TeX root = ../../main-arxiv.tex
First the \((\Rightarrow)\) direction.
If \(
    \iltikzfig{strings/category/f}[g][white]
    \rewrite[\mcr]
    \iltikzfig{strings/category/f}[h][white]
\) then we have \(
    \iltikzfig{strings/category/f}[g][white]
    =
    \iltikzfig{strings/rewriting/rewrite-l}
\) and \(
    \iltikzfig{strings/rewriting/rewrite-r}
    =
    \iltikzfig{strings/category/f}[h][white].
\)
Define the following cospans:
\begin{alignat}{3}
    \label{gath:l-cospan}
    \cospan{0}{L}{i+j}
    &:=
    \termandfrobtohypsigma[\foldinterfaces[\iltikzfig{strings/category/f}[l][white]]]
    &&=
    \termandfrobtohypsigma[\iltikzfig{strings/rewriting/l-folded}]
    \\
    \cospan{0}{R}{i+j}
    &:=
    \termandfrobtohypsigma[\foldinterfaces[\iltikzfig{strings/category/f}[r][white]]]
    &&=
    \termandfrobtohypsigma[\iltikzfig{strings/rewriting/r-folded}]
    \\
    \cospan{0}{G}{m+n}
    &:=
    \termandfrobtohypsigma[\foldinterfaces[\iltikzfig{strings/category/f}[f][white]]]
    &&=
    \termandfrobtohypsigma[\iltikzfig{strings/rewriting/lc-folded}]
    \\
    \label{gath:h-cospan}
    \cospan{0}{H}{m+n}
    &:=
    \termandfrobtohypsigma[\foldinterfaces[\iltikzfig{strings/category/f}[h][white]]]
    &&=
    \termandfrobtohypsigma[\iltikzfig{strings/rewriting/rc-folded}]
    \\
    \cospan{i+j}{C}{m+n}
    &:=
    \termandfrobtohypsigma[\iltikzfig{strings/rewriting/c-folded}]
    &&
\end{alignat}
By functoriality, since \(
    \foldinterfaces[\iltikzfig{strings/category/f}[f][white]]
    =
    \iltikzfig{strings/rewriting/l-folded}
    \seq
    \iltikzfig{strings/rewriting/c-folded}
\) then \[
    \cospan{0}{G}{m+n} = \cospan{0}{L}{i+j} \seq \cospan{i+j}{C}{m+n}.
\]
Cospan composition is pushout, so \(\cospan{L}{G}{C}\) is a pushout.
Using the same reasoning, \(\cospan{R}{G}{C}\) is also a pushout: this
gives us the DPO diagram.
All that remains is to check that the aforementioned pushouts are traced
boundary complements: this follows by inspecting components.

Now the \(\ifdir\) direction: assume we have a DPO diagram (\ref{def:dpo})
where \(L \leftarrow i + j\), \(i + j \rightarrow R\), \(m + n \to G\) and
\(m + n \to H\) are defined as in (\ref{gath:l-cospan}-\ref{gath:h-cospan})
above.
We must show that \(
    \iltikzfig{strings/category/f}[f][white]
    =
    \iltikzfig{strings/rewriting/rewrite-l}
\) and \(
    \iltikzfig{strings/category/f}[h][white]
    =
    \iltikzfig{strings/rewriting/rewrite-r}
\).
By definition of traced boundary complement \(\cospan{j+m}{C}{i+n}\) is a
partial monogamous cospan, so by fullness of \(\termandfrobtohypsigma\),
there exists a term \(
    \iltikzfig{strings/category/f-2-2}[c][white][j][m][i][n]
    \in \smcsigma
\) such that \(
    \termtohyp[
        \iltikzfig{strings/category/f-2-2}[c][white]
    ]{\Sigma}
    =
    \cospan{j+m}{C}{i+n}
\).
By traced decomposition (\cref{lem:traced-decomposition}), we have that for any
traced boundary complement \(\cospan{i+j}{C}{m+n}\) and morphism
\(L \to G\), \(\cospan{m}{G}{n}\) can be factored as in
(\ref{gath:decomposition}), i.e.\ \[
    \termandfrobtohypsigma[\iltikzfig{strings/category/f}[f][white]]
    =
    \trace{j}{\termandfrobtohypsigma[\iltikzfig{strings/category/f}[l][white]]
    \tensor
    \id[n]
    \seq
    \termandfrobtohypsigma[\iltikzfig{strings/category/f-2-2}[c][white]]}.
\]
So by functoriality, we have that \(
    \iltikzfig{strings/category/f}[f][white]
    =
    \iltikzfig{strings/rewriting/rewrite-l}
\).
The same reasoning follows for \(
    \iltikzfig{strings/category/f}[h][white]
    =
    \iltikzfig{strings/rewriting/rewrite-r}
\).