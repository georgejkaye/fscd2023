% !TeX root = ../../main-conf.tex
To show that \(\tracedtosymandfrob[\termtohyp[f]{\Sigma}]{\Sigma}\) is partial
monogamous for any \(f \in \stmc{\Sigma}\) we use induction on the structure of
\(f\).
Generators, identities and symmetries are partial monogamous, as
semi-monogamicity is preserved by composition, tensor and trace.
So \(\termtohyp[f]{\Sigma}\) is partial monogamous.

Now we show that any partial monogamous cospan \(
    \cospan{m}[f]{F}[g]{n}
\)
must be in the image of \(
    \termtohyp{\Sigma} \circ \tracedtosymandfrob{\Sigma}
\).
To do this, we will now construct a cospan that is isomorphic to
\(\cospan{m}[f]{F}[g]{n}\), but from which it is possible to read off a
unique term in \(\stmc{\Sigma}\).
The components of the new cospan are as follows:
\begin{itemize}
    \item let \(L\) be the hypergraph containing vertices with degree
            \((0,0)\) that are not in the image of \(f\) or \(g\);
    \item let \(E\) be the hypergraph containing hyperedges of \(F\) and
            their source and target vertices, but disconnected;
    \item let \(V\) be the discrete hypergraph containing all the
            vertices of \(F\); and
    \item let \(S\) and \(T\) be the discrete hypergraphs containing
            the source and target vertices of hyperedges in \(F\)
            respectively, with the ordering determined by some order
            \(e_1,e_2,\cdots,e_n\) on the edges in \(F\).
\end{itemize}

These parts can be composed and a trace applied to obtain the follow
cospan:
\begin{gather}
    \trace{T}{
        \cospan{T + m}[\id + f]{V}[\id + g]{S + n}
        \,\seq\,
        \cospan{\emptyset + S + n}[\id]{L + E + n}[\id]{\emptyset + T + n}
    }
    \label{gat:cospan}
\end{gather}

This can be checked to be isomorphic to the original cospan
\(\cospan{m}[f]{F}[g]{n}\) by applying the pushouts.
From this we can read off a term in \(\stmc{\Sigma}\):
Since the first cospan is monogamous, it corresponds to a term \(
    \iltikzfig{strings/category/f-2-2}[f][white][|\vertices{T}|][m][|\vertices{S}|][n]
\) by \cref{lem:monog-discrete-cospan}.
The second cospan corresponds to \(
    \iltikzfig{strings/category/f}[g][white][|\vertices{S}][\vertices{T}]
    :=
    \bigtensor_{v \in \vertices{L}}
    \iltikzfig{strings/traced/trace-id}
    \tensor
    \bigtensor_{e \in 0 \leq i \leq n}
    \iltikzfig{graphs/isomorphism/label-e}
    \tensor
    \iltikzfig{strings/category/identity}[white][n]
\), where \(\elabel{}(e)\) is the generator in \(\generators\) that \(e\) is
labelled with.
Putting this all together yields \(
    h := \termtohypsigma[\iltikzfig{graphs/isomorphism/construction}]
\).
While there may be multiple orderings on the edges, the possible terms
are equal by sliding and the naturality of symmetry, so there is one
unique term \(
    \iltikzfig{strings/category/f}[H][white]
\) that corresponds to cospan (\ref{gat:cospan}).
It is clear by definition that \(
    \termtohypsigma[\iltikzfig{strings/category/f}[h][white]]
\) produces (\ref{gat:cospan}), which is isomorphic to the original
cospan \(\cospan{m}[f]{F}[g]{n}\), so it is in the image of
\(\termtohypsigma \circ \tracedtosymandfrob{\Sigma}\).