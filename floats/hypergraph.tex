\begin{figure}
    \centering
    \begin{minipage}{0.35\textwidth}
        \begin{equation}
            \tag{\(\mathsf{H1}\)}
            \iltikzfig{strings/structure/hypergraph/monoid-resp-lhs}[X]
            =
            \iltikzfig{strings/structure/hypergraph/monoid-resp-rhs}[X]
            \label{eq:hypergraph-monoid-resp}
        \end{equation}
    \end{minipage}
    \begin{minipage}{0.35\textwidth}
        \begin{equation}
            \tag{\(\mathsf{H2}\)}
            \iltikzfig{strings/structure/hypergraph/comonoid-resp-lhs}[X]
            =
            \iltikzfig{strings/structure/hypergraph/comonoid-resp-rhs}[X]
            \label{eq:hypergraph-comonoid-resp}
        \end{equation}
    \end{minipage}
    \begin{minipage}{0.3\textwidth}
        \begin{equation}
            \tag{\(\mathsf{H3}\)}
            \iltikzfig{strings/structure/hypergraph/unit-resp-lhs}[X]
            =
            \iltikzfig{strings/structure/hypergraph/unit-resp-rhs}[X]
            \label{eq:hypergraph-unit-resp}
        \end{equation}
    \end{minipage}
    \begin{minipage}{0.3\textwidth}
        \begin{equation}
            \tag{\(\mathsf{H4}\)}
            \iltikzfig{strings/structure/hypergraph/counit-resp-lhs}[X]
            =
            \iltikzfig{strings/structure/hypergraph/counit-resp-rhs}[X]
            \label{eq:hypergraph-counit-resp}
        \end{equation}
    \end{minipage}
    \caption{
        Equations \(\equations[\mathbf{Hyp}]\) of a
        \emph{hypergraph category}, in addition to those in
        \cref{fig:monoid-equations,fig:comonoid-equations,fig:frobenius-equations}.
    }
    \label{fig:hypergraph-category}
\end{figure}