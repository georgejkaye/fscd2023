% !TeX root = main-conf.tex
% custom macros
\input{macros/letters}
\input{macros/sets}
\input{macros/graphs}
\input{macros/category}
\input{macros/proofs}
% !TeX root = ../main-conf.tex

\subsection{Digital circuits}
\label{sec:digital-circuits}

As mentioned above, traced Cartesian categories are useful for reasoning in
settings with fixpoint operators.
One such setting is that of \emph{digital circuits} built from primitive logic
gates: in \cite{ghica2023compositional}, digital circuits are modelled as
morphisms in a STMC.
Here, the trace models a feedback loop, and the comonoid structure represents
forking wires.
The semantics of digital circuits can be expressed as a monoidal
theory~\cite[Sec. 6]{ghica2023compositional}.

\begin{definition}[Gate-level circuits]
    Let the monoidal theory of \emph{gate-level sequential circuits} be defined
    as \(
        (\generators[\mathbf{SCirc}], \equations[\mathbf{SCirc}])
    \), where \[
        \generators[\mathbf{SCirc}]
        :=
        \{
            \iltikzfig{circuits/components/gates/and},
            \iltikzfig{circuits/components/gates/or},
            \iltikzfig{circuits/components/gates/not},
            \iltikzfig{strings/structure/monoid/init}[comb]
            \iltikzfig{strings/structure/comonoid/copy}[comb],
            \iltikzfig{strings/structure/monoid/merge}[comb],
            \iltikzfig{strings/structure/comonoid/discard}[comb],
            \iltikzfig{circuits/components/values/v}[\belnaptrue],
            \iltikzfig{circuits/components/values/v}[\belnapfalse],
            \iltikzfig{circuits/components/values/v}[\belnapboth],
            \iltikzfig{circuits/components/waveforms/delay}
        \}
    \] and the equations of \(
        \equations[\mathbf{SCirc}]
    \) are listed in
    \cref{fig:monoid-equations,fig:comonoid-equations,fig:bialgebra-equations,fig:circuit-equations},
    where
    \(
        \gateinterpretation
    \) maps gates to the corresponding truth table, \(\ljoin\) is the join in a
    lattice structure on \(\{\bullet, \belnaptrue, \belnapfalse, \belnapboth\}\),
    and \(
        \iltikzfig{circuits/components/circuits/f-1-2}[F^n][comb][m][x][n]
    \) is defined inductively as \(
        \iltikzfig{circuits/instant-feedback/f0-box}
        :=
        \iltikzfig{circuits/instant-feedback/f0-definition}
    \) and \(
        \iltikzfig{circuits/instant-feedback/fkp1-box}
        :=
        \iltikzfig{circuits/instant-feedback/fkp1-definition}
    \).
\end{definition}

\begin{figure}
    \centering
    \begin{minipage}{0.28\textwidth}
        \begin{equation}
            \tag{\(\mathsf{B1}\)}
            \iltikzfig{strings/structure/bialgebra/merge-copy-lhs}
            =
            \iltikzfig{strings/structure/bialgebra/merge-copy-rhs}
            \label{eq:bialgebra-merge-copy}
        \end{equation}
    \end{minipage}
    \begin{minipage}{0.23\textwidth}
        \begin{equation}
            \tag{\(\mathsf{B2}\)}
            \iltikzfig{strings/structure/bialgebra/init-copy-lhs}
            =
            \iltikzfig{strings/structure/bialgebra/init-copy-rhs}
            \label{eq:bialgebra-init-copy}
        \end{equation}
    \end{minipage}
    \begin{minipage}{0.23\textwidth}
        \begin{equation}
            \tag{\(\mathsf{B3}\)}
            \iltikzfig{strings/structure/bialgebra/merge-discard-lhs}
            =
            \iltikzfig{strings/structure/bialgebra/merge-discard-rhs}
            \label{eq:bialgebra-merge-discard}
        \end{equation}
    \end{minipage}
    \begin{minipage}{0.2\textwidth}
        \begin{equation}
            \tag{\(\mathsf{B4}\)}
            \iltikzfig{strings/structure/bialgebra/init-discard-lhs}
            =
            \iltikzfig{strings/structure/bialgebra/init-discard-rhs}
            \label{eq:bialgebra-init-discard}
        \end{equation}
    \end{minipage}
    \caption{
        Equations \(\equations[\bialg]\) of a \emph{bialgebra}, in
        addition to those in
        \cref{fig:monoid-equations,fig:comonoid-equations}.
    }
    \label{fig:bialgebra-equations}
\end{figure}
\begin{figure}[t]
    \centering
    \iltikzfig{circuits/axioms/gate-lhs}
    \(=\)
    \iltikzfig{circuits/axioms/gate-rhs}
    \quad
    \iltikzfig{circuits/axioms/fork-lhs}[v]
    \(=\)
    \iltikzfig{circuits/axioms/fork-rhs}[v]
    \quad
    \iltikzfig{circuits/axioms/join-lhs}[v][w]
    \(=\)
    \iltikzfig{circuits/axioms/join-rhs}[v][w]
    \quad
    \iltikzfig{circuits/axioms/stub-lhs}[v]
    \(=\)
    \iltikzfig{strings/monoidal/empty}

    \vspace{1em}

    \iltikzfig{circuits/axioms/fork-gate-lhs}
    \(=\)
    \iltikzfig{circuits/axioms/fork-gate-rhs}
    \quad
    \iltikzfig{circuits/axioms/gate-stub-lhs}
    \(=\)
    \iltikzfig{circuits/axioms/gate-stub-rhs}
    \quad
    \iltikzfig{circuits/axioms/unobservable-lhs}
    \(=\)
    \iltikzfig{circuits/axioms/unobservable-rhs}
    \quad
    \iltikzfig{strings/structure/frobenius/copy-merge-lhs}
    \(=\)
    \iltikzfig{strings/structure/frobenius/copy-merge-rhs}
    \quad
    \iltikzfig{circuits/axioms/streaming-lhs-verbose}[g][v][m]
    \(=\)
    \iltikzfig{circuits/axioms/streaming-rhs}[g][v][n]

    \vspace{1em}

    \iltikzfig{circuits/axioms/bottom-delay-lhs}
    \(=\)
    \iltikzfig{circuits/axioms/bottom-delay-rhs}
    \quad
    \iltikzfig{circuits/axioms/delay-fork-lhs}
    \(=\)
    \iltikzfig{circuits/axioms/delay-fork-rhs}
    \quad
    \iltikzfig{circuits/axioms/delay-join-lhs}
    \(=\)
    \iltikzfig{circuits/axioms/delay-join-rhs}
    \quad
    \iltikzfig{circuits/instant-feedback/equation-lhs}[F][m][n][x]
    \(=\)
    \iltikzfig{circuits/instant-feedback/fixpoint-concrete}
    \quad
    \iltikzfig{circuits/axioms/delay-discard-lhs}[F][m][x]
    \(=\)
    \iltikzfig{circuits/axioms/delay-discard-rhs}[m]
    \caption{
        The equations of \(\equations[\mathbf{SCirc}]\), from the monoidal
        theory of gate-level sequential circuits.
    }
    \label{fig:circuit-equations}
\end{figure}
\begin{figure}[t]
    \centering
    \iltikzfig{circuits/productivity/productive-lhs-verbose}[F][s][v][m][n]
    \(=\)
    \iltikzfig{circuits/productivity/productive-step-9}[F][s][v][m][n]
    \caption{
        The cycle equation, which is derivable from the equations in
        \(\equations[\mathbf{SCirc}]\).
    }
    \label{fig:cycle}
\end{figure}

The generators in \(\generators[\mathbf{SCirc}]\) are, respectively:
\(\andgate\), \(\orgate\) and \(\notgate\) gates; constructs for introducing,
forking, joining and stubbing wires; \emph{values} representing a true signal,
a false signal, and both signals at once; and a delay of one unit of time.
Note that while the equations in \(\equations[\mathbf{Scirc}]\) contain those
of a commutative comonoid, they do \emph{not} explicitly contain the general
Cartesian equations: instead, these are derived from smaller equations.

Using graph rewriting, we can sketch out an \emph{operational semantics} for
sequential circuits.
For the interests of brevity, we will only consider circuits of the form \(
    \iltikzfig{circuits/productivity/mealy-form-verbose}[F][v][m][n]
\): circuits with no `non-delay-guarded feedback' in which the registers of the
circuit have been isolated from a core containing only `blue'
(\emph{combinational}) components, which models a function.
Any sequential circuit can be translated into such a form by the equational
theory.

We can `apply' such a circuit to an input as shown in the left-hand side of
\cref{fig:cycle}; the equations in
\(\equations[\mathbf{SCirc}]\) can be used to derive the right-hand side.
The four equations in the top row of \cref{fig:circuit-equations} can then be
repeatedly applied to reduce the two `new' cores down to values, representing
the output and new state of the circuit.

When the circuits are interpreted as hypergraphs and the equations as rewrites,
a computer could perform this sequence of rewrites in order to evaluate circuits
in a step-by-step manner.


\newcommand{\concat}{\mathbin{+\hspace{-3pt}+}}
\newcommand{\stmcequationslink}{
    \hyperref[fig:stmc-axioms]{\stmcequations}
}
\newcommand{\productiveequationsdelaylink}{
    \hyperref[def:productive-equations-delay]{\productiveequationsdelay}
}\newcommand{\productiveequationslink}{
    \hyperref[def:productive-equations]{\productiveequations}
}
\newcommand{\combinationalequationslink}{
    \hyperref[lem:combinational-equations]{\combinationalequations}
}
\newcommand{\reductiveequationslink}{
    \hyperref[def:reductive-equations]{\reductiveequations}
}
\newcommand{\mealyequationslink}{
    \hyperref[def:structural-equations]{\mealyequations}
}
\newcommand{\abstractionequationslink}{
    \hyperref[def:abstraction-equations]{\abstractionequations}
}
\newcommand{\bialgebraequationslink}{
    \hyperref[fig:bialgebra-axioms]{\bialgebraequations}
}
\newcommand{\localequationslink}{
    \hyperref[fig:bialgebra-axioms]{\localequations}
}
\newcommand{\unfoldingeqn}{
    \begin{equation}
        \tag{\(\mathsf{UF}\)}
        \iltikzfig{strings/traced/trace-rhs}[F][seq][m][n][x]
        =
        \iltikzfig{circuits/examples/reasoning/unfolding/unfolding-3}
        \label{eq:unfolding}
    \end{equation}
}
\newcommand{\streamingeqn}{
    \begin{equation}
        \tag{\(\mathsf{Str}\)}
        \iltikzfig{circuits/axioms/streaming-lhs}
        =
        \iltikzfig{circuits/axioms/streaming-rhs}
        \label{eq:streaming}
    \end{equation}
}
\newcommand{\genstreamingeqn}{
    \begin{equation*}
        \tag{\(\mathsf{GStr}\)}
        \iltikzfig{circuits/axioms/generalised-streaming-lhs}
        =
        \iltikzfig{circuits/axioms/generalised-streaming-rhs}
        \label{eq:generalised-streaming}
    \end{equation*}
}
\newcommand{\bisimulationequationcontent}{
    \begin{equation}
        \tag{\(\bisimulationequation\)}
        \left(
            \forall (\overline{r}, \overline{u}) \in B(
                \iltikzfig{circuits/components/circuits/f-2-2}[F][comb],
                \overline{s},
                \iltikzfig{circuits/components/circuits/f-2-2}[G][comb],
                \overline{t},
            ), \forall \overline{v} \in \valuetuple{m}.
            \quad
            \iltikzfig{circuits/full-abstraction/output}[F][r][v]
            \eqaxioms
            \iltikzfig{circuits/full-abstraction/output}[G][u][v]
        \right)
        \quad
        \Rightarrow
        \quad
        \iltikzfig{circuits/productivity/mealy-form}[F][s]
        =
        \iltikzfig{circuits/productivity/mealy-form}[G][t]
        \label{eq:bisimulation}
    \end{equation}
}
\newcommand{\instantfeedbackeqn}{
    \begin{equation}
        \tag{\(\instantfeedbackequation\)}
        \iltikzfig{circuits/instant-feedback/equation-lhs}
        =
        \iltikzfig{circuits/instant-feedback/equation-rhs}
        \label{eq:instant-feedback}
    \end{equation}
}
\newcommand{\combinationalequationslist}{
    \begin{minipage}[b]{0.24\textwidth}
        \begin{equation}
            \tag{\(\mathsf{Gate}_\interpretation\)}
            \iltikzfig{circuits/axioms/gate-lhs}
            =
            \iltikzfig{circuits/axioms/gate-rhs}
            \label{eq:gate}
        \end{equation}
    \end{minipage}%
    \begin{minipage}[b]{0.22\textwidth}
        \begin{equation}\textbf{}
            \tag{\(\mathsf{Fork}\)}
            \iltikzfig{circuits/axioms/fork-lhs}[v]
            =
            \iltikzfig{circuits/axioms/fork-rhs}[v]
            \label{eq:fork}
        \end{equation}
    \end{minipage}%
    \begin{minipage}[b]{0.22\textwidth}
        \begin{equation}
            \tag{\(\mathsf{Join}\)}
            \iltikzfig{circuits/axioms/join-lhs}[v][w]
            =
            \iltikzfig{circuits/axioms/join-rhs}[v][w]
            \label{eq:join}
        \end{equation}
    \end{minipage}%%
    \begin{minipage}[b]{0.20\textwidth}
        \begin{equation}
            \tag{\(\mathsf{Stub}\)}
            \iltikzfig{circuits/axioms/stub-lhs}[v]
            =
            \iltikzfig{strings/monoidal/empty}
            \label{eq:stub}
        \end{equation}
    \end{minipage}
}
\newcommand{\disconnecteqn}{
    \begin{equation}
        \tag{\(\mathsf{Disc}\)}
        \iltikzfig{circuits/axioms/disconnect-lhs}
        =
        \iltikzfig{circuits/axioms/disconnect-rhs}
        \label{eq:disconnect}
    \end{equation}
}
\newcommand{\mealyformequations}{
    \begin{minipage}[b]{0.3\textwidth}
        \disconnecteqn
    \end{minipage}%
    \begin{minipage}[b]{0.3\textwidth}
        \begin{equation}\textbf{}
            \tag{\(\mathsf{Unit}_l\)}
            \iltikzfig{strings/structure/monoid/unitality-l-lhs}
            =
            \iltikzfig{strings/structure/monoid/unitality-l-rhs}
            \label{eq:mealy-monoid-unitality-l}
        \end{equation}
    \end{minipage}%
    \begin{minipage}[b]{0.3\textwidth}
        \begin{equation}
            \tag{\(\mathsf{Unit}_r\)}
            \iltikzfig{strings/structure/monoid/unitality-r-lhs}
            =
            \iltikzfig{strings/structure/monoid/unitality-r-rhs}
            \label{eq:mealy-monoid-unitality-r}
        \end{equation}
    \end{minipage}
}
\newcommand{\forkgateeqn}{
    \begin{equation}
        \tag{\(\mathsf{GFork}\)}
        \iltikzfig{circuits/axioms/fork-gate-lhs}
        =
        \iltikzfig{circuits/axioms/fork-gate-rhs}
        \label{eq:fork-gate}
    \end{equation}
}
\newcommand{\stubgateeqn}{
    \begin{equation}
        \tag{\(\mathsf{GStub}\)}
        \iltikzfig{circuits/axioms/stub-gate-lhs}
        =
        \iltikzfig{circuits/axioms/stub-gate-rhs}
        \label{eq:stub-gate}
    \end{equation}
}
\newcommand{\forkdelayeqn}{
    \begin{equation}
        \tag{\(\mathsf{DFork}\)}
        \iltikzfig{circuits/axioms/fork-delay-lhs}
        =
        \iltikzfig{circuits/axioms/fork-delay-rhs}
        \label{eq:fork-delay}
    \end{equation}
}
\newcommand{\joindelayeqn}{
    \begin{equation}
        \tag{\(\mathsf{DJoin}\)}
        \iltikzfig{circuits/axioms/join-delay-lhs}
        =
        \iltikzfig{circuits/axioms/join-delay-rhs}
        \label{eq:join-delay}
    \end{equation}
}
\newcommand{\stubdelayeqn}{
    \begin{equation}
        \tag{\(\mathsf{DStub}\)}
        \iltikzfig{circuits/axioms/unobservable-lhs}
        =
        \iltikzfig{circuits/axioms/unobservable-rhs}
        \label{eq:stub-delay}
    \end{equation}
}
\newcommand{\cycleeqn}{
    \begin{equation}
        \tag{\(\mathsf{Cycle}\)}
        \iltikzfig{circuits/productivity/productive-lhs}[F][s][v]
        =
        \iltikzfig{circuits/productivity/productive-step-9}
        \label{eq:cycle}
    \end{equation}
}
\newcommand{\cartesiancopyeqn}{
    \begin{equation}
        \tag{\(\mathsf{Copy}\)}
        \iltikzfig{strings/structure/cartesian/naturality-copy-lhs}[F][seq][m][n]
        =
        \iltikzfig{strings/structure/cartesian/naturality-copy-rhs}[F][seq][m][n]
        \label{eq:cartesian-naturality-copy}
    \end{equation}
}
\newcommand{\joinforkinverseeqn}{
    \begin{equation}
        \tag{\(\mathsf{JF}\)}
        \iltikzfig{circuits/axioms/join-fork-inverses-lhs}
        =
        \iltikzfig{circuits/axioms/join-fork-inverses-rhs}
        \label{eq:join-fork-inverse}
    \end{equation}
}
\newcommand{\forkjoininverseeqn}{
    \begin{equation}
        \tag{\(\mathsf{FJ}\)}
        \iltikzfig{circuits/axioms/fork-join-inverse-lhs}
        =
        \iltikzfig{circuits/axioms/fork-join-inverse-rhs}
        \label{eq:fork-join-inverse}
    \end{equation}
}
\newcommand{\delaydiscardeqn}{
    \begin{equation}
        \tag{\(\mathsf{UDelay}\)}
        \iltikzfig{circuits/axioms/delay-discard-lhs}
        =
        \iltikzfig{circuits/axioms/delay-discard-rhs}
        \label{eq:delay-discard}
    \end{equation}
}
\newcommand{\mealyeqn}{
    \begin{equation}
        \tag{\(\mathsf{Mealy}\)}
        \iltikzfig{circuits/productivity/trace-delay}[F][v][m][n][x][y][z]
        =
        \iltikzfig{circuits/productivity/mealy-form/mealy-form-4}
        \label{eq:mealy}
    \end{equation}
}

% !TeX root = ../main-conf.tex

\subsection{Digital circuits}
\label{sec:digital-circuits}

As mentioned above, traced Cartesian categories are useful for reasoning in
settings with fixpoint operators.
One such setting is that of \emph{digital circuits} built from primitive logic
gates: in \cite{ghica2023compositional}, digital circuits are modelled as
morphisms in a STMC.
Here, the trace models a feedback loop, and the comonoid structure represents
forking wires.
The semantics of digital circuits can be expressed as a monoidal
theory~\cite[Sec. 6]{ghica2023compositional}.

\begin{definition}[Gate-level circuits]
    Let the monoidal theory of \emph{gate-level sequential circuits} be defined
    as \(
        (\generators[\mathbf{SCirc}], \equations[\mathbf{SCirc}])
    \), where \[
        \generators[\mathbf{SCirc}]
        :=
        \{
            \iltikzfig{circuits/components/gates/and},
            \iltikzfig{circuits/components/gates/or},
            \iltikzfig{circuits/components/gates/not},
            \iltikzfig{strings/structure/monoid/init}[comb]
            \iltikzfig{strings/structure/comonoid/copy}[comb],
            \iltikzfig{strings/structure/monoid/merge}[comb],
            \iltikzfig{strings/structure/comonoid/discard}[comb],
            \iltikzfig{circuits/components/values/v}[\belnaptrue],
            \iltikzfig{circuits/components/values/v}[\belnapfalse],
            \iltikzfig{circuits/components/values/v}[\belnapboth],
            \iltikzfig{circuits/components/waveforms/delay}
        \}
    \] and the equations of \(
        \equations[\mathbf{SCirc}]
    \) are listed in
    \cref{fig:monoid-equations,fig:comonoid-equations,fig:bialgebra-equations,fig:circuit-equations},
    where
    \(
        \gateinterpretation
    \) maps gates to the corresponding truth table, \(\ljoin\) is the join in a
    lattice structure on \(\{\bullet, \belnaptrue, \belnapfalse, \belnapboth\}\),
    and \(
        \iltikzfig{circuits/components/circuits/f-1-2}[F^n][comb][m][x][n]
    \) is defined inductively as \(
        \iltikzfig{circuits/instant-feedback/f0-box}
        :=
        \iltikzfig{circuits/instant-feedback/f0-definition}
    \) and \(
        \iltikzfig{circuits/instant-feedback/fkp1-box}
        :=
        \iltikzfig{circuits/instant-feedback/fkp1-definition}
    \).
\end{definition}

\begin{figure}
    \centering
    \begin{minipage}{0.28\textwidth}
        \begin{equation}
            \tag{\(\mathsf{B1}\)}
            \iltikzfig{strings/structure/bialgebra/merge-copy-lhs}
            =
            \iltikzfig{strings/structure/bialgebra/merge-copy-rhs}
            \label{eq:bialgebra-merge-copy}
        \end{equation}
    \end{minipage}
    \begin{minipage}{0.23\textwidth}
        \begin{equation}
            \tag{\(\mathsf{B2}\)}
            \iltikzfig{strings/structure/bialgebra/init-copy-lhs}
            =
            \iltikzfig{strings/structure/bialgebra/init-copy-rhs}
            \label{eq:bialgebra-init-copy}
        \end{equation}
    \end{minipage}
    \begin{minipage}{0.23\textwidth}
        \begin{equation}
            \tag{\(\mathsf{B3}\)}
            \iltikzfig{strings/structure/bialgebra/merge-discard-lhs}
            =
            \iltikzfig{strings/structure/bialgebra/merge-discard-rhs}
            \label{eq:bialgebra-merge-discard}
        \end{equation}
    \end{minipage}
    \begin{minipage}{0.2\textwidth}
        \begin{equation}
            \tag{\(\mathsf{B4}\)}
            \iltikzfig{strings/structure/bialgebra/init-discard-lhs}
            =
            \iltikzfig{strings/structure/bialgebra/init-discard-rhs}
            \label{eq:bialgebra-init-discard}
        \end{equation}
    \end{minipage}
    \caption{
        Equations \(\equations[\bialg]\) of a \emph{bialgebra}, in
        addition to those in
        \cref{fig:monoid-equations,fig:comonoid-equations}.
    }
    \label{fig:bialgebra-equations}
\end{figure}
\begin{figure}[t]
    \centering
    \iltikzfig{circuits/axioms/gate-lhs}
    \(=\)
    \iltikzfig{circuits/axioms/gate-rhs}
    \quad
    \iltikzfig{circuits/axioms/fork-lhs}[v]
    \(=\)
    \iltikzfig{circuits/axioms/fork-rhs}[v]
    \quad
    \iltikzfig{circuits/axioms/join-lhs}[v][w]
    \(=\)
    \iltikzfig{circuits/axioms/join-rhs}[v][w]
    \quad
    \iltikzfig{circuits/axioms/stub-lhs}[v]
    \(=\)
    \iltikzfig{strings/monoidal/empty}

    \vspace{1em}

    \iltikzfig{circuits/axioms/fork-gate-lhs}
    \(=\)
    \iltikzfig{circuits/axioms/fork-gate-rhs}
    \quad
    \iltikzfig{circuits/axioms/gate-stub-lhs}
    \(=\)
    \iltikzfig{circuits/axioms/gate-stub-rhs}
    \quad
    \iltikzfig{circuits/axioms/unobservable-lhs}
    \(=\)
    \iltikzfig{circuits/axioms/unobservable-rhs}
    \quad
    \iltikzfig{strings/structure/frobenius/copy-merge-lhs}
    \(=\)
    \iltikzfig{strings/structure/frobenius/copy-merge-rhs}
    \quad
    \iltikzfig{circuits/axioms/streaming-lhs-verbose}[g][v][m]
    \(=\)
    \iltikzfig{circuits/axioms/streaming-rhs}[g][v][n]

    \vspace{1em}

    \iltikzfig{circuits/axioms/bottom-delay-lhs}
    \(=\)
    \iltikzfig{circuits/axioms/bottom-delay-rhs}
    \quad
    \iltikzfig{circuits/axioms/delay-fork-lhs}
    \(=\)
    \iltikzfig{circuits/axioms/delay-fork-rhs}
    \quad
    \iltikzfig{circuits/axioms/delay-join-lhs}
    \(=\)
    \iltikzfig{circuits/axioms/delay-join-rhs}
    \quad
    \iltikzfig{circuits/instant-feedback/equation-lhs}[F][m][n][x]
    \(=\)
    \iltikzfig{circuits/instant-feedback/fixpoint-concrete}
    \quad
    \iltikzfig{circuits/axioms/delay-discard-lhs}[F][m][x]
    \(=\)
    \iltikzfig{circuits/axioms/delay-discard-rhs}[m]
    \caption{
        The equations of \(\equations[\mathbf{SCirc}]\), from the monoidal
        theory of gate-level sequential circuits.
    }
    \label{fig:circuit-equations}
\end{figure}
\begin{figure}[t]
    \centering
    \iltikzfig{circuits/productivity/productive-lhs-verbose}[F][s][v][m][n]
    \(=\)
    \iltikzfig{circuits/productivity/productive-step-9}[F][s][v][m][n]
    \caption{
        The cycle equation, which is derivable from the equations in
        \(\equations[\mathbf{SCirc}]\).
    }
    \label{fig:cycle}
\end{figure}

The generators in \(\generators[\mathbf{SCirc}]\) are, respectively:
\(\andgate\), \(\orgate\) and \(\notgate\) gates; constructs for introducing,
forking, joining and stubbing wires; \emph{values} representing a true signal,
a false signal, and both signals at once; and a delay of one unit of time.
Note that while the equations in \(\equations[\mathbf{Scirc}]\) contain those
of a commutative comonoid, they do \emph{not} explicitly contain the general
Cartesian equations: instead, these are derived from smaller equations.

Using graph rewriting, we can sketch out an \emph{operational semantics} for
sequential circuits.
For the interests of brevity, we will only consider circuits of the form \(
    \iltikzfig{circuits/productivity/mealy-form-verbose}[F][v][m][n]
\): circuits with no `non-delay-guarded feedback' in which the registers of the
circuit have been isolated from a core containing only `blue'
(\emph{combinational}) components, which models a function.
Any sequential circuit can be translated into such a form by the equational
theory.

We can `apply' such a circuit to an input as shown in the left-hand side of
\cref{fig:cycle}; the equations in
\(\equations[\mathbf{SCirc}]\) can be used to derive the right-hand side.
The four equations in the top row of \cref{fig:circuit-equations} can then be
repeatedly applied to reduce the two `new' cores down to values, representing
the output and new state of the circuit.

When the circuits are interpreted as hypergraphs and the equations as rewrites,
a computer could perform this sequence of rewrites in order to evaluate circuits
in a step-by-step manner.

% !TeX root = ../main-conf.tex

\subsection{Digital circuits}
\label{sec:digital-circuits}

As mentioned above, traced Cartesian categories are useful for reasoning in
settings with fixpoint operators.
One such setting is that of \emph{digital circuits} built from primitive logic
gates: in \cite{ghica2023compositional}, digital circuits are modelled as
morphisms in a STMC.
Here, the trace models a feedback loop, and the comonoid structure represents
forking wires.
The semantics of digital circuits can be expressed as a monoidal
theory~\cite[Sec. 6]{ghica2023compositional}.

\begin{definition}[Gate-level circuits]
    Let the monoidal theory of \emph{gate-level sequential circuits} be defined
    as \(
        (\generators[\mathbf{SCirc}], \equations[\mathbf{SCirc}])
    \), where \[
        \generators[\mathbf{SCirc}]
        :=
        \{
            \iltikzfig{circuits/components/gates/and},
            \iltikzfig{circuits/components/gates/or},
            \iltikzfig{circuits/components/gates/not},
            \iltikzfig{strings/structure/monoid/init}[comb]
            \iltikzfig{strings/structure/comonoid/copy}[comb],
            \iltikzfig{strings/structure/monoid/merge}[comb],
            \iltikzfig{strings/structure/comonoid/discard}[comb],
            \iltikzfig{circuits/components/values/v}[\belnaptrue],
            \iltikzfig{circuits/components/values/v}[\belnapfalse],
            \iltikzfig{circuits/components/values/v}[\belnapboth],
            \iltikzfig{circuits/components/waveforms/delay}
        \}
    \] and the equations of \(
        \equations[\mathbf{SCirc}]
    \) are listed in
    \cref{fig:monoid-equations,fig:comonoid-equations,fig:bialgebra-equations,fig:circuit-equations},
    where
    \(
        \gateinterpretation
    \) maps gates to the corresponding truth table, \(\ljoin\) is the join in a
    lattice structure on \(\{\bullet, \belnaptrue, \belnapfalse, \belnapboth\}\),
    and \(
        \iltikzfig{circuits/components/circuits/f-1-2}[F^n][comb][m][x][n]
    \) is defined inductively as \(
        \iltikzfig{circuits/instant-feedback/f0-box}
        :=
        \iltikzfig{circuits/instant-feedback/f0-definition}
    \) and \(
        \iltikzfig{circuits/instant-feedback/fkp1-box}
        :=
        \iltikzfig{circuits/instant-feedback/fkp1-definition}
    \).
\end{definition}

\begin{figure}
    \centering
    \begin{minipage}{0.28\textwidth}
        \begin{equation}
            \tag{\(\mathsf{B1}\)}
            \iltikzfig{strings/structure/bialgebra/merge-copy-lhs}
            =
            \iltikzfig{strings/structure/bialgebra/merge-copy-rhs}
            \label{eq:bialgebra-merge-copy}
        \end{equation}
    \end{minipage}
    \begin{minipage}{0.23\textwidth}
        \begin{equation}
            \tag{\(\mathsf{B2}\)}
            \iltikzfig{strings/structure/bialgebra/init-copy-lhs}
            =
            \iltikzfig{strings/structure/bialgebra/init-copy-rhs}
            \label{eq:bialgebra-init-copy}
        \end{equation}
    \end{minipage}
    \begin{minipage}{0.23\textwidth}
        \begin{equation}
            \tag{\(\mathsf{B3}\)}
            \iltikzfig{strings/structure/bialgebra/merge-discard-lhs}
            =
            \iltikzfig{strings/structure/bialgebra/merge-discard-rhs}
            \label{eq:bialgebra-merge-discard}
        \end{equation}
    \end{minipage}
    \begin{minipage}{0.2\textwidth}
        \begin{equation}
            \tag{\(\mathsf{B4}\)}
            \iltikzfig{strings/structure/bialgebra/init-discard-lhs}
            =
            \iltikzfig{strings/structure/bialgebra/init-discard-rhs}
            \label{eq:bialgebra-init-discard}
        \end{equation}
    \end{minipage}
    \caption{
        Equations \(\equations[\bialg]\) of a \emph{bialgebra}, in
        addition to those in
        \cref{fig:monoid-equations,fig:comonoid-equations}.
    }
    \label{fig:bialgebra-equations}
\end{figure}
\begin{figure}[t]
    \centering
    \iltikzfig{circuits/axioms/gate-lhs}
    \(=\)
    \iltikzfig{circuits/axioms/gate-rhs}
    \quad
    \iltikzfig{circuits/axioms/fork-lhs}[v]
    \(=\)
    \iltikzfig{circuits/axioms/fork-rhs}[v]
    \quad
    \iltikzfig{circuits/axioms/join-lhs}[v][w]
    \(=\)
    \iltikzfig{circuits/axioms/join-rhs}[v][w]
    \quad
    \iltikzfig{circuits/axioms/stub-lhs}[v]
    \(=\)
    \iltikzfig{strings/monoidal/empty}

    \vspace{1em}

    \iltikzfig{circuits/axioms/fork-gate-lhs}
    \(=\)
    \iltikzfig{circuits/axioms/fork-gate-rhs}
    \quad
    \iltikzfig{circuits/axioms/gate-stub-lhs}
    \(=\)
    \iltikzfig{circuits/axioms/gate-stub-rhs}
    \quad
    \iltikzfig{circuits/axioms/unobservable-lhs}
    \(=\)
    \iltikzfig{circuits/axioms/unobservable-rhs}
    \quad
    \iltikzfig{strings/structure/frobenius/copy-merge-lhs}
    \(=\)
    \iltikzfig{strings/structure/frobenius/copy-merge-rhs}
    \quad
    \iltikzfig{circuits/axioms/streaming-lhs-verbose}[g][v][m]
    \(=\)
    \iltikzfig{circuits/axioms/streaming-rhs}[g][v][n]

    \vspace{1em}

    \iltikzfig{circuits/axioms/bottom-delay-lhs}
    \(=\)
    \iltikzfig{circuits/axioms/bottom-delay-rhs}
    \quad
    \iltikzfig{circuits/axioms/delay-fork-lhs}
    \(=\)
    \iltikzfig{circuits/axioms/delay-fork-rhs}
    \quad
    \iltikzfig{circuits/axioms/delay-join-lhs}
    \(=\)
    \iltikzfig{circuits/axioms/delay-join-rhs}
    \quad
    \iltikzfig{circuits/instant-feedback/equation-lhs}[F][m][n][x]
    \(=\)
    \iltikzfig{circuits/instant-feedback/fixpoint-concrete}
    \quad
    \iltikzfig{circuits/axioms/delay-discard-lhs}[F][m][x]
    \(=\)
    \iltikzfig{circuits/axioms/delay-discard-rhs}[m]
    \caption{
        The equations of \(\equations[\mathbf{SCirc}]\), from the monoidal
        theory of gate-level sequential circuits.
    }
    \label{fig:circuit-equations}
\end{figure}
\begin{figure}[t]
    \centering
    \iltikzfig{circuits/productivity/productive-lhs-verbose}[F][s][v][m][n]
    \(=\)
    \iltikzfig{circuits/productivity/productive-step-9}[F][s][v][m][n]
    \caption{
        The cycle equation, which is derivable from the equations in
        \(\equations[\mathbf{SCirc}]\).
    }
    \label{fig:cycle}
\end{figure}

The generators in \(\generators[\mathbf{SCirc}]\) are, respectively:
\(\andgate\), \(\orgate\) and \(\notgate\) gates; constructs for introducing,
forking, joining and stubbing wires; \emph{values} representing a true signal,
a false signal, and both signals at once; and a delay of one unit of time.
Note that while the equations in \(\equations[\mathbf{Scirc}]\) contain those
of a commutative comonoid, they do \emph{not} explicitly contain the general
Cartesian equations: instead, these are derived from smaller equations.

Using graph rewriting, we can sketch out an \emph{operational semantics} for
sequential circuits.
For the interests of brevity, we will only consider circuits of the form \(
    \iltikzfig{circuits/productivity/mealy-form-verbose}[F][v][m][n]
\): circuits with no `non-delay-guarded feedback' in which the registers of the
circuit have been isolated from a core containing only `blue'
(\emph{combinational}) components, which models a function.
Any sequential circuit can be translated into such a form by the equational
theory.

We can `apply' such a circuit to an input as shown in the left-hand side of
\cref{fig:cycle}; the equations in
\(\equations[\mathbf{SCirc}]\) can be used to derive the right-hand side.
The four equations in the top row of \cref{fig:circuit-equations} can then be
repeatedly applied to reduce the two `new' cores down to values, representing
the output and new state of the circuit.

When the circuits are interpreted as hypergraphs and the equations as rewrites,
a computer could perform this sequence of rewrites in order to evaluate circuits
in a step-by-step manner.


\title{Graph rewriting for digital circuits}

\begin{document}

    \maketitle

    \begin{abstract}
        Sequential circuits are circuits with feedback and delay constructed
        from primitive logic gates.
        There has been recent work into developing an \emph{equational theory}
        for sequential circuits using \emph{string diagrams}, giving an
        intuitive graphical framework for comparing and evaluating circuits.
        However, while they are suitable for pen-and-paper reasoning, string
        diagrams are not an efficient data structure for reasoning
        computationally.
        In this paper we extend previous work on rewriting string diagrams using
        \emph{hypergraphs} to solve this problem, by considering the case when
        the underlying category has a traced comonoid structure, both features
        of the category of sequential circuits.
        We show that certain subclasses of hypergraphs are sound and complete
        for traced comonoid categories: that is to say, every term in such a
        category has a unique corresponding hypergraph up to isomorphism, and
        from every hypergraph with the desired properties, a unique term in the
        category can be retrieved up to the axioms of traced comonoid
        categories.
        We also show how the framework of double pushout rewriting (DPO) can be
        adapted for traced comonoid categories by characterising the valid
        pushout complements for rewriting in our setting.
        Finally, we conclude by demonstrating how an operational semantics for
        sequential circuits can be implemented using the previously developed
        equational theory and graph rewriting.
    \end{abstract}

    % !TeX root = ../main-conf.tex
\section{Introduction}

\begin{itemize}
    \item We want to reason/rewrite in traced categories with comonoid structure
            (e.g. dataflow categories are traced comonoid + cartesian,
            digital circuits is traced + comonoid + circuit axioms)
    \item Previous work on rewriting for traced monoidal categories with
            \emph{string diagrams} \cite{kissinger2012pictures,dixon2013opengraphs}
            but for these you have to work with rewriting modulo wire homeomorphisms
            which is a pain
    \item More recent work on rewriting with \emph{Frobenius} \cite{bonchi2022string}
            structure with hypergraphs, and subsequently for \emph{symmetric monoidal}
            by restricting the type of graphs and rewrites \cite{bonchi2022stringa}
    \item Comonoid structure is `half' Frobenius, (co)monoid structure has been studied
        too \cite{fritz2022free,milosavljevic2022string}
    \item The trace sits in the middle of compact closed/frobenius and symmetric
        monoidal (notion of causal vs relational)
    \item The trace can be constructed using the Frobenius structure (and this
        is fine in the land of hypergraphs because e.g. Cartesian equations do
        not hold by default, and since we do the trace all in one go we don't
        end up with degeneracies)
    \item \textbf{Contribution 1:} show that a subclass of cospans of hypergraphs
        called \emph{partial monogamous} are in correspondence with
        traced terms, and \emph{partial left-monogamous} cospans are in
        correspondence with traced comonoid terms.
    \item \textbf{Design choice:} what precisely should a partial monogamous
        hypergraph be? Sticking point e.g. how to represent trace of the
        identity (closed loop). In the frobenius realm this is a disconnected
        vertex not in the interfaces: is this okay for traced? When adding
        comonoid structure does this still work with the trace of the fork?
    \item When rewriting with Frobenius, every pushout complement is valid.
        When rewriting with symmetric monoidal, exactly one pushout complement
        is valid.
        In the traced case, \emph{some} are valid.
    \item \textbf{Contribution 2:} show that \emph{traced boundary complements}
        correspond to valid traced rewrites
    \item \textbf{Example}: there is a pushout complement below that is valid
        in Frobenius because it uses the monoid structure, but not in a
        traced or even traced comonoid setting
    \begin{itemize}
        \item Imagine we have a rule \(\rrule{
            \iltikzfig{graphs/dpo/non-valid/rule-lhs}
        }{
            \iltikzfig{graphs/dpo/non-valid/rule-rhs}
        }\) and a term \(
            \iltikzfig{graphs/dpo/non-valid/term}
        \).
        When performing graph rewriting modulo \emph{Frobenius}, the following
        DPO rewrite is valid:
    \end{itemize}
    \begin{center}
        \includestandalone{figures/graphs/dpo/non-valid/rewrite}
    \end{center}
    This corresponds to the term rewrite \(
        \iltikzfig{graphs/dpo/non-valid/term}
        =
        \iltikzfig{graphs/dpo/non-valid/term-rewriting}
        =
        \iltikzfig{graphs/dpo/non-valid/term-rewritten}
    \), which holds in a Frobenius setting, but not a setting without a
    commutative monoid structure.
    On the other hand, the rewriting system for symmetric monoidal categories~\cite{bonchi2022stringa}
    is too restricting, and prevents matchings such as \(
        \iltikzfig{graphs/dpo/matchings/trace-rule}
    \) in \(
        \iltikzfig{graphs/dpo/matchings/trace-match}
    \).
    We need something in the middle.
    \item \textbf{Design choices:} what precisely is a traced boundary
        complement? In the end it turned out to be very similar to the symmetric
        monoidal case but everything was partial monogamous instead of vanilla
        monogamous. What happens when we make the matching non-mono (it is mono
        for symmetric monoidal)
\end{itemize}
    % !TeX root = ../main-conf.tex

\subsection{Digital circuits}
\label{sec:digital-circuits}

As mentioned above, traced Cartesian categories are useful for reasoning in
settings with fixpoint operators.
One such setting is that of \emph{digital circuits} built from primitive logic
gates: in \cite{ghica2023compositional}, digital circuits are modelled as
morphisms in a STMC.
Here, the trace models a feedback loop, and the comonoid structure represents
forking wires.
The semantics of digital circuits can be expressed as a monoidal
theory~\cite[Sec. 6]{ghica2023compositional}.

\begin{definition}[Gate-level circuits]
    Let the monoidal theory of \emph{gate-level sequential circuits} be defined
    as \(
        (\generators[\mathbf{SCirc}], \equations[\mathbf{SCirc}])
    \), where \[
        \generators[\mathbf{SCirc}]
        :=
        \{
            \iltikzfig{circuits/components/gates/and},
            \iltikzfig{circuits/components/gates/or},
            \iltikzfig{circuits/components/gates/not},
            \iltikzfig{strings/structure/monoid/init}[comb]
            \iltikzfig{strings/structure/comonoid/copy}[comb],
            \iltikzfig{strings/structure/monoid/merge}[comb],
            \iltikzfig{strings/structure/comonoid/discard}[comb],
            \iltikzfig{circuits/components/values/v}[\belnaptrue],
            \iltikzfig{circuits/components/values/v}[\belnapfalse],
            \iltikzfig{circuits/components/values/v}[\belnapboth],
            \iltikzfig{circuits/components/waveforms/delay}
        \}
    \] and the equations of \(
        \equations[\mathbf{SCirc}]
    \) are listed in
    \cref{fig:monoid-equations,fig:comonoid-equations,fig:bialgebra-equations,fig:circuit-equations},
    where
    \(
        \gateinterpretation
    \) maps gates to the corresponding truth table, \(\ljoin\) is the join in a
    lattice structure on \(\{\bullet, \belnaptrue, \belnapfalse, \belnapboth\}\),
    and \(
        \iltikzfig{circuits/components/circuits/f-1-2}[F^n][comb][m][x][n]
    \) is defined inductively as \(
        \iltikzfig{circuits/instant-feedback/f0-box}
        :=
        \iltikzfig{circuits/instant-feedback/f0-definition}
    \) and \(
        \iltikzfig{circuits/instant-feedback/fkp1-box}
        :=
        \iltikzfig{circuits/instant-feedback/fkp1-definition}
    \).
\end{definition}

\begin{figure}
    \centering
    \begin{minipage}{0.28\textwidth}
        \begin{equation}
            \tag{\(\mathsf{B1}\)}
            \iltikzfig{strings/structure/bialgebra/merge-copy-lhs}
            =
            \iltikzfig{strings/structure/bialgebra/merge-copy-rhs}
            \label{eq:bialgebra-merge-copy}
        \end{equation}
    \end{minipage}
    \begin{minipage}{0.23\textwidth}
        \begin{equation}
            \tag{\(\mathsf{B2}\)}
            \iltikzfig{strings/structure/bialgebra/init-copy-lhs}
            =
            \iltikzfig{strings/structure/bialgebra/init-copy-rhs}
            \label{eq:bialgebra-init-copy}
        \end{equation}
    \end{minipage}
    \begin{minipage}{0.23\textwidth}
        \begin{equation}
            \tag{\(\mathsf{B3}\)}
            \iltikzfig{strings/structure/bialgebra/merge-discard-lhs}
            =
            \iltikzfig{strings/structure/bialgebra/merge-discard-rhs}
            \label{eq:bialgebra-merge-discard}
        \end{equation}
    \end{minipage}
    \begin{minipage}{0.2\textwidth}
        \begin{equation}
            \tag{\(\mathsf{B4}\)}
            \iltikzfig{strings/structure/bialgebra/init-discard-lhs}
            =
            \iltikzfig{strings/structure/bialgebra/init-discard-rhs}
            \label{eq:bialgebra-init-discard}
        \end{equation}
    \end{minipage}
    \caption{
        Equations \(\equations[\bialg]\) of a \emph{bialgebra}, in
        addition to those in
        \cref{fig:monoid-equations,fig:comonoid-equations}.
    }
    \label{fig:bialgebra-equations}
\end{figure}
\begin{figure}[t]
    \centering
    \iltikzfig{circuits/axioms/gate-lhs}
    \(=\)
    \iltikzfig{circuits/axioms/gate-rhs}
    \quad
    \iltikzfig{circuits/axioms/fork-lhs}[v]
    \(=\)
    \iltikzfig{circuits/axioms/fork-rhs}[v]
    \quad
    \iltikzfig{circuits/axioms/join-lhs}[v][w]
    \(=\)
    \iltikzfig{circuits/axioms/join-rhs}[v][w]
    \quad
    \iltikzfig{circuits/axioms/stub-lhs}[v]
    \(=\)
    \iltikzfig{strings/monoidal/empty}

    \vspace{1em}

    \iltikzfig{circuits/axioms/fork-gate-lhs}
    \(=\)
    \iltikzfig{circuits/axioms/fork-gate-rhs}
    \quad
    \iltikzfig{circuits/axioms/gate-stub-lhs}
    \(=\)
    \iltikzfig{circuits/axioms/gate-stub-rhs}
    \quad
    \iltikzfig{circuits/axioms/unobservable-lhs}
    \(=\)
    \iltikzfig{circuits/axioms/unobservable-rhs}
    \quad
    \iltikzfig{strings/structure/frobenius/copy-merge-lhs}
    \(=\)
    \iltikzfig{strings/structure/frobenius/copy-merge-rhs}
    \quad
    \iltikzfig{circuits/axioms/streaming-lhs-verbose}[g][v][m]
    \(=\)
    \iltikzfig{circuits/axioms/streaming-rhs}[g][v][n]

    \vspace{1em}

    \iltikzfig{circuits/axioms/bottom-delay-lhs}
    \(=\)
    \iltikzfig{circuits/axioms/bottom-delay-rhs}
    \quad
    \iltikzfig{circuits/axioms/delay-fork-lhs}
    \(=\)
    \iltikzfig{circuits/axioms/delay-fork-rhs}
    \quad
    \iltikzfig{circuits/axioms/delay-join-lhs}
    \(=\)
    \iltikzfig{circuits/axioms/delay-join-rhs}
    \quad
    \iltikzfig{circuits/instant-feedback/equation-lhs}[F][m][n][x]
    \(=\)
    \iltikzfig{circuits/instant-feedback/fixpoint-concrete}
    \quad
    \iltikzfig{circuits/axioms/delay-discard-lhs}[F][m][x]
    \(=\)
    \iltikzfig{circuits/axioms/delay-discard-rhs}[m]
    \caption{
        The equations of \(\equations[\mathbf{SCirc}]\), from the monoidal
        theory of gate-level sequential circuits.
    }
    \label{fig:circuit-equations}
\end{figure}
\begin{figure}[t]
    \centering
    \iltikzfig{circuits/productivity/productive-lhs-verbose}[F][s][v][m][n]
    \(=\)
    \iltikzfig{circuits/productivity/productive-step-9}[F][s][v][m][n]
    \caption{
        The cycle equation, which is derivable from the equations in
        \(\equations[\mathbf{SCirc}]\).
    }
    \label{fig:cycle}
\end{figure}

The generators in \(\generators[\mathbf{SCirc}]\) are, respectively:
\(\andgate\), \(\orgate\) and \(\notgate\) gates; constructs for introducing,
forking, joining and stubbing wires; \emph{values} representing a true signal,
a false signal, and both signals at once; and a delay of one unit of time.
Note that while the equations in \(\equations[\mathbf{Scirc}]\) contain those
of a commutative comonoid, they do \emph{not} explicitly contain the general
Cartesian equations: instead, these are derived from smaller equations.

Using graph rewriting, we can sketch out an \emph{operational semantics} for
sequential circuits.
For the interests of brevity, we will only consider circuits of the form \(
    \iltikzfig{circuits/productivity/mealy-form-verbose}[F][v][m][n]
\): circuits with no `non-delay-guarded feedback' in which the registers of the
circuit have been isolated from a core containing only `blue'
(\emph{combinational}) components, which models a function.
Any sequential circuit can be translated into such a form by the equational
theory.

We can `apply' such a circuit to an input as shown in the left-hand side of
\cref{fig:cycle}; the equations in
\(\equations[\mathbf{SCirc}]\) can be used to derive the right-hand side.
The four equations in the top row of \cref{fig:circuit-equations} can then be
repeatedly applied to reduce the two `new' cores down to values, representing
the output and new state of the circuit.

When the circuits are interpreted as hypergraphs and the equations as rewrites,
a computer could perform this sequence of rewrites in order to evaluate circuits
in a step-by-step manner.

    % !TeX root = ../main-conf.tex
\section{Hypergraphs}

Hypergraphs are formally defined as objects in a functor category.

\begin{definition}[Hypergraph]
    Let \(\mathbf{X}\) be the category containing objects \((k, l)\) for
    \(k, l \in \nat\) and one additional object \(\star\).
    For each \((k, l)\) there are \(k + l\) morphisms from \((k, l) \to \star\).
    Let \(\hyp\) be the functor category \([\mathbf{X},\set]\).
\end{definition}

An object in \(\hyp\) maps \(\star\) to a set of vertices, and each \((k,l)\) to
a set of hyperedges with \(k\) sources and \(l\) targets,
Given a hypergraph \(H \in \hyp\), we write \(\vertices{H}\) for its set of
vertices and \(\edges{H}{k}{l}\) for the set of edges with \(k\) sources and
\(l\) targets.

\begin{definition}[Hypergraph signature]
    For a given monoidal signature \(\Sigma\), its corresponding
    \emph{hypergraph signature} \(\hypsignature{\Sigma}\) is the hypergraph with
    edges \(\phi \in \edges{\hypsignature{\Sigma}}{k}{l}\) for each
    \((\phi, k, l) \in \sigma\), and a single vertex acting as the source and
    target of all edges.
\end{definition}

\begin{definition}[Labelled hypergraph]
    For a monoidal signature \(\Sigma\), let the category \(\hypsigma\) be
    defined as the slice category \(\hyp \setminus \hypsignature{\Sigma}\).
\end{definition}

While (labelled) hypergraphs may have dangling vertices, they do not have \emph{interfaces}.
These can be provided using \emph{cospans}.

\begin{definition}[Categories of cospans~\cite{bonchi2021string}]\label{def:cospans}
    For a finitely cocomplete category \(\mcc\), a \emph{cospan} from \(X \to Y\) is a pair of arrows \(X \to A \leftarrow Y\).
    A \emph{cospan morphism} \((X \to A \leftarrow Y) \to (X \to B \leftarrow Y)\) is a morphism in \(C \to D \in \mcc\) such that the following diagram commutes:

    \begin{center}
        \includestandalone{figures/graphs/cospans/morphism}
    \end{center}

    \noindent
    Two cospans \(\cospan{X}{A}{Y}\) and \(\cospan{X}{B}{Y}\) are \emph{isomorphic} if there exists a morphism of cospans as above where \(\alpha\) is an isomorphism.
    Composition is by pushout:

    \begin{center}
        \includestandalone{figures/graphs/cospans/composition}
    \end{center}

    \noindent
    The identity is \(X \xrightarrow{\id[X]} X \xleftarrow{\id[X]} X\).
    The category of cospans over \(\mcc\), denoted \(\csp{\mcc}\) has as objects the objects of \(\mcc\) and as morphisms the isomorphism classes of cospans.
    This category has monoidal product given by the coproduct in \(\mcc\) with unit the initial object \(0 \in \mcc\).
\end{definition}

\begin{definition}[Discrete hypergraph]
    A hypergraph is called \emph{discrete} if it has no edges.
\end{definition}

\noindent
A discrete hypergraph \(H\) with \(|\vertices{H}| = n\) is often written as \(n\) when clear from context.
Discrete hypergraphs are used as the `legs' of a cospan to select the inputs and outputs of hypergraphs.
There still needs to be a notion of \emph{ordering} on the interfaces, which is obtained using a functor from \(\finset\), the prop of finite sets and functions.

\begin{theorem}[\cite{bonchi2022string}, Thm. 3.8]
    \label{thm:cospan-homomorphism}
    Let \(\mathbb{X}\) be a prop whose monoidal product is a coproduct, \(\mcc\) a category with finite colimits, and \(\morph{F}{\mathbb{X}}{\mcc}\) a coproduct-preserving functor.
    Then there is a homomorphism of props \(\morph{\tilde{F}}{\csp{\mathbb{X}}}{\csp[F]{\mcc}}\) that sends \(\cospan{m}[f]{X}[g]{n}\) to \(\cospan{Fm}[Ff]{FX}[Fg]{Fn}\).
    If \(F\) is full and faithful, then \(\tilde{F}\) is faithful.
\end{theorem}

\begin{definition}
    Let \(\morph{D}{\finset}{\hypsigma}\) be the faithful, coproduct-preserving functor that sends each object \(m \in \finset\) to the discrete hypergraph \(m \in \hypsigma\) and each morphism to the induced homomorphism of discrete hypergraphs.
\end{definition}

\noindent
From this we define the category \(\cspdhyp\) with objects \emph{discrete cospans of hypergraphs}.
Since the legs of each cospan are discrete hypergraphs containing some number of vertices, the objects of this category can be viewed as natural numbers, making this a prop.

\subsection{Adding a trace}

Our goal is to adapt this hypergraph framework for a setting with a \emph{trace}.
Fortunately, the category of interfaced hypergraphs we have just defined already contains a great deal of useful structure.

\begin{definition}[Hypergraph category \cite{fong2019hypergraph}]
    A \emph{hypergraph category} is a symmetric monoidal category in which each object \(X\) is equipped with a Frobenius structure.
\end{definition}

\begin{proposition}
    \(\cspdhyp\) is a hypergraph category.
\end{proposition}
\iftoggle{proofs}{
    \begin{proof}
        The Frobenius structure is defined as follows:
        \begin{gather*}
            \iltikzfig{strings/structure/monoid/merge}[white]
            :=
            \cospan{m + m}{m}{m}
            \quad
            \iltikzfig{strings/structure/monoid/init}[white]
            :=
            \cospan{0}{m}{m}
            \\
            \iltikzfig{strings/structure/comonoid/copy}[white]
            :=
            \cospan{m}{m}{m+m}
            \quad
            \iltikzfig{strings/structure/comonoid/discard}[white]
            :=
            \cospan{m}{m}{0}
        \end{gather*}
        It is a simple exercise to check the axioms are satisfied.
    \end{proof}
}{}

\begin{corollary}[\cite{carboni1987cartesian}]
    \(\cspdhyp\) is compact closed.
\end{corollary}

It is well known that every compact closed category is equipped with a \emph{canonical trace}~\cite{joyal1996traced}, constructed by combining the cup, cap and an identity.
So in some sense, \(\cspfihyp\) is already traced.
However, it is useful to consider a(n equivalent) formulation in which the trace is constructed directly.

To take the \((f,g)\)-trace of a cospan \(\cospan{x+m}[f+h]{H}[g+k]{x+n}\), we coalesce the \(i\)th elements of the image of \(f\) and \(g\).
\[
    \trace{1}{\iltikzfig{graphs/trace/before-trace}}
    \equiv
    \iltikzfig{graphs/trace/after-trace}
\]
This is defined formally using a pushout, in a similar vein
to~\cite{dixon2013opengraphs}.

\begin{definition}
    For a cospan of hypergraphs \(
        \cospan{x+m}[f+h]{H}[g+k]{x+n},
    \) its trace on \(x\) is \(
        \cospan{m}[h \seq p]{H^\prime}[k \seq p]{n},
    \) where \(p\) and \(H^\prime\) are computed by the following pushout:
    \begin{center}
        \tikzfig{graphs/trace/trace-pushout}
    \end{center}
\end{definition}

\begin{example}
    Below is an example of tracing a simple hypergraph in this manner.
    \begin{center}
        \iltikzfig{graphs/trace/trace-pushout-example}
    \end{center}
\end{example}

\subsection{Monogamy}

In~\cite{bonchi2016rewriting}, it is shown that terms in a (non-traced)
symmetric monoidal category are interpreted as monogamous acyclic graphs, in
which each vertex has an `in' connection and an 'out' connection, be it to an
edge or the interface.
Clearly, to model trace we must drop the acyclicity condition.
However, we must also tweak the monogamicity condition.
This stems from considering the trace of the identity, which is defined as
\(
    \trace{1}{\iltikzfig{strings/category/identity}[white]}
    =
    \iltikzfig{strings/traced/trace-id}
\).
One might think that such a closed loop can be discarded.
However, this is not always the case, such as the category of finite dimensional
vector spaces~\cite[Sec. 6.1]{hasegawa1997recursion}.
This means that closed loops must be represented in the hypergraph framework:
fortunately, there is a natural representation of such a loop as a lone vertex
disconnected from either interface.
This is not compatible with monogamy, so a weakening must be considered.

\begin{definition}
    For a hypergraph \(H \in \hyp\), the \emph{degree} of a vertex \(v \in \vertices{H}\) is a tuple \((i,o)\) where \(i\) is the number of pairs \((e,i)\) where \(e\) is a hyperedge with \(v\) as its \(i\)th target, and \(o\) is similarly the number of pairs \((e,j)\) where \(e\) is a hyperedge with \(v\) as its \(j\)th target.
\end{definition}

\begin{definition}
    For a cospan \(\cospan{m}[f]{H}[g]{n}\), we say it is \emph{partial monogamous} if \(f\) and \(g\) are mono and, for all nodes \(v \in H_\star\),
    \begin{center}
        the degree of \(v\) is $\begin{cases}
            (0,0) & \text{if}\ v \in f_\star \wedge v \in g_\star \\
            (0,1) & \text{if}\ v \in f_\star \\
            (1,0) & \text{if}\ v \in g_\star \\
            (0,0) \text{ or } (1,1) & otherwise \\
        \end{cases}$
    \end{center}
\end{definition}

\begin{lemma}\label{lem:trace-degree}
    Given a cospan \(\cospan{x+m}[f+h]{F}[g+k]{x+n}\), and its corresponding
    traced cospan \(\cospan{m}[g \seq p]{F^\prime}[h \seq p]{n}\).
    For \(i < x\), let \((k_1,l_1)\) be the degree of \(f(i)\) and \((k_2,l_2)\)
    be the degree of \(g(i)\).
    Then, the degree of \(p(f(i)) = p(h(i))\) is \((k_1 + k_2, l_1 + l_2)\).
\end{lemma}
\iftoggle{proofs}{
    \begin{proof}
        \(p(f(i))\) and \(p(g(i))\) are coalesced by the pushout.
        The pushout cannot coalesce the edges in \(H\) as \(x\) contains no edges,
        so the degree of \(f(i)\) and \(h(i)\) must be preserved.
    \end{proof}
}{}

\begin{lemma}\label{lem:trace-interface}
    Given a partial monogamous cospan \(\cospan{x+m}[f+h]{F}[g+k]{x+n}\) and its
    trace \(\cospan{m}[h \seq p]{F^\prime}[k \seq p]{n},\) any vertex in the
    image of \(f\) is not in the image of \(h \seq p\), and similarly any vertex
    in the image of \(g\) is not in the image of \(k \seq p\).
\end{lemma}
\iftoggle{proofs}{
    \begin{proof}
        Since this is a partial monogamous cospan, the images of \(f\) and \(h\) are
        disjoint, as are the images of \(g\) and \(k\).
        Therefore, \(f(v)\) is not in the image of \(h\), so it cannot be in the
        image of \(h \seq p\), and similarly for \(g(v)\) and \(k\).
    \end{proof}
}{}

\begin{lemma}~\label{lem:partial monogamous-ops}
    Let \(\cospan{m}{F}{n}\), \(\cospan{n}{G}{p}\), \(\cospan{p}{H}{q}\) and
    \(\cospan{x+m}{K}{x+n}\) be partial monogamous cospans.
    Then,
    \begin{itemize}
        \item Identities and symmetries are partial monogamous.
        \item \(\cospan{m}{F}{n} \seq \cospan{n}{G}{p}\) is partial monogamous.
        \item \(\cospan{m}{F}{n} \tensor \cospan{p}{H}{q}\) is partial
        monogamous.
        \item \(
            \trace{x}{\cospan{x+m}[f+h]{K}[g+k]{x+n}}
            =
            \cospan{m}[h \seq p]{pK}[k \seq p]{n}
        \) is partial monogamous.
    \end{itemize}
\end{lemma}
\iftoggle{proofs}{
    \begin{proof}
        Since any monogamous hypergraph is also partial monogamous, the first three
        points hold due to~\cite[Prop.16]{bonchi2022string}, dropping the acyclicity
        condition.
        For the final condition, consider the image of \(f\) and \(g\).
        For each \(i \in x\), there are two cases to consider: \(f(i) = g(i)\) and
        \(f(i) \neq g(i)\).

        In the former, the degree of \(v^\prime = f(v) = h(v)\) must be \((0,0)\)
        by definition of semi-monogamicity.
        Therefore in the traced hypergraph \(
            \cospan{m}[g \seq p]{K^\prime}[h \seq p]{n}
        \), \(v^\prime\) will still have degree \((0,0)\), and will not be in the
        image of \(g \seq p\) or \(h \seq p\) by \cref{lem:trace-interface}.
        So the cospan is partial monogamous.

        In the latter case, \(f(i)\) must have degree of \((0,0)\) if it in the
        image of \(k\) or \((0,1)\) otherwise.
        Similarly \(g(i)\) either has degree \((0,0)\) or \((1,0)\).
        Let \(v := p(f(i)) = p(g(i))\); we now consider the degree of \(v\) computed
        using \cref{lem:trace-degree}:
        \begin{itemize}
            \item If \(f(i)\) is in the image of \(k\) and \(g(i)\) is in the image
                    of \(h\), then \(v\) has degree \((0,0)\).
                    \(v\) is in the image of \(h \seq p\) and \(h \seq p\), so the
                    cospan is partial monogamous.
            \item If \(f(i)\) is in the image of \(k\), then \(v\) has degree
                    \((1, 0)\); since \(v\) is in the image of \(k \seq p\), the
                    cospan is partial monogamous.
            \item If \(g(i)\) is in the image of \(h\), then \(v\) has degree
                    \((0, 1)\); since \(v\) is in the image of \(h \seq p\), the
                    cospan is partial monogamous.
            \item Otherwise, \(v\) will have degree \((1, 1)\), and is not in the
                    image of either interface so the cospan is partial monogamous.
        \end{itemize}
    \end{proof}
}{}

\noindent
This means that the partial monogamous hypergraphs form a sub-prop of
\(\cspfihyp\), which we name \(\pmcspfihyp\).

\subsection{From terms to graphs}

Since \(\stmc{\Sigma}\) and \(\pmcspfihyp\) are freely generated, to define a
morphism between them it suffices to define it on the generators in the former.

\begin{definition}
    Let \(\morph{\termtohyp{\Sigma}}{\stmc{\Sigma}}{\pmcspfihyp}\) be a prop
    morphism defined on the generators of \(\Sigma\) as shown in
    \cref{fig:termtohyp}.
    \begin{figure}
        \begin{gather*}
            \termtohyp[\iltikzfig{strings/category/f}[\phi][white][m][n]]{\Sigma}
            :=
            \iltikzfig{graphs/terms/generator}
            \quad
            \termtohyp[\iltikzfig{strings/category/identity}[white][m][n]]{\Sigma}
            :=
            \iltikzfig{graphs/terms/identity}[white]
            \\[1em]
            \termtohyp[\iltikzfig{strings/symmetric/symmetry}[white][m][n]]{\Sigma}
            :=
            \iltikzfig{graphs/terms/symmetry}[white]
        \end{gather*}
        \caption{Definition of \(\termtohyp{\Sigma}\) on generators in \(\smc{\Sigma}\).}
        \label{fig:termtohyp}
    \end{figure}
\end{definition}

\noindent
Recall one of the key theorems from~\cite{bonchi2021string}.

\begin{theorem}[\cite{bonchi2021string}, Corollary 20]\label{thm:smc-graph-iso}
    \(\smc{\Sigma} \cong \macspfihyp\).
\end{theorem}

\noindent
Our goal is to extend this to terms in \(\stmcsigma\) and
\emph{partial monogamous hypergraphs}.
We first characterise the image of \(\termtohypsigma\) and show that it is
exactly \(\pmcspfihyp\).
Before progressing to the main theorem, we must show some results about terms in
\(\smc{}\), i.e.\ terms constructed from just symmetries and identities.
In particular, there is a correspondence between \(\smc{}\) and \(\perms\), the
subprop of the prop of finite sets and functions containing only the bijective
functions.


\begin{lemma}\label{lem:symmetries-prop}
    \(\smc{} \cong \perms\).
\end{lemma}
\iftoggle{proofs}{
    \begin{proof}
        The morphism \(\morph{\phi}{\smc{}}{\perms}\) is defined over generators in
        \(\smc{}\) as \[
            \phi(\iltikzfig{strings/monoidal/empty}) = \{\}
            \quad
            \phi(\iltikzfig{strings/category/identity}[white])
            =
            \{0 \mapsto 0\}
            \quad
            \phi(\iltikzfig{strings/symmetric/symmetry}[white])
            =
            \{0 \mapsto 1, 1 \mapsto 0\}
        \]
        Since any term in \(\smc{}\) can be expressed using these generators, this
        defines the complete transformation.

        The reverse morphism \(\morph{\psi}{\finset}{\smc{}}\) is inductively over
        the size of \(m\).
        For the base case \(\morph{f}{[0]}{[0]}\), let \(
            \phi(f) := \iltikzfig{strings/monoidal/empty}
        \).
        For \(
            \morph{f}{[k+1]}{[k+1]}
        \), let \(i\) such that \(f(i) = k+1\), and define the function \(
            \morph{f^\prime}{\nat_{k}}{\nat_{k}}
        \) as the function such that \(
            f^\prime(j) = f(j)
        \) if \(j < i\), and \(f(j+1)\) otherwise.
        Then \[
            \psi(f) := \iltikzfig{strings/symmetric/f-construction}.
        \]

        \noindent
        These are shown to be inverses by a simple induction in both directions.
    \end{proof}
}{}

\noindent
This allows us to establish a correspondence between terms in \(\smc{}\) and
cospans of discrete hypergraphs.

\begin{lemma}\label{lem:monog-discrete-cospan}
    Given a monogamous cospan \(\cospan{m}[f]{m}[g]{m}\), there exists a unique
    term \(\morph{h}{m}{m} \in \smc{}\) up to the axioms of SMCs such that
    \(\termtohyp[h]{\Sigma} = \cospan{m}[f]{m}[g]{m}\).
\end{lemma}
\iftoggle{proofs}{
    \begin{proof}
        Since the cospan is monogamous, \(f\) and \(g\) are mono.
        As the cospan is also discrete, there exists a (unique) bijective function
        \(\morph{h^\prime}{[m]}{[m]}\) such that \(h^\prime(i) = j\) if
        \(f(i) = g(j)\).
        By \cref{lem:symmetries-prop}, there is a corresponding term
        \(h \in \smc{}\) that is unique up to SMC axioms: a simple induction shows
        that \(\termtohyp[h]{\Sigma} = \cospan{m}[f]{m}[g]{m}\).
    \end{proof}
}{}

\noindent
We now proceed to the first main result.

\begin{proposition}\label{prop:termtohyp-image}
    A cospan \(\cospan{m}{H}{n}\) is in the image of \(\termtohyp{\Sigma}\) if
    and only if it is partial monogamous.
\end{proposition}
\iftoggle{proofs}{
    \begin{proof}
        To show that \(\termtohyp[f]{\Sigma}\) is partial monogamous for any
        \(f \in \smc{\Sigma}\) we use induction on the structure of \(f\).
        Generators, identities and symmetries are partial monogamous, as
        semi-monogamicity is preserved by composition, tensor and trace by
        \cref{lem:partial monogamous-ops}.
        So \(\termtohyp[f]{\Sigma}\) is partial monogamous.

        Now we show that any partial monogamous cospan \(\cospan{m}[f]{F}[g]{n}\)
        must be in the image of \(\termtohyp{\Sigma}\).
        To do this, we will now construct a cospan that is isomorphic to
        \(\cospan{m}[f]{F}[g]{n}\), but from which it is possible to read off a
        unique term in \(\smc{\Sigma}\).
        The components of the new cospan are as follows:
        \begin{itemize}
            \item let \(L\) be the hypergraph containing vertices with degree
                    \((0,0)\) that are not in the image of \(f\) or \(g\);
            \item let \(E\) be the hypergraph containing hyperedges of \(F\) and
                    their source and target vertices, but disconnected;
            \item let \(V\) be the discrete hypergraph containing all the vertices
                    of \(F\); and
            \item let \(S\) and \(T\) be the discrete hypergraphs containing the
                    source and target vertices of hyperedges in \(F\) respectively,
                    with the ordering determined by some order
                    \(e_1,e_2,\cdots,e_n\) on the edges in \(F\).
        \end{itemize}

        \noindent
        These parts can be composed and a trace applied to obtain the follow cospan:
        \begin{gather}
            \trace{T}{
                \cospan{T + m}[\id + f]{V}[\id + g]{S + n}
                \,\seq\,
                \cospan{\emptyset + S + n}[\id]{L + E + n}[\id]{\emptyset + T + n}
            }
            \label{gat:cospan}
        \end{gather}

        \noindent
        This can be checked to be isomorphic to the original cospan
        \(\cospan{m}[f]{F}[g]{n}\) by applying the pushouts.
        From this we can read off a term in \(\smc{\Sigma}\):
        Since the first cospan is monogamous, it corresponds to a term \(
            \iltikzfig{graphs/isomorphism/construction-f}
        \) by \cref{lem:monog-discrete-cospan}.
        The second cospan corresponds to \(
            \iltikzfig{graphs/isomorphism/construction-g}
            :=
            \bigtensor_{v \in \vertices{L}}
            \iltikzfig{strings/traced/trace-id}
            \tensor
            \bigtensor_{e \in 0 \leq i \leq n}
            \iltikzfig{graphs/isomorphism/label-e}
            \tensor
            \iltikzfig{strings/category/identity}[white][n]
        \).
        Putting this all together yields \(
            h := \termtohypsigma[\iltikzfig{graphs/isomorphism/construction}]
        \).
        While there may be multiple orderings on the edges, the possible terms
        are equal by sliding and the naturality of symmetry, so there is one
        unique term \(
            \iltikzfig{strings/category/f}[H][white]
        \) that corresponds to cospan (\ref{gat:cospan}).

        It is clear by definition that \(
            \termtohypsigma[\iltikzfig{strings/category/f}[H][white]]
        \) produces (\ref{gat:cospan}), which is isomorphic to the original cospan
        \(\cospan{m}[f]{F}[g]{n}\), so it is in the image of \(\termtohypsigma\).
    \end{proof}
}{}

\noindent
The final step is to show that \(\termtohypsigma\) is \emph{faithful}, i.e.\
each unique term in \(\stmcsigma\) up to the equations of STMCs corresponds to a
unique cospan of hypergraphs.
First we show that any term in \(\stmcsigma\) not equal to a term in
\(\smcsigma\) cannot correspond to a cospan in \(\macspdhyp\).

\begin{lemma}
    If \(\iltikzfig{strings/category/f}[H][white]\) is a term in \(
        \stmc{\Sigma} \setminus \smc{\Sigma}
    \), then \(
        \termtohypsigma[\iltikzfig{strings/category/f}[H][white]]
    \) either fails to be monogamous, acyclic, or both.
\end{lemma}
\iftoggle{proofs}{
    \begin{proof}
        Assume that \(\iltikzfig{strings/category/f}[H][white]\) is monogamous
        acyclic, so it is in \(\macspfihyp\).
        By \cref{thm:smc-graph-iso}, any graph in \(\macspfihyp\) corresponds to a
        unique term in \(\smc{\Sigma}\) up to axioms of SMCs: a contradiction, as \(
            \iltikzfig{strings/category/f}[H][white]
            \in
            \stmc{\Sigma} \setminus \smc{\Sigma}
        \).
    \end{proof}
}{}

\begin{corollary}
    The images of \(\smc{\Sigma}\) and \(\stmc{\Sigma} \setminus \smc{\Sigma}\)
    are disjoint under \(\termtohypsigma\).
\end{corollary}

\begin{corollary}[\cite{bonchi2022string}, Corollary 3.11]\label{cor:termtohyp-faithful-smc}
    \(\morph{\termtohypsigma}{\stmc{\Sigma}}{\pmcspfihyp}\) is faithful when the
    domain is restricted to \(\smc{\Sigma}\).
\end{corollary}

\begin{theorem}
    \(\termtohypsigma\) is faithful when the domain is restricted to \(
        \stmc{\Sigma} \setminus \smc{\Sigma}
    \).
\end{theorem}
\iftoggle{proofs}{
    \begin{proof}
        Let \(
            \iltikzfig{strings/category/f}[F][white]
        \) and \(
            \iltikzfig{strings/category/f}[G][white]
        \) be terms in \(\stmc{\Sigma}\) such that \(
            \iltikzfig{strings/category/f}[F][white]
            \neq
            \iltikzfig{strings/category/f}[G][white]
        \).
        By applying axioms of STMCs, these terms can be rewritten as \(
            \iltikzfig{strings/traced/trace-rhs}[f^\prime][white]
        \) and \(
            \iltikzfig{strings/traced/trace-rhs}[g^\prime][white]
        \) respectively, where \(
            \iltikzfig{strings/traced/trace-lhs}[f^\prime][white]
        \) and \(
            \iltikzfig{strings/traced/trace-lhs}[g^\prime][white]
        \) are terms in \(\smc{\Sigma}\).
        Since \(
            \iltikzfig{strings/category/f}[F][white]
            \neq
            \iltikzfig{strings/category/f}[F][white]
        \), then \(
            \iltikzfig{strings/traced/trace-lhs}[f^\prime][white]
            \neq
            \iltikzfig{strings/traced/trace-lhs}[g^\prime][white]
        \).
        Assume that \(
            \termtohypsigma[\iltikzfig{strings/category/f}[F][white]]
        \) and \(
            \termtohypsigma[\iltikzfig{strings/category/f}[F][white]]
        \) are isomorphic as cospans.
        Then \(
            \termtohypsigma[\iltikzfig{strings/traced/trace-rhs}[f^\prime][white]]
        \) and \(
            \termtohypsigma[\iltikzfig{strings/traced/trace-rhs}[g^\prime][white]]
        \) are also isomorphic.
        However, since \(\termtohypsigma\) is faithful on terms in \(\smc{\Sigma}\)
        by \cref{cor:termtohyp-faithful-smc}, \(
            \termtohypsigma[\iltikzfig{strings/traced/trace-lhs}[f^\prime][white]]
        \) and \(
            \termtohypsigma[\iltikzfig{strings/traced/trace-lhs}[g^\prime][white]]
        \) are not isomorphic.

        The only axiom of STMCs that allows for the trace of non-equal morphisms to
        be equal is the sliding axiom.
        But for such a situation to be valid here, the original terms must also be
        equal by sliding, a contradiction.
        So \(
            \termtohypsigma[\iltikzfig{strings/category/f}[F][white]]
        \) and \(
            \termtohypsigma[\iltikzfig{strings/category/f}[G][white]]
        \) are not isomorphic: \(\termtohypsigma\) is faithful.
    \end{proof}
}{}

\begin{corollary}
    \(\termtohypsigma\) is faithful.
\end{corollary}

\begin{corollary}
    \(\stmc{\Sigma} \cong \pmcspfihyp\).
\end{corollary}
    % !TeX root = ../main-conf.tex
\section{Comonoid structure}

Digital circuits do not just have a traced structure, but moreover the ability
to \emph{copy} and \emph{discard} wires: \(\scircsigmal\) has a
\emph{comonoid structure}.
Such categories are also known as \emph{gs-monoidal} (\emph{garbage-sharing})
categories~\cite{fritz2022free}.

There has been recent work into hypergraphs at the `halfway point' between
symmetric monoidal and Frobenius terms: \cite{fritz2022free} examines a general
construction for constructing free gs-categories as hypergraphs, but does not
continue as far as graph rewriting.
Conversely, \cite{milosavljevic2022string} presensts results for rewriting with
a \emph{monoid} structure: similar results can be extracted for a comonoid
structure by flipping all the directions.
Both of these papers consider \emph{acyclic} hypergraphs: we must ensure that
removing the acyclicity condition does not lead to any degeneracies.

\begin{definition}[Partial left-monogamy]
    For a cospan \(\cospan{m}[f]{H}[g]{n}\), we say it is
    \emph{partial left-monogamous} if \(f\) is mono and, for all nodes
    \(v \in H_\star\), the degree of \(v\) is \((0,m)\) if \(v \in f_\star\) and
    \((0,m)\) or \((1,m)\) otherwise, for some \(m \in \nat\).
\end{definition}

\begin{lemma}
    Any partial monogamous cospan is also partial left-monogamous.
\end{lemma}

\begin{lemma}
    \label{lem:trace-in-degree}
    Given a partial left-monogamous cospan \(\cospan{x+m}[f+h]{K}[g+k]{x+n}\)
    and its trace \(\cospan{m}[h \seq p]{pK}[k \seq p]{n}\), let vertices
    \(v_0, v_1, \cdots, v_n\) such that each \(v_i\) is in the image of \(g\)
    and \(p(v_0) = p(v_1) = \cdots = p(v_n)\).
    Then, there exists at most one \(v_i\) with in-degree \(1\).
\end{lemma}
\iftoggle{proofs}{
    \begin{proof}
        Assume that there exist vertices \(g(i),g(j)\) with in-degree \(1\).
        For \(p(g(i)) = p(g(j))\) to hold, then there must either exist a sequence
        \(f(i) = g(i_0), f(i_0) = g(i_1), \cdots, f(i_n) = g(j)\), or vice versa.
        But \(f(i_n) = g(j)\) must have in-degree \(0\) by partial left-monogamy, a
        contradiction.
        Therefore at most one \(v_i\) can have in-degree \(1\).
    \end{proof}
}{}

\begin{lemma}
    \label{lem:partial-monogamous-ops}
    Let \(\cospan{m}{F}{n}\), \(\cospan{n}{G}{p}\), \(\cospan{p}{H}{q}\) and
    \(\cospan{x+m}{K}{x+n}\) be partial left-monogamous cospans.
    Then,
    \begin{itemize}
        \item Identities and symmetries are partial left-monogamous.
        \item \(\cospan{m}[f]{F}[g]{n} \seq \cospan{n}[h]{G}[k]{p}\) is partial
                left-monogamous.
        \item \(\cospan{m}{F}{n} \tensor \cospan{p}{H}{q}\) is partial
                left-monogamous.
        \item \(
                \trace{x}{\cospan{x+m}[f+h]{K}[g+k]{x+n}}
                =
                \cospan{m}[h \seq p]{pK}[k \seq p]{n}
              \) is partial left-monogamous.
    \end{itemize}
\end{lemma}
\iftoggle{proofs}{
    \begin{proof}
        Identities and symmetries are monogamous, and as such they are also
        partial left-monogamous.
        For composition, the vertices in the image of \(g\) and \(h\) are
        identified.
        Let \(v = p(g(i)) = p(h(i))\).
        We must show that \(v\) has in-degree \(0\) if it is in the image of
        \(f\), and \(0\) or \(1\) otherwise.
        \(h(i)\) has in-degree \(1\) by definition, so the in-degree of \(v\) is
        entirely determined by \(g(i)\).
        If \(v\) is in the image of \(f\), then \(g(i)\) must also be in the
        image of \(f\), so it has in-degree \(0\), and hence so does \(v\).
        Conversely, if \(v\) is not in the image of \(f\), \(g(i)\) has
        in-degree of either \(0\) or \(1\), and hence so does \(v\).

        For tensor, the elements of the original two graphs are unaffected so
        the degrees remain unchanged.

        For trace, let \(v = p(f(i)) = p(g(i))\).
        \(v\) cannot be in the image of \(h \seq p\) as this would mean that
        \(f + h\) is not mono.
        Therefore we must show that \(v\) has either degree of either \((0,m)\)
        or \((1,m)\).
        The degree of \(v\) is the sum of the degrees of each \(v_{fi}\) and
        \(v_{gi}\).
        Let \(v_{f0},\cdots,v_{fn}\) be the vertices in the image of \(f\) such
        that \(p(v_{fi}) = v\), and similarly for \(v_{gi}\).
        The in-degree of each \(v_{fi}\) must be \(0\) so all the in-degree is
        contributed by each \(v_{gi}\).
        By \cref{lem:trace-in-degree}, at most \(v_{fi}\) can have in-degree
        \(1\), so the in-degree of \(v\) can either be \(0\) or \(1\).
        Therefore the cospan is partial left-monogamous.
    \end{proof}
}{}

Therefore cospans of partial left-monogamous hypergraphs form a category, which
we name \(\plmcspfihyp\).
This category can be equipped with a comonoid structure.
Let \(\ccomon\) be the prop freely generated over signature \(
    \{
        \iltikzfig{strings/structure/comonoid/copy}[white],
        \iltikzfig{strings/structure/comonoid/discard}[white]
    \}
\).

\begin{proposition}[\cite{lack2004composing}, Example 5.2]
    \label{prop:ccomon-iso-finsetop}
    \(\ccomon \cong \op{\finset}\).
\end{proposition}

\noindent
The following is a corollary of \cref{thm:cospan-homomorphism}.

\begin{corollary}
    \label{cor:prop-homomorphism-finset}
    There is a faithful prop homomorphism \(
        \morph{\tilde{D}}{\csp{\op{\finset}}}{\csp[D]{\hypsigma}}
    \).
\end{corollary}

\begin{definition}
    Let \(\morph{\comonoidtohyp}{\ccomon}{\cspfihyp}\) be the homomorphism
    obtained by composing the isomorphism of \cref{prop:ccomon-iso-finsetop}
    with the homomorphism of \cref{cor:prop-homomorphism-finset}.
    Concretely, it is defined on objects in the obvious way and on morphisms as
    \(
        \comonoidtohyp[\iltikzfig{strings/structure/comonoid/copy}[white]]
        :=
        \cospan{1}{1}{2}
    \) and \(
        \comonoidtohyp[\iltikzfig{strings/structure/comonoid/discard}[white]]
        :=
        \cospan{1}{1}{0}
    \).
\end{definition}

\begin{lemma}
    The image of \(\comonoidtohyp\) is in \(\plmcspfihyp\).
\end{lemma}
\begin{proof}
    By definition.
\end{proof}

\begin{definition}
    Let \(\morph{\termandfrobtohypsigma}{\stmcsigma + \ccomon}{\plmcspfihyp}\) be
    defined as the copairing of \(\termtohypsigma\) and \(\comonoidtohyp\).
\end{definition}

\begin{lemma}
    Given a left-monogamous cospan \(\cospan{m}[f]{m}[g]{n}\), there exists a
    unique term \(\morph{h}{m}{n} \in \ccomon\) up to the axioms of SMCs such
    that \(\comonoidtohyp\) = \(\cospan{m}[f]{m}[g]{n}\).
\end{lemma}
\iftoggle{proofs}{
    \begin{proof}

    \end{proof}
}{}

\begin{theorem}
    \(\stmcsigma + \ccomon \cong \plmcspfihyp\).
\end{theorem}
\iftoggle{proofs}{
    \begin{proof}
        Since \(\termandfrobtohypsigma\) and \(\comonoidtohyp\) are faithful, it
        suffices to show that every cospan \(\cospan{m}{F}{n} \in \plmcspfihyp\)
        can be decomposed in such a way that each component is in the image of
        either \(\termandfrobtohypsigma\) or \(\comonoidtohyp\).
    \end{proof}
}{}

    % !TeX root = ../main-conf.tex
\section{Graph rewriting}

\begin{definition}[Rewrite rule]
    Given cospans \(
        \cospan{i}[a_1]{L}[a_2]{j}
    \) and \(
        \cospan{i}[b_1]{R}[b_2]{j}
    \), their rewrite rule is a span in \(\hypsigma\) \(
        \spann{L}[[a_1,a_2]]{i+j}[[b_1,b_2]]{R}
    \).
\end{definition}

\begin{lemma}
    Given terms \(
        \iltikzfig{strings/rewriting/l}
    \), \(
        \iltikzfig{strings/rewriting/r}
    \) in \(\stmcsigma\) and their corresponding rewrite rule \[
        \spann{
            \termtohypsigma[\foldinterfaces[\iltikzfig{strings/rewriting/l}]]
        }[[a_1, a_2]]{
            i + j
        }[[b_1, b_2]]{
            \termtohypsigma[\foldinterfaces[\iltikzfig{strings/rewriting/r}]]
        }
    \] in \(\hypsigma\), \(a_1\), \(a_2\), \(b_1\) and \(b_2\) are mono.
\end{lemma}
\iftoggle{proofs}{
    \begin{proof}
        By definition of partial monogamous cospan.
    \end{proof}
}{}

\begin{definition}[DPO]
    \begin{gather}
        \label{gath:dpo}
        \includestandalone{figures/graphs/dpo/dpo}
    \end{gather}
\end{definition}

\begin{definition}[Boundary complement \cite{bonchi2021string}, Definition 23]\label{def:boundary-complement}
    For \(
        \cospan{i}[a_1]{L}[a_2]{j}
    \) and \(
        \cospan{n}[b_1]{G}[b_2]{m} \in \macspfihyp
    \) and mono \(
        \morph{f}{L}{G} \in \hypsigma
    \), a pushout complement as below
    \begin{gather}
        \includestandalone{figures/graphs/dpo/boundary-complement}
        \label{gath:boundary-complement}
    \end{gather}
    is called a \emph{boundary complement in \(\mathcal{C}\)} if \([c_1, c_2]\)
    is mono and there exist morphisms \(
        \morph{d_1}{n}{\pushoutcomplement{L}}
    \) and \(
        \morph{d_2}{m}{\pushoutcomplement{L}}
    \) making the above triangle commute and such that \[
        \cospan{j+n}[[c_2,d_1]]{\pushoutcomplement{L}}[[c_1,d_2]]{{m+i}}
    \] is a monogamous cospan.
\end{definition}

\begin{proposition}[\cite{bonchi2021string}, Proposition 24]\label{prop:boundary-complement-unique}
    Boundary complements in \(\macspfihyp\) are unique, when they exist.
\end{proposition}

\noindent
Crucially, boundary complements rely on the matching \(L \to G\) being mono.
But restricting to these matchings cuts off potential rewrites in the
\emph{traced} setting, such as the occurrence of a rewrite rule inside a loop:
\[
    \iltikzfig{graphs/dpo/matchings/non-mono-matching}
\]

\begin{definition}
    For partial monogamous cospans \(
        \cospan{i}[a_1]{L}[a_2]{j}
    \) and \(
        \cospan{n}[b_1]{G}[b_2]{m} \in \mathcal{C}
    \) and (not necessarily mono) \(
        \morph{f}{L}{G} \in \hypsigma
    \), a pushout complement as in \cref{gath:boundary-complement}
    is called a \emph{traced boundary complement} if \(c_1\) and \(c_2\) are
    mono, \(f\) is injective on edges and vertices not in the image of \(a\),
    and there exist morphisms \(
        \morph{d_1}{m}{\pushoutcomplement{L}}
    \) and \(
        \morph{d_2}{n}{\pushoutcomplement{L}}
    \) making the above triangle commute such that
    \begin{gather}
        \cospan{j+m}[[c_2,d_1]]{\pushoutcomplement{L}}[[d_2,c_1]]{{n+i}}
        \label{gat:traced-complement}
    \end{gather} is a partial monogamous cospan.
\end{definition}

\noindent
Using the semi-monogamicity of (\ref{gat:traced-complement}) we can prove some
useful lemmas regarding when vertices can be coalesced.

\begin{lemma}
    In a traced boundary complement, if \(
        a_1(v) = a_2(w)
    \), then \(
        c_1(v) = c_2(w)
    \) if and only if \(c_1(v)\) and \(c_2(w)\) are not in the image of \(d\)
    and \(f(a_1(v)) = f(a_2(w))\) has degree \((0,0)\).
\end{lemma}
\iftoggle{proofs}{
    \begin{proof}
        For the \(\onlyifdir\) direction, assume that \(c_1(v) = c_2(w)\): let this
        vertex be \(v^\prime\).
        Since (\ref{gat:traced-complement}) is partial monogamous, \([c_2,d_1]\) and
        \([c_1,d_2]\) must be mono, and thus the images of \(c_1\) and \(d_2\) must
        be disjoint, as must the images of \(c_2\) and \(d_1\).
        Therefore \(v^\prime\) cannot be in the image of \(d\); moreover, \(c_1(v)\)
        must have out-degree \(0\) and \(c_2(w)\) must have in-degree \(0\), so
        \(v^\prime\) has degree \((0,0)\).
        Let \(v^{\prime\prime}\) be \(a_1(v) = a_2(w)\): since \(
            \cospan{i}[a_1]{\iltikzfig{strings/rewriting/l}}[a_2]{j}
        \) is a partial monogamous cospan \(v^\prime\) must have degree \((0,0)\).
        Since \(G\) is computed by pushout, \(
            f(v^{\prime\prime}) = g(v^{\prime})
        \) and the degree of this vertex is contributed wholly by \(v^\prime\) and
        \(v^{\prime\prime}\).
        So \(f(a_1(v)) = f(a_2(w))\) also has degree \((0,0)\).

        Now for the \(\ifdir\) direction, assume that \(c_1(v)\) and \(c_2(w)\) are
        not in the image of \(d\), and \(f(a_1(v)) = f(a_2(w))\) has degree
        \((0,0)\).
        Assume that \(c_1(v) \neq c_2(w)\).
        By semi-monogamy of (\ref{gat:traced-complement}), \(c_1(v)\) must have
        degree \((0,1)\) since it is solely in the image of \([d_2, c_1]\).
        But \(f(a_1(v)) = f(a_2(w)) = g(c_1(v)) = g(c_2(w))\), \(c_1(v)\) and
        \(c_2(w)\) must have degree \((0,0)\), a contradiction.
        The same holds for \(c_2(w)\), so \(c_1(v) = c_2(w)\).
    \end{proof}
}{}

A crucial part of~\cite{bonchi2021string} is that (non-traced) boundary
complements are \emph{unique}.
One might assume the same for traced boundary complements, but this is not the
case.

\begin{example}
    Consider the rule and its interpretation.
    \begin{gather}
        \rrule{
            \iltikzfig{graphs/dpo/non-unique/rule-lhs}
        }{
            \iltikzfig{graphs/dpo/non-unique/rule-rhs}
        }
        \qquad
        \raisebox{-2.1em}{\includestandalone{figures/graphs/dpo/non-unique/rule}}
        \label{gath:non-unique-rule}
    \end{gather}

    Now consider the following hypergraph with interface \(
        \iltikzfig{graphs/dpo/non-unique/g-unlabelled}
        \leftarrow
        \iltikzfig{graphs/dpo/non-unique/j-unlabelled}
    \).
    There are \emph{two} valid traced boundary complements for the above rule in
    this graph!

    \begin{center}
        \scalebox{0.95}{
            \includestandalone{figures/graphs/dpo/non-unique/rewrite-1}
            \includestandalone{figures/graphs/dpo/non-unique/rewrite-2}
        }
    \end{center}
    \noindent
    Both derivations are valid and arise since we are rewriting modulo
    \emph{yanking}:
    \begin{gather*}
        \iltikzfig{graphs/dpo/non-unique/derivation-1}
        =
        \iltikzfig{graphs/dpo/non-unique/derivation-2}
        =
        \iltikzfig{graphs/dpo/non-unique/derivation-3a}
        =
        \iltikzfig{graphs/dpo/non-unique/derivation-4a}
        =
        \iltikzfig{graphs/dpo/non-unique/derivation-5a}
        \\
        \iltikzfig{graphs/dpo/non-unique/derivation-1}
        =
        \iltikzfig{graphs/dpo/non-unique/derivation-2}
        =
        \iltikzfig{graphs/dpo/non-unique/derivation-3b}
        =
        \iltikzfig{graphs/dpo/non-unique/derivation-4b}
        =
        \iltikzfig{graphs/dpo/non-unique/derivation-5b}
    \end{gather*}
\end{example}

\noindent
Rewriting modulo yanking also eliminates another foible of rewriting modulo
(non-traced) symmetric monoidal structure.
In the latter, matchings must be \emph{convex}: there cannot be a path from the
outputs of a match back to its inputs.
However, in the traced case this is redundant, as illustrated in the following
example.

\begin{example}
    Consider the following rewrite rule and its interpretation.
    %
    \begin{gather}
        \rrule{
            \iltikzfig{graphs/dpo/convex/example-l}
        }{
            \iltikzfig{graphs/dpo/convex/example-r}
        }
        \qquad
        \iltikzfig{graphs/dpo/convex/example-rule-graph}
        \label{gath:convex-rule}
    \end{gather}
    %
    \noindent
    Now consider the following term and interpretation:
    %
    \begin{gather}
        \iltikzfig{graphs/dpo/convex/example-g}
        \qquad
        \iltikzfig{graphs/dpo/convex/example-g-graph}
        \label{gath:convex-term}
    \end{gather}
    %
    \noindent
    Although it is not obvious in the original string diagram, there is in fact
    a matching of (\ref{gath:convex-rule}) in (\ref{gath:convex-term}).
    Performing the DPO procedure yields the following:
    %
    \begin{gather}
        \iltikzfig{graphs/dpo/convex/example-h-graph}
        \qquad
        \iltikzfig{graphs/dpo/convex/example-h}
    \end{gather}
    %
    \noindent
    In a non-traced setting this is an invalid rule!
    However, it is possible with yanking.
    \begin{gather*}
        \iltikzfig{graphs/dpo/convex/example-g}
        =
        \iltikzfig{graphs/dpo/convex/rewrite-1}
        =
        \iltikzfig{graphs/dpo/convex/rewrite-2}
        =
        \iltikzfig{graphs/dpo/convex/rewrite-3}
        \\[1em]
        =
        \iltikzfig{graphs/dpo/convex/rewrite-4}
        =
        \iltikzfig{graphs/dpo/convex/rewrite-5}
        =
        \iltikzfig{graphs/dpo/convex/example-h}
    \end{gather*}
\end{example}

\begin{definition}[Traced DPO]
    For morphisms \(G \leftarrow m+n\) and \(H \leftarrow m+n\) in
    \(\hypsigma\), there is a traced rewrite \(G \trgrewrite{\mcr} H\) if there
    exists a rule \(
        \spann{L}{i+j}{G} \in \mcr
    \) and cospan \(
        \cospan{i+j}{C}{n+m} in \hypsigma
    \) such that the following diagram commutes:
    \begin{gather}
        \includestandalone{figures/graphs/dpo/dpo}
    \end{gather}
    and \(i+j \to C\) is a traced boundary complement.
\end{definition}

\noindent
We are almost ready to show the soundness and completeness of this DPO rewriting
system.
The final prerequisite is a decomposition lemma, akin to a similar result
in~\cite{bonchi2021string} for the symmetric monoidal case.

\begin{lemma}[Traced decomposition]\label{lem:decomposition}
    Given partial monogamous cospans \(\cospan{m}[p_1]{G}[p_2]{n}\) and \(
        \cospan{i}[a_1]{L}[a_2]{j}
    \), along with a morphism \(
        L \xrightarrow{f} G
    \) satisfying the no-dangling-hyperedges condition, then \(
        \cospan{m}[p_1]{G}[p_2]{n}
    \) can be factored as
    \begin{gather}
        \trace{i}{
            \begin{array}{cc}
                \cospan{i}[a_1]{L}[a_2]{j} \\
                \tensor \\
                \cospan{m}{m}{m}
            \end{array}
            \seq
            \cospan{j+m}{C}{i+n}
        }
        \label{gath:decomposition}
    \end{gather}
    where all cospans are partial monogamous and \(C\) is a traced boundary
    complement.
\end{lemma}
\iftoggle{proofs}{
    \begin{proof}
        Let \(C\) be defined as a traced boundary complement of \(
            i+j \xrightarrow{[a_1,a_2]} L \xrightarrow{f} G
        \), which exists as the no-dangling-hyperedges condition is satisfied.
        We assign names to the various cospans in play, and reason string
        diagrammatically:
        \begin{align*}
            \iltikzfig{graphs/dpo/lhat} &:= \cospan{i}{L}{j}
            &
            \iltikzfig{graphs/dpo/ltilde} &:= \cospan{0}{L}{i+j} \\
            \iltikzfig{graphs/dpo/chat} &:= \cospan{j+m}{L}{i+n}
            &
            \iltikzfig{graphs/dpo/ctilde} &:= \cospan{i+j}{C}{m+n} \\
            \iltikzfig{graphs/dpo/ghat} &:= \cospan{m}{G}{n}
            &
            \iltikzfig{graphs/dpo/gtilde} &:= \cospan{0}{G}{m+n}
        \end{align*}
        By using the compact closed structure of \(\cspfihyp\), we also have that \(
            \iltikzfig{graphs/dpo/ltilde} = \iltikzfig{graphs/dpo/lhat-bent}
        \), \(
            \iltikzfig{graphs/dpo/ctilde} = \iltikzfig{graphs/dpo/chat-bent}
        \) and \(
            \iltikzfig{graphs/dpo/gtilde} = \iltikzfig{graphs/dpo/ghat-bent}
        \).
        Since \(
            \iltikzfig{graphs/dpo/gtilde} = \iltikzfig{graphs/dpo/lctilde}
        \), it follows that \(
            \iltikzfig{graphs/dpo/ghat-bent} = \iltikzfig{graphs/dpo/lchat-bent}
        \) and subsequently \(
            \iltikzfig{graphs/dpo/ghat} = \iltikzfig{graphs/dpo/lchat}
        \).
        The `loop' is constructed in the same manner as the canonical trace on
        \(\cspfihyp\), so this is a term in the form of (\ref{gath:decomposition}).
        Moreover, all cospans involved are partial monogamous by definition of
        rewrite rules and traced boundary complements.
    \end{proof}
}{}

\begin{theorem}
    Let \(\mcr\) be a rewriting system on \(\stmcsigma\).
    Then, \(
        \iltikzfig{strings/category/f}[F][white]
        \rewrite[\mcr]
        \iltikzfig{strings/category/f}[H][white]
    \) if and only if \(
        \termandfrobtohypsigma[\foldinterfaces[\iltikzfig{strings/category/f}[F][white]]]
        \grewrite[\termandfrobtohypsigma[\foldinterfaces[\mcr]]]
        \termandfrobtohypsigma[\foldinterfaces[\iltikzfig{strings/category/f}[H][white]]].
    \)
\end{theorem}
\iftoggle{proofs}{
    \begin{proof}
        First the \((\Rightarrow)\) direction.
        If \(
            \iltikzfig{strings/category/f}[F][white]
            \rewrite[\mcr]
            \iltikzfig{strings/category/f}[H][white]
        \) then we have \(
            \iltikzfig{strings/category/f}[F][white]
            =
            \iltikzfig{strings/rewriting/rewrite-l}
        \) and \(
            \iltikzfig{strings/rewriting/rewrite-r}
            =
            \iltikzfig{strings/category/f}[H][white].
        \)
        Define the following cospans:
        \begin{align}
            \label{gath:l-cospan}
            \cospan{0}{L}{i+j}
            &:=
            \termandfrobtohypsigma[\foldinterfaces[\iltikzfig{strings/rewriting/l}]]
            &&=
            \termandfrobtohypsigma[\iltikzfig{strings/rewriting/l-folded}]
            \\
            \cospan{0}{R}{i+j}
            &:=
            \termandfrobtohypsigma[\foldinterfaces[\iltikzfig{strings/rewriting/r}]]
            &&=
            \termandfrobtohypsigma[\iltikzfig{strings/rewriting/r-folded}]
            \\
            \cospan{0}{G}{m+n}
            &:=
            \termandfrobtohypsigma[\foldinterfaces[\iltikzfig{strings/category/f}[F][white]]]
            &&=
            \termandfrobtohypsigma[\iltikzfig{strings/rewriting/lc-folded}]
            \\
            \label{gath:h-cospan}
            \cospan{0}{H}{m+n}
            &:=
            \termandfrobtohypsigma[\foldinterfaces[\iltikzfig{strings/category/f}[H][white]]]
            &&=
            \termandfrobtohypsigma[\iltikzfig{strings/rewriting/rc-folded}]
            \\
            \cospan{i+j}{C}{m+n}
            &:=
            \termandfrobtohypsigma[\iltikzfig{strings/rewriting/c-folded}]
            &&
        \end{align}
        By functoriality, since \(
            \foldinterfaces[\iltikzfig{strings/category/f}[F][white]]
            =
            \iltikzfig{strings/rewriting/l-folded}
            \seq
            \iltikzfig{strings/rewriting/c-folded}
        \) then \[
            \cospan{0}{G}{m+n} = \cospan{0}{L}{i+j} \seq \cospan{i+j}{C}{m+n}.
        \]
        Cospan composition is pushout, so \(\cospan{L}{G}{C}\) is a pushout.
        Using the same reasoning, \(\cospan{R}{G}{C}\) is also a pushout: this gives
        us the DPO diagram.
        All that remains is to check that the aforementioned pushouts are traced
        boundary complements: this follows by inspecting components.

        Now the \(\ifdir\) direction: assume we have a DPO diagram (\ref{gath:dpo})
        where \(L \leftarrow i + j\), \(i + j \rightarrow R\), \(m + n \to G\) and
        \(m + n \to H\) are defined as in (\ref{gath:l-cospan}-\ref{gath:h-cospan})
        above.
        We must show that \(
            \iltikzfig{strings/category/f}[F][white]
            =
            \iltikzfig{strings/rewriting/rewrite-l}
        \) and \(
            \iltikzfig{strings/category/f}[H][white]
            =
            \iltikzfig{strings/rewriting/rewrite-r}
        \).
        By definition of traced boundary complement \(\cospan{j+m}{C}{i+n}\) is a
        partial monogamous cospan, so by fullness of \(\termandfrobtohypsigma\),
        there exists a term \(\iltikzfig{strings/rewriting/c} \in \stmcsigma\) such
        that \(
            \termandfrobtohypsigma[\iltikzfig{strings/rewriting/c}]
            =
            \cospan{j+m}{C}{i+n}
        \).
        By traced decomposition (\cref{lem:decomposition}), we have that for any
        traced boundary complement \(\cospan{i+j}{C}{m+n}\) and morphism
        \(L \to G\), \(\cospan{m}{G}{n}\) can be factored as in
        (\ref{gath:decomposition}), i.e.\ \[
            \termandfrobtohypsigma[\iltikzfig{strings/category/f}[F][white]]
            =
            \trace{j}{\termandfrobtohypsigma[\iltikzfig{strings/rewriting/l}]
            \tensor
            \id[n]
            \seq
            \termandfrobtohypsigma[\iltikzfig{strings/rewriting/c}]}.
        \]
        So by functoriality, we have that \(
            \iltikzfig{strings/category/f}[F][white]
            =
            \iltikzfig{strings/rewriting/rewrite-l}
        \).
        The same reasoning follows for \(
            \iltikzfig{strings/category/f}[H][white]
            =
            \iltikzfig{strings/rewriting/rewrite-r}
        \).
    \end{proof}
}{}

    \input{sections/operational}
    \section{Conclusion, related and further work}

    \iftoggle{conf}{
        \bibliography{refs/refs}
    }{
        \printbibliography
    }
\end{document}