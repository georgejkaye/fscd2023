% !TeX root = main-arxiv.tex
% custom macros
\input{macros/letters}
\input{macros/sets}
\input{macros/graphs}
\input{macros/category}
\input{macros/proofs}
% !TeX root = ../main-conf.tex

\subsection{Digital circuits}
\label{sec:digital-circuits}

A traced monoidal theory with a comonoid structure that is of particular
interest to us is the \emph{local theory of sequential circuits} from
\cite[Sec. VI]{ghica2022compositional}.

\begin{definition}[Gate-level circuits]
    Let the monoidal theory of \emph{gate-level sequential circuits} be defined
    as \(
        (\generators[\mathbf{SCirc}], \equations[\mathbf{SCirc}])
    \), where \[
        \generators[\mathbf{SCirc}]
        :=
        \{
            \iltikzfig{circuits/components/gates/and},
            \iltikzfig{circuits/components/gates/or},
            \iltikzfig{circuits/components/gates/not},
            \iltikzfig{strings/structure/comonoid/copy}[comb],
            \iltikzfig{strings/structure/monoid/merge}[comb],
            \iltikzfig{strings/structure/comonoid/discard}[comb],
            \iltikzfig{circuits/components/values/v}[\belnapnone],
            \iltikzfig{circuits/components/values/v}[\belnaptrue],
            \iltikzfig{circuits/components/values/v}[\belnapfalse],
            \iltikzfig{circuits/components/values/v}[\belnapboth],
            \iltikzfig{circuits/components/waveforms/delay}
        \}
    \] and the equations of \(
        \equations[\mathbf{SCirc}]
    \) are listed in \cref{app:equations}, \cref{fig:circuit-equations}, where
    \(
        \gateinterpretation
    \) maps gates to the corresponding truth table in \cref{app:belnap},
    \(\ljoin\) is the join in the information lattice in \cref{app:belnap}, and
    \(
        \iltikzfig{circuits/components/circuits/f-1-2}[F^n][comb][m][x][n]
    \) is defined inductively as \(
        \iltikzfig{circuits/instant-feedback/f0-box}
        :=
        \iltikzfig{circuits/instant-feedback/f0-definition}
    \) and \(
        \iltikzfig{circuits/instant-feedback/fkp1-box}
        :=
        \iltikzfig{circuits/instant-feedback/fkp1-definition}
    \).
\end{definition}

The generators in \(\generators[\mathbf{SCirc}]\) are, respectively:
\(\andgate\), \(\orgate\) and \(\notgate\) gates; constructs for forking,
joining and stubbing wires; \emph{values} representing no signal, a true signal,
a false signal, and both signals at once; and a delay of one unit of time.


The equations of \(\equations[\mathbf{SCirc}]\) contain the equations of a
commutative comonoid, so this is a perfect use case for rewriting modulo
trace commutative comonoid structure.
Using graph rewriting, we can sketch out an \emph{operational semantics} for
sequential circuits.
For the interests of brevity, we will only consider circuits of the form \(
    \iltikzfig{circuits/productivity/mealy-form-verbose}
\): circuits with no `non-delay-guarded feedback' in which the registers of the
circuit have been isolated from a core \(
    \iltikzfig{strings/category/f-2-2}[F][comb]
\) containing only `blue' (\emph{combinational}) components, which models a
function.

We can `apply' such a circuit to an input as shown in the left-hand side of
\cref{fig:cycle}; \cite[Thm. 104]{ghica2022compositional} shows that the
equations in \(\equations[\mathbf{SCirc}]\) can be used to derive the right-hand
side.
The equations \eqref{eq:gate}, \eqref{eq:fork}, \eqref{eq:join}, \eqref{eq:stub}
can then be applied to reduce the two `new' cores down to values, which
represent the output and new state of the circuit.

When the circuits are interpreted as hypergraphs and the equations as rewrites,
it would be possible for a computer to perform this sequence of rewrites to
evaluate circuits while being able to `peek inside' and see what is going on.

\begin{remark}
    This is another framework which would benefit from a way of formalising
    subgraphs in rewrite rules.
\end{remark}

\begin{figure*}
    \centering
    \begin{equation*}
        \tag{\(\mathsf{Cycle}\)}
        \iltikzfig{circuits/productivity/productive-lhs-verbose}[F][s][v]
        =
        \iltikzfig{circuits/productivity/productive-step-9}
        \label{eq:cycle}
    \end{equation*}
    \caption{
        The cycle equation, which is derivable from the equations in
        \(\equations[\mathbf{SCirc}]\)
    }
    \label{fig:cycle}
\end{figure*}


\newcommand{\concat}{\mathbin{+\hspace{-3pt}+}}
\newcommand{\stmcequationslink}{
    \hyperref[fig:stmc-axioms]{\stmcequations}
}
\newcommand{\productiveequationsdelaylink}{
    \hyperref[def:productive-equations-delay]{\productiveequationsdelay}
}\newcommand{\productiveequationslink}{
    \hyperref[def:productive-equations]{\productiveequations}
}
\newcommand{\combinationalequationslink}{
    \hyperref[lem:combinational-equations]{\combinationalequations}
}
\newcommand{\reductiveequationslink}{
    \hyperref[def:reductive-equations]{\reductiveequations}
}
\newcommand{\mealyequationslink}{
    \hyperref[def:structural-equations]{\mealyequations}
}
\newcommand{\abstractionequationslink}{
    \hyperref[def:abstraction-equations]{\abstractionequations}
}
\newcommand{\bialgebraequationslink}{
    \hyperref[fig:bialgebra-axioms]{\bialgebraequations}
}
\newcommand{\localequationslink}{
    \hyperref[fig:bialgebra-axioms]{\localequations}
}
\newcommand{\unfoldingeqn}{
    \begin{equation}
        \tag{\(\mathsf{UF}\)}
        \iltikzfig{strings/traced/trace-rhs}[F][seq][m][n][x]
        =
        \iltikzfig{circuits/examples/reasoning/unfolding/unfolding-3}
        \label{eq:unfolding}
    \end{equation}
}
\newcommand{\streamingeqn}{
    \begin{equation}
        \tag{\(\mathsf{Str}\)}
        \iltikzfig{circuits/axioms/streaming-lhs-verbose}[g][v][m]
        =
        \iltikzfig{circuits/axioms/streaming-rhs}[g][v][n]
        \label{eq:streaming}
    \end{equation}
}
\newcommand{\genstreamingeqn}{
    \begin{equation*}
        \tag{\(\mathsf{GStr}\)}
        \iltikzfig{circuits/axioms/generalised-streaming-lhs}
        =
        \iltikzfig{circuits/axioms/generalised-streaming-rhs}
        \label{eq:generalised-streaming}
    \end{equation*}
}
\newcommand{\bisimulationequationcontent}{
    \begin{equation}
        \tag{\(\bisimulationequation\)}
        \left(
            \forall (\overline{r}, \overline{u}) \in B(
                \iltikzfig{circuits/components/circuits/f-2-2}[F][comb],
                \overline{s},
                \iltikzfig{circuits/components/circuits/f-2-2}[G][comb],
                \overline{t},
            ), \forall \overline{v} \in \valuetuple{m}.
            \quad
            \iltikzfig{circuits/full-abstraction/output}[F][r][v]
            \eqaxioms
            \iltikzfig{circuits/full-abstraction/output}[G][u][v]
        \right)
        \quad
        \Rightarrow
        \quad
        \iltikzfig{circuits/productivity/mealy-form}[F][s]
        =
        \iltikzfig{circuits/productivity/mealy-form}[G][t]
        \label{eq:bisimulation}
    \end{equation}
}
\newcommand{\instantfeedbackeqn}{
    \begin{equation}
        \tag{\(\instantfeedbackequation\)}
        \iltikzfig{circuits/instant-feedback/equation-lhs}[F][m][n][x]
        =
        \iltikzfig{circuits/instant-feedback/fixpoint-concrete}
        \label{eq:instant-feedback}
    \end{equation}
}
\newcommand{\disconnecteqn}{
    \begin{equation}
        \tag{\(\mathsf{Disc}\)}
        \iltikzfig{circuits/axioms/bottom-delay-lhs}
        =
        \iltikzfig{circuits/axioms/bottom-delay-rhs}
        \label{eq:disconnect}
    \end{equation}
}
\newcommand{\mealyformequations}{
    \begin{minipage}[b]{0.3\textwidth}
        \disconnecteqn
    \end{minipage}%
    \begin{minipage}[b]{0.3\textwidth}
        \begin{equation}\textbf{}
            \tag{\(\mathsf{Unit}_l\)}
            \iltikzfig{strings/structure/monoid/unitality-l-lhs}
            =
            \iltikzfig{strings/structure/monoid/unitality-l-rhs}
            \label{eq:mealy-monoid-unitality-l}
        \end{equation}
    \end{minipage}%
    \begin{minipage}[b]{0.3\textwidth}
        \begin{equation}
            \tag{\(\mathsf{Unit}_r\)}
            \iltikzfig{strings/structure/monoid/unitality-r-lhs}
            =
            \iltikzfig{strings/structure/monoid/unitality-r-rhs}
            \label{eq:mealy-monoid-unitality-r}
        \end{equation}
    \end{minipage}
}
\newcommand{\forkgateeqn}{
    \begin{equation}
        \tag{\(\mathsf{GFork}\)}
        \iltikzfig{circuits/axioms/fork-gate-lhs}
        =
        \iltikzfig{circuits/axioms/fork-gate-rhs}
        \label{eq:fork-gate}
    \end{equation}
}
\newcommand{\stubgateeqn}{
    \begin{equation}
        \tag{\(\mathsf{GStub}\)}
        \iltikzfig{circuits/axioms/gate-stub-lhs}
        =
        \iltikzfig{circuits/axioms/gate-stub-rhs}
        \label{eq:stub-gate}
    \end{equation}
}
\newcommand{\forkdelayeqn}{
    \begin{equation}
        \tag{\(\mathsf{DFork}\)}
        \iltikzfig{circuits/axioms/delay-fork-lhs}
        =
        \iltikzfig{circuits/axioms/delay-fork-rhs}
        \label{eq:fork-delay}
    \end{equation}
}
\newcommand{\joindelayeqn}{
    \begin{equation}
        \tag{\(\mathsf{DJoin}\)}
        \iltikzfig{circuits/axioms/delay-join-lhs}
        =
        \iltikzfig{circuits/axioms/delay-join-rhs}
        \label{eq:join-delay}
    \end{equation}
}
\newcommand{\stubdelayeqn}{
    \begin{equation}
        \tag{\(\mathsf{DStub}\)}
        \iltikzfig{circuits/axioms/unobservable-lhs}
        =
        \iltikzfig{circuits/axioms/unobservable-rhs}
        \label{eq:stub-delay}
    \end{equation}
}
\newcommand{\cycleeqn}{
    \begin{equation}
        \tag{\(\mathsf{Cycle}\)}
        \iltikzfig{circuits/productivity/productive-lhs}[F][s][v][m][n]
        =
        \iltikzfig{circuits/productivity/productive-step-9}[F][s][v][m][n]
        \label{eq:cycle}
    \end{equation}
}
\newcommand{\cartesiancopyeqn}{
    \begin{equation}
        \tag{\(\mathsf{Copy}\)}
        \iltikzfig{strings/structure/cartesian/naturality-copy-lhs}[F][seq][m][n]
        =
        \iltikzfig{strings/structure/cartesian/naturality-copy-rhs}[F][seq][m][n]
        \label{eq:cartesian-naturality-copy}
    \end{equation}
}
\newcommand{\joinforkinverseeqn}{
    \begin{equation}
        \tag{\(\mathsf{JF}\)}
        \iltikzfig{circuits/axioms/join-fork-inverses-lhs}
        =
        \iltikzfig{circuits/axioms/join-fork-inverses-rhs}
        \label{eq:join-fork-inverse}
    \end{equation}
}
\newcommand{\forkjoininverseeqn}{
    \begin{equation}
        \tag{\(\mathsf{FJ}\)}
        \iltikzfig{strings/structure/frobenius/copy-merge-lhs}
        =
        \iltikzfig{strings/structure/frobenius/copy-merge-rhs}
        \label{eq:fork-join-inverse}
    \end{equation}
}
\newcommand{\delaydiscardeqn}{
    \begin{equation}
        \tag{\(\mathsf{UDelay}\)}
        \iltikzfig{circuits/axioms/delay-discard-lhs}[F][m][x]
        =
        \iltikzfig{circuits/axioms/delay-discard-rhs}[m]
        \label{eq:delay-discard}
    \end{equation}
}
\newcommand{\mealyeqn}{
    \begin{equation}
        \tag{\(\mathsf{Mealy}\)}
        \iltikzfig{circuits/productivity/trace-delay}[F][v][m][n][x][y][z]
        =
        \iltikzfig{circuits/productivity/mealy-form/mealy-form-4}
        \label{eq:mealy}
    \end{equation}
}

% !TeX root = ../main-conf.tex

\subsection{Digital circuits}
\label{sec:digital-circuits}

A traced monoidal theory with a comonoid structure that is of particular
interest to us is the \emph{local theory of sequential circuits} from
\cite[Sec. VI]{ghica2022compositional}.

\begin{definition}[Gate-level circuits]
    Let the monoidal theory of \emph{gate-level sequential circuits} be defined
    as \(
        (\generators[\mathbf{SCirc}], \equations[\mathbf{SCirc}])
    \), where \[
        \generators[\mathbf{SCirc}]
        :=
        \{
            \iltikzfig{circuits/components/gates/and},
            \iltikzfig{circuits/components/gates/or},
            \iltikzfig{circuits/components/gates/not},
            \iltikzfig{strings/structure/comonoid/copy}[comb],
            \iltikzfig{strings/structure/monoid/merge}[comb],
            \iltikzfig{strings/structure/comonoid/discard}[comb],
            \iltikzfig{circuits/components/values/v}[\belnapnone],
            \iltikzfig{circuits/components/values/v}[\belnaptrue],
            \iltikzfig{circuits/components/values/v}[\belnapfalse],
            \iltikzfig{circuits/components/values/v}[\belnapboth],
            \iltikzfig{circuits/components/waveforms/delay}
        \}
    \] and the equations of \(
        \equations[\mathbf{SCirc}]
    \) are listed in \cref{app:equations}, \cref{fig:circuit-equations}, where
    \(
        \gateinterpretation
    \) maps gates to the corresponding truth table in \cref{app:belnap},
    \(\ljoin\) is the join in the information lattice in \cref{app:belnap}, and
    \(
        \iltikzfig{circuits/components/circuits/f-1-2}[F^n][comb][m][x][n]
    \) is defined inductively as \(
        \iltikzfig{circuits/instant-feedback/f0-box}
        :=
        \iltikzfig{circuits/instant-feedback/f0-definition}
    \) and \(
        \iltikzfig{circuits/instant-feedback/fkp1-box}
        :=
        \iltikzfig{circuits/instant-feedback/fkp1-definition}
    \).
\end{definition}

The generators in \(\generators[\mathbf{SCirc}]\) are, respectively:
\(\andgate\), \(\orgate\) and \(\notgate\) gates; constructs for forking,
joining and stubbing wires; \emph{values} representing no signal, a true signal,
a false signal, and both signals at once; and a delay of one unit of time.


The equations of \(\equations[\mathbf{SCirc}]\) contain the equations of a
commutative comonoid, so this is a perfect use case for rewriting modulo
trace commutative comonoid structure.
Using graph rewriting, we can sketch out an \emph{operational semantics} for
sequential circuits.
For the interests of brevity, we will only consider circuits of the form \(
    \iltikzfig{circuits/productivity/mealy-form-verbose}
\): circuits with no `non-delay-guarded feedback' in which the registers of the
circuit have been isolated from a core \(
    \iltikzfig{strings/category/f-2-2}[F][comb]
\) containing only `blue' (\emph{combinational}) components, which models a
function.

We can `apply' such a circuit to an input as shown in the left-hand side of
\cref{fig:cycle}; \cite[Thm. 104]{ghica2022compositional} shows that the
equations in \(\equations[\mathbf{SCirc}]\) can be used to derive the right-hand
side.
The equations \eqref{eq:gate}, \eqref{eq:fork}, \eqref{eq:join}, \eqref{eq:stub}
can then be applied to reduce the two `new' cores down to values, which
represent the output and new state of the circuit.

When the circuits are interpreted as hypergraphs and the equations as rewrites,
it would be possible for a computer to perform this sequence of rewrites to
evaluate circuits while being able to `peek inside' and see what is going on.

\begin{remark}
    This is another framework which would benefit from a way of formalising
    subgraphs in rewrite rules.
\end{remark}

\begin{figure*}
    \centering
    \begin{equation*}
        \tag{\(\mathsf{Cycle}\)}
        \iltikzfig{circuits/productivity/productive-lhs-verbose}[F][s][v]
        =
        \iltikzfig{circuits/productivity/productive-step-9}
        \label{eq:cycle}
    \end{equation*}
    \caption{
        The cycle equation, which is derivable from the equations in
        \(\equations[\mathbf{SCirc}]\)
    }
    \label{fig:cycle}
\end{figure*}

% !TeX root = ../main-conf.tex

\subsection{Digital circuits}
\label{sec:digital-circuits}

A traced monoidal theory with a comonoid structure that is of particular
interest to us is the \emph{local theory of sequential circuits} from
\cite[Sec. VI]{ghica2022compositional}.

\begin{definition}[Gate-level circuits]
    Let the monoidal theory of \emph{gate-level sequential circuits} be defined
    as \(
        (\generators[\mathbf{SCirc}], \equations[\mathbf{SCirc}])
    \), where \[
        \generators[\mathbf{SCirc}]
        :=
        \{
            \iltikzfig{circuits/components/gates/and},
            \iltikzfig{circuits/components/gates/or},
            \iltikzfig{circuits/components/gates/not},
            \iltikzfig{strings/structure/comonoid/copy}[comb],
            \iltikzfig{strings/structure/monoid/merge}[comb],
            \iltikzfig{strings/structure/comonoid/discard}[comb],
            \iltikzfig{circuits/components/values/v}[\belnapnone],
            \iltikzfig{circuits/components/values/v}[\belnaptrue],
            \iltikzfig{circuits/components/values/v}[\belnapfalse],
            \iltikzfig{circuits/components/values/v}[\belnapboth],
            \iltikzfig{circuits/components/waveforms/delay}
        \}
    \] and the equations of \(
        \equations[\mathbf{SCirc}]
    \) are listed in \cref{app:equations}, \cref{fig:circuit-equations}, where
    \(
        \gateinterpretation
    \) maps gates to the corresponding truth table in \cref{app:belnap},
    \(\ljoin\) is the join in the information lattice in \cref{app:belnap}, and
    \(
        \iltikzfig{circuits/components/circuits/f-1-2}[F^n][comb][m][x][n]
    \) is defined inductively as \(
        \iltikzfig{circuits/instant-feedback/f0-box}
        :=
        \iltikzfig{circuits/instant-feedback/f0-definition}
    \) and \(
        \iltikzfig{circuits/instant-feedback/fkp1-box}
        :=
        \iltikzfig{circuits/instant-feedback/fkp1-definition}
    \).
\end{definition}

The generators in \(\generators[\mathbf{SCirc}]\) are, respectively:
\(\andgate\), \(\orgate\) and \(\notgate\) gates; constructs for forking,
joining and stubbing wires; \emph{values} representing no signal, a true signal,
a false signal, and both signals at once; and a delay of one unit of time.


The equations of \(\equations[\mathbf{SCirc}]\) contain the equations of a
commutative comonoid, so this is a perfect use case for rewriting modulo
trace commutative comonoid structure.
Using graph rewriting, we can sketch out an \emph{operational semantics} for
sequential circuits.
For the interests of brevity, we will only consider circuits of the form \(
    \iltikzfig{circuits/productivity/mealy-form-verbose}
\): circuits with no `non-delay-guarded feedback' in which the registers of the
circuit have been isolated from a core \(
    \iltikzfig{strings/category/f-2-2}[F][comb]
\) containing only `blue' (\emph{combinational}) components, which models a
function.

We can `apply' such a circuit to an input as shown in the left-hand side of
\cref{fig:cycle}; \cite[Thm. 104]{ghica2022compositional} shows that the
equations in \(\equations[\mathbf{SCirc}]\) can be used to derive the right-hand
side.
The equations \eqref{eq:gate}, \eqref{eq:fork}, \eqref{eq:join}, \eqref{eq:stub}
can then be applied to reduce the two `new' cores down to values, which
represent the output and new state of the circuit.

When the circuits are interpreted as hypergraphs and the equations as rewrites,
it would be possible for a computer to perform this sequence of rewrites to
evaluate circuits while being able to `peek inside' and see what is going on.

\begin{remark}
    This is another framework which would benefit from a way of formalising
    subgraphs in rewrite rules.
\end{remark}

\begin{figure*}
    \centering
    \begin{equation*}
        \tag{\(\mathsf{Cycle}\)}
        \iltikzfig{circuits/productivity/productive-lhs-verbose}[F][s][v]
        =
        \iltikzfig{circuits/productivity/productive-step-9}
        \label{eq:cycle}
    \end{equation*}
    \caption{
        The cycle equation, which is derivable from the equations in
        \(\equations[\mathbf{SCirc}]\)
    }
    \label{fig:cycle}
\end{figure*}


\begin{document}

    \maketitle

    \begin{abstract}
        In this paper we adapt previous work on rewriting string diagrams using
        hypergraphs to the case where the underlying category has a
        \emph{traced comonoid structure}, in which wires can be forked and
        the outputs of a morphism can be connected to its input.
        Such a structure is particularly interesting because any traced
        Cartesian (dataflow) category has an underlying traced comonoid
        structure.
        We show that certain subclasses of hypergraphs are fully complete
        for traced comonoid categories: that is to say, every term in such a
        category has a unique corresponding hypergraph up to isomorphism, and
        from every hypergraph with the desired properties, a unique term in the
        category can be retrieved up to the axioms of traced comonoid
        categories.
        We also show how the framework of double pushout rewriting (DPO) can be
        adapted for traced comonoid categories by characterising the valid
        pushout complements for rewriting in our setting.
        We conclude by presenting a case study in the form of recent work on
        an equational theory for \emph{sequential circuits}: circuits built from
        primitive logic gates with delay and feedback.
        The graph rewriting framework allows for the definition of an
        \emph{operational semantics} for sequential circuits.
    \end{abstract}

    \section{Introduction}
    % !TeX root = ../main-arxiv.tex
\section{Monoidal theories and hypergraphs}

When modelling a system using monoidal categories, its components and
properties are specified using a \emph{monoidal theory}.
A class of SMCs particularly interesting to us is that of
\emph{PROPs}~\cite{maclane1965categorical} (`categories of \emph{PRO}ducts and
\emph{P}ermutations'), which have natural numbers as objects and addition as
tensor product.

\begin{definition}[Symmetric monoidal theory]
    A \emph{(single-sorted) symmetric monoidal theory} (SMT) is a tuple \(
        (\generators,\equations)
    \) where \(\generators\) is a set of \emph{generators} with associated
    (co)arities in \(\dom,\cod \in \nat\) and \(\equations\) is a set of
    equations.
    Given a SMT \((\generators,\equations)\), let \(
        \smc{\generators}
    \) be the symmetric monoidal category freely generated over \(\generators\)
    and let \(
        \smc{\generators,\equations}
    \) be \(\smc{\generators}\) quotiented by the equations in \(\equations\).
    We write \(\smc{} := \smc{\emptyset}\) for the SMC with terms constructed
    solely from identities and symmetries.
\end{definition}

\begin{remark}
    It is important to distinguish between the \emph{syntactic} category
    \(\smc{\generators}\) and the \emph{semantic} category
    \(\smc{\generators,\equations}\).
    In the former, only `structural' equalities of the axioms of SMCs hold:
    moving boxes around while retaining connectivity.
    In the latter, more equations hold that mean terms with completely different
    boxes and connectivity can be potentially equal.
\end{remark}

\begin{remark}
    One can also define a \emph{multi-sorted} SMT, in which wires can be of
    multiple colours.
    For brevity, we will only consider the single-sorted case, but the results
    generalise easily using the results
    of~\cite{bonchi2022string,bonchi2022stringa}.
\end{remark}

While one could reason in \(\smc{\generators}\) using the
one-dimensional categorical term language, it is more intuitive to reason with
\emph{string diagrams}~\cite{joyal1991geometry,selinger2011survey}, which
represent \emph{equivalence classes} of terms up to the axioms of SMCs.
In the language of string diagrams, a generator \(\morph{\phi}{m}{n}\) is drawn
as a box \(
    \iltikzfig{strings/category/generator}[\phi][white][m][n]
\), the identity \(\id[x]\) as \(
    \iltikzfig{strings/category/identity}[white][x]
\), and the symmetry \(\swap{X}{Y}\) as \(
    \iltikzfig{strings/symmetric/symmetry}[white][x][y]
\).
Composite terms will be illustrated as wider boxes \(
    \iltikzfig{strings/category/f}[f][white][m][n]
\) to distinguish them from generators: then (diagrammatic order) composition
\(f \seq g\) is defined as horizontal juxtaposition \(
    \iltikzfig{strings/category/composition}[f][g][white][x][y][z]
\) and tensor \(f \tensor g\) as vertical juxtaposition \(
    \iltikzfig{strings/monoidal/tensor}[f][g][white][x][y][z][w]
\).

\begin{example}\label{ex:frobenius}
    The monoidal theory of
    \emph{special commutative Frobenius algebras} is defined as \(
        (\generators[\frob], \equations[\frob])
    \) where \(
        \generators[\frob] := \{
            \iltikzfig{strings/structure/monoid/merge}[white],
            \iltikzfig{strings/structure/monoid/init}[white]
            \iltikzfig{strings/structure/comonoid/copy}[white],
            \iltikzfig{strings/structure/comonoid/discard}[white]
        \}
    \) and the equations of \(\equations[\frob]\) are listed in
    \iftoggle{conf}{\cref{app:equations},}{}
    \cref{fig:monoid-equations,fig:comonoid-equations,fig:frobenius-equations}.
    We write \(\frob := \smc{\generators[\frob], \equations[\frob]}\).
\end{example}

\iftoggle{conf}{}{
    \begin{figure}
    \centering
    \begin{minipage}{0.21\textwidth}
        \begin{equation}
            \tag{\(\mathsf{M1}\)}
            \iltikzfig{strings/structure/monoid/unitality-l-lhs}
            =
            \iltikzfig{strings/structure/monoid/unitality-l-rhs}
            \label{eq:monoid-unitality-l}
        \end{equation}
    \end{minipage}
    \begin{minipage}{0.26\textwidth}
        \begin{equation}
            \tag{\(\mathsf{M2}\)}
            \iltikzfig{strings/structure/monoid/associativity-lhs}
            =
            \iltikzfig{strings/structure/monoid/associativity-rhs}
            \label{eq:monoid-associativity}
        \end{equation}
    \end{minipage}
    \begin{minipage}{0.26\textwidth}
        \begin{equation}
            \tag{\(\mathsf{M3}\)}
            \iltikzfig{strings/structure/monoid/commutativity-lhs}
            =
            \iltikzfig{strings/structure/monoid/commutativity-rhs}
            \label{eq:monoid-commutativity}
        \end{equation}
    \end{minipage}
    \caption{Equations \(\equations[\cmon]\) of a \emph{commutative monoid}.}
    \label{fig:monoid-equations}
\end{figure}
    \begin{figure}
    \centering
    \begin{minipage}{0.21\textwidth}
        \begin{equation}
            \tag{\(\mathsf{C1}\)}
            \iltikzfig{strings/structure/comonoid/unitality-l-lhs}
            =
            \iltikzfig{strings/structure/comonoid/unitality-l-rhs}
            \label{eq:comonoid-unitality-l}
        \end{equation}
    \end{minipage}
    \begin{minipage}{0.26\textwidth}
        \begin{equation}
            \tag{\(\mathsf{C2}\)}
            \iltikzfig{strings/structure/comonoid/associativity-lhs}
            =
            \iltikzfig{strings/structure/comonoid/associativity-rhs}
            \label{eq:comonoid-associativity}
        \end{equation}
    \end{minipage}
    \begin{minipage}{0.26\textwidth}
        \begin{equation}
            \tag{\(\mathsf{C3}\)}
            \iltikzfig{strings/structure/comonoid/commutativity-lhs}
            =
            \iltikzfig{strings/structure/comonoid/commutativity-rhs}
            \label{eq:comonoid-commutativity}
        \end{equation}
    \end{minipage}
    \caption{Equations \(\equations[\ccomon]\) of a \emph{commutative comonoid}.}
    \label{fig:comonoid-equations}
\end{figure}
    \begin{figure}[t]
    \centering
    \iltikzfig{strings/structure/frobenius/frobenius-l}[X]
    \(=\)
    \iltikzfig{strings/structure/bialgebra/merge-copy-lhs}[X]
    \quad
    \iltikzfig{strings/structure/frobenius/frobenius-r}[X]
    \(=\)
    \iltikzfig{strings/structure/bialgebra/merge-copy-lhs}[X]
    \quad
    \iltikzfig{strings/structure/frobenius/copy-merge-lhs}[X]
    \(=\)
    \iltikzfig{strings/structure/frobenius/copy-merge-rhs}[X]
    \caption{
        Equations \(\equations[\frob]\) of a
        \emph{special commutative Frobenius algebra}, in addition to those in
        \cref{fig:monoid-equations,fig:comonoid-equations}.
    }
    \label{fig:frobenius-equations}
\end{figure}
}

Reasoning equationally using string diagrams is certainly attractive
as a pen-and-paper method, but for larger systems it quickly becomes intractible
to do this by hand.
Instead, it is desirable to perform equational reasoning \emph{computationally}.
Unfortunately, string diagrams as topological objects are not particularly
suited for this purpose; instead, we require a combinatorial representation.
Fortunately, this has been well studied
recently, first with
\emph{string graphs}~\cite{dixon2013opengraphs,kissinger2012pictures}
and later with
\emph{hypergraphs}~\cite{bonchi2022string,bonchi2022stringa,bonchi2022stringb},
a generalisation of regular graphs in which edges can be the source or target of
an arbitrary number of vertices.
In this paper we are concerned with the latter.

Hypergraphs are formally defined as objects in a functor category.

\begin{definition}[Hypergraph]
    Let \(\mathbf{X}\) be the category containing objects \((k, l)\) for
    \(k, l \in \nat\) and one additional object \(\star\).
    For each \((k, l)\) there are \(k + l\) morphisms \((k, l) \to \star\).
    Let \(\hyp\) be the functor category \([\mathbf{X},\set]\).
\end{definition}

An object in \(\hyp\) maps \(\star\) to a set of vertices, and each \((k,l)\) to
a set of hyperedges with \(k\) sources and \(l\) targets,
Given a hypergraph \(F \in \hyp\), we write \(\vertices{F}\) for its set of
vertices and \(\edges{F}{k}{l}\) for the set of edges with \(k\) sources and
\(l\) targets.
A morphism of hypergraphs \(\morph{f}{F}{G} \in \hyp\) consists of functions
\(\vertices{f}\) and \(\edges{f}{k}{l}\) for each \(k,l \in \nat\) preserving
sources and targets in the obvious way.
With such morphisms a hypergraph can be \emph{labelled}.

\begin{definition}[Slice category~\cite{lawvere1963functorial}]
    For a category \(\mathbf{C}\) and an object \(C \in \mathbf{C}\), the
    \emph{slice category} \(\mathbb{C} / C\) is the category with objects the
    morphisms of \(\mathbf{C}\) with target \(C\), and morphisms \(
        (\morph{f}{X}{C}) \to (\morph{f^\prime}{X^\prime}{C})
    \) are the morphisms \(\morph{g}{X}{X^\prime} \in \mathbf{C}\) such that
    \(f^\prime \circ g = f\).
\end{definition}


\begin{definition}[Hypergraph signature~\cite{bonchi2022string}]
    For a given monoidal signature \(\signature\), its corresponding
    \emph{hypergraph signature} \(\hypsignature{\Sigma}\) is the hypergraph with
    edges \(
        e_\phi \in \edges{\hypsignature{\Sigma}}{\dom{\phi}}{\cod{\phi}}
    \) for each \(\phi \in \Sigma\), and a vertex \(v\).
    For a hyperedge \(e_\phi\), \(i < \dom{\phi}\) and \(j < \cod{\phi}\), \(
        \sources{i}(e_\phi) = \targets{j}(e_\phi) = v
    \).
\end{definition}

\begin{definition}[Labelled hypergraph~\cite{bonchi2022string}]
    For a monoidal signature \(\Sigma\), let the category \(\hypsigma\) be
    defined as the slice category \(\hyp / \hypsignature{\Sigma}\).
\end{definition}

While (labelled) hypergraphs may have dangling vertices, they do not have
\emph{interfaces} specifying the order of inputs and outputs.
These can be provided using \emph{cospans}.

\begin{definition}[Categories of cospans~\cite{bonchi2022stringa}]\label{def:cospans}
    For a finitely cocomplete category \(\mathbf{C}\), a \emph{cospan} from
    \(X \to Y\) is a pair of arrows \(X \to A \leftarrow Y\).
    A \emph{cospan morphism} \(
        (\cospan{X}[f]{A}[g]{Y}) \to (\cospan{X}[h]{B}[k]{Y})
    \) is a morphism \(\morph{\alpha}{A}{B} \in \mathbf{C}\)%
    \iftoggle{conf}{
        such that \(\alpha \circ f = h\) and \(\alpha \seq g = k\).
    }{
        such that the following diagram commutes:
        \begin{center}
            \includestandalone{figures/graphs/cospans/morphism}
        \end{center}
    }

    Two cospans \(\cospan{X}{A}{Y}\) and \(\cospan{X}{B}{Y}\) are
    \emph{isomorphic} if there exists a morphism of cospans as above where
    \(\alpha\) is an isomorphism.%
    \iftoggle{conf}{
        Composition is by pushout.
    }{
        Composition is by pushout:

        \begin{center}
            \includestandalone{figures/graphs/cospans/composition}
        \end{center}
    }
    The identity is \(X \xrightarrow{\id[X]} X \xleftarrow{\id[X]} X\).
    The category of cospans over \(\mathbf{C}\), denoted \(\csp{\mathbf{C}}\)
    has as objects the objects of \(\mathbf{C}\) and as morphisms the
    isomorphism classes of cospans.
    This category has monoidal product given by the coproduct in \(\mathbf{C}\)
    with unit the initial object \(0 \in \mathbf{C}\).
\end{definition}

The interfaces of a hypergraph can be specified as cospans by having the `legs'
of the cospan pick vertices in the graph at the apex.

\begin{definition}[Discrete hypergraph]
    A hypergraph is called \emph{discrete} if it has no edges.
\end{definition}

A discrete hypergraph \(F\) with \(|\vertices{F}| = n\) is written as
\(n\) when clear from context.
Morphisms from discrete hypergraphs to a main graph pick out the vertices in the
interface: to assign an order to these vertices some more categorical machinery
is required.

\begin{theorem}[\cite{bonchi2022string}, Thm. 3.6]
    Let \(\mathbb{X}\) be a PROP whose monoidal product is a coproduct,
    \(\mathbf{C}\) a category with finite colimits, and \(
        \morph{F}{\mathbb{X}}{\mathbf{C}}
    \) a coproduct-preserving functor.
    Then there exists a PROP \(\csp[F]{\mathbf{C}}\) whose arrows \(m \to n\)
    are isomorphism classes of \(\mathbf{C}\) cospans \(\cospan{Fm}{C}{Fn}\).
\end{theorem}

\iftoggle{conf}{}{
    \begin{theorem}[\cite{bonchi2022string}, Thm. 3.8]
        \label{thm:cospan-homomorphism}
        Let \(\mathbb{X}\) be a PROP whose monoidal product is a coproduct,
        \(\mathbf{C}\) a category with finite colimits, and
        \(\morph{F}{\mathbb{X}}{\mathbf{C}}\) a coproduct-preserving functor.
        Then there is a homomorphism of PROPs \(
            \morph{\tilde{F}}{\csp{\mathbb{X}}}{\csp[F]{\mathbf{C}}}
        \) that sends \(\cospan{m}[f]{X}[g]{n}\) to \(\cospan{Fm}[Ff]{FX}[Fg]{Fn}\).
        If \(F\) is full and faithful, then \(\tilde{F}\) is faithful.
    \end{theorem}
}

\begin{definition}
    Let \(\finset\) be the PROP with morphisms \(m \to n\) the functions
    between finite sets \([m] \to [n]\).
\end{definition}

\begin{definition}[\cite{bonchi2022string}]
    Let \(\morph{D}{\finset}{\hypsigma}\) be the faithful, coproduct-preserving
    functor that sends each object \(m \in \finset\) to the discrete hypergraph
    \(m \in \hypsigma\) and each morphism to the induced homomorphism of
    discrete hypergraphs.
\end{definition}

From this we define the category \(\cspdhyp\) with objects
\emph{discrete cospans of hypergraphs}.
Since the legs of each cospan are discrete hypergraphs containing some number of
vertices, the objects of this category can be viewed as natural numbers, making
this another PROP.
    % !TeX root = ../main-conf.tex
\section{Hypergraphs for traced categories}

We wish to use the hypergraph framework for a setting with a \emph{trace}.

\begin{definition}[Symmetric traced monoidal category \cite{joyal1996traced,hasegawa2009traced}]
    A \emph{symmetric traced monoidal category} (STMC) is a symmetric monoidal
    category \(\mathbf{C}\) equipped with \iftoggle{conf}{a family of }{}
    functions \(
        \morph{
            \trace{X}[A][B]{-}
        }{
            \mathbf{C}(X \tensor A, X \tensor B)
        }{
            \mathbf{C}(A, B)
        }
    \) for any objects \(A,B\) and \(X\) satisfying the axioms of STMCs listed
    in \cref{fig:stmc-axioms}.
\end{definition}

\begin{figure}
    \centering
    \begin{minipage}{0.52\textwidth}
        \begin{align*}
            \tag{\(\mathsf{Tighten}\)}
            \iltikzfig{strings/traced/naturality-lhs}
            &=
            \iltikzfig{strings/traced/naturality-rhs}
            \label{eq:tightening}
            \\[0.75em]
            \tag{\(\mathsf{Slide}\)}
            \iltikzfig{strings/traced/sliding-lhs}
            &=
            \iltikzfig{strings/traced/sliding-rhs}
            \label{eq:sliding}
            \\[0.75em]
            \tag{\(\mathsf{Vanish}\)}
            \iltikzfig{strings/traced/vanishing-lhs}
            &=
            \iltikzfig{strings/traced/vanishing-rhs}
            \label{eq:vanishing}
        \end{align*}
    \end{minipage}
    \hspace{-1em}
    \begin{minipage}{0.455\textwidth}
        \begin{align*}
            \tag{\(\mathsf{Superpose}\)}
            \iltikzfig{strings/traced/superposing-lhs}
            &=
            \iltikzfig{strings/traced/superposing-rhs}
            \label{eq:superposing}
            \\[0.75em]
            \tag{\(\mathsf{Yank}\)}
            \iltikzfig{strings/traced/yanking-lhs}
            &=
            \iltikzfig{strings/traced/yanking-rhs}
            \label{eq:yanking}
        \end{align*}
    \end{minipage}
    \caption{
        Equations that hold in any \emph{symmetric traced monoidal category.}
    }
    \label{fig:stmc-axioms}
\end{figure}

In string diagrams, the trace is represented by joining output wires to input
wires:
%
\begin{center}
    \(
        \trace{X}[A][B]{\iltikzfig{strings/traced/trace-lhs}[f][white][X][A][B]}
        \stackrel{\text{def}}{=}
        \iltikzfig{strings/traced/trace-rhs}[f][white][A][B]
    \)
\end{center}

Traced monoidal categories are not the only kind of category in which wires can
`bend'.

\begin{definition}[Compact closed category]
    A \emph{compact closed category} (CCC) is a symmetric monoidal category in
    which every object \(X\) has a \emph{dual} \(\dual{X}\) equipped with
    morphisms called the \emph{unit} \(
        \iltikzfig{strings/compact-closed/cup}[white][\dual{X}][X]
    \) (`cup') and the \emph{counit} \(
        \iltikzfig{strings/compact-closed/cap}[white][X][\dual{X}]
    \) (`cap') satisfying the equations of CCCs listed in \cref{fig:ccc-axioms}.
\end{definition}

\begin{figure}
    \centering
    \begin{minipage}{0.29\textwidth}
        \begin{equation}
            \tag{\(\mathsf{CC1}\)}
            \iltikzfig{strings/compact-closed/snake-l}[X]
            =
            \iltikzfig{strings/compact-closed/snake-c}[X]
            \label{eq:snake-l}
        \end{equation}
    \end{minipage}
    \begin{minipage}{0.29\textwidth}
        \begin{equation}
            \tag{\(\mathsf{CC2}\)}
            \iltikzfig{strings/compact-closed/snake-r}[X]
            =
            \iltikzfig{strings/compact-closed/snake-c}[X]
            \label{eq:snake-r}
        \end{equation}
    \end{minipage}
    \caption{
        Equations that hold in any \emph{compact closed category}.
    }
    \label{fig:ccc-axioms}
\end{figure}

Dual objects are conventionally drawn as wires flowing the `other way', but in
this paper this is not necessary as all categories will be \emph{self-dual}: any
object \(X\) is isomorphic to \(\dual{X}\).

\begin{proposition}[Canonical trace (\cite{joyal1996traced}, Prop. 3.1)]
    \label{prop:canonical-trace}
    Any CCC has a trace \(
        \trace{X}[A][B]{\iltikzfig{strings/category/f-2-2}[f][white][X][A][X][B]}
    \) called the \emph{canonical trace}, defined for the self-dual case as \[
        \left(
            \iltikzfig{strings/compact-closed/cup}[white][X][X]
            \tensor
            \iltikzfig{strings/category/identity}[white][A]
        \right)
        \seq
        \left(
            \iltikzfig{strings/category/identity}[white][X]
            \tensor
            \iltikzfig{strings/category/f-2-2}[f][white][X][A][X][B]
        \right)
        \seq
        \left(
            \iltikzfig{strings/compact-closed/cap}[white][X][X]
            \tensor
            \iltikzfig{strings/category/identity}[white][B]
        \right).
    \]
\end{proposition}

The category of interfaced hypergraphs as defined in the previous section
already contains the structure necessary to define a trace.

\begin{definition}[Hypergraph category \cite{fong2019hypergraph}]
    A \emph{hypergraph category} is a symmetric monoidal category in which
    each object \(X\) has a special commutative Frobenius structure in the sense
    of \cref{ex:frobenius} satisfying the equations in
    \cref{fig:hypergraph-category}.
\end{definition}

\begin{figure}
    \centering
    \iltikzfig{strings/structure/hypergraph/monoid-resp-lhs}[X]
    \(=\)
    \iltikzfig{strings/structure/hypergraph/monoid-resp-rhs}[X]
    \quad
    \iltikzfig{strings/structure/hypergraph/comonoid-resp-lhs}[X]
    \(=\)
    \iltikzfig{strings/structure/hypergraph/comonoid-resp-rhs}[X]
    \quad
    \iltikzfig{strings/structure/hypergraph/unit-resp-lhs}[X]
    \(=\)
    \iltikzfig{strings/structure/hypergraph/unit-resp-rhs}[X]
    \quad
    \iltikzfig{strings/structure/hypergraph/counit-resp-lhs}[X]
    \(=\)
    \iltikzfig{strings/structure/hypergraph/counit-resp-rhs}[X]
    \caption{
        Equations \(\equations[\mathbf{Hyp}]\) of a
        \emph{hypergraph category}, in addition to those in
        \cref{fig:monoid-equations,fig:comonoid-equations,fig:frobenius-equations}.
    }
    \label{fig:hypergraph-category}
\end{figure}

\begin{proposition}[\cite{rosebrugh2005generic}]
    Any hypergraph category is self-dual compact closed.
\end{proposition}
\begin{proof}
    The cup is constructed as \(
        \iltikzfig{strings/structure/monoid/init}[white]
        \seq
        \iltikzfig{strings/structure/comonoid/copy}[white]
    \) and the cap as \(
        \iltikzfig{strings/structure/monoid/merge}[white]
        \seq
        \iltikzfig{strings/structure/comonoid/discard}[white]
    \).
\end{proof}

A generic `hypergraph category' should not be confused with the
category of hypergraphs \(\hyp\), which is not itself a hypergraph category.
However, the category of \emph{cospans} of hypergraphs is such a category.

\begin{proposition}[\cite{carboni1987cartesian,bonchi2022string}]\label{prop:frobenius-map}
    \(\cspdhyp\) is a hypergraph category.
\end{proposition}
\begin{proof}
    A Frobenius structure can be defined on \(\cspdhyp\) for each \(m \in \nat\)
    as follows:
    \begin{gather*}
        \iltikzfig{strings/structure/monoid/merge}[white]
        :=
        \cospan{m + m}{m}{m}
        \quad
        \iltikzfig{strings/structure/monoid/init}[white]
        :=
        \cospan{0}{m}{m}
        \\
        \iltikzfig{strings/structure/comonoid/copy}[white]
        :=
        \cospan{m}{m}{m+m}
        \quad
        \iltikzfig{strings/structure/comonoid/discard}[white]
        :=
        \cospan{m}{m}{0}
        \qedhere
    \end{gather*}
\end{proof}

\begin{corollary}
    \(\cspdhyp\) is compact closed.
\end{corollary}

\begin{corollary}
    \(\cspdhyp\) has a trace.
\end{corollary}

This means that a STMC freely generated over a signature faithfully embeds into
a CCC generated over the same signature, mapping the trace in the former to the
canonical trace in the latter.
However, this mapping is not \emph{full}: there are terms in a CCC that are not
terms in a STMC, such as \(\iltikzfig{strings/traced/invalid}\).
This means we must still restrict the cospans of hypergraphs in \(\cspdhyp\) we
use for \emph{traced} terms.

\subsection{Monogamy}

In~\cite{bonchi2016rewriting}, it is shown that terms in a (non-traced)
symmetric monoidal category are interpreted via a faithful functor into a
sub-PROP of \(\cspdhyp\).
One condition on this sub-PROP is that all hypergraphs are \emph{acyclic}.
Clearly, to model trace this condition must be dropped.

However, there is also another condition known as \emph{monogamy}: informally,
this means that every vertex has exactly one `in' and `out' connection, be it to
an edge or an interface.
For the most part, this condition also applies to the traced case: wires cannot
arbitrarily fork and join.
There is one nuance: the trace of the identity.
This is depicted as a closed loop \(
    \trace{1}{\iltikzfig{strings/category/identity}[white]}
    =
    \iltikzfig{strings/traced/trace-id}
\), and one might think that it can be discarded, i.e. \(
    \iltikzfig{strings/traced/trace-id}
    =
    \iltikzfig{strings/monoidal/empty}
\).
This is \emph{not} always the case, such as in the category of finite
dimensional vector spaces~\cite[Sec. 6.1]{hasegawa1997recursion}.
These closed loops must be represented in the hypergraph framework:
there is a natural representation as a lone vertex disconnected
from either interface.
In fact, this is exactly how the canonical trace applied to an identity is
interpreted in \(\cspdhyp\).

\begin{definition}
    For a hypergraph \(F \in \hyp\), the \emph{degree} of a vertex
    \(v \in \vertices{F}\) is a tuple \((i,o)\) where \(i\) is the number of
    pairs \((e,i)\) where \(e\) is a hyperedge with \(v\) as its \(i\)th target,
    and \(o\) is similarly the number of pairs \((e,j)\) where \(e\) is a
    hyperedge with \(v\) as its \(j\)th target.
\end{definition}

\begin{definition}
    For a cospan \(\cospan{m}[f]{F}[g]{n} \in \cspdhyp\), we say it is
    \emph{partial monogamous} if \(f\) and \(g\) are mono and, for all nodes
    \(v \in \vertices{F}\), the degree of \(v\) is

    \begin{center}
        \begin{tabular}{rlcrl}
            \((0,0)\)
            &
            if \(v \in f_\star \wedge v \in g_\star\)
            &
            \quad
            &
            \((0,1)\)
            &
            if \(v \in f_\star\)
            \\
            \((1,0)\)
            &
            if \(v \in g_\star\)
            &
            \quad
            &
            \((0,0)\)
            or \((1,1)\)
            &
            otherwise
        \end{tabular}
    \end{center}
\end{definition}

Intuitively, partial monogamy means that each vertex has exactly one `in' and
`out' connection, be it to an edge or to an interface.

\begin{figure}
    \centering
    \[
        \underbrace{
            \iltikzfig{graphs/monogamy/yes-comonoid-0}
            \iltikzfig{graphs/monogamy/yes-comonoid-1}
        }_{\text{partial monogamous}}
        \quad
        \underbrace{
            \iltikzfig{graphs/monogamy/no-comonoid-0}
            \iltikzfig{graphs/monogamy/no-comonoid-1}
        }_{\text{not partial monogamous}}
    \]
    \caption{Examples of cospans that are and are not partial monogamous.}
    \label{fig:partial-monogamous-examples}
\end{figure}

\begin{example}
    Examples of cospans that are and are not partial monogamous are shown
    in \cref{fig:partial-monogamous-examples}
\end{example}

\iftoggle{conf}{}{
    \begin{lemma}\label{lem:trace-degree}
        Given a cospan \(\cospan{x+m}[h+f]{F}[k+g]{x+n} \in \cspdhyp\), let \(p\) be
        the map sending vertices in \(\vertices{F}\) to the corresponding
        vertices after constructing \(\trace{x}{\cospan{x+m}{F}{x+n}}\).
        Then if the degree of \(h(i)\) is \((i_1, j_1)\) and the degree of
        \(k(i)\) is \((i_2, j_2)\), then the degree of \(p(h(i)) = p(k(i))\) is
        \((i_1 + i_2, j_1 + j_2)\).
    \end{lemma}
    \iftoggle{proofs}{
        \begin{proof}
            By computing the pushouts.
        \end{proof}
    }{}
}

In order to establish a correspondence between cospans of partial monogamous
hypergraphs, they need to be assembled into a sub-PROP of \(\cspdhyp\).

\begin{theorem}~\label{thm:partial-monogamous-ops}
    Let \(\cospan{m}{F}{n}\), \(\cospan{n}{G}{p}\), \(\cospan{p}{H}{q}\) and
    \(\cospan{x+m}{K}{x+n}\) be partial monogamous cospans in \(\cspdhyp\).
    Then,
    \begin{itemize}
        \item identities and symmetries are partial monogamous;
        \item \(\cospan{m}{F}{n} \seq \cospan{n}{G}{p}\) is partial monogamous;
        \item \(\cospan{m}{F}{n} \tensor \cospan{p}{H}{q}\) is partial
                monogamous; and
        \item \(\trace{x}{\cospan{x+m}{K}{x+n}} \) is partial monogamous.
    \end{itemize}
\end{theorem}
\iftoggle{proofs}{
    \begin{proof}
        Since any monogamous hypergraph is also partial monogamous, the first
        three points hold due to~\cite[Prop.16]{bonchi2022string}, dropping the
        acyclicity condition.
        For the final condition, consider the image of \(f\) and \(g\).
        For each \(i \in x\), there are two cases to consider: \(f(i) = g(i)\)
        and \(f(i) \neq g(i)\).

        In the former, the degree of \(v^\prime = f(v) = h(v)\) must be
        \((0,0)\) by definition of partial monogamicity.
        Therefore in the traced hypergraph \(
            \cospan{m}{K^\prime}{n}
        \), \(v^\prime\) will still have degree \((0,0)\) by
        \cref{lem:trace-degree}, and cannot be in the interfaces, so the
        hypergraph is still partial monogamous.

        In the latter case, \(f(i)\) must have degree of \((0,0)\) if it in the
        image of \(k\) or \((0,1)\) otherwise.
        Similarly \(g(i)\) either has degree \((0,0)\) or \((1,0)\).
        Let \(v := p(f(i)) = p(g(i))\); we now consider the degree of \(v\)
        computed using \cref{lem:trace-degree}:
        \begin{itemize}
            \item If \(f(i)\) is in the image of \(k\) and \(g(i)\) is in the
                    image of \(h\), then \(v\) has degree \((0,0)\).
                    \(v\) is in the image of \(h \seq p\) and \(h \seq p\), so
                    the cospan is partial monogamous.
            \item If \(f(i)\) is in the image of \(k\), then \(v\) has degree
                    \((1, 0)\); since \(v\) is in the image of \(k \seq p\), the
                    cospan is partial monogamous.
            \item If \(g(i)\) is in the image of \(h\), then \(v\) has degree
                    \((0, 1)\); since \(v\) is in the image of \(h \seq p\), the
                    cospan is partial monogamous.
            \item Otherwise, \(v\) will have degree \((1, 1)\), and is not in
                    the image of either interface so the cospan is partial
                    monogamous.
        \end{itemize}
    \end{proof}
}{
    \begin{proof}
        Since any monogamous hypergraph is also partial monogamous, the first
        three points hold due to~\cite[Prop.16]{bonchi2022string}, dropping the
        acyclicity condition.
        The final condition is routine by case analysis on the interfaces that
        each vertex is in the image of.
    \end{proof}
}

\begin{definition}
    Let \(\pmcspdhyp\) be the sub-PROP of \(\cspdhyp\) containing only the
    partial monogamous cospans of hypergraphs.
\end{definition}

Crucially, while we leave \(\pmcspdhyp\) in order to construct the trace using
the cup and cap, the resulting cospan \emph{is} in \(\pmcspdhyp\).

\subsection{From terms to graphs}

\begin{definition}
    For a SMT \((\generators,\equations)\), let
    \(\stmc{\generators}\) be the strict STMC freely generated over the
    generators in \(\generators\).
    Let \(\stmc{\generators, \equations}\) be \(\stmc{\generators}\) quotiented
    by equations in \(\equations\).
\end{definition}

A \emph{(traced) PROP morphism} is a strict (traced) symmetric monoidal functor
between PROPs.
For \(\pmcspfihyp\) to be suitable for reasoning with a traced category
\(\stmc{\Sigma}\) for some given signature, there must be a
\emph{fully complete} PROP morphism \(\stmc{\Sigma} \to \pmcspfihyp\): a full
and faithful functor from terms to cospans of hypergraphs.

We exploit the interplay between compact closed and traced categories in
order to reuse the existing PROP morphisms from~\cite{bonchi2022string} for the
traced case.
Since \(\smc{\Sigma}\) and \(\pmcspdhyp\) are freely generated, to define a
PROP morphism between them it suffices to define it on the generators in the
former.

\begin{definition}[\cite{bonchi2022string}]\label{def:hyp-morphisms}
    Let \(\morph{\termtohyp{\Sigma}}{\smc{\Sigma}}{\cspdhyp}\) be a PROP
    morphism defined as \begin{gather*}
        \termtohyp[\iltikzfig{strings/category/generator}[\phi][white][m][n]]{\Sigma}
            :=
            \cospan{m}{
                \iltikzfig{graphs/terms/generator}
            }{n}
        \\
        \termtohyp[\iltikzfig{strings/category/identity}[white][n][n]]{\Sigma}
        :=
        \cospan{n}[\id]{n}[\id]{n}
        \qquad
        \termtohyp[\iltikzfig{strings/symmetric/symmetry}[white][m][n]]{\Sigma}
            :=
        \cospan{m+n}[[\id,\id]]{m+n}[[\id,\id]]{n+m}
    \end{gather*}
    Let \(\morph{\frobtohyp{\Sigma}}{\frob}{\cspdhyp}\) be a PROP morphism
    defined as in \cref{prop:frobenius-map}.
    \iftoggle{conf}{
        Then, let \(
            \morph{\termandfrobtohypsigma}{\smc{\Sigma} + \frob}{\cspdhyp}
        \)
        be the copairing of \(\termtohyp{\Sigma}\) and \(\frobtohyp{\Sigma}\).
    }{
        Then, let \[
            \morph{\termandfrobtohypsigma}{\smc{\Sigma} + \frob}{\cspdhyp}
        \]
        be the copairing of \(\termtohyp{\Sigma}\) and \(\frobtohyp{\Sigma}\).
    }
\end{definition}

\begin{lemma}\label{lem:smc-core}
    Let \(\iltikzfig{strings/category/f}[f][white][m][n]\) be a term in
    \(\stmc{\Sigma}\).
    Then there exists at least one \(
        \iltikzfig{strings/category/f-2-2}[g][white][x][m][x][n] \in \smc{\Sigma}
    \) such that \(
        \trace{x}{\iltikzfig{strings/category/f-2-2}[g][white]}
        =
        \iltikzfig{strings/category/f}[f][white]
    \).
\end{lemma}
\iftoggle{proofs}{
    \begin{proof}
        Any trace can be made a `global trace' by applying \eqref{eq:tightening}
        and \eqref{eq:superposing}.
    \end{proof}
}{}

\begin{proposition}\label{prop:trace-to-sym-frob}
    There exists a faithful PROP morphism \(
        \morph{\tracedtosymandfrob{\Sigma}}{\stmc{\Sigma}}{\smc{\Sigma} + \frob}
    \).
\end{proposition}
\begin{proof}
    \cref{lem:smc-core} is used to isolate a term in \(\smc{\Sigma}\).
    The corresponding term in \(\smc{\Sigma} + \frob\) is then the canonical
    trace of this term.
    There may be many such terms in \(\smc{\Sigma}\), but the canonical trace
    being a trace means that any possible outcomes post-trace are all equal.
    The equations of \(\frob\) do not merge any morphisms
    since the only use of the generators of \(\frob\) is in the canonical trace,
    to which the Frobenius equations do not apply.
\end{proof}

Before progressing to the main theorem, we must show a result about terms in
\(\smc{}\), i.e.\ terms constructed from just symmetries and identities: there
is a correspondence between \(\smc{}\) and \emph{bijective} functions.

\begin{definition}
    Let \(\perms\) be the sub-PROP of \(\finset\) containing only the
    bijective functions.
\end{definition}

\begin{lemma}\label{lem:symmetries-prop}
    \(\smc{} \cong \perms\).
\end{lemma}
\iftoggle{proofs}{
    \begin{proof}
        The morphism \(\morph{\phi}{\smc{}}{\perms}\) is defined over
        generators in \(\smc{}\) as \[
            \phi(\iltikzfig{strings/monoidal/empty}) = \{\}
            \quad
            \phi(\iltikzfig{strings/category/identity}[white])
            =
            \{0 \mapsto 0\}
            \quad
            \phi(\iltikzfig{strings/symmetric/symmetry}[white])
            =
            \{0 \mapsto 1, 1 \mapsto 0\}
        \]
        Since any term in \(\smc{}\) can be expressed using these generators,
        this defines the complete transformation.

        The reverse morphism \(\morph{\psi}{\finset}{\smc{}}\) is inductively
        over the size of \(m\).
        For the base case \(\morph{f}{[0]}{[0]}\), let \(
            \phi(f) := \iltikzfig{strings/monoidal/empty}
        \).
        For \(
            \morph{f}{[k+1]}{[k+1]}
        \), let \(i\) such that \(f(i) = k+1\), and define the function \(
            \morph{f^\prime}{\nat_{k}}{\nat_{k}}
        \) as the function such that \(
            f^\prime(j) = f(j)
        \) if \(j < i\), and \(f(j+1)\) otherwise.
        Then \[
            \psi(f) := \iltikzfig{strings/symmetric/f-construction}.
        \]

        These are shown to be inverses by a simple induction in both directions.
    \end{proof}
}{}

\begin{lemma}\label{lem:monog-discrete-cospan}
    Given a monogamous cospan \(\cospan{m}[f]{m}[g]{m}\), there exists a unique
    term \(\morph{h}{m}{m} \in \smc{}\) up to the axioms of SMCs such that
    \(\termtohyp[h]{\Sigma} = \cospan{m}[f]{m}[g]{m}\).
\end{lemma}
\iftoggle{proofs}{
    \begin{proof}
        Since the cospan is monogamous, \(f\) and \(g\) are mono.
        As the cospan is also discrete, there exists a (unique) bijective
        function \(\morph{h^\prime}{[m]}{[m]}\) such that \(h^\prime(i) = j\) if
        \(f(i) = g(j)\).
        By \cref{lem:symmetries-prop}, there is a corresponding term
        \(h \in \smc{}\) that is unique up to SMC axioms: a simple induction
        shows that \(\termtohyp[h]{\Sigma} = \cospan{m}[f]{m}[g]{m}\).
    \end{proof}
}{}

\begin{theorem}\label{thm:termtohyp-image}
    A cospan \(\cospan{m}{F}{n}\) is in the image of \(
        \termandfrobtohypsigma \circ \tracedtosymandfrob{\Sigma}\) if
    and only if it is partial monogamous.
\end{theorem}
\iftoggle{proofs}{
    \begin{proof}
        % !TeX root = ../../main-conf.tex
To show that \(\termtohyp[f]{\Sigma}\) is partial monogamous for any
\(f \in \smc{\Sigma}\) we use induction on the structure of \(f\).
Generators, identities and symmetries are partial monogamous, as
semi-monogamicity is preserved by composition, tensor and trace.
So \(\termtohyp[f]{\Sigma}\) is partial monogamous.

Now we show that any partial monogamous cospan \(
    \cospan{m}[f]{F}[g]{n}
\)
must be in the image of \(\termtohyp{\Sigma}\).
To do this, we will now construct a cospan that is isomorphic to
\(\cospan{m}[f]{F}[g]{n}\), but from which it is possible to read off a
unique term in \(\smc{\Sigma}\).
The components of the new cospan are as follows:
\begin{itemize}
    \item let \(L\) be the hypergraph containing vertices with degree
            \((0,0)\) that are not in the image of \(f\) or \(g\);
    \item let \(E\) be the hypergraph containing hyperedges of \(F\) and
            their source and target vertices, but disconnected;
    \item let \(V\) be the discrete hypergraph containing all the
            vertices of \(F\); and
    \item let \(S\) and \(T\) be the discrete hypergraphs containing
            the source and target vertices of hyperedges in \(F\)
            respectively, with the ordering determined by some order
            \(e_1,e_2,\cdots,e_n\) on the edges in \(F\).
\end{itemize}

These parts can be composed and a trace applied to obtain the follow
cospan:
\begin{gather}
    \trace{T}{
        \cospan{T + m}[\id + f]{V}[\id + g]{S + n}
        \,\seq\,
        \cospan{\emptyset + S + n}[\id]{L + E + n}[\id]{\emptyset + T + n}
    }
    \label{gat:cospan}
\end{gather}

This can be checked to be isomorphic to the original cospan
\(\cospan{m}[f]{F}[g]{n}\) by applying the pushouts.
From this we can read off a term in \(\smc{\Sigma}\):
Since the first cospan is monogamous, it corresponds to a term \(
    \iltikzfig{strings/category/f-2-2}[f][white][|\vertices{T}|][m][|\vertices{S}|][n]
\) by \cref{lem:monog-discrete-cospan}.
The second cospan corresponds to \(
    \iltikzfig{strings/category/f}[g][white][|\vertices{S}][\vertices{T}]
    :=
    \bigtensor_{v \in \vertices{L}}
    \iltikzfig{strings/traced/trace-id}
    \tensor
    \bigtensor_{e \in 0 \leq i \leq n}
    \iltikzfig{graphs/isomorphism/label-e}
    \tensor
    \iltikzfig{strings/category/identity}[white][n]
\), where \(\elabel{e}\) is the generator in \(\generators\) that \(e\) is
labelled with.
Putting this all together yields \(
    h := \termtohypsigma[\iltikzfig{graphs/isomorphism/construction}]
\).
While there may be multiple orderings on the edges, the possible terms
are equal by sliding and the naturality of symmetry, so there is one
unique term \(
    \iltikzfig{strings/category/f}[H][white]
\) that corresponds to cospan (\ref{gat:cospan}).

It is clear by definition that \(
    \termtohypsigma[\iltikzfig{strings/category/f}[H][white]]
\) produces (\ref{gat:cospan}), which is isomorphic to the original
cospan \(\cospan{m}[f]{F}[g]{n}\), so it is in the image of
\(\termtohypsigma\).
    \end{proof}
}{
    \begin{proof}[Proof (Sketch)]
        The \(\onlyifdir\) direction is by induction on the structure of the term.
        For the \(\ifdir\) direction, a cospan isomorphic to the original
        cospan can be constructed from which a term in \(\stmcsigma\) can be
        read off.
        Informally, this cospan is
        \begin{gather}
            \trace{}{
                \cospan{x + m}{V}{x + n}
                \seq
                \cospan{x + n}{L + E + n}{x + n}
            }
        \end{gather}
        where \(V\) is all the vertices in \(F\), \(L\) is the vertices with
        degree \((0, 0)\) not in the image of the interfaces, and \(E\) is the
        all the hyperedges in \(F\).
        The first component corresponds to a term in \(\smc{}\) by
        \cref{lem:symmetries-prop}, and the stack of edges to a tensor of
        generators in \(\stmc{\Sigma}\).
    \end{proof}
}

\begin{proposition}[\cite{bonchi2022string}]
    \(\termtohypsigma\) and \(\frobtohyp{\Sigma}\) are faithful.
\end{proposition}

\begin{corollary}\label{cor:stmc-graph-iso}
    \(\stmc{\Sigma} \cong \pmcspfihyp\).
\end{corollary}
    % !TeX root = ../main-conf.tex
\section{Hypergraphs for traced commutative comonoid categories}

We are interested in another element of structure in addition to the trace: the
ability to \emph{copy} and \emph{discard} wires.
This is known as a \emph{(commutative) comonoid structure}: categories equipped
with such a structure are also known as \emph{gs-monoidal}
(\emph{garbage-sharing}) categories~\cite{fritz2022free}.

\begin{definition}
    Let \((\generators[\ccomon], \equations[\ccomon])\) be the symmetric
    monoidal theory of \emph{commutative comonoids}, with \(\Sigma_\ccomon := \{
        \iltikzfig{strings/structure/comonoid/copy}[white],
        \iltikzfig{strings/structure/comonoid/discard}[white]
    \}\) and \(\mce_\ccomon\) defined as in \cref{app:equations},
    \cref{fig:comonoid-equations}.
    We write \(\ccomon := \smc{\generators[\ccomon], \equations[\ccomon]}\).
\end{definition}

From now on, we write `comonoid' to mean `commutative comonoid'.
There has already been work using hypergraphs for PROPs with a (co)monoid
structure~\cite{fritz2022free,milosavljevic2022string} but these consider
\emph{acyclic} hypergraphs: we must ensure that removing the acyclicity
condition does not lead to any degeneracies.

\begin{definition}[Partial left-monogamy]
    For a cospan \(\cospan{m}[f]{H}[g]{n}\), we say it is
    \emph{partial left-monogamous} if \(f\) is mono and, for all nodes
    \(v \in H_\star\), the degree of \(v\) is \((0,m)\) if \(v \in f_\star\) and
    \((0,m)\) or \((1,m)\) otherwise, for some \(m \in \nat\).
\end{definition}

\begin{remark}
    As with the vertices not in the interfaces with degree \((0, 0)\) in the
    vanilla traced case, the vertices not in the interface with degree
    \((0, m)\) allow for terms such as \(
        \trace{}{\iltikzfig{strings/structure/comonoid/copy}[white]}
    \).
\end{remark}

\iftoggle{conf}{}{
    \begin{lemma}
        \label{lem:trace-in-degree}
        Given a partial left-monogamous cospan \(\cospan{x+m}[f+h]{K}[g+k]{x+n}\)
        and its trace \(\cospan{m}[h \seq p]{pK}[k \seq p]{n}\), let vertices
        \(v_0, v_1, \cdots, v_n\) such that each \(v_i\) is in the image of \(g\)
        and \(p(v_0) = p(v_1) = \cdots = p(v_n)\).
        Then, there exists at most one \(v_i\) with in-degree \(1\).
    \end{lemma}
    \iftoggle{proofs}{
        \begin{proof}
            Assume that there exist vertices \(g(i),g(j)\) with in-degree \(1\).
            For \(p(g(i)) = p(g(j))\) to hold, then there must either exist a sequence
            \(f(i) = g(i_0), f(i_0) = g(i_1), \cdots, f(i_n) = g(j)\), or vice versa.
            But \(f(i_n) = g(j)\) must have in-degree \(0\) by partial left-monogamy, a
            contradiction.
            Therefore at most one \(v_i\) can have in-degree \(1\).
        \end{proof}
    }{}
}

\begin{lemma}
    \label{lem:partial-monogamous-ops}
    Let \(\cospan{m}{F}{n}\), \(\cospan{n}{G}{p}\), \(\cospan{p}{H}{q}\) and
    \(\cospan{x+m}{K}{x+n}\) be partial left-monogamous cospans.
    Then,
    \begin{itemize}
        \item identities and symmetries are partial left-monogamous;
        \item \(\cospan{m}{F}{n} \seq \cospan{n}{G}{p}\) is partial
                left-monogamous;
        \item \(\cospan{m}{F}{n} \tensor \cospan{p}{H}{q}\) is partial
                left-monogamous; and
        \item \(\trace{x}{\cospan{x+m}{K}{x+n}}\) is partial left-monogamous.
    \end{itemize}
\end{lemma}
\iftoggle{proofs}{
    \begin{proof}
        Identities and symmetries are monogamous, and as such they are also
        partial left-monogamous.
        For composition, the vertices in the image of \(g\) and \(h\) are
        identified.
        Let \(v = p(g(i)) = p(h(i))\).
        We must show that \(v\) has in-degree \(0\) if it is in the image of
        \(f\), and \(0\) or \(1\) otherwise.
        \(h(i)\) has in-degree \(1\) by definition, so the in-degree of \(v\) is
        entirely determined by \(g(i)\).
        If \(v\) is in the image of \(f\), then \(g(i)\) must also be in the
        image of \(f\), so it has in-degree \(0\), and hence so does \(v\).
        Conversely, if \(v\) is not in the image of \(f\), \(g(i)\) has
        in-degree of either \(0\) or \(1\), and hence so does \(v\).

        For tensor, the elements of the original two graphs are unaffected so
        the degrees remain unchanged.

        For trace, let \(v = p(f(i)) = p(g(i))\).
        \(v\) cannot be in the image of \(h \seq p\) as this would mean that
        \(f + h\) is not mono.
        Therefore we must show that \(v\) has either degree of either \((0,m)\)
        or \((1,m)\).
        The degree of \(v\) is the sum of the degrees of each \(v_{fi}\) and
        \(v_{gi}\).
        Let \(v_{f0},\cdots,v_{fn}\) be the vertices in the image of \(f\) such
        that \(p(v_{fi}) = v\), and similarly for \(v_{gi}\).
        The in-degree of each \(v_{fi}\) must be \(0\) so all the in-degree is
        contributed by each \(v_{gi}\).
        By \cref{lem:trace-in-degree}, at most \(v_{fi}\) can have in-degree
        \(1\), so the in-degree of \(v\) can either be \(0\) or \(1\).
        Therefore the cospan is partial left-monogamous.
    \end{proof}
}{}

\begin{definition}
    Let \(\plmcspdhyp\) be the sub-PROP of \(\cspdhyp\) containing only the
    partial left-monogamous cospans of hypergraphs.
\end{definition}

This category can be equipped with a comonoid structure.

\iftoggle{conf}{
    \begin{definition}
        Let \(
            \morph{
                \comonoidtofrob
            }{
                \ccomon
            }{
                \frob
            }
        \) be the obvious embedding of \(\ccomon\) into \(\frob\), and let \(
            \morph{
                \tracedandcomonoidtofrob{\Sigma}
            }{
                \stmc{\Sigma} + \comon
            }{
                \smc{\Sigma} + \frob
            }
        \) be the copairing of \(\tracedtosymandfrob{\Sigma}\) and
        \(\comonoidtofrob\).
    \end{definition}
}{
    \begin{proposition}[\cite{lack2004composing}, Example 5.2]
        \label{prop:ccomon-iso-finsetop}
        \(\ccomon \cong \op{\finset}\).
    \end{proposition}

    \iftoggle{conf}{}{
        The following is a corollary of \cref{thm:cospan-homomorphism}.
    }

    \begin{corollary}
        \label{cor:prop-homomorphism-finset}
        There is a faithful PROP homomorphism \(
            \morph{\tilde{D}}{\csp{\op{\finset}}}{\csp[D]{\hypsigma}}
        \).
    \end{corollary}

    \begin{definition}
        Let \(\morph{\comonoidtohyp{\Sigma}}{\ccomon}{\cspdhyp}\) be the homomorphism
        obtained by composing the isomorphism of \cref{prop:ccomon-iso-finsetop}
        with the homomorphism of \cref{cor:prop-homomorphism-finset}.
        Concretely, it is defined on objects in the obvious way and on morphisms as
        \(
            \comonoidtohyp[\iltikzfig{strings/structure/comonoid/copy}[white]]
            :=
            \cospan{1}{1}{2}
        \) and \(
            \comonoidtohyp[\iltikzfig{strings/structure/comonoid/discard}[white]]
            :=
            \cospan{1}{1}{0}
        \).
    \end{definition}
}

\begin{lemma}
    The image of \(\frobtohyp{\Sigma} \circ \comonoidtofrob\) is in \(\plmcspdhyp\)
\end{lemma}
\iftoggle{proofs}{
    \begin{proof}
        By definition.
    \end{proof}
}{}

\begin{corollary}
    The image of \(
        \termandfrobtohypsigma \circ \tracedandcomonoidtofrob{\Sigma}
    \) is in \(\plmcspdhyp\).
\end{corollary}

\begin{lemma}\label{lem:ccomon-term}
    Given a partial left-monogamous cospan \(\cospan{m}[f]{m}[g]{n}\), there
    exists a unique term \(\morph{h}{m}{n} \in \ccomon\) up to the axioms of
    SMCs and comonoids such that \(
        \frobtohyp{\Sigma} \circ \comonoidtofrob
        =
        \cospan{m}[f]{m}[g]{n}
    \).
\end{lemma}
\iftoggle{proofs}{
    \begin{proof}

    \end{proof}
}{}

\begin{theorem}
    \(\stmcsigma + \ccomon \cong \plmcspfihyp\).
\end{theorem}
\begin{proof}
    Since \(\termandfrobtohypsigma\) and \(\comonoidtohyp{\Sigma}\) are faithful,
    it suffices to show that every cospan \(\cospan{m}{F}{n}\) in
    \(\plmcspdhyp\) can be decomposed in such a way that each component is in
    the image of either \(\termandfrobtohypsigma\) or
    \(\comonoidtohyp{\Sigma}\).
    This is achieved by taking the construction of \cref{thm:termtohyp-image}
    and allowing the first component to be partial left-monogamous; by
    \cref{lem:ccomon-term} a term in \(\ccomon\) can be retrieved from this.
\end{proof}

    % !TeX root = ../main-arxiv.tex
\section{Graph rewriting}

We have now shown that we can reason up to the axioms of symmetric traced
categories with a comonoid structure using hypergraphs: string diagrams equal by
topological deformations are translated into isomorphic hypergraphs.
However, to reason about an \emph{monoidal theory} with extra equations we must
actually rewrite the components of the graph.
In the syntactic realm this is performed with \emph{term rewriting}.

\begin{definition}[Term rewriting]\label{def:term-rewriting}
    A \emph{rewriting system} \(\mcr\) for a traced PROP \(\stmc{\Sigma}\)
    consists of a set of \emph{rewrite rules} \(
        \rrule{
            \iltikzfig{strings/category/f}[l][white][i][j]
        }{
            \iltikzfig{strings/category/f}[r][white][i][j]
        }
    \).
    Given terms \(
        \iltikzfig{strings/category/f}[g][white][m][n]
    \) and \(
        \iltikzfig{strings/category/f}[h][white][m][n]
    \) in \(\stmc{\generators}\) we write \(
        \iltikzfig{strings/category/f}[g][white]
        \rewrite[\mcr]
        \iltikzfig{strings/category/f}[h][white]
    \) if there exists rewrite rule \((
        \iltikzfig{strings/category/f}[l][white][i][j],
        \iltikzfig{strings/category/f}[r][white][i][j]
    )\) in \(\mcr\) and \(
        \iltikzfig{strings/category/f-2-2}[c][white][j][m][i][n]
    \) in \(\stmc{\Sigma}\) such that \(
        \iltikzfig{strings/category/f}[g][white]
        =
        \iltikzfig{strings/rewriting/rewrite-l}
    \) and \(
        \iltikzfig{strings/category/f}[h][white]
        =
        \iltikzfig{strings/rewriting/rewrite-r}
    \) by axioms of STMCs.
\end{definition}

The graph equivalent is \emph{graph rewriting}.
A common framework is that of \emph{double pushout rewriting} (DPO rewriting).
We use an extension, known as \emph{double pushout rewriting with interfaces}
(DPOI rewriting).

\begin{definition}[DPO rule]
    Given interfaced hypergraphs \(
        \cospan{i}[a_1]{L}[a_2]{j}
    \) and \(
        \cospan{i}[b_1]{R}[b_2]{j}
    \), their \emph{DPO rule} in \(\hypsigma\) is a span \(
        \spann{L}[[a_1,a_2]]{i+j}[[b_1,b_2]]{R}
    \).
\end{definition}

\begin{definition}[DPO(I) rewriting]\label{def:dpo}
    Let \(\mcr\) be a set of DPO rules.
    Then, for morphisms \(G \leftarrow m+n\) and \(H \leftarrow m+n\) in
    \(\hypsigma\), there is a rewrite \(G \trgrewrite[\mcr] H\) if there
    exists a rule \(
        \spann{L}{i+j}{R} \in \mcr
    \) and cospan \(
        \cospan{i+j}{C}{n+m} \in \hypsigma
    \) such that diagram in the left of \cref{fig:dpo} commutes.
\end{definition}

\begin{figure}
    \centering
    \raisebox{1em}{\includestandalone{figures/graphs/dpo/dpo}}
    \qquad
    \includestandalone{figures/graphs/dpo/pushout-complement}
    \caption{The DPO diagram and a pushout complement.}
    \label{fig:dpo}
\end{figure}

The first thing to note is that the graphs in the DPO diagram have a
\emph{single} interface \(G \leftarrow m + n\) instead of the cospans \(
    \cospan{m}{G}{n}
\) we are used to.
Before performing DPO rewriting in \(\hypsigma\), the interfaces must be
`folded' into one.

\begin{definition}[\cite{bonchi2022stringa}]
    Let \(\morph{\foldinterfaces}{\smc{\Sigma} + \frob}{\smc{\Sigma} + \frob}\)
    be defined as having action \(
        \iltikzfig{strings/category/f}[f][white][m][n]
        \mapsto
        \iltikzfig{strings/rewriting/folding}[f][white][m][n]
    \).
\end{definition}

Note that the result of applying \(\foldinterfaces\) is not in the image of \(
    \termandfrobtohypsigma \circ \tracedtosymandfrob{\Sigma}
\) any more.
This is not an issue, so long as we `unfold' the interfaces once rewriting is
completed.

\begin{proposition}[\cite{bonchi2022string}, Prop. 4.8]
    If \(
        \termandfrobtohypsigma[\iltikzfig{strings/category/f}[f][white][m][n]]
        =
        \cospan{m}[i]{F}[o]{n}
    \) then \(
        \foldinterfaces[
            \termandfrobtohypsigma[\iltikzfig{strings/category/f}[f][white]]
        ]
    \) is isomorphic to \(
        \cospan{0}[]{F}[i+o]{m+n}
    \).
\end{proposition}

In order to apply a given DPO rule \(\spann{L}{i+j}{R}\) in some larger
graph \(\cospan{m}{G}{n}\), a morphism \(L \to G\) must first be identified.
The next step is to `cut out' the components of \(L\) that exist in \(G\).

\begin{definition}[Pushout complement]\label{def:pushout-complement}
    Let \(i + j \to L \to G \rightarrow m + n\) be morphisms in \(\hypsigma\).
    Then the \emph{pushout complement} of these morphisms is an object \(C\)
    with morphisms \(i + j \to C \to G\) such that \(\cospan{L}{G}{C}\) is a
    pushout and the diagram on the right of \cref{fig:dpo} commutes.
\end{definition}

Once a pushout complement \(C\) is computed, the pushout of
\(\spann{C}{i+j}{R}\) can be performed to obtain the completed rewrite \(H\).
However, a pushout complement may not exist for a given rule and matching.

\begin{definition}[\cite{bonchi2022string}, Def. 3.16]
    Let \(i + j \xrightarrow{a} L \xrightarrow{f} G\) be morphisms in
    \(\hypsigma\).
    The morphisms satisfy the \emph{no-dangling} condition if, for every
    hyperedge not in the image of \(f\), each of its source and target vertices
    is either not in the image of \(f\) or are in the image of \(f \circ a\).
    The morphisms satisfy the \emph{no-identification} condition if any two
    distinct elements merged by \(f\) are also in the image of \(f \circ a\).
\end{definition}

\begin{proposition}[\cite{bonchi2022string}, Prop. 3.17]
    \label{prop:pushout-complement}
    The morphisms \(i + j \to L \to G\) have at least one pushout complement if
    and only if they satisfy the no-dangling and no-identification conditions.
\end{proposition}

\begin{definition}
    Given a partial monogamous cospan \(\cospan{i}{L}{j}\), a morphism
    \(L \to G\) is called a \emph{matching} if it has at least one pushout
    complement.
\end{definition}

In certain settings, known as
\emph{adhesive categories}~\cite{lack2004adhesive}, it is possible to be more
precise about the number of pushout complements for a given matching and rewrite
rule.

\begin{proposition}[\cite{lack2004adhesive}]
    In an adhesive category, pushout complements of \(
        i + j \xrightarrow{a} L \to G\)
    are unique if they exist and \(a\) is mono.
\end{proposition}

\begin{proposition}[\cite{lack2005adhesive}]
    \(\hypsigma\) is adhesive.
\end{proposition}

Since a given pushout complement uniquely determines the rewrite performed, it
might seem advantageous to always have exactly one pushout complement.
However, as shall be detailed later, when writing modulo traced comonoid
structure there are settings where having multiple pushout complements is
beneficial.

\subsection{Rewriting with traced structure}

While in the Frobenius case considered in~\cite{bonchi2022string}, all valid
pushout complements correspond to a valid rewrite, this is not the case for the
traced monoidal case.
In \cite{bonchi2022stringa}, pushout complements that correspond to a valid
rewrite in the non-traced symmetric monoidal case are identified as
\emph{boundary complements}.
We will use a weakening of this definition.

\begin{definition}[Traced boundary complement]
    \label{def:traced-boundary-complement}
    For partial monogamous cospans \(
        \cospan{i}[a_1]{L}[a_2]{j}
    \) and \(
        \cospan{n}[b_1]{G}[b_2]{m} \in \hypsigma
    \), a pushout complement as in \cref{def:pushout-complement} is called a
    \emph{traced boundary complement} if \(c_1\) and \(c_2\) are mono and \(
        \cospan{j+m}[[c_2,d_1]]{C}[[d_2,c_1]]{{n+i}}
    \) is a partial monogamous cospan.
\end{definition}

Unlike~\cite{bonchi2022stringa}, we do not enforce that the matching is mono,
as restricting to these matchings actually cuts off potential rewrites in the
\emph{traced} setting, such as the occurrence of a rewrite rule inside a loop: \(
    \iltikzfig{graphs/dpo/matchings/trace-l}
    \to
    \iltikzfig{graphs/dpo/matchings/trace-g}
\).

\begin{definition}[Traced DPO]
    For morphisms \(G \leftarrow m+n\) and \(H \leftarrow m+n\) in
    \(\hypsigma\), there is a traced rewrite \(G \trgrewrite[\mcr] H\) if there
    exists a rule \(
        \spann{L}{i+j}{G} \in \mcr
    \) and cospan \(
        \cospan{i+j}{C}{n+m} \in \hypsigma
    \) such that diagram in \cref{def:dpo} commutes and \(i+j \to C\) is a
    traced boundary complement.
\end{definition}

Some intuition on the construction of traced boundary complements may be
required: this will be provided through a lemma and some examples.

\begin{lemma}
    In a traced boundary complement, let \(v \in i\) and
    \(w_0, w_1, \cdots w_k\) such that \(f(a_1(v)) = f(a_2(w_0))\),
    \(f(a_1(v)) = f(a_2(w_1))\) and so on.
    Then either (1) there exists exactly one \(w_l\) not in the image of \(d_1\)
    such that \(c_1(v) = c_2(w_l)\); (2) \(c_1(v)\) is in the image of \(d_1\);
    or (3) \(c_1(v)\) has degree \((1,0)\).
    The same also holds for \(w \in j\), with the interface map as \(d_2\) and
    the degree as \((0,1)\).
\end{lemma}
\iftoggle{proofs}{
    \begin{proof}
        Since \(c_1(v)\) is in the image of \(c\), it must have either degree
        \((0,0)\) or \((1,0)\) by partial monogamy.
        For it to have degree \((0, 0)\), it must either be in the image of one
        of \(c_2\) or \(d_2\).
        In the case of the former, this means that there must be a \(w_l\) such
        that \(c_1(v) = c_2(w_l)\), and only one such vertex as \(c_2\) is mono,
        so (1) is satisfied.
        In the case of the latter, (2) is immediately satisfied.
        Otherwise, (3) is satisfied.

        The proof for the flipped case is exactly the same.
    \end{proof}
}{}

Often there can be valid rewrites in the realm of graphs that are non-obvious
in the term language.
This is because we are rewriting modulo \emph{yanking}.

\begin{example}
    Consider the rule \(
        \rrule{
            \iltikzfig{graphs/dpo/split-loop/rule-lhs}
        }{
            \iltikzfig{graphs/dpo/split-loop/rule-rhs}
        }
    \).
    The interpretation of this as a DPO rule in a valid traced boundary
    complement is illustrated below.
    \begin{center}
        \includestandalone{figures/graphs/dpo/split-loop/rewrite}
    \end{center}
    This corresponds to a valid term rewrite:
    \[
        \iltikzfig{graphs/dpo/split-loop/derivation-1}
        =
        \iltikzfig{graphs/dpo/split-loop/derivation-2}
        =
        \iltikzfig{graphs/dpo/split-loop/derivation-3}
        =
        \iltikzfig{graphs/dpo/split-loop/derivation-4}
    \]
\end{example}

Unlike regular boundary complements, traced boundary complements need not be
unique.
This is not an issue, as all pushout complements can be enumerated for a given
rule and matching~\cite{heumuller2011construction}.

\begin{example}
    Consider the rule \(
        \rrule{
            \iltikzfig{graphs/dpo/non-unique/rule-lhs}
        }{
            \iltikzfig{graphs/dpo/non-unique/rule-rhs}
        }
    \).
    Below are two valid traced boundary complements involving a matching of this
    rule.

    \begin{center}
        \includestandalone{figures/graphs/dpo/non-unique/rewrite-1}
        \includestandalone{figures/graphs/dpo/non-unique/rewrite-2}
    \end{center}

    These derivations arise since we are rewriting modulo \emph{yanking}:
    \begin{gather*}
        \iltikzfig{graphs/dpo/non-unique/derivation-1}
        =
        \iltikzfig{graphs/dpo/non-unique/derivation-2}
        =
        \iltikzfig{graphs/dpo/non-unique/derivation-3a}
        =
        \iltikzfig{graphs/dpo/non-unique/derivation-4a}
        =
        \iltikzfig{graphs/dpo/non-unique/derivation-5a}
        \\
        \iltikzfig{graphs/dpo/non-unique/derivation-1}
        =
        \iltikzfig{graphs/dpo/non-unique/derivation-2}
        =
        \iltikzfig{graphs/dpo/non-unique/derivation-3b}
        =
        \iltikzfig{graphs/dpo/non-unique/derivation-4b}
        =
        \iltikzfig{graphs/dpo/non-unique/derivation-5b}
    \end{gather*}
\end{example}


Rewriting modulo yanking also eliminates another foible of rewriting modulo
(non-traced) symmetric monoidal structure.
In the SMC case, the image of the matching must be \emph{convex}: any
path between vertices must also be captured.
This is not necessary in the traced case.

\begin{example}
    Consider the following rewrite rule and its interpretation.
    %
    \begin{gather}
        \rrule{
            \iltikzfig{graphs/dpo/convex/example-l}
        }{
            \iltikzfig{graphs/dpo/convex/example-r}
        }
        \qquad
        \iltikzfig{graphs/dpo/convex/example-rule-graph}
        \label{gath:convex-rule}
    \end{gather}
    %
    Now consider the following term and interpretation:
    %
    \begin{gather}
        \iltikzfig{graphs/dpo/convex/example-g}
        \qquad
        \iltikzfig{graphs/dpo/convex/example-g-graph}
        \label{gath:convex-term}
    \end{gather}
    Although it is not obvious in the original string diagram, there is in fact
    a matching of (\ref{gath:convex-rule}) in (\ref{gath:convex-term}).
    Performing the DPO procedure yields the following:
    %
    \begin{gather}
        \iltikzfig{graphs/dpo/convex/example-h-graph}
        \qquad
        \iltikzfig{graphs/dpo/convex/example-h}
    \end{gather}
    In a non-traced setting this is an invalid rule!
    However, it is possible with yanking.
    \begin{gather*}
        \iltikzfig{graphs/dpo/convex/example-g}
        =
        \iltikzfig{graphs/dpo/convex/rewrite-2}
        =
        \iltikzfig{graphs/dpo/convex/rewrite-4}
        =
        \iltikzfig{graphs/dpo/convex/rewrite-5}
        =
        \iltikzfig{graphs/dpo/convex/example-h}
    \end{gather*}
\end{example}

We are almost ready to show the soundness and completeness of this DPO rewriting
system.
The final prerequisite is a decomposition lemma, akin to a similar result
in~\cite{bonchi2022string}.

\begin{lemma}[Traced decomposition]\label{lem:traced-decomposition}
    Given partial monogamous cospans \(
        \cospan{m}[d_1]{G}[d_2]{n}
    \) and \(
        \cospan{i}[a_1]{L}[a_2]{j}
    \), along with a morphism \(
        L \xrightarrow{f} G
    \) such that \(i+j \rightarrow L \rightarrow G\) satisfies the no-dangling
    and no-identification conditions, then there exists \(
        \cospan{j+m}[[c_2,d_1]]{C}[[c_1,d_2]]{i+n}
    \) such that \(
        \cospan{m}{G}{n}
    \) can be factored as
    \begin{gather}
        \trace{i}{
            \begin{array}{cc}
                \cospan{i}[a_1]{L}[a_2]{j} \\
                \tensor \\
                \cospan{m}{m}{m}
            \end{array}
            \seq
            \cospan{j+m}[[c_2,d_1]]{C}[[c_1,d_2]]{i+n}
        }
        \label{gath:decomposition}
    \end{gather}
    where all cospans are partial monogamous and \(
        \cospan{j+m}[c_2,d_1]{C}[c_1,d_2]{i+n}
    \) is a traced boundary complement.
\end{lemma}
\iftoggle{proofs}{
    \begin{proof}
        Let \(
    i + j \xrightarrow{[c_1, c_2]} C \xleftarrow{[d_1, d_2]} m+n
\) be defined as a traced boundary complement of \(
    i+j \xrightarrow{[a_1,a_2]} L \xrightarrow{f} G
\), which exists as the no-dangling and no-identification condition is
satisfied.
We assign names to the various cospans in play, and reason string
diagrammatically:
\begin{align*}
    \iltikzfig{strings/category/f}[l][white][i][j] &:= \cospan{i}{L}{j}
    &
    \iltikzfig{strings/category/f-0-2}[\hat{l}][white][i][j]
    &:=
    \cospan{0}{L}{i+j}
    \\
    \iltikzfig{strings/category/f-2-2}[c][white][j][m][i][n]
    &:=
    \cospan{j+m}[[c_2, d_1]]{C}[[c_1, d_2]]{i+n}
    &
    \iltikzfig{strings/category/f-2-2}[\hat{c}][white][i][j][m][n]
    &:=
    \cospan{i+j}[[c_1, c_2]]{C}[[d_1, d_2]]{m+n}
    \\
    \iltikzfig{strings/category/f}[g][white][m][n]
    &:=
    \cospan{m}{G}{n}
    &
    \iltikzfig{strings/category/f-0-2}[\hat{g}][white][m][n]
    &:=
    \cospan{0}{G}{m+n}
\end{align*}
Note that the cospans in the left column are partial monogamous by definition
of rewrite rules and traced boundary complements.
We will show that  \(
    \iltikzfig{strings/category/f}[g][white]
\) can be decomposed into a form using the two cospans \(
    \iltikzfig{strings/category/f}[l][white]
\) and \(
    \iltikzfig{strings/category/f-2-2}[c][white]
\), along with identities.

By using the compact closed structure of \(\cspdhyp\), we have the following:
\begin{gather*}
    \iltikzfig{strings/category/f}[g][white][m][n]
    =
    \iltikzfig{graphs/dpo/g-bent}
    \quad
    \iltikzfig{strings/category/f-2-2}[\hat{c}][white][i][j][m][n]
    =
    \iltikzfig{graphs/dpo/cprime-as-c}
    \quad
    \iltikzfig{strings/category/f-0-2}[\hat{l}][white][i][j]
    =
    \iltikzfig{strings/compact-closed/f-bent-input}[l][white][i][j]
\end{gather*}
Since \(G\) is the pushout of \(
    L \xleftarrow{[a_1, a_2]} i+j \xrightarrow{[c_1, c_2]} C
\) and pushout is cospan composition, we also have that \(
    \iltikzfig{strings/category/f-0-2}[\hat{g}][white][m][n]
    =
    \iltikzfig{graphs/dpo/lctilde}
\).

Putting this all together we can show that
\begin{gather*}
    \iltikzfig{strings/category/f}[g][white][m][n]
    =
    \iltikzfig{graphs/dpo/g-bent}
    =
    \iltikzfig{graphs/dpo/l-c-bent}
    =
    \iltikzfig{graphs/dpo/l-c-bent-1}
    =
    \iltikzfig{graphs/dpo/lc-bent-2}
    =
    \iltikzfig{strings/rewriting/rewrite-l}[m][n]
\end{gather*}
Since the `loop' is constructed in the same manner as the canonical trace on
\(\cspdhyp\) (and is therefore identical in the graphical notation), this is a
term in the form of (\ref{gath:decomposition}).
    \end{proof}
}{}

We write \(
    \foldinterfaces[\tracedtosymandfrob[\mcr]{\Sigma}]
\) for the pointwise map \(
    (
        \iltikzfig{strings/category/f}[l][white],
        \iltikzfig{strings/category/f}[r][white]
    )
    \mapsto
    (
        \foldinterfaces[
            \tracedtosymandfrob[
                \iltikzfig{strings/category/f}[l][white]
            ]{\Sigma}
        ],
        \foldinterfaces[
            \tracedtosymandfrob[
                \iltikzfig{strings/category/f}[r][white]
            ]{\Sigma}
        ]
    ).
\)

\begin{theorem}\label{thm:traced-rewriting}
    Let \(\mcr\) be a rewriting system on \(\stmcsigma\).
    \iftoggle{conf}{
        Then, \(
            \iltikzfig{strings/category/f}[g][white][m][n]
            \rewrite[\mcr]
            \iltikzfig{strings/category/f}[h][white][m][n]
        \) if and only if \(
            \termandfrobtohypsigma[
                \foldinterfaces[
                    \tracedtosymandfrob[
                        \iltikzfig{strings/category/f}[g][white]
                    ]{\Sigma}
                ]
            ]
            \grewrite[
                \termandfrobtohypsigma[
                    \foldinterfaces[
                        \tracedtosymandfrob[\mcr]{\Sigma}
                    ]
                ]
            ]
            \termandfrobtohypsigma[
                \foldinterfaces[
                    \tracedtosymandfrob[
                        \iltikzfig{strings/category/f}[h][white]
                    ]{\Sigma}
                ]
            ].
        \)
    }{
        Then,
        \begin{gather*}
            \iltikzfig{strings/category/f}[g][white][m][n]
            \rewrite[\mcr]
            \iltikzfig{strings/category/f}[h][white][m][n]
            \quad
            \text{if and only if}
            \quad
            \termandfrobtohypsigma[
                \foldinterfaces[
                    \tracedtosymandfrob[
                        \iltikzfig{strings/category/f}[g][white]
                    ]{\Sigma}
                ]
            ]
            \grewrite[
                \termandfrobtohypsigma[
                    \foldinterfaces[
                        \tracedtosymandfrob[\mcr]{\Sigma}
                    ]
                ]
            ]
            \termandfrobtohypsigma[
                \foldinterfaces[
                    \tracedtosymandfrob[
                        \iltikzfig{strings/category/f}[h][white]
                    ]{\Sigma}
                ]
            ].
        \end{gather*}
    }
\end{theorem}
\iftoggle{proofs}{
    \begin{proof}
        First the \((\Rightarrow)\) direction.
If \(
    \iltikzfig{strings/category/f}[g][white]
    \rewrite[\mcr]
    \iltikzfig{strings/category/f}[H][white]
\) then we have \(
    \iltikzfig{strings/category/f}[g][white]
    =
    \iltikzfig{strings/rewriting/rewrite-l}
\) and \(
    \iltikzfig{strings/rewriting/rewrite-r}
    =
    \iltikzfig{strings/category/f}[H][white].
\)
Define the following cospans:
\begin{alignat}{3}
    \label{gath:l-cospan}
    \cospan{0}{L}{i+j}
    &:=
    \termandfrobtohypsigma[\foldinterfaces[\iltikzfig{strings/category/f}[l][white]]]
    &&=
    \termandfrobtohypsigma[\iltikzfig{strings/rewriting/l-folded}]
    \\
    \cospan{0}{R}{i+j}
    &:=
    \termandfrobtohypsigma[\foldinterfaces[\iltikzfig{strings/category/f}[r][white]]]
    &&=
    \termandfrobtohypsigma[\iltikzfig{strings/rewriting/r-folded}]
    \\
    \cospan{0}{G}{m+n}
    &:=
    \termandfrobtohypsigma[\foldinterfaces[\iltikzfig{strings/category/f}[f][white]]]
    &&=
    \termandfrobtohypsigma[\iltikzfig{strings/rewriting/lc-folded}]
    \\
    \label{gath:h-cospan}
    \cospan{0}{H}{m+n}
    &:=
    \termandfrobtohypsigma[\foldinterfaces[\iltikzfig{strings/category/f}[H][white]]]
    &&=
    \termandfrobtohypsigma[\iltikzfig{strings/rewriting/rc-folded}]
    \\
    \cospan{i+j}{C}{m+n}
    &:=
    \termandfrobtohypsigma[\iltikzfig{strings/rewriting/c-folded}]
    &&
\end{alignat}
By functoriality, since \(
    \foldinterfaces[\iltikzfig{strings/category/f}[f][white]]
    =
    \iltikzfig{strings/rewriting/l-folded}
    \seq
    \iltikzfig{strings/rewriting/c-folded}
\) then \[
    \cospan{0}{G}{m+n} = \cospan{0}{L}{i+j} \seq \cospan{i+j}{C}{m+n}.
\]
Cospan composition is pushout, so \(\cospan{L}{G}{C}\) is a pushout.
Using the same reasoning, \(\cospan{R}{G}{C}\) is also a pushout: this
gives us the DPO diagram.
All that remains is to check that the aforementioned pushouts are traced
boundary complements: this follows by inspecting components.

Now the \(\ifdir\) direction: assume we have a DPO diagram (\ref{def:dpo})
where \(L \leftarrow i + j\), \(i + j \rightarrow R\), \(m + n \to G\) and
\(m + n \to H\) are defined as in (\ref{gath:l-cospan}-\ref{gath:h-cospan})
above.
We must show that \(
    \iltikzfig{strings/category/f}[f][white]
    =
    \iltikzfig{strings/rewriting/rewrite-l}
\) and \(
    \iltikzfig{strings/category/f}[H][white]
    =
    \iltikzfig{strings/rewriting/rewrite-r}
\).
By definition of traced boundary complement \(\cospan{j+m}{C}{i+n}\) is a
partial monogamous cospan, so by fullness of \(\termandfrobtohypsigma\),
there exists a term \(
    \iltikzfig{strings/category/f-2-2}[c][white][j][m][i][n]
    \in \smcsigma
\) such that \(
    \termtohyp[
        \iltikzfig{strings/category/f-2-2}[c][white]
    ]{\Sigma}
    =
    \cospan{j+m}{C}{i+n}
\).
By traced decomposition (\cref{lem:traced-decomposition}), we have that for any
traced boundary complement \(\cospan{i+j}{C}{m+n}\) and morphism
\(L \to G\), \(\cospan{m}{G}{n}\) can be factored as in
(\ref{gath:decomposition}), i.e.\ \[
    \termandfrobtohypsigma[\iltikzfig{strings/category/f}[f][white]]
    =
    \trace{j}{\termandfrobtohypsigma[\iltikzfig{strings/category/f}[l][white]]
    \tensor
    \id[n]
    \seq
    \termandfrobtohypsigma[\iltikzfig{strings/category/f-2-2}[c][white]]}.
\]
So by functoriality, we have that \(
    \iltikzfig{strings/category/f}[f][white]
    =
    \iltikzfig{strings/rewriting/rewrite-l}
\).
The same reasoning follows for \(
    \iltikzfig{strings/category/f}[H][white]
    =
    \iltikzfig{strings/rewriting/rewrite-r}
\).
    \end{proof}
}{
    \begin{proof}
        \(\onlyifdir\) follows by defining cospans corresponding each part
        of \cref{def:term-rewriting} and composing them together:
        since composition of cospans is by pushout, the DPO diagram can be
        recovered and the pushouts checked to be traced boundary complements.
        \(\ifdir\) follows by using \cref{lem:traced-decomposition} and the
        fullness of \(\termandfrobtohypsigma\) to obtain the pieces of
        \cref{def:term-rewriting}.
    \end{proof}
}

\subsection{Rewriting with traced comonoid structure}

To extend rewriting with traced structure to the comonoid case, the traced
boundary complement conditions need to be weakened to the case of
\emph{left}-monogamous cospans.

\begin{definition}[Traced left-boundary complement]
    \label{def:traced-left-boundary-complement}
    For partial left-monogamous cospans \(
        \cospan{i}[a_1]{L}[a_2]{j}
    \) and \(
        \cospan{n}[b_1]{G}[b_2]{m} \in \hypsigma
    \), a pushout complement as in \cref{def:traced-boundary-complement}
    is called a \emph{traced left-boundary complement} if \(c_2\)
    is mono and \(
        \cospan{j+m}[[c_2,d_1]]{C}[[c_1,d_2]]{{i+n}}
    \) is a partial left-monogamous cospan.
\end{definition}

\begin{definition}[Traced comonoid DPO]
    For morphisms \(G \leftarrow m+n\) and \(H \leftarrow m+n\) in
    \(\hypsigma\), there is a traced comonoid rewrite \(G \trgrewrite[\mcr] H\)
    if there exists a rule \(
        \spann{L}{i+j}{G} \in \mcr
    \) and cospan \(
        \cospan{i+j}{C}{n+m} \in \hypsigma
    \) such that diagram in \cref{def:dpo} commutes and \(i+j \to C \to G\) is a
    traced left-boundary complement.
\end{definition}

\begin{lemma}[Traced comonoid decomposition]\label{lem:traced-comonoid-decomposition}
    \cref{lem:traced-decomposition} holds when all cospans are partial
    left-monogamous and \(
        \cospan{j+m}[c_2,d_1]{C}[c_1,d_2]{i+n}
    \) is a traced left-boundary complement.
\end{lemma}
\iftoggle{proofs}{
    \begin{proof}
        As \cref{lem:traced-decomposition}, but with partial left-monogamous
        cospans.
    \end{proof}
}{}

\begin{theorem}\label{thm:traced-comonoid-rewriting}
    Let \(\mcr\) be a rewriting system on \(\stmcsigma + \ccomon\).
    Then, \(
        \iltikzfig{strings/category/f}[g][white]
        \rewrite[\mcr]
        \iltikzfig{strings/category/f}[h][white]
    \) in \(\stmcsigma + \ccomon\) if and only if \(
        \termandfrobtohypsigma[
            \foldinterfaces[
                \tracedandcomonoidtofrob[
                    \iltikzfig{strings/category/f}[g][white]
                ]{\Sigma}
            ]
        ]
        \grewrite[
            \termandfrobtohypsigma[
                \foldinterfaces[
                    \tracedandcomonoidtofrob[\mcr]{\Sigma}
                ]
            ]
        ]
        \termandfrobtohypsigma[
            \foldinterfaces[
                \tracedandcomonoidtofrob[
                    \iltikzfig{strings/category/f}[h][white]
                ]{\Sigma}
            ]
        ].
    \).
\end{theorem}
\begin{proof}
    This is the same as for \cref{thm:traced-rewriting} but with the replacement
    of all traced boundary complements with traced left-boundary complements.
\end{proof}

\begin{example}
    As with the traced case, there may be multiple valid rewrites given a
    particular interface.
    The comonoid structure adds more possibilities, as there are the equations
    of commutative comonoids to consider.
    Consider the following rule and its interpretation.
    \begin{gather}
        \rrule{
            \iltikzfig{graphs/dpo/non-unique-comonoid/rule-lhs}
        }{
            \iltikzfig{graphs/dpo/non-unique-comonoid/rule-rhs}
        }
        \qquad
        \raisebox{-1.7em}{\includestandalone{figures/graphs/dpo/non-unique-comonoid/rule}}
        \label{gath:non-unique-rule-comonoid}
    \end{gather}
    Two valid rewrites are as follows:
    \begin{center}
        \includestandalone{figures/graphs/dpo/non-unique-comonoid/rewrite-1}
        \quad
        \includestandalone{figures/graphs/dpo/non-unique-comonoid/rewrite-2}
    \end{center}
    The first rewrite is the `obvious' one, but the second also holds by
    cocommutativity:
    \begin{gather*}
        \iltikzfig{graphs/dpo/non-unique-comonoid/rewrite-1}
        =
        \iltikzfig{graphs/dpo/non-unique-comonoid/rewrite-2a}
        \qquad
        \iltikzfig{graphs/dpo/non-unique-comonoid/rewrite-1}
        =
        \iltikzfig{graphs/dpo/non-unique-comonoid/rewrite-2b}
        =
        \iltikzfig{graphs/dpo/non-unique-comonoid/rewrite-3b}
    \end{gather*}
\end{example}
    % !TeX root = ../main-conf.tex
\section{Case studies}

\subsection{Cartesian structure}

One important class of categories with a traced comonoid structure are
\emph{traced Cartesian categories}~\cite{cazanescu1990new,hasegawa1997recursion}.
These categories are interesting when considering \emph{data flow} because any
traced Cartesian category admits a fixpoint
operator~\cite[Thm. 3.1]{hasegawa1997recursion}.

\begin{definition}[Cartesian category~\cite{fox1976coalgebras}]
    A monoidal category is \emph{Cartesian} if its tensor is given by the
    Cartesian product.
\end{definition}

As a result of this, the unit is a terminal object in any Cartesian category,
and any object has a comonoid structure.
Cartesian categories can also be expressed as a monoidal theory:

\begin{definition}
    For a given base PROP \(\stmc{\Sigma_\mathbf{C}}\) with a comonoid
    structure, the monoidal theory \((
        \generators[\mathbf{Cart}_\mathbf{C}],
        \equations[\mathbf{Cart}_\mathbf{C}]
    )\) is defined with \(
        \generators[\mathbf{Cart}_\mathbf{C}] := \Sigma_\mathbf{C}
    \) and \(
        \equations[\mathbf{Cart}_\mathbf{C}]
    \) as the equations in \cref{fig:cartesian-equations}.
\end{definition}

Note that as the equations in \(\equations[\mathbf{Cart}_\mathbf{C}]\) are
parameterised over any morphism
\(\iltikzfig{strings/category/f}[f][white][m][n]\), a separate
DPO rewrite rule is required for every combination of generators as in
\cref{fig:cartesian-equations}.
However, as is the case in the next section, it is often possible to
characterise the copying behaviour through a finite number of equations.

\begin{remark}
    The combination of Cartesian equations with the underlying compact closed
    structure of \(\cspdhyp\) may prompt alarm bells, as a compact closed
    category in which the tensor is the Cartesian product is trivial.
    However, it is important to note that \(\cspdhyp\) is \emph{not} subject to
    these equations: it is only a setting for performing graph
    rewrites.
\end{remark}

\begin{figure}
    \centering
    \begin{tabular}{cc}
        \iltikzfig{strings/structure/cartesian/naturality-copy-lhs}[f][white][m][n]
        \(=\)
        \iltikzfig{strings/structure/cartesian/naturality-copy-rhs}[f][white][m][n]
        &
        \iltikzfig{strings/structure/cartesian/naturality-discard-lhs}[f][white][m]
        \(=\)
        \iltikzfig{strings/structure/cartesian/naturality-discard-rhs}[m]
        \\[2em]
        \includestandalone{figures/graphs/dpo/cartesian/copy/rule}
        &
        \raisebox{1em}{\includestandalone{figures/graphs/dpo/cartesian/discard/rule}}
    \end{tabular}
    \caption{
        Equations of the monoidal theory \(\mathbf{Cart}_\mathcal{C}\),
        where \(\iltikzfig{strings/category/f}[f][white][m][n]\) is an arbitrary
        morphism in \(\mathcal{C}\), and the interpretations of these equations
        as rewrite rules for an arbitrary generator \(e\).
    }
    \label{fig:cartesian-equations}
\end{figure}

Fixpoints can be reasoned about using the \emph{unfolding} rule, which holds in
any traced Cartesian category.

\begin{center}
    \iltikzfig{strings/traced/trace-rhs}[f][white][m][n][x]
    \(=\)
    \iltikzfig{circuits/examples/reasoning/unfolding/unfolding-1}[f][white][m][n][x]
    \(=\)
    \iltikzfig{circuits/examples/reasoning/unfolding/unfolding-2}[f][white][m][n][x]
    \(=\)
    \iltikzfig{circuits/examples/reasoning/unfolding/unfolding-3}[f][white][m][n][x]
\end{center}

In the syntactic setting, this requires the application of multiple
equations: the two counitality equations followed by the copy equation and
finally some axioms of STMCs for `housekeeping'.
However, if we interpret this in the hypergraph setting, the comonoid equations
are absorbed into the notation so only the copy equation needs to be applied.

\begin{center}
    \includestandalone{figures/graphs/dpo/unfolding/rewrite-1}
\end{center}



The dual notion of traced \emph{cocartesian}
categories~\cite{bainbridge1976feedback} are also important in computer science:
a trace in a traced cocartesian category corresponds to \emph{iteration} in the
context of \emph{control flow}.
The details of this section could also be applied to the cocartesian case by
flipping all the directions and working with partial \emph{right}-monogamous
cospans.

However, attempting to combine the product and coproduct approaches for settings
with a \emph{biproduct} would simply yield the category \(\cspdhyp\), a
hypergraph category (\cref{prop:frobenius-map}) subject to the Frobenius
equations in \cref{fig:frobenius-equations}.
A category with biproducts is not necessarily subject to such equations, so this
would not be a suitable approach.
    % !TeX root = ../main-conf.tex

\subsection{Digital circuits}
\label{sec:digital-circuits}

A traced monoidal theory with a comonoid structure that is of particular
interest to us is the \emph{local theory of sequential circuits} from
\cite[Sec. VI]{ghica2022compositional}.

\begin{definition}[Gate-level circuits]
    Let the monoidal theory of \emph{gate-level sequential circuits} be defined
    as \(
        (\generators[\mathbf{SCirc}], \equations[\mathbf{SCirc}])
    \), where \[
        \generators[\mathbf{SCirc}]
        :=
        \{
            \iltikzfig{circuits/components/gates/and},
            \iltikzfig{circuits/components/gates/or},
            \iltikzfig{circuits/components/gates/not},
            \iltikzfig{strings/structure/comonoid/copy}[comb],
            \iltikzfig{strings/structure/monoid/merge}[comb],
            \iltikzfig{strings/structure/comonoid/discard}[comb],
            \iltikzfig{circuits/components/values/v}[\belnapnone],
            \iltikzfig{circuits/components/values/v}[\belnaptrue],
            \iltikzfig{circuits/components/values/v}[\belnapfalse],
            \iltikzfig{circuits/components/values/v}[\belnapboth],
            \iltikzfig{circuits/components/waveforms/delay}
        \}
    \] and the equations of \(
        \equations[\mathbf{SCirc}]
    \) are listed in \cref{app:equations}, \cref{fig:circuit-equations}, where
    \(
        \gateinterpretation
    \) maps gates to the corresponding truth table in \cref{app:belnap},
    \(\ljoin\) is the join in the information lattice in \cref{app:belnap}, and
    \(
        \iltikzfig{circuits/components/circuits/f-1-2}[F^n][comb][m][x][n]
    \) is defined inductively as \(
        \iltikzfig{circuits/instant-feedback/f0-box}
        :=
        \iltikzfig{circuits/instant-feedback/f0-definition}
    \) and \(
        \iltikzfig{circuits/instant-feedback/fkp1-box}
        :=
        \iltikzfig{circuits/instant-feedback/fkp1-definition}
    \).
\end{definition}

The generators in \(\generators[\mathbf{SCirc}]\) are, respectively:
\(\andgate\), \(\orgate\) and \(\notgate\) gates; constructs for forking,
joining and stubbing wires; \emph{values} representing no signal, a true signal,
a false signal, and both signals at once; and a delay of one unit of time.


The equations of \(\equations[\mathbf{SCirc}]\) contain the equations of a
commutative comonoid, so this is a perfect use case for rewriting modulo
trace commutative comonoid structure.
Using graph rewriting, we can sketch out an \emph{operational semantics} for
sequential circuits.
For the interests of brevity, we will only consider circuits of the form \(
    \iltikzfig{circuits/productivity/mealy-form-verbose}
\): circuits with no `non-delay-guarded feedback' in which the registers of the
circuit have been isolated from a core \(
    \iltikzfig{strings/category/f-2-2}[F][comb]
\) containing only `blue' (\emph{combinational}) components, which models a
function.

We can `apply' such a circuit to an input as shown in the left-hand side of
\cref{fig:cycle}; \cite[Thm. 104]{ghica2022compositional} shows that the
equations in \(\equations[\mathbf{SCirc}]\) can be used to derive the right-hand
side.
The equations \eqref{eq:gate}, \eqref{eq:fork}, \eqref{eq:join}, \eqref{eq:stub}
can then be applied to reduce the two `new' cores down to values, which
represent the output and new state of the circuit.

When the circuits are interpreted as hypergraphs and the equations as rewrites,
it would be possible for a computer to perform this sequence of rewrites to
evaluate circuits while being able to `peek inside' and see what is going on.

\begin{remark}
    This is another framework which would benefit from a way of formalising
    subgraphs in rewrite rules.
\end{remark}

\begin{figure*}
    \centering
    \begin{equation*}
        \tag{\(\mathsf{Cycle}\)}
        \iltikzfig{circuits/productivity/productive-lhs-verbose}[F][s][v]
        =
        \iltikzfig{circuits/productivity/productive-step-9}
        \label{eq:cycle}
    \end{equation*}
    \caption{
        The cycle equation, which is derivable from the equations in
        \(\equations[\mathbf{SCirc}]\)
    }
    \label{fig:cycle}
\end{figure*}

    % !TeX root = ../main-conf.tex
\section{Conclusion, related and further work}

\begin{figure}
    \centering
    \includestandalone{figures/graphs/roadmap}
    \caption{The various PROP morphisms at play.}
    \label{fig:roadmap}
\end{figure}

We have shown how the frameworks for rewriting string diagrams modulo
Frobenius~\cite{bonchi2022string} and symmetric
monoidal~\cite{bonchi2022stringa} structure using hypergraphs can also be
adapted for rewriting modulo traced comonoid structure, by using
hypergraphs that sit between the two settings.

Graphical languages for traced categories have seen many applications, such as
to illustrate cyclic lambda calculi~\cite{hasegawa1997recursion}, or to reason
graphically about programs~\cite{schweimeier1999categorical}.
The presentation of traced categories as \emph{string diagrams} has existed
since the 90s~\cite{joyal1991geometry,joyal1996traced}; a soundness and
completeness theorem for traced string diagrams, folklore for many years
but only proven for certain signatures~\cite{selinger2011survey}, was finally
shown in~\cite{kissinger2014abstract}.
Combinatorial languages predate even this, having existed since at least the 80s
in the guise of
\emph{flowchart schemes}~\cite{stefanescu1990feedback,cazanescu1990new,cazanescu1994feedback}.
These diagrams have also been used to show the completeness of finite dimensional
vector spaces~\cite{hasegawa2008finite} with respect to traced categories and,
when equipped with a dagger, Hilbert spaces~\cite{selinger2012finite}.

We are not just concerned with diagrammatic languages as a standalone concept:
we are interested in performing \emph{graph rewriting} with them to reason about
monoidal theories.
This has been been studied in the context of traced categories before using
\emph{string graphs}~\cite{kissinger2012pictures,dixon2013opengraphs}.
We have instead opted to use the \emph{hypergraph} framework
of~\cite{bonchi2022string,bonchi2022stringa,bonchi2022stringb} instead, as it
allows rewriting modulo \emph{yanking}, is more extensible for rewriting modulo
comonoid structure, and one does not need to awkwardly reason modulo wire
homeomorphisms.

As mentioned during the case studies, there are still elements of the rewriting
framework that are somewhat informal.
One such issue involves defining rewrite spans for arbitrary subgraphs: this is
hard to do at a general level because the edges must be concretely specified in
DPO rewriting.
However, if we performed rewriting with
\emph{hierarchical hypergraphs}~\cite{alvarez-picallo2021functorial}, in
which edges can have hypergraphs as labels, we could `compress' the subgraph
into a single edge that can be rewritten: this is future work.

In regular PROP notation, wires are annotated with numbers in order to avoid
drawing multiple wires in parallel: when interpreted as hypergraphs a vertex is
created for each wire, and simple diagrams can quickly get very large.
The results of \cite{bonchi2022stringa} also extend to the multi-sorted case, in
which vertices are labelled in addition to wires.
We could use this in combination with the \emph{strictifiers}
of~\cite{wilson2022string}: these are additional generators for transforming
buses of wires into thinner or thicker ones.
This could drastically reduce the number of elements in a hypergraph, which is
ideal from a computational point of view.
Work has already begun on implementing the rewriting system for digital circuits
using these techniques.

\section*{Acknowledgements}

Thanks to Chris Barrett for helpful comments.

    \iftoggle{conf}{
        \bibliography{refs/refs}
    }{
        \printbibliography
    }

    \appendix

    \iftoggle{conf}{
        \section{Equations}
\label{app:equations}

Equations that hold in any symmetric traced monoidal category are listed in
\cref{fig:stmc-axioms}.
These axioms were originally presented in~\cite{joyal1996traced}: an additional
vanishing axiom was originally also included but it was shown in
\cite{hasegawa2009traced} that this was redundant.

\begin{figure}
    \centering
    \begin{minipage}{0.54\textwidth}
        \begin{align*}
            \tag{\(\mathsf{Tightening}\)}
            \iltikzfig{strings/traced/naturality-lhs}
            &=
            \iltikzfig{strings/traced/naturality-rhs}
            \label{eq:tightening}
            \\[0.75em]
            \tag{\(\mathsf{Sliding}\)}
            \iltikzfig{strings/traced/sliding-lhs}
            &=
            \iltikzfig{strings/traced/sliding-rhs}
            \label{eq:sliding}
            \\[0.75em]
            \tag{\(\mathsf{Vanishing}\)}
            \iltikzfig{strings/traced/vanishing-lhs}
            &=
            \iltikzfig{strings/traced/vanishing-rhs}
            \label{eq:vanishing}
        \end{align*}
    \end{minipage}
    \begin{minipage}{0.45\textwidth}
        \begin{align*}
            \tag{\(\mathsf{Superposing}\)}
            \iltikzfig{strings/traced/superposing-lhs}
            &=
            \iltikzfig{strings/traced/superposing-rhs}
            \label{eq:superposing}
            \\[0.75em]
            \tag{\(\mathsf{Yanking}\)}
            \iltikzfig{strings/traced/yanking-lhs}
            &=
            \iltikzfig{strings/traced/yanking-rhs}
            \label{eq:yanking}
        \end{align*}
    \end{minipage}
    \caption{Equations that hold in any STMC.}
    \label{fig:stmc-axioms}
\end{figure}
    }{}
    % !TeX root = ../../main-conf.tex
\section{Belnap logic}
\label{app:belnap}

The equational theory of gate-level sequential circuits in
\cref{sec:digital-circuits} is based on the four-valued
\emph{Belnap logic}~\cite{belnap1977useful}.
The four values in question, \(\belnapnone\), \(\belnapfalse\), \(\belnaptrue\)
and \(\belnapboth\) form a lattice in two ways.
The first is the lattice of \emph{information order} shown in the left of
\cref{tab:truths}, which dictates how much `information content' a particular value
has.
The second is the lattice of \emph{logical order}, which can be seen as the
lattice in \cref{tab:truths} rotated by 90 degrees clockwise.
This lattice determines the outputs of the gates: the \(\andgate\) gate \(
    \iltikzfig{circuits/components/gates/and}
\) is the meet in this lattice and the \(\orgate\) gate \(
    \iltikzfig{circuits/components/gates/or}
\) is the join.
The \(\notgate\) gate \(
    \iltikzfig{circuits/components/gates/not}
\) acts in the usual way on \(\belnapfalse\) and \(\belnaptrue\), and as the
identity otherwise.
The truth tables are shown in the right of \cref{tab:truths}.

\begin{figure*}
    \centering
    \tikzfig{circuits/a4}
    \quad
    \begin{tabular}{|c|cccc|}
        \hline
        \(\iltikzfig{circuits/components/gates/and}\) & \(\belnapnone\) & \(\belnapfalse\) & \(\belnaptrue\) & \(\belnapboth\) \\
        \hline
        \(\belnapnone\)  & \(\belnapnone\) & \(\belnapfalse\) & \(\belnapnone\) & \(\belnapfalse\) \\
        \(\belnapfalse\) & \(\belnapfalse\) & \(\belnapfalse\) & \(\belnapfalse\) & \(\belnapfalse\) \\
        \(\belnaptrue\) & \(\belnapnone\) & \(\belnapfalse\) & \(\belnaptrue\) & \(\belnapboth\) \\
        \(\belnapboth\) & \(\belnapfalse\) & \(\belnapfalse\) & \(\belnapboth\) & \(\belnapboth\) \\
        \hline
    \end{tabular}
    \quad
    \begin{tabular}{|c|c|}
        \hline
        \(\iltikzfig{circuits/components/gates/not}\) & \\
        \hline
        \(\belnapnone\) & \(\belnapnone\) \\
        \(\belnaptrue\) & \(\belnapfalse\) \\
        \(\belnapfalse\) & \(\belnaptrue\) \\
        \(\belnapboth\) & \(\belnapboth\) \\
        \hline
    \end{tabular}
    \quad
    \begin{tabular}{|c|cccc|}
        \hline
        \(\iltikzfig{circuits/components/gates/or}\) & \(\belnapnone\) & \(\belnapfalse\) & \(\belnaptrue\) & \(\belnapboth\) \\
        \hline
        \(\belnapnone\)  & \(\belnapnone\) & \(\belnapnone\) & \(\belnaptrue\) & \(\belnaptrue\) \\
        \(\belnapfalse\) & \(\belnapnone\) & \(\belnapfalse\) & \(\belnaptrue\) & \(\belnapboth\) \\
        \(\belnaptrue\) & \(\belnaptrue\) & \(\belnaptrue\) & \(\belnaptrue\) & \(\belnaptrue\) \\
        \(\belnapboth\) & \(\belnaptrue\) & \(\belnapboth\) & \(\belnaptrue\) & \(\belnapboth\) \\
        \hline
    \end{tabular}

    \caption{The lattice structure on Belnap values, and the truth tables
    of Belnap logic gates~\cite{belnap1977useful}.}
    \label{tab:truths}
\end{figure*}
    \iftoggle{conf}{
        \section{Omitted proofs}
\label{app:omitted-proofs}

\begin{proof}[Proof of \cref{thm:termtohyp-image}]
    % !TeX root = ../../main-conf.tex
To show that \(\termtohyp[f]{\Sigma}\) is partial monogamous for any
\(f \in \smc{\Sigma}\) we use induction on the structure of \(f\).
Generators, identities and symmetries are partial monogamous, as
semi-monogamicity is preserved by composition, tensor and trace.
So \(\termtohyp[f]{\Sigma}\) is partial monogamous.

Now we show that any partial monogamous cospan \(
    \cospan{m}[f]{F}[g]{n}
\)
must be in the image of \(\termtohyp{\Sigma}\).
To do this, we will now construct a cospan that is isomorphic to
\(\cospan{m}[f]{F}[g]{n}\), but from which it is possible to read off a
unique term in \(\smc{\Sigma}\).
The components of the new cospan are as follows:
\begin{itemize}
    \item let \(L\) be the hypergraph containing vertices with degree
            \((0,0)\) that are not in the image of \(f\) or \(g\);
    \item let \(E\) be the hypergraph containing hyperedges of \(F\) and
            their source and target vertices, but disconnected;
    \item let \(V\) be the discrete hypergraph containing all the
            vertices of \(F\); and
    \item let \(S\) and \(T\) be the discrete hypergraphs containing
            the source and target vertices of hyperedges in \(F\)
            respectively, with the ordering determined by some order
            \(e_1,e_2,\cdots,e_n\) on the edges in \(F\).
\end{itemize}

These parts can be composed and a trace applied to obtain the follow
cospan:
\begin{gather}
    \trace{T}{
        \cospan{T + m}[\id + f]{V}[\id + g]{S + n}
        \,\seq\,
        \cospan{\emptyset + S + n}[\id]{L + E + n}[\id]{\emptyset + T + n}
    }
    \label{gat:cospan}
\end{gather}

This can be checked to be isomorphic to the original cospan
\(\cospan{m}[f]{F}[g]{n}\) by applying the pushouts.
From this we can read off a term in \(\smc{\Sigma}\):
Since the first cospan is monogamous, it corresponds to a term \(
    \iltikzfig{strings/category/f-2-2}[f][white][|\vertices{T}|][m][|\vertices{S}|][n]
\) by \cref{lem:monog-discrete-cospan}.
The second cospan corresponds to \(
    \iltikzfig{strings/category/f}[g][white][|\vertices{S}][\vertices{T}]
    :=
    \bigtensor_{v \in \vertices{L}}
    \iltikzfig{strings/traced/trace-id}
    \tensor
    \bigtensor_{e \in 0 \leq i \leq n}
    \iltikzfig{graphs/isomorphism/label-e}
    \tensor
    \iltikzfig{strings/category/identity}[white][n]
\), where \(\elabel{e}\) is the generator in \(\generators\) that \(e\) is
labelled with.
Putting this all together yields \(
    h := \termtohypsigma[\iltikzfig{graphs/isomorphism/construction}]
\).
While there may be multiple orderings on the edges, the possible terms
are equal by sliding and the naturality of symmetry, so there is one
unique term \(
    \iltikzfig{strings/category/f}[H][white]
\) that corresponds to cospan (\ref{gat:cospan}).

It is clear by definition that \(
    \termtohypsigma[\iltikzfig{strings/category/f}[H][white]]
\) produces (\ref{gat:cospan}), which is isomorphic to the original
cospan \(\cospan{m}[f]{F}[g]{n}\), so it is in the image of
\(\termtohypsigma\).
\end{proof}

\begin{proof}[Proof of \cref{thm:traced-rewriting}]
    First the \((\Rightarrow)\) direction.
If \(
    \iltikzfig{strings/category/f}[g][white]
    \rewrite[\mcr]
    \iltikzfig{strings/category/f}[H][white]
\) then we have \(
    \iltikzfig{strings/category/f}[g][white]
    =
    \iltikzfig{strings/rewriting/rewrite-l}
\) and \(
    \iltikzfig{strings/rewriting/rewrite-r}
    =
    \iltikzfig{strings/category/f}[H][white].
\)
Define the following cospans:
\begin{alignat}{3}
    \label{gath:l-cospan}
    \cospan{0}{L}{i+j}
    &:=
    \termandfrobtohypsigma[\foldinterfaces[\iltikzfig{strings/category/f}[l][white]]]
    &&=
    \termandfrobtohypsigma[\iltikzfig{strings/rewriting/l-folded}]
    \\
    \cospan{0}{R}{i+j}
    &:=
    \termandfrobtohypsigma[\foldinterfaces[\iltikzfig{strings/category/f}[r][white]]]
    &&=
    \termandfrobtohypsigma[\iltikzfig{strings/rewriting/r-folded}]
    \\
    \cospan{0}{G}{m+n}
    &:=
    \termandfrobtohypsigma[\foldinterfaces[\iltikzfig{strings/category/f}[f][white]]]
    &&=
    \termandfrobtohypsigma[\iltikzfig{strings/rewriting/lc-folded}]
    \\
    \label{gath:h-cospan}
    \cospan{0}{H}{m+n}
    &:=
    \termandfrobtohypsigma[\foldinterfaces[\iltikzfig{strings/category/f}[H][white]]]
    &&=
    \termandfrobtohypsigma[\iltikzfig{strings/rewriting/rc-folded}]
    \\
    \cospan{i+j}{C}{m+n}
    &:=
    \termandfrobtohypsigma[\iltikzfig{strings/rewriting/c-folded}]
    &&
\end{alignat}
By functoriality, since \(
    \foldinterfaces[\iltikzfig{strings/category/f}[f][white]]
    =
    \iltikzfig{strings/rewriting/l-folded}
    \seq
    \iltikzfig{strings/rewriting/c-folded}
\) then \[
    \cospan{0}{G}{m+n} = \cospan{0}{L}{i+j} \seq \cospan{i+j}{C}{m+n}.
\]
Cospan composition is pushout, so \(\cospan{L}{G}{C}\) is a pushout.
Using the same reasoning, \(\cospan{R}{G}{C}\) is also a pushout: this
gives us the DPO diagram.
All that remains is to check that the aforementioned pushouts are traced
boundary complements: this follows by inspecting components.

Now the \(\ifdir\) direction: assume we have a DPO diagram (\ref{def:dpo})
where \(L \leftarrow i + j\), \(i + j \rightarrow R\), \(m + n \to G\) and
\(m + n \to H\) are defined as in (\ref{gath:l-cospan}-\ref{gath:h-cospan})
above.
We must show that \(
    \iltikzfig{strings/category/f}[f][white]
    =
    \iltikzfig{strings/rewriting/rewrite-l}
\) and \(
    \iltikzfig{strings/category/f}[H][white]
    =
    \iltikzfig{strings/rewriting/rewrite-r}
\).
By definition of traced boundary complement \(\cospan{j+m}{C}{i+n}\) is a
partial monogamous cospan, so by fullness of \(\termandfrobtohypsigma\),
there exists a term \(
    \iltikzfig{strings/category/f-2-2}[c][white][j][m][i][n]
    \in \smcsigma
\) such that \(
    \termtohyp[
        \iltikzfig{strings/category/f-2-2}[c][white]
    ]{\Sigma}
    =
    \cospan{j+m}{C}{i+n}
\).
By traced decomposition (\cref{lem:traced-decomposition}), we have that for any
traced boundary complement \(\cospan{i+j}{C}{m+n}\) and morphism
\(L \to G\), \(\cospan{m}{G}{n}\) can be factored as in
(\ref{gath:decomposition}), i.e.\ \[
    \termandfrobtohypsigma[\iltikzfig{strings/category/f}[f][white]]
    =
    \trace{j}{\termandfrobtohypsigma[\iltikzfig{strings/category/f}[l][white]]
    \tensor
    \id[n]
    \seq
    \termandfrobtohypsigma[\iltikzfig{strings/category/f-2-2}[c][white]]}.
\]
So by functoriality, we have that \(
    \iltikzfig{strings/category/f}[f][white]
    =
    \iltikzfig{strings/rewriting/rewrite-l}
\).
The same reasoning follows for \(
    \iltikzfig{strings/category/f}[H][white]
    =
    \iltikzfig{strings/rewriting/rewrite-r}
\).
\end{proof}
    }{}
\end{document}